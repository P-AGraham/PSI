\documentclass[10pt, a4paper]{article}

%%%%%%%%%%%%%%
%  Packages  %
%%%%%%%%%%%%%%


\usepackage{page_format}
\usepackage{special}
\usepackage{hyperref}
\usepackage{tikz}
\usepackage[compat=1.1.0]{tikz-feynman}
\input{math_func}

% References
\usepackage{biblatex}
\addbibresource{ref.bib}


%%%%%%%%%%%%
%  Colors  %
%%%%%%%%%%%%
% ! EDIT HERE !
\colorlet{chaptercolor}{red!70!black} % Foreground color.
\colorlet{chaptercolorback}{red!10!white} % Background color


%%%%%%%%%%%%%%
% Page titre %
%%%%%%%%%%%%%%
\title{Homework 1} % Title of the assignement.
\author{\PA} % Your name(s).
\teacher{Gang Xu} % Your teacher's name.
\class{Quantum Field Theory I} % The class title.

\university{Perimeter Institute for Theoretical Physics} % University
\faculty{Perimeter Scholars International} % Faculty
%\departement{<Departement>} % Departement
\date{\today} % Date.


%%%%%%%%%%%%%%%%%%%%%%
% Begin the document %
%%%%%%%%%%%%%%%%%%%%%%
\begin{document}

% Make the title page.
\maketitlepage

% Make table of contents
\maketableofcontents

% Assignment starts here ----------------------------

\section{The Poincaré Algebra}
\begin{enumerate}
  \item[(a)] The Poincaré is the group of transformation of minkowski space that preserve the spacetime interval between all events. This group contains spacetime translations and Lorentz transformation (boosts and rotations). In a coordinate system where events hapenning at $x$ with four-coordinate $x^{\mu}$, translation by a constant four-vector $a$ with components  $a^\mu$ reads $x' = x + a$ ($x^{\mu'} = x^{\mu} + a^{\mu}$). The lorentz transformation $\Lambda$ with components $\Lambda^{\nu}{}_{\mu}$ act as $x' = \Lambda x$ ($x^{\mu'} = \Lambda^{\mu}{}_{\nu} x^{\nu}$, following the matrix multiplication convention $x^\nu$ can be written as a column with $\nu$ as a row index and $\Lambda^{\mu}{}_{\nu}$ as a square matrix with $\mu$ row index and $\nu$ column index). 
  
  We want to find the caracteristic of the unitary operator $U$ representing Poincaré transformation near the identity $\delta$ (with components $\delta^{\mu}{}_{\nu}$). To do this, we write the first order Taylor expansions $\Lambda = \delta + \omega + O(\omega^2)$ and $a = \epsilon$ (exact even for large $\epsilon$) with respect to an infinitesimal Lorentz shift $\omega$ with components $\omega_{\mu \nu}$ (combining infinitesimal rotation angles, and boost angles) and translation $\epsilon$ with components $\epsilon^\mu$. The first order in $\omega$ and $\epsilon$ expansion of the unitary is $U(\delta+\omega, \epsilon)=\mathbf{1}+\frac{i}{2} \omega_{\mu \nu} J^{\mu \nu}+i \epsilon_\mu P^\mu + O(\omega^2, \epsilon^2)$ where $J^{\mu \nu}$, $P^\mu$ are the hermitian matrices generating the Poincaré transformation. 
  
  Since $\Lambda = \delta + \omega$ is a lorentz transformation, we have that it preserves space time intervals. The spacetime interval between events $x$ and $y$ is $(x_\mu-y_\mu)(x^\mu-y^\mu) = x_\mu x^\mu + y_\mu y^\mu - 2 y_\mu x^\mu $. Since the first two terms are themselves spacetime intervals between $x, y$ and $0$, they are individually preserved by a Poincaré transformation. This forces the invariance of the lorentzian product $y_\mu x^\mu$ for any $x, y$ under Poincaré transformations ($x^\mu y_\mu = x^{\mu'} y_\mu'$).
  
  In general, we can Taylor expand $x^{\mu'} y_\mu'$ around $x^\mu y_\mu$ in powers of $\omega$ as 
  \begin{align*}
    x^{\mu'} y_\mu' = (x^\mu + \omega^{\mu \sigma} x_{\sigma} + O(\omega^2))(y_\mu + \omega_{\mu \nu} y^{\nu} + O(\omega^2)) = x^\mu y_\mu + \omega_{\mu \nu}x^{\mu}y^{\nu} + \omega^{\mu \sigma}x_{\sigma} y_{\mu} + O(\omega^2 )
  \end{align*}
  but we know that length is preserved, and the unique Taylor series stops at $O(1)$ forcing all other orders to vanish. This implies that  $0 = \omega_{\mu \nu}x^{\mu}y^{\nu} + \omega_{\mu \nu}x^{\nu} y^{\mu} = (\omega_{\mu \nu} + \omega_{\nu \mu})x^{\mu}y^{\nu}$. This finally implies that $\omega_{\mu \nu} = - \omega_{\nu \mu}$ since it is true for any $x, y$ which shows $\omega_{\nu \mu}$ is antisymmetric. 

  \item[(b)] The unitary $U$ is meant to represent a Poincaré symmetry transformation of the quantum states of a module Hilbert space. In quantum mechanics, A symmetry transformation preserves all the statistical properties of observables $O$. This is equivalent to saying that for any two states $\ket{\phi}$ and $\ket{\psi}$ the quantities $\bra{\phi}O\ket{\psi}$ are left unchanged by the symmetry. We have the following transformation of states $\ket{\psi'} = U\ket{\psi}$ and $\bra{\phi'} =\bra{\phi} U^\dagger$. The transformed $O'$ operator is such that
  \begin{align*}
    \bra{\phi}O\ket{\psi} = \bra{\phi'}O'\ket{\psi'} = \bra{\phi} U^\dagger O' U\ket{\psi},\quad \forall \bra{\phi}, \ket{\psi} \iff  O = U^\dagger O' U \iff O' = U O U^{\dagger}.
  \end{align*}
  \item[(c)] Following the result of the previous item, we take $O = U(\delta + \omega, \epsilon)$ and compute the operator $O'$ associated to the general $U(\Lambda, a)$ unitary Poincaré transformation representing the combined lorentz and translation transformation $T(\Lambda, a)$. With the same notation, we write $T(\delta + \omega, \epsilon)$ to reference the infinitesimal Poincaré transformation.  Because the representation is an homomorphism, we have 
  \begin{align*}
    O' = U(\Lambda, a) U(T(\delta + \omega, \epsilon)) U^{\dagger}(\Lambda, a) = U(T(\Lambda, a)  T(\delta + \omega, \epsilon) T^{-1}(\Lambda, a)). 
  \end{align*}
  To make this expression more precise, we look for $\Lambda'$ and $a'$ such that $T^{-1}(\Lambda, a) = T(\Lambda', a')$. Acting with the identity on an arbitrary four-vector $x$ leads to 
  \begin{align*}
    x &= T^{-1}(\Lambda, a) T(\Lambda, a) x = T(\Lambda', a') T(\Lambda, a) x = \Lambda'(\Lambda x + a) + a' \\
    &= \Lambda'\Lambda x + \Lambda' a + a', \quad \forall x \iff \Lambda'\Lambda = 1\ \& \ \Lambda' a + a' = 0 
  \end{align*}
  With these relations in hand, the action on $x$ of the product Poincaré transformation represented by $O'$ is expanded as follows:
  \begin{align*}
    T(\Lambda, a)  T(\delta + \omega, \epsilon) T^{-1}(\Lambda, a)x &= T(\Lambda, a)  T(\delta + \omega, \epsilon)(\Lambda^{-1} x - \Lambda^{-1} a)\\
    &= T(\Lambda, a) ((\Lambda^{-1} x - \Lambda^{-1} a) + \omega(\Lambda^{-1} x - \Lambda^{-1} a) + \epsilon)\\
    &= T(\Lambda, a) ((\Lambda^{-1} + \omega \Lambda^{-1})  x -\omega \Lambda^{-1} a - \Lambda^{-1} a + \epsilon)\\
    &= ((\delta + \Lambda \omega \Lambda^{-1})  x -\Lambda\omega \Lambda^{-1} a - a + \Lambda \epsilon + a) = T(\delta + \Lambda \omega \Lambda^{-1}, -\Lambda\omega \Lambda^{-1} a + \Lambda \epsilon )x.
  \end{align*}
  This relation holds for all $x$ and we can finally write $O' = U(\delta + \Lambda \omega \Lambda^{-1}, -\Lambda\omega \Lambda^{-1} a + \Lambda \epsilon)$.
\newpage
  \item[(d)] Combining the hermitian generator expansion of $U(\delta + \omega, a)$ (we omit Landau order notation in the next calculations) given in item (a) to the expression for the transformed operator $O'$ of item (c), we find
  \begin{align*}
    O' = U(\Lambda, a) U(\delta+\omega, \epsilon) U^\dagger(\Lambda, a)&=\mathbf{1}+\frac{i}{2} \omega_{\mu \nu} U(\Lambda, a)J^{\mu \nu}U^\dagger(\Lambda, a)+i \epsilon_\mu U(\Lambda, a)P^\mu U^\dagger(\Lambda, a)\\
    &= \mathbf{1}+\frac{i}{2} (\Lambda \omega \Lambda^{-1})_{\mu \nu} J^{\mu \nu}+i (-\Lambda\omega \Lambda^{-1} a + \Lambda \epsilon)_\mu P^\mu\\
    &= \mathbf{1}+\frac{i}{2} (\Lambda_{\mu}{}^{\rho}  \omega_{\rho\sigma} \Lambda_{\nu}{}^{\sigma}) J^{\mu \nu}+i (-\Lambda_{\mu}{}^{\rho}   \Lambda_{\nu}{}^{\sigma} a^{\nu} \omega_{\rho\sigma} + \Lambda_{\mu}{}^{\nu} \epsilon_{\nu}) P^\mu \\
    &= \mathbf{1}+\frac{i}{2} (\Lambda_{\rho}{}^{\mu}  \omega_{\mu\nu} \Lambda_{\sigma}{}^{\nu}) J^{\rho \sigma}+i (-\Lambda_{\rho}{}^{\mu}   \Lambda_{\sigma}{}^{\nu} a^{\sigma} \omega_{\mu\nu} P^\rho +  \epsilon_{\mu} \Lambda_{\nu}{}^{\mu} P^\nu)\\ 
    &= \mathbf{1}+\frac{i}{2} (\Lambda_{\rho}{}^{\mu}  \omega_{\mu\nu} \Lambda_{\sigma}{}^{\nu}) J^{\rho \sigma}+i \epsilon_{\mu} \Lambda_{\nu}{}^{\mu} P^\nu + \frac{i}{2}(-\Lambda_{\rho}{}^{\mu}   \Lambda_{\sigma}{}^{\nu} a^{\sigma} \omega_{\mu\nu} P^\rho +\Lambda_{\rho}{}^{\mu}   \Lambda_{\sigma}{}^{\nu} a^{\sigma} \omega_{\nu\mu} P^\rho)\\ 
    &= \mathbf{1}+\frac{i}{2} (\Lambda_{\rho}{}^{\mu}  \omega_{\mu\nu} \Lambda_{\sigma}{}^{\nu}) J^{\rho \sigma}+i \epsilon_{\mu} \Lambda_{\nu}{}^{\mu} P^\nu + \frac{i}{2}(-\Lambda_{\rho}{}^{\mu}   \Lambda_{\sigma}{}^{\nu} a^{\sigma} \omega_{\mu\nu} P^\rho + \Lambda_{\sigma}{}^{\nu}   \Lambda_{\rho}{}^{\mu} a^{\rho} \omega_{\mu\nu} P^\sigma)\tag{$\star$}
  \end{align*}
  where we expanded the result of item (c) at $O(\omega, \epsilon)$ in the second line and used antisymmetry of $\omega_{\mu\nu}$ in the second last line. To obtain the component representation of the previous result, the vector/matrix multiplication was written and then converted to appropriated index structure:
  \begin{align*}
    &(-\Lambda\omega \Lambda^{-1} a + \Lambda \epsilon)^\mu = -\Lambda^{\mu}{}_{\rho} \omega^{\rho}{}_{\sigma}  (\Lambda^{-1})^{\sigma}{}_{\nu} a^{\nu} + \Lambda^{\mu}{}_{\nu} \epsilon^{\nu}\\ &\iff (-\Lambda\omega \Lambda^{-1} a + \Lambda \epsilon)_\mu = -\Lambda_{\mu}{}^{\rho} \omega_{\rho\sigma}  (\Lambda^{-1})^{\sigma}{}_{\nu} a^{\nu} + \Lambda_{\mu}{}^{\nu} \epsilon_{\nu} =  -\Lambda_{\mu}{}^{\rho} \Lambda_{\nu}{}^{\sigma} a_{\nu} \omega_{\rho\sigma} + \Lambda_{\mu}{}^{\nu} \epsilon_{\nu}\\
    &(\Lambda \omega \Lambda^{-1})^{\mu}{}_{\nu} = \Lambda^{\mu}{}_{\rho}  \omega^{\rho}{}_{\sigma}  (\Lambda^{-1})^{\sigma}{}_{\nu}\\
    &\iff (\Lambda \omega \Lambda^{-1})_{\mu\nu} = \eta_{\mu\lambda} (\Lambda \omega \Lambda^{-1})^{\lambda}{}_{\nu} = \Lambda_{\mu}{}^{\rho}  \omega_{\rho\sigma}  (\Lambda^{-1})^{\sigma}{}_{\nu} = \Lambda_{\mu}{}^{\rho}  \omega_{\rho\sigma} \Lambda_{\nu}{}^{\sigma}.
  \end{align*}
  The inverse transformation components could be related to the direct components because the Lorentz matrices preserve the Lorentzian product of arbitrary $x, y$. Indeed, this property implies 
  \begin{align*}
    &\eta_{\rho \sigma} x^{\rho} y^{\sigma}= \eta_{\mu \nu}(\Lambda^{\mu}{}_{\rho} x^\rho  \Lambda^{\nu}{}_{\sigma} y^{\sigma}), \ \forall x, y \iff \eta_{\rho \sigma}= \eta_{\mu \nu}(\Lambda^{\mu}{}_{\rho} \Lambda^{\nu}{}_{\sigma}) \\
    &\iff \eta_{\rho \sigma} (\Lambda^{-1})^{\sigma}{}_{\lambda}= \eta_{\mu \nu} \Lambda^{\mu}{}_{\rho} \Lambda^{\nu}{}_{\sigma} (\Lambda^{-1})^{\sigma}{}_{\lambda} = \eta_{\mu \lambda} \Lambda^{\mu}{}_{\rho}
    \iff (\Lambda^{-1})^{\nu}{}_{\lambda} = \eta^{\nu \rho}\eta_{\rho \sigma} (\Lambda^{-1})^{\sigma}{}_{\lambda}= \eta_{\mu \lambda} \eta^{\nu \rho} \Lambda^{\mu}{}_{\rho} = \Lambda_{\lambda}{}^{\nu}.
  \end{align*}
  Equality of the first and last of $(\star)$ for all $\omega, \epsilon$ ensures that the tensors contracted with $\omega$ and $\epsilon$ are equal. Regrouping terms proportionnal to $\omega_{\mu\nu}$ and $\epsilon_\mu$ We have 
  \begin{align*}
   U(\Lambda, a) J^\mu U^{\dagger}(\Lambda, a) & =\Lambda_\rho{ }^\mu \Lambda_o{ }^\nu\left(J^{\rho \sigma}+a^\rho P^\sigma-a^\sigma P^\rho\right) \\ U(\Lambda, a) P^\mu U^{\dagger}(\Lambda, a) & =\Lambda_\rho{ }^\mu P^\rho .
  \end{align*}

  \item[(e)] 
  \item[(f)]
  \item[(g)]

\end{enumerate}




\section{Acknowledgement}


% References
\makereferences
%-------------------------------------------------------


%%%%%%%%%%%%%%%%%%%%%%%%
% Terminer le document %
%%%%%%%%%%%%%%%%%%%%%%%%
\end{document}