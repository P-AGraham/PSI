\documentclass[10pt, a4paper]{article}

%%%%%%%%%%%%%%
%  Packages  %
%%%%%%%%%%%%%%


\usepackage{page_format}
\usepackage{special}
\usepackage{hyperref}
\usepackage{tikz}
\usepackage[compat=1.1.0]{tikz-feynman}
\input{math_func}

% References
\usepackage{biblatex}
\addbibresource{ref.bib}


%%%%%%%%%%%%
%  Colors  %
%%%%%%%%%%%%
% ! EDIT HERE !
\colorlet{chaptercolor}{red!70!black} % Foreground color.
\colorlet{chaptercolorback}{red!10!white} % Background color


%%%%%%%%%%%%%%
% Page titre %
%%%%%%%%%%%%%%
\title{Homework 2} % Title of the assignement.
\author{\PA} % Your name(s).
\teacher{Gang Xu} % Your teacher's name.
\class{Quantum Field Theory I} % The class title.

\university{Perimeter Institute for Theoretical Physics} % University
\faculty{Perimeter Scholars International} % Faculty
%\departement{<Departement>} % Departement
\date{\today} % Date.


%%%%%%%%%%%%%%%%%%%%%%
% Begin the document %
%%%%%%%%%%%%%%%%%%%%%%
\begin{document}

% Make the title page.
\maketitlepage

% Make table of contents
\maketableofcontents

% Assignment starts here ----------------------------

\section{The commutator/anti-commutator}
The construction of the spinor field starts with creation (and associated anihilation) operators $(b_\mathbf{p}^s)^\dagger$, $(c_\mathbf{p}^s)^\dagger$ (respectively creating a spinor and an anti-spinor). They act on the free vacuum $\ket{0}$ to create eigenstates (respectively $\ket{\mathbf{p}, s}$ and $\ket{\overline{\mathbf{p}, s}}$) of the space translation generators (with eigenvalues $\mathbf{p}$) and intrinsic $z$ rotation generators (with eigenvalues $s$). Using the Casimir operator (mass $m$ times identity or four-norm of the translation generators) of the Poincare group, we can use the eigenvalues $\mathbf{p}$ to extract the energy $p^0$ to form the four-momentum $p$. 

The creation and anihilation operators can be linearly combined to form spinor components of creation and anihilation fields at four-position $x$ respectively denoted $\psi^{+}_\ell(x)$ and $\psi^{-}_\ell(x)$. The combination is done with coefficients $u_\ell^s(\mathbf{p}) e^{-i p \cdot x}$ and $v_\ell^s(\mathbf{p}) e^{+i p \cdot x}$ with momentum integrated with the Lorentz invariant measure $\text{d}V_\mathbf{p}$ and spin summed. It reads 
\begin{align*}
  \psi^{-}_\ell(x) = \sum_s \int \text{d}V_\mathbf{p} \ u_l^s(\mathbf{p}) e^{-i p \cdot x} b_\mathbf{p}^s  \quad \&\quad \psi^{+}_\ell(x) = \sum_s \int \text{d}V_\mathbf{p} \ v_l^s(\mathbf{p}) e^{+i p \cdot x} (c_\mathbf{p}^s)^\dagger.
\end{align*}
The coefficient of the decomposition of the fields are chosen so that the Poincare transformations of $(b_\mathbf{p}^s)^\dagger$, $(c_\mathbf{p}^s)^\dagger$ are consistent with a representation of the Poincare group acting on the spinor component space. More precisely, this imposes relations between the spinor index transformation of the coefficients and their spin index transformation inherited from operator transformations. Linear independance of the anihilation and creation operators (they create/anihilate a full basis of the Hilbert space) provides a relation for every pair $\mathbf{p}, s$ and each element of the lorentz group acting on the pair. Progress is made by restricting this relation to coefficients evaluated at $\mathbf{p} = 0$ (the same relation links $\mathbf{p} = 0$ to non-zero $\mathbf{p}$ through consistency of boost Poincare transformations) and to rotation transformations for these \textit{rest} coefficients. The fact the $\mathbf{p}=0$ vector transforms trivially under rotation leaves all the interesting transformation properties to the spin index. We get the relations 
\begin{align*}
  \sum_{s'} u_{\ell'}^{s'} \mathbf{J}_{s's}^{j} (R) = \sum_{\ell} \mathcal{J}_{\ell'\ell} (R) u_{\ell}^{s'}\quad \& \quad \sum_{s'} v_{\ell'}^{s'} \mathbf{J}_{s's}^{j\ \star} (R) = -\sum_{\ell} \mathcal{J}_{\ell'\ell} (R) u_{\ell}^{s'}
\end{align*}
where $\mathbf{J}^{j}$ is the matrix representing the effect of rotations on the spin $j$ index and $\mathcal{J}$ is the matrix representing the of rotations on the spinor index.  Here $\mathbf{J}^j$ is restricted to be the irreductible representation of complex dimension $2j+1$ of $\text{SU}(2)$ (because the created particles have spin the same $j$). Using matrix notation, the relation for $u_{m, \pm}^{s}$ becomes 
$
  \mathbf{U}\mathbf{J}^{j} = \mathcal{J}\mathbf{U}
$
with components $[\mathbf{U}] = u_\ell^s(0)$. 

By Schur's lemma, $\mathbf{U}$ is either a square matrix or the representation $\mathcal{J}$ is reducible. If $\mathbf{U}$ was a square matrix, the spin index would have the same dimensionnality as the spinor index leading to $2j+1=4 \iff j = 3/2$. If $\mathcal{J}$ is reducible, we can find a basis where it is block diagonal with blocks $\mathcal{J}_+$, $\mathcal{J}_-$ (we select the Weyl block diagonal representation). The relation for $\pm$ blocks reads 
$
  \mathbf{U}_{\pm}\mathbf{J}^{j} = \mathcal{J}_\pm \mathbf{U}_{\pm}
$
where $\mathbf{U}_{\pm}$ restricts the $\mathbf{U}$ matrix to the blocks of the reducible representation $\mathcal{J}$. 

Now, Schur's Lemma implies that $\mathbf{U}_{\pm}$ are square matrices and this forces $2(2j+1) = 4 \iff j = 1/2$ consistent with spin $1/2$. For the Dirac representation, we get
$
  \mathbf{U}_{\pm}\mathbf{\sigma} = \mathbf{\sigma} \mathbf{U}_{\pm}
$
where $\mathbf{\sigma}$ is any pauli matrix. The only matrix $U_{\pm}$ that commutes with all the pauli matrices is a constant $c_\pm$ multiple of the identity. Restoring the internal indices of $\mathbf{U}_{\pm}$ ($s$ spin index and $m$ internal block index) leads to $u_{m, \pm}^{s} = c_\pm \delta_{ms}$ and similar considerations lead to $v_{m, \pm}^{s} = -id_\pm (\sigma_2)_{ms}$ where $d_\pm$ is another proportionality constant. 

% $\beta = i \gamma^0$
Finally, we impose that the parity transformation of the spinor fields be compatible with the parity transformation of its constituant creation/anihilation operator (inherited from the parity transformation of the states). This constraint allows to make $c_\pm, d_\pm$ more precise by reducing them to sign factors $b_u, b_v$ (with $b_u^2 = b_v^2 = 1$). Explicitely, we can write 
\begin{align*}
  &u^{(1/2)}(0) = 
  \begin{pmatrix}
    1& 0& b_u& 0
  \end{pmatrix}^T,\quad
   u^{(-1/2)}(0) = 
  \begin{pmatrix}
    1& 0& 0& b_u
  \end{pmatrix}^T\\
  &v^{(1/2)}(0) = 
  \begin{pmatrix}
    1& 0& 0& b_v
  \end{pmatrix}^T,\quad
   v^{(-1/2)}(0) = 
  \begin{pmatrix}
    1& 0& b_v& 0
  \end{pmatrix}^T.
\end{align*}
\newpage
\begin{enumerate}
  \item[(a)] The next step in the construction of the spinor field consists in combining the anihilation and creation field to produce a total field $\psi_\ell(x) = \kappa \psi^-(x) + \lambda \psi^+(x)$ with $\kappa, \lambda \in \mathbb{C}$. Causality requires that the commutator/anti-commutator (respectively denoted $[\circ,\circ]_-$ and $[\circ,\circ]_+$) of the field at $x$ with its adjoint at $y$ vanishes for spacelike $x-y$. The commutator/anti-commutator is computed in general by using the fundamental commutator/anti-commutator
  \begin{align*}
    [b^{s}_\mathbf{p}, (b^{s'}_\mathbf{q})^\dagger]_{\pm} = E_\mathbf{q} (2\pi)^3 \delta(\mathbf{p} - \mathbf{q})\delta_{ss'} \quad \& \quad [c^{s}_\mathbf{p}, (c^{s'}_\mathbf{q})^\dagger]_{\pm} = E_\mathbf{q} (2\pi)^3 \delta(\mathbf{p} - \mathbf{q})\delta_{ss'} 
  \end{align*}
  with all other combinations having $[\circ,\circ]_\pm = 0$. We have the expansion 
  \begin{align*}
    &[\kappa \psi^-_\ell(x) + \lambda \psi^+_\ell(x), \kappa^\star (\psi^-(y))_{\ell'}^{\dagger} + \lambda^\star (\psi^+(y))_{\ell'}^\dagger]_{\pm}\\
    &= [\kappa \psi^-_{\ell}(x), \kappa^\star (\psi^-_{\ell'}(y))^{\dagger}]_{\pm} + [\kappa \psi^-_{\ell}(x),\lambda^\star (\psi^+_{\ell'}(y))^\dagger]_{\pm} + [\lambda \psi^+_{\ell}(x), \kappa^\star (\psi^-_{\ell'}(y))^{\dagger}]_{\pm} + [\lambda \psi^+_{\ell}(x), \lambda^\star (\psi^+(y)_{\ell'})^\dagger]_{\pm}\\
    &=|\kappa|^2[\psi^-_{\ell}(x), (\psi^-_{\ell'}(y))^{\dagger}]_{\pm} + |\lambda|^2[ \psi^+_{\ell}(x), (\psi^+_{\ell'}(y))^\dagger]_{\pm} +\sim \left[ [b,c]_{\pm} + [c^\dagger, b^{\dagger}]_{\pm}\right]\\
    &= 
  \end{align*}
  \item[(b)]
  \item[(c)]
  \item[(d)] 
  \item[(e)]
  \item[(f)]
  \item[(g)]
  \item[(h)]     
\end{enumerate}

\section{Causality condition}
\begin{enumerate}
  \item[(a)]
  \item[(b)]
  \item[(c)]
  \item[(d)] 
  \item[(e)]   
\end{enumerate}


\section{Acknowledgement}



% References
\makereferences
%-------------------------------------------------------


%%%%%%%%%%%%%%%%%%%%%%%%
% Terminer le document %
%%%%%%%%%%%%%%%%%%%%%%%%
\end{document}