\documentclass[10pt, a4paper]{article}

%%%%%%%%%%%%%%
%  Packages  %
%%%%%%%%%%%%%%


\usepackage{page_format}
\usepackage{special}
\usepackage{hyperref}
\usepackage{tikz}
\usepackage[compat=1.1.0]{tikz-feynman}
%----------------------------------------------------------------------
%\usepackage{amssymb} % Mathematical fonts.
%\usepackage{amsfonts} % Mathematical fonts.
\usepackage[nice]{nicefrac} % Nicer fractions
\usepackage{braket} % Dirac Notation.
\usepackage{bbm} % More bold fonts.
%\usepackage{mathrsfs} % Mathematical fonts.
\usepackage{esint} % Integrals
\usepackage{cancel} % Allows to scratch expressions.
\usepackage{mathtools} % Tools for math formating.
\usepackage{slashed} % Allows to slash individual characters.
\usepackage{xargs} % Better handling of optional arguments for commands
%----------------------------------------------------------------------
%\usepackage{lmodern} % Fonts.
\usepackage{feyn} % Feynman Diagrams in mathmode

%%%%%%%%%%%%%%%%%%%%%%%%%%%
% Mathématiques et physique
%%%%%%%%%%%%%%%%%%%%%%%%%%%%
% SI Units -----------------------
% The package 'siunitx' causes unresolved crashes (as of 22/08/31)
\newcommand{\ampere}{\text{A}}
\newcommand{\bell}{\text{B}}
\newcommand{\celsius}{\degree\text{C}}
\newcommand{\coulomb}{\text{C}}
\newcommand{\degree}{\,^{\circ}}
\newcommand{\farad}{\text{F}}
\newcommand{\electro}{\text{e}}
\newcommand{\gram}{\text{g}}
\newcommand{\henry}{\text{H}}
\newcommand{\hertz}{\text{Hz}}
\newcommand{\hour}{\text{h}}
\newcommand{\joule}{\text{J}}
\newcommand{\kelvin}{\text{K}}
\newcommand{\meter}{\text{m}}
\newcommand{\minute}{\text{m}}
\newcommand{\mole}{\text{mol}}
\newcommand{\newton}{\text{N}}
\newcommand{\ohm}{\Omega}
\newcommand{\pascal}{\text{Pa}}
\newcommand{\rad}{\text{rad}}
\newcommand{\second}{\text{s}}
\newcommand{\tesla}{\text{T}}
\newcommand{\torr}{\text{Torr}}
\newcommand{\volt}{\text{V}}
\newcommand{\watt}{\text{W}}
%
\newcommand{\tera}{\text{T}}
\newcommand{\giga}{\text{G}}
\newcommand{\mega}{~\text{M}}
\newcommand{\kilo}{~\text{k}}
\newcommand{\deci}{\text{d}}
\newcommand{\centi}{\text{c}}
\newcommand{\milli}{\text{m}}
\newcommand{\micro}{\mu}
\newcommand{\nano}{\text{n}}
\newcommand{\pico}{\text{p}}
\newcommand{\femto}{\text{f}}
%
\newcommand{\units}[1]{\text{#1}}
\newcommand{\tothe}[1]{\textsuperscript{#1}}
%
\newcommand{\per}{\text{/}}
%
\newcommand{\Time}[3]{#1\hour~#2\minute~#3\second} % TODO Optional arguments.
\newcommand{\Angle}[3]{#1^{\circ}~#2'~#3''} % TODO Optional arguments.


% Better epsilon -----------------------
\let\oldepsilon\epsilon
\let\epsilon\varepsilon
\let\varepsilon\oldepsilon


% Better \bar -----------------------
\renewcommand{\bar}[1]{\mkern 1.5mu\overline{\mkern-1.5mu#1\mkern-1.5mu}\mkern 1.5mu}


% Équations -----------------------
\newcommand{\al}[1]{\begin{align} #1 \end{align}} % Numbered equation(s),
\newcommand{\eqn}[1]{\begin{align*} #1 \end{align*}} % Number-less equation(s),
\newcommand{\sys}[1]{\begin{dcases*} #1 \end{dcases*}} % System of equations.


% Exponents -----------------------
\newcommand{\Exp}[1]{\text{e}^{#1}}		% e^#
\newcommand{\E}[1]{\times 10^{#1}}		% X 10^#


% Delimiters -----------------------
\newcommand{\p}[1]{\left( #1 \right)}	% (#)
\newcommand{\cro}[1]{\left[ #1 \right]}	% [#]
\newcommand{\abs}[1]{\left| #1\right|}	% |#|
\newcommand{\avg}[1]{\left\langle #1 \right\rangle} % <#>
\newcommand{\acc}[1]{\left\lbrace #1 \right\rbrace} % {#}


% Vectors -----------------------
\newcommand{\ve}[1]{\mathbf{#1}} % Upright bold face.
\newcommand{\vu}[1]{\hat{\ve{#1}}} % Hat vector upright bold face
\newcommand{\tens}{\otimes} % Tensor product
\newcommand{\nablav}{\bm{\nabla}} % Bold gradient


% Trig. functions with automatic formating  -----------------------
\newcommandx{\Sin}[2][1={}]{\text{sin}^{#1}\!\p{#2}}
\newcommandx{\Cos}[2][1={}]{\text{cos}^{#1}\!\p{#2}}
\newcommandx{\Tan}[2][1={}]{\text{tan}^{#1}\!\p{#2}}
\newcommandx{\Csc}[2][1={}]{\text{csc}^{#1}\!\p{#2}}
\newcommandx{\Sec}[2][1={}]{\text{sec}^{#1}\!\p{#2}}
\newcommandx{\Cot}[2][1={}]{\text{cot}^{#1}\!\p{#2}}
\newcommandx{\Arcsin}[2][1={}]{\text{arcsin}^{#1}\!\p{#2}}
\newcommandx{\Arccos}[2][1={}]{\text{arccos}^{#1}\!\p{#2}}
\newcommandx{\Arctan}[2][1={}]{\text{arctan}^{#1}\!\p{#2}}
\newcommandx{\Sinh}[2][1={}]{\text{sinh}^{#1}\!\p{#2}}
\newcommandx{\Cosh}[2][1={}]{\text{cosh}^{#1}\!\p{#2}}
\newcommandx{\Tanh}[2][1={}]{\text{tanh}^{#1}\!\p{#2}}


% Matrices -----------------------
\newcommand{\mat}[1]{\begin{bmatrix} #1 \end{bmatrix}} % Matrices with hooks.
\newcommand{\pmat}[1]{\begin{pmatrix} #1 \end{pmatrix}} % Matrices with parentheses.
\newcommand{\deter}[1]{\abs{\begin{matrix} #1 \end{matrix}}} % Determinant.
\newcommandx{\mO}[2][1={}, 2={}]{ \def\temp{#2}\ifx\temp\empty\ve{O}_{#1}\else\ve{O}_{#1\times #2}\fi}% Zero matrix.
\newcommandx{\mI}[2][1={}, 2={}]{ \def\temp{#2}\ifx\temp\empty\ve{I}_{#1}\else\ve{O}_{#1\times #2}\fi}%  Identity matrix.
\newcommand{\Det}[1]{\text{det}\p{#1}} % det(#)
\newcommand{\Tr}[1]{\text{Tr}\p{#1}} % Tr(#)


% Derivatives -----------------------
\newcommand{\D}{\text{d}} % Differential 'd'.
\newcommandx{\dd}[3][1={},3={}]{\frac{\D^{#3}#1}{\D{#2}^{#3}}} % Total derivative according to #2, #1 is the function and #3 is the order.
\newcommand{\del}{\partial} % Partial 'd'.
\newcommandx{\ddp}[3][1={},3={}]{\frac{\del^{#3}#1}{\del{#2}^{#3}}} % Dérivée partielle selon #2, #1 est la fonction est #3 est l'ordre.
\newcommand{\eval}[1]{\left. {#1} \right|} % Bar on the right of expression.
\newcommand{\delbar}{\slashed{\del}} % Partial Inexact differential.
\newcommand{\dbar}{\dj}% Inexact differential.


% Integrals -----------------------
\newcommand{\intinf}{\int\displaylimits_{-\infty}^{\infty}} % From -00 to 00.
\newcommandx{\Int}[2][1={},2={}]{\int\displaylimits_{#1}^{#2}} % Faster bounded integrals.


% Complex numbers -----------------------
\renewcommand{\Re}[1]{\text{Re}\acc{#1}} % Re{#}
\renewcommand{\Im}[1]{\text{Im}\acc{#1}} % Im{#}


% Sets -----------------------
\newcommand{\N}{\mathbbm{N}} % Natural numbers.
\newcommand{\Z}{\mathbbm{Z}} % Integers.
\newcommand{\Q}{\mathbbm{Q}} % Rational numbers.
\newcommandx{\R}[1][1={}]{\mathbbm{R}^{#1}} % Real numbers.
\newcommandx{\C}[1][1={}]{\mathbbm{C}^{#1}} % Complex numbers.
\newcommandx{\F}[1][1={}]{\mathbbm{F}^{#1}} % Some field.
\newcommand{\M}[3]{\mathbb{M}_{#1\times#2}(#3)}	% Matrices.
\newcommand{\Po}[2]{\mathbb{P}_{#1}(#2)} % Polynomials.
\newcommand{\Lin}{\mathbb{L}} % Linear maps.


% Constants and physical symbols -----------------------
\newcommand{\eo}{\epsilon_0} % epsilon 0.
\renewcommand{\L}{\mathcal{L}} % Lagrangian.

% References
\usepackage{biblatex}
\addbibresource{ref.bib}


%%%%%%%%%%%%
%  Colors  %
%%%%%%%%%%%%
% ! EDIT HERE !
\colorlet{chaptercolor}{red!70!black} % Foreground color.
\colorlet{chaptercolorback}{red!10!white} % Background color


%%%%%%%%%%%%%%
% Page titre %
%%%%%%%%%%%%%%
\title{Homework 1} % Title of the assignement.
\author{\PA} % Your name(s).
\teacher{Gang Xu} % Your teacher's name.
\class{Quantum Field Theory I} % The class title.

\university{Perimeter Institute for Theoretical Physics} % University
\faculty{Perimeter Scholars International} % Faculty
%\departement{<Departement>} % Departement
\date{\today} % Date.


%%%%%%%%%%%%%%%%%%%%%%
% Begin the document %
%%%%%%%%%%%%%%%%%%%%%%
\begin{document}

% Make the title page.
\maketitlepage

% Make table of contents
\maketableofcontents

% Assignment starts here ----------------------------

\section{The Poincaré Algebra}
\begin{enumerate}
  \item[(a)] The Poincaré is the group of transformation of Minkowski space that preserves the spacetime interval between all events. This group contains spacetime translations and Lorentz transformation (boosts and rotations). In a coordinate system where events hapenning at $x$ with four-coordinate $x^{\mu}$, translation by a constant four-vector $a$ with components  $a^\mu$ reads $x' = x + a$ ($x^{\mu'} = x^{\mu} + a^{\mu}$). The lorentz transformation $\Lambda$ with components $\Lambda^{\nu}{}_{\mu}$ act as $x' = \Lambda x$ ($x^{\mu'} = \Lambda^{\mu}{}_{\nu} x^{\nu}$, following the matrix multiplication convention $x^\nu$ can be written as a column with $\nu$ as a row index and $\Lambda^{\mu}{}_{\nu}$ as a square matrix with $\mu$ row index and $\nu$ column index). We want to find the characteristic of the unitary operator $U$ representing Poincaré transformation near the identity $\delta$ (with components $\delta^{\mu}{}_{\nu}$). To do this, we write the first order Taylor expansions $\Lambda = \delta + \omega + O(\omega^2)$ and $a = \epsilon$ (exact even for large $\epsilon$) with respect to an infinitesimal Lorentz shift $\omega$ with components $\omega_{\mu \nu}$ (combining infinitesimal rotation angles and boost angles) and translation $\epsilon$ with components $\epsilon^\mu$. The first order in $\omega$ and $\epsilon$ expansion of the unitary is $U(\delta+\omega, \epsilon)=\mathbf{1}+\frac{i}{2} \omega_{\mu \nu} J^{\mu \nu}+i \epsilon_\mu P^\mu + O(\omega^2, \epsilon^2)$ where $J^{\mu \nu}$, $P^\mu$ are the hermitian matrices generating the Poincaré transformation. Since $\Lambda = \delta + \omega$ is a Lorentz transformation, we have that it preserves space-time intervals. The spacetime interval between events $x$ and $y$ is $(x_\mu-y_\mu)(x^\mu-y^\mu) = x_\mu x^\mu + y_\mu y^\mu - 2 y_\mu x^\mu $. Since the first two terms are themselves spacetime intervals between $x, y$, and $0$, they are individually preserved by a Poincaré transformation. This forces the invariance of the lorentzian product $y_\mu x^\mu$ for any $x, y$ under Poincaré transformations ($x^\mu y_\mu = x^{\mu'} y_\mu'$).
  
  In general, we can Taylor expand $x^{\mu'} y_\mu'$ around $x^\mu y_\mu$ in powers of $\omega$ as 
  \begin{align*}
    x^{\mu'} y_\mu' = (x^\mu + \omega^{\mu \sigma} x_{\sigma} + O(\omega^2))(y_\mu + \omega_{\mu \nu} y^{\nu} + O(\omega^2)) = x^\mu y_\mu + \omega_{\mu \nu}x^{\mu}y^{\nu} + \omega^{\mu \sigma}x_{\sigma} y_{\mu} + O(\omega^2 )
  \end{align*}
  but we know that length is preserved, and the unique Taylor series stops at $O(1)$ forcing all other orders to vanish. This implies that  $0 = \omega_{\mu \nu}x^{\mu}y^{\nu} + \omega_{\mu \nu}x^{\nu} y^{\mu} = (\omega_{\mu \nu} + \omega_{\nu \mu})x^{\mu}y^{\nu}$. This finally implies that $\omega_{\mu \nu} = - \omega_{\nu \mu}$ since it is true for any $x, y$ which shows $\omega_{\nu \mu}$ is antisymmetric. 

  \item[(b)] The unitary $U$ is meant to represent a Poincaré symmetry transformation of the quantum states of a module Hilbert space. In quantum mechanics, A symmetry transformation preserves all the statistical properties of observables $O$. This is equivalent to saying that for any two states $\ket{\phi}$ and $\ket{\psi}$ the quantities $\bra{\phi}O\ket{\psi}$ are left unchanged by the symmetry. We have the following transformation of states $\ket{\psi'} = U\ket{\psi}$ and $\bra{\phi'} =\bra{\phi} U^\dagger$. The transformed $O'$ operator is such that
  \begin{align*}
    \bra{\phi}O\ket{\psi} = \bra{\phi'}O'\ket{\psi'} = \bra{\phi} U^\dagger O' U\ket{\psi},\quad \forall \bra{\phi}, \ket{\psi} \iff  O = U^\dagger O' U \iff O' = U O U^{\dagger}.
  \end{align*}
  \item[(c)] Following the result of the previous item, we take $O = U(\delta + \omega, \epsilon)$ and compute the operator $O'$ associated to the general $U(\Lambda, a)$ unitary Poincaré transformation representing the combined Lorentz and translation transformation $T(\Lambda, a)$. With the same notation, we write $T(\delta + \omega, \epsilon)$ to reference the infinitesimal Poincaré transformation.  Because the representation is a homomorphism, we have 
  \begin{align*}
    O' = U(\Lambda, a) U(T(\delta + \omega, \epsilon)) U^{\dagger}(\Lambda, a) = U(T(\Lambda, a)  T(\delta + \omega, \epsilon) T^{-1}(\Lambda, a)). 
  \end{align*}
  To make this expression more precise, we look for $\Lambda'$ and $a'$ such that $T^{-1}(\Lambda, a) = T(\Lambda', a')$. Acting with the identity on an arbitrary four-vector $x$ leads to 
  \begin{align*}
    x &= T^{-1}(\Lambda, a) T(\Lambda, a) x = T(\Lambda', a') T(\Lambda, a) x = \Lambda'(\Lambda x + a) + a' \\
    &= \Lambda'\Lambda x + \Lambda' a + a', \quad \forall x \iff \Lambda'\Lambda = 1\ \& \ \Lambda' a + a' = 0 
  \end{align*}
  With these relations in hand, the action on $x$ of the product Poincaré transformation represented by $O'$ is expanded as follows:
  \begin{align*}
    T(\Lambda, a)  T(\delta + \omega, \epsilon) T^{-1}(\Lambda, a)x &= T(\Lambda, a)  T(\delta + \omega, \epsilon)(\Lambda^{-1} x - \Lambda^{-1} a)\\
    &= T(\Lambda, a) ((\Lambda^{-1} x - \Lambda^{-1} a) + \omega(\Lambda^{-1} x - \Lambda^{-1} a) + \epsilon)\\
    &= T(\Lambda, a) ((\Lambda^{-1} + \omega \Lambda^{-1})  x -\omega \Lambda^{-1} a - \Lambda^{-1} a + \epsilon)\\
    &= ((\delta + \Lambda \omega \Lambda^{-1})  x -\Lambda\omega \Lambda^{-1} a - a + \Lambda \epsilon + a) = T(\delta + \Lambda \omega \Lambda^{-1}, -\Lambda\omega \Lambda^{-1} a + \Lambda \epsilon )x.
  \end{align*}
  This relation holds for all $x$ and we can finally write $O' = U(\delta + \Lambda \omega \Lambda^{-1}, -\Lambda\omega \Lambda^{-1} a + \Lambda \epsilon)$.
\newpage
  \item[(d)] Combining the hermitian generator expansion of $U(\delta + \omega, a)$ (we omit Landau order notation in the next calculations) given in item (a) to the expression for the transformed operator $O'$ of item (c), we find
  \begin{align*}
    O' &= U(\Lambda, a) U(\delta+\omega, \epsilon) U^\dagger(\Lambda, a)\tag{$\star$}\\&=\mathbf{1}+\frac{i}{2} \omega_{\mu \nu} U(\Lambda, a)J^{\mu \nu}U^\dagger(\Lambda, a)+i \epsilon_\mu U(\Lambda, a)P^\mu U^\dagger(\Lambda, a)\\
    &= \mathbf{1}+\frac{i}{2} (\Lambda \omega \Lambda^{-1})_{\mu \nu} J^{\mu \nu}+i (-\Lambda\omega \Lambda^{-1} a + \Lambda \epsilon)_\mu P^\mu\\
    &= \mathbf{1}+\frac{i}{2} (\Lambda_{\mu}{}^{\rho}  \omega_{\rho\sigma} \Lambda_{\nu}{}^{\sigma}) J^{\mu \nu}+i (-\Lambda_{\mu}{}^{\rho}   \Lambda_{\nu}{}^{\sigma} a^{\nu} \omega_{\rho\sigma} + \Lambda_{\mu}{}^{\nu} \epsilon_{\nu}) P^\mu \\
    &= \mathbf{1}+\frac{i}{2} (\Lambda_{\rho}{}^{\mu}  \omega_{\mu\nu} \Lambda_{\sigma}{}^{\nu}) J^{\rho \sigma}+i (-\Lambda_{\rho}{}^{\mu}   \Lambda_{\sigma}{}^{\nu} a^{\sigma} \omega_{\mu\nu} P^\rho +  \epsilon_{\mu} \Lambda_{\nu}{}^{\mu} P^\nu)\\ 
    &= \mathbf{1}+\frac{i}{2} (\Lambda_{\rho}{}^{\mu}  \omega_{\mu\nu} \Lambda_{\sigma}{}^{\nu}) J^{\rho \sigma}+i \epsilon_{\mu} \Lambda_{\nu}{}^{\mu} P^\nu + \frac{i}{2}(-\Lambda_{\rho}{}^{\mu}   \Lambda_{\sigma}{}^{\nu} a^{\sigma} \omega_{\mu\nu} P^\rho +\Lambda_{\rho}{}^{\mu}   \Lambda_{\sigma}{}^{\nu} a^{\sigma} \omega_{\nu\mu} P^\rho)\\ 
    &= \mathbf{1}+\frac{i}{2} (\Lambda_{\rho}{}^{\mu}  \omega_{\mu\nu} \Lambda_{\sigma}{}^{\nu}) J^{\rho \sigma}+i \epsilon_{\mu} \Lambda_{\rho}{}^{\mu} P^\rho + \frac{i}{2}(-\Lambda_{\rho}{}^{\mu}   \Lambda_{\sigma}{}^{\nu} a^{\sigma} \omega_{\mu\nu} P^\rho + \Lambda_{\sigma}{}^{\nu}   \Lambda_{\rho}{}^{\mu} a^{\rho} \omega_{\mu\nu} P^\sigma)
  \end{align*}
  where we expanded the result of item (c) at $O(\omega, \epsilon)$ in the third line and used antisymmetry of $\omega_{\mu\nu}$ in the second last line. To obtain the component representation of the previous result, the vector/matrix multiplication was written and then converted to the appropriated index structure:
  \begin{align*}
    &(-\Lambda\omega \Lambda^{-1} a + \Lambda \epsilon)^\mu = -\Lambda^{\mu}{}_{\rho} \omega^{\rho}{}_{\sigma}  (\Lambda^{-1})^{\sigma}{}_{\nu} a^{\nu} + \Lambda^{\mu}{}_{\nu} \epsilon^{\nu}\\ &\iff (-\Lambda\omega \Lambda^{-1} a + \Lambda \epsilon)_\mu = -\Lambda_{\mu}{}^{\rho} \omega_{\rho\sigma}  (\Lambda^{-1})^{\sigma}{}_{\nu} a^{\nu} + \Lambda_{\mu}{}^{\nu} \epsilon_{\nu} =  -\Lambda_{\mu}{}^{\rho} \Lambda_{\nu}{}^{\sigma} a_{\nu} \omega_{\rho\sigma} + \Lambda_{\mu}{}^{\nu} \epsilon_{\nu}\\
    &(\Lambda \omega \Lambda^{-1})^{\mu}{}_{\nu} = \Lambda^{\mu}{}_{\rho}  \omega^{\rho}{}_{\sigma}  (\Lambda^{-1})^{\sigma}{}_{\nu}\\
    &\iff (\Lambda \omega \Lambda^{-1})_{\mu\nu} = \eta_{\mu\lambda} (\Lambda \omega \Lambda^{-1})^{\lambda}{}_{\nu} = \Lambda_{\mu}{}^{\rho}  \omega_{\rho\sigma}  (\Lambda^{-1})^{\sigma}{}_{\nu} = \Lambda_{\mu}{}^{\rho}  \omega_{\rho\sigma} \Lambda_{\nu}{}^{\sigma}.
  \end{align*}
  The inverse transformation components could be related to the direct components because the Lorentz matrices preserve the Lorentzian product of arbitrary $x, y$. Indeed, this property implies 
  \begin{align*}
    &\eta_{\rho \sigma} x^{\rho} y^{\sigma}= \eta_{\mu \nu}(\Lambda^{\mu}{}_{\rho} x^\rho  \Lambda^{\nu}{}_{\sigma} y^{\sigma}), \ \forall x, y \iff \eta_{\rho \sigma}= \eta_{\mu \nu}(\Lambda^{\mu}{}_{\rho} \Lambda^{\nu}{}_{\sigma}) \\
    &\iff \eta_{\rho \sigma} (\Lambda^{-1})^{\sigma}{}_{\lambda}= \eta_{\mu \nu} \Lambda^{\mu}{}_{\rho} \Lambda^{\nu}{}_{\sigma} (\Lambda^{-1})^{\sigma}{}_{\lambda} = \eta_{\mu \lambda} \Lambda^{\mu}{}_{\rho}
    \iff (\Lambda^{-1})^{\nu}{}_{\lambda} = \eta^{\nu \rho}\eta_{\rho \sigma} (\Lambda^{-1})^{\sigma}{}_{\lambda}= \eta_{\mu \lambda} \eta^{\nu \rho} \Lambda^{\mu}{}_{\rho} = \Lambda_{\lambda}{}^{\nu}.
  \end{align*}
  Equality of the second and last lines of $(\star)$ for all $\omega, \epsilon$ ensures that the tensors contracted with $\omega$ and $\epsilon$ are equal. Regrouping terms proportional to $\omega_{\mu\nu}$ and $\epsilon_\mu$ We have 
  \begin{align*}
   &U(\Lambda, a) J^{\mu\nu} U^{\dagger}(\Lambda, a) =\Lambda_\rho{ }^\mu \Lambda_\rho{ }^\nu\left(J^{\rho \sigma}+a^\rho P^\sigma-a^\sigma P^\rho\right), \\ &U(\Lambda, a) P^\mu U^{\dagger}(\Lambda, a) =\Lambda_\rho{ }^\mu P^\rho .
  \end{align*}

  \item[(e)] To extract the commutation relations of the generators $J^{\mu \nu}, P^{\mu}$, we start by setting $\Lambda = \delta + \omega$, $a = \epsilon$ in the final result of item (d). At first order in $\omega, \epsilon$, we get
  \begin{align*}
    P^\mu + \eta^{\mu \sigma}\omega_{\rho \sigma} P^\rho = \Lambda_\rho{ }^\mu P^\rho = U(\Lambda, a) P^\mu U^{\dagger}(\Lambda, a) &= \left(\mathbf{1}+\frac{i}{2} \omega_{\lambda \epsilon} J^{\lambda \epsilon}+i \epsilon_\nu P^\nu\right) P^\mu \left(\mathbf{1}-\frac{i}{2} \omega_{\rho \sigma} J^{\rho \sigma}-i \epsilon_\sigma P^\sigma\right)\\
    &= P^\mu + \frac{i}{2} \left(\omega_{\lambda \epsilon} J^{\lambda \epsilon}P^{\mu} - P^{\mu} \omega_{\rho \sigma}  J^{\rho \sigma}\right)+i (\epsilon_\nu P^\nu P^\mu - P^\mu \epsilon_\nu P^\nu)\\
    &= P^\mu + \frac{i}{2} \omega_{\rho \sigma} [J^{\rho \sigma}, P^{\mu}]  +i \epsilon_\nu [P^{\nu}, P^{\mu}].
   \end{align*}
   where $U^{\dagger}(\Lambda, a) = U((\delta + \omega)^{-1}, -(\delta + \omega)^{-1} a) = U(\delta - \omega, -\epsilon + \omega \epsilon) = U(\delta - \omega, -\epsilon)$ was used. Invoking the validity of the last set of equalities for all $\omega, \epsilon$, we find: 
    \begin{align*}
      [P^{\nu}, P^{\mu}] = 0 \quad \& \quad \dfrac{i}{2}[J^{\rho \sigma}, P^{\mu}] =  \dfrac{1}{2}\left(\eta^{\mu \sigma}P^\rho- \eta^{\mu \rho} P^\sigma\right).
    \end{align*}
    with the antisymmetry of $\omega$ used to write $$2\eta^{\mu \sigma}\omega_{\rho \sigma} P^\rho = \eta^{\mu \sigma}\omega_{\rho \sigma} P^\rho - \eta^{\mu \sigma}\omega_{\sigma \rho} P^\rho = \eta^{\mu \sigma}\omega_{\rho \sigma} P^\rho - \eta^{\mu \rho}\omega_{\rho \sigma} P^\sigma. $$
    The expanded transformation of $J^{\mu\nu}$ reads 
    \begin{align*}
      J^{\mu \nu} + \dfrac{i}{2} \omega_{\rho\sigma} [J^{\rho \sigma}, J^{\mu \nu}] + O(\epsilon) &= \left(\mathbf{1}+\frac{i}{2} \omega_{\lambda \epsilon} J^{\lambda \epsilon}+i \epsilon_\alpha P^\alpha\right) J^{\mu \nu} \left(\mathbf{1}-\frac{i}{2} \omega_{\rho \sigma} J^{\rho \sigma}-i \epsilon_\sigma P^\sigma\right)\\
      &= U(\Lambda, a) J^{\mu\nu} U^{\dagger}(\Lambda, a)\\
      &= \Lambda_\rho{ }^\mu \Lambda_\sigma{ }^\nu\left(J^{\rho \sigma}+\epsilon^\rho P^\sigma-\epsilon^\sigma P^\rho\right) \\
      &= (\delta_\rho{ }^\mu + \omega_\rho{ }^\mu)(\delta_\sigma{ }^\nu+\omega_\sigma{ }^\nu) J^{\rho \sigma} + O(\epsilon)\\
      &= J^{\mu \nu} + J^{\mu \sigma} \omega_\sigma{ }^\nu + J^{\rho \nu}\omega_\rho{ }^\mu + O(\epsilon)\\
      &= J^{\mu \nu} + \dfrac{1}{2} \omega_{\rho\sigma} \left(\eta^{\nu \sigma}J^{\mu \rho} \omega_{\rho\sigma} -\eta^{\nu \rho}J^{\mu \sigma} + \eta^{\mu \sigma} J^{\rho \nu} - \eta^{\mu \rho} J^{\sigma \nu}\right) + O(\epsilon)
    \end{align*}
    The expansion in $\epsilon$ was not explicit because it only serves to determine $[J^{\rho \sigma}, P^{\mu}]$ which is already known at that point. Finally, using the fact $\eta$ is symmetric and $\omega$ arbitrary, we get 
    \begin{align*}
      \dfrac{i}{2} [J^{\rho \sigma}, J^{\mu \nu}] = \dfrac{1}{2} \left(\eta^{\nu \sigma} J^{\mu \rho} -\eta^{\nu \rho} J^{\mu \sigma} + \eta^{\sigma\mu} J^{\rho \nu} - \eta^{\rho\mu} J^{\sigma \nu}\right).
    \end{align*}
  \item[(f)] We now define the angular momentum vector $\mathbf{J} = (J^{23}, J^{31}, J^{12}) \equiv (J^1, J^2, J^3)$.  The commutation relation of these operators is given by
  \begin{align*}
    &[J^1, J^2] = [J^{23}, J^{31}] = -i \left(\eta^{1 3} J^{3 2} -\eta^{1 2} J^{3 3} + \eta^{33} J^{2 1} - \eta^{23} J^{3 1}\right) = -i \left( \eta^{33} J^{2 1}\right) = -i (J^{12}) = -i J^3\\
    &[J^2, J^3] = [J^{31}, J^{12}] =   -i\left(\eta^{2 1} J^{1 3} -\eta^{2 3} J^{1 1} + \eta^{1 1} J^{3 2} - \eta^{3 1} J^{1 2}\right) = -i\left( \eta^{1 1} J^{3 2}\right) = -i\left(J^{2 3}\right) = -i J^{1} \\
    &[J^3, J^1] = [J^{12}, J^{23}] = -i\left(\eta^{3 2} J^{2 1} -\eta^{3 1} J^{2 2} + \eta^{2 2} J^{1 3} - \eta^{1 2} J^{2 3}\right) = -i\left(\eta^{2 2} J^{1 3}\right) = -i\left(J^{3 1}\right) = -iJ^2 
  \end{align*}  
  Combining these results with the antisymmetry of the commutator, we recover $\left[J^i, J^j\right]=-i \epsilon^{i j}{}_{k} J^k$.
  \item[(g)] We finally consider the commutation of the contraction $P^2 = P^{\mu} P_\mu$ with all other generators of the Poincaré group. Since $[P^{\nu}, P^{\mu}] = 0$ the $P^2$ commutes with all translation generators. For Lorentz transformation generators, we get 
  \begin{align*}
    [P^{\mu} P_\mu, J^{\rho \sigma}] &= P^{\mu}[P_\mu, J^{\rho \sigma}] + [P^{\mu}, J^{\rho \sigma}] P_\mu\\
    &= P_{\mu}[P^\mu, J^{\rho \sigma}] + [P^{\mu}, J^{\rho \sigma}] P_\mu\\
    &=  -i \left(P_{\mu}\eta^{\mu \sigma}P^\rho- P_{\mu}\eta^{\mu \rho} P^\sigma\right) -i \left(\eta^{\mu \sigma}P^\rho P_{\mu}- \eta^{\mu \rho} P^\sigma P_{\mu}\right)\\
    &= -i\left(P^{\sigma}P^\rho- P^{\rho} P^\sigma\right) -i \left(P^\rho P^{\sigma}- P^\sigma P^{\rho}\right) = 0
  \end{align*} 
  so $P^2$ commutes with all generators.  

\end{enumerate}




\section{Acknowledgement}
I worked on my own for this assignment.


% References
\makereferences
%-------------------------------------------------------


%%%%%%%%%%%%%%%%%%%%%%%%
% Terminer le document %
%%%%%%%%%%%%%%%%%%%%%%%%
\end{document}