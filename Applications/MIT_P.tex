\documentclass[12pt]{article}
\usepackage[T1]{fontenc}
\usepackage[utf8]{inputenc}
\usepackage[a4paper, total={6in, 10.5in}]{geometry}

\newcommand{\HRule}{\rule{\linewidth}{0.5mm}}
\newcommand{\Hrule}{\rule{\linewidth}{0.3mm}}

\makeatletter% since there's an at-sign (@) in the command name
\renewcommand{\@maketitle}{%
  \parindent=0pt% don't indent paragraphs in the title block
  \centering
  {\Large \bfseries\textsc{\@title}}
  \HRule\par%
  \textit{\@author \hfill \@date}
  \par
}
\makeatother% resets the meaning of the at-sign (@)

\title{Statement of Objectives}
\author{Pierre-Antoine Graham}
\date{MIT physics Ph.D.}

\begin{document}

\maketitle
\vspace{0.5cm}

Interaction between quantum matter and quantum information have sparked countless ideas shaping a second quantum revolution. From high temperature superconducors to superconducting qubits, these ideas reached me as my journey trough academia started and they showed me how modern technology can build on the full elegant machinery of theoretical physics. I aim to become a leader participating in the second quantum revolution with no compromise between practical application and theoretical sophisication. At the age of 14, I won a writting contest about career ambitions by describing a future where I would play an important role in the developpement of physics. This dream stayed with me and at the dawn of specialization, working at MIT would make it a reality. \vspace{0.4cm}

My first contact with science took the form of four participations to the Hydro-Québec science fair, followed by a summer job at an astronomical observatory. These opportunities to share my growing passion with the general public made me value scitentifc collaboration deeply. This was reflected trough my three participations and organization of populariszation events at Universite de Sherbrooke as a member of the diversity commitee. I also joined tutoring programs where I helped students individually and in groups troughout undergrad. Lately, I am oragnizing a seminar series for students of my program to share their interests. I plan to carry my desire for excellence in scientific communicaton by contributing to the impressive outreach programs at MIT notably material sciences related. \vspace{0.4cm} 

While outreach projects fostered my interest in the foundations of modern physics, internships made me explore and refine my aspirations. Eager to go beyond courses and start contributing to science directly, I took an internship during the first summer of undergraduate. I joined Prof. Jeffrey Quilliam's group to conceive an interface treating nuclear magnetic resonance data generated by experiments on frustrated Kagome crystals at high pressure. Troughout the summer, I developped a rigorous work ethic as I implemented Fourier analysis of spin echos, curve fitting tools and efficient data storing. Along the way, interactions with experimentalists tainted my theorist view of physics with an appreciation of the work behind data. \vspace{0.4cm}

Wanting to continue my exploration of material sciences opportunities, I started working as a theorist in Prof. André-Marie Tremblay's group. I was tasked to produce and analyse data from two-particle self-consistent approach simulations of electron-doped cuprates. At the beginning of the project, I demonstrated my self-learning abilities by rapidly gaining working understanding of Matsubara frequencies and their role in the evaluation of many-body functions. Meticulousely exploring simulation data, I devised a procedure to locate the quantum critical point of the electron-dopped phase diagram using an approximate temperature for antiferromagnetic pseudogap appearence. My analysis allowed me to qualify the effect of a simple implementation of disorder on the quantum critical point and helped orient further research that would lead to a publication in Physical Review B. Looking back, I see this internship as the origin of my current ambitions in the material sciences. It is the idea that my work could lead to new technologies that made it so engaging for me. \vspace{0.4cm}

The thrill of material sciences became even more vivid with my next internship project. This time, I was working with Prof. Ion Garate on an adaptation of the Van Roosbroeck system of partial differential equations providing a semi-classical description of Weyl semimetals. The goal of the project was to solve the equations to gain insight into the semi-classical role of the chiral anomaly. Having more independance, I produced animations of the numerical evolution of charge densities through time and found a way to decouple the equations using Ampère's law. In the end, we identified the presence of photoinduced plasma oscillations leading to a publication in Physical Review B. This internship introduced me to the intricacies of non-equillibrium phenomenas and the challenging task of testing the consistency of a model with nature. \vspace{0.4cm}  

Although my first three internships were all connected to quantum materials, my curiosity pushed me to join Prof. Valerio Faraoni's Group to work with modified theories of gravity. The prototypical idea of the project was to map scalar-tensor gravity to a dissipative fluid with a finite temperature. In the first part of my work, I used perturbation theory to explain why a family of cosmological solutions have ill-defined temperatures. These results led to a publication in Physical Review D. In the second half of the internship, I proposed an extension of the formalism to multi-scalar tensor gravity, by introducing the idea of an effective multi-fluid described in a paper published in  The European Physical Journal Plus. While exploring this extension, I accessed the core of the research and took part in the group discussions. These discussions made me a more confident scientist by showing me how to take my place in an active research group.\vspace{0.4cm}

Sincerely,

\vspace{0.4cm}
Pierre-Antoine Graham
%\includegraphics[scale = 0.4]{Signed.png}

\end{document}