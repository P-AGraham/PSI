\documentclass[12pt]{article}
\usepackage[T1]{fontenc}
\usepackage[utf8]{inputenc}
\usepackage[a4paper, total={6in, 10.5in}]{geometry}

\newcommand{\HRule}{\rule{\linewidth}{0.5mm}}
\newcommand{\Hrule}{\rule{\linewidth}{0.3mm}}

\makeatletter% since there's an at-sign (@) in the command name
\renewcommand{\@maketitle}{%
  \parindent=0pt% don't indent paragraphs in the title block
  \centering
  {\Large \bfseries\textsc{\@title}}
  \HRule\par%
  \textit{\@author \hfill \@date}
  \par
}
\makeatother% resets the meaning of the at-sign (@)

\title{Statement of Objectives}
\author{Pierre-Antoine Graham}
\date{MIT physics Ph.D.}

\begin{document}

\maketitle
\vspace{0.5cm}

Interaction between quantum matter and quantum information has sparked countless ideas shaping a second quantum revolution. From high-temperature superconductors to superconducting qubits, these ideas reached me as my journey through academia started and they showed me how modern technology can build on the full elegant machinery of theoretical physics. I aim to become a leader participating in the second quantum revolution with no compromise between concrete application and theoretical sophistication. At the age of 14, I won a writing contest about career ambitions by describing a future where I would play an important role in the development of physics. This dream stayed with me and at the dawn of specialization, working at MIT would help make it a reality.\\ 

My first contact with science took the form of four participations at the Hydro-Québec science fair, followed by a summer job at an astronomical observatory. These opportunities to share my growing passion with the general public made me value scientific communication deeply. This was reflected through my three participations and organization of popularisation events at Universite de Sherbrooke as a member of the diversity committee. I also joined tutoring programs where I helped students individually and in groups throughout undergrad. Lately, I have been contributing to the organization of a seminar series for students of my program to share their interests. I plan to carry my desire for excellence in scientific communication to the impressive outreach programs at MIT notably in material sciences.\\  

While outreach projects fostered my interest in the foundations of modern physics, internships made me explore and refine my aspirations. Eager to go beyond courses and start contributing to science directly, I took an internship during the first summer of my undergraduate. I joined Prof. Jeffrey Quilliam's group to conceive an interface treating nuclear magnetic resonance data generated by experiments on frustrated Kagome crystals at high pressure. Throughout the summer, I developed a rigorous work ethic as I implemented Fourier analysis of spin echos, curve fitting tools, and efficient data storage. Along the way, interactions with experimentalists tainted my theorist view of physics with an appreciation of the work behind data.\\

With the reality of experimentalists in mind, My theoretical physics journey journey started in Prof. André-Marie Tremblay's group. I was tasked to produce and analyze data from two-particle self-consistent approach simulations of electron-doped cuprates. At the beginning of the project, I demonstrated efficient self-learning by rapidly gaining a working understanding of Matsubara frequencies and their role in the evaluation of many-body functions. Meticulously exploring simulation data, I devised a procedure to locate the quantum critical point of the electron-doped phase diagram using an approximate temperature for antiferromagnetic pseudogap appearance. My analysis allowed me to qualify the effect of a simple implementation of disorder on the quantum critical point and helped orient further research that would lead to a publication in Physical Review B. Looking back, I see this internship as the origin of my current ambitions in the material sciences. It is the idea that my work could lead to new technologies that made it so engaging for me.\\ 

\newpage

The thrill of material sciences became even more vivid with my next internship project. This time, I was working with Prof. Ion Garate on an adaptation of the Van Roosbroeck system of partial differential equations providing a semi-classical description of Weyl semimetals. The goal of the project was to solve the equations to gain insight into the semi-classical role of the chiral anomaly. Having more independence, I produced animations of the numerical evolution of charge densities through time and found a way to decouple the equations using Ampère's law. In the end, we identified the presence of photoinduced plasma oscillations leading to a publication in Physical Review B. This internship introduced me to the intricacies of non-equilibrium phenomena and the challenging task of bringing a theoretical model closer to an experiment.\\ 

As my interest in condensed matter became more established, I started exploring other areas of physics. To satisfy my curiosity about gravitational physics, I worked with Prof. Valerio Faraoni's group on a map from scalar-tensor gravity to a dissipative fluid with an associated temperature. This final undergrad research experience brought me to a bigger research group where I gained confidence as a collaborating scientist. The next step in my exploration unfolded in Prof. Alexander Blais's group where I worked with a master's student on the mixed regular/chaotic dynamics of a driven superconducting cat qubit and its classical limit. The fact we were working with the sophisticated ideas of chaos theory to ultimately improve quantum gates was fulfilling and laid the basis of my current career ambitions. Following these ambitions, I will participate in a winter research project on a new qubit implementation relying on the Unruh effect.\\ 


Paragraph about Perimeter

Paragraph about Professors

In a general sense, I look forward to making meaningful connections with the great minds of MIT. Collaboration and exchange are central to launching a stimulating career I will be proud of. I would be honored to take part in the cutting-edge research happening at MIT and I am convinced MIT's professors and students will allow me to reach my full potential in this century of great scientific opportunities.\\

Sincerely,

Pierre-Antoine Graham
%\includegraphics[scale = 0.4]{Signed.png}

\end{document}