\documentclass[12pt]{article}
\usepackage[T1]{fontenc}
\usepackage[utf8]{inputenc}
\usepackage[a4paper, total={6in, 10.5in}]{geometry}

\newcommand{\HRule}{\rule{\linewidth}{0.5mm}}
\newcommand{\Hrule}{\rule{\linewidth}{0.3mm}}

\makeatletter% since there's an at-sign (@) in the command name
\renewcommand{\@maketitle}{%
  \parindent=0pt% don't indent paragraphs in the title block
  \centering
  {\Large \bfseries\textsc{\@title}}
  \HRule\par%
  \textit{\@author \hfill \@date}
  \par
}
\makeatother% resets the meaning of the at-sign (@)

\title{Statement of Objectives}
\author{Pierre-Antoine Graham}
\date{MIT physics Ph.D.}

\begin{document}

\maketitle
\vspace{0.5cm}

The interaction between quantum matter and quantum computing has sparked countless ideas, shaping a second quantum revolution. From high-temperature superconductors to superconducting qubits, these ideas reached me as my journey through academia started and they showed me how modern technology can build on the full elegant machinery of theoretical physics. I aim to become a leader participating in the second quantum revolution with no compromise between concrete application and theoretical sophistication. At the age of 14, I won a writing contest about career ambitions by describing a future where I would play an important role in the development of physics. This dream stayed with me and, at the dawn of specialization, working at MIT would help make it a reality.\\[0.2cm]

While outreach projects fostered my interest in the foundations of modern physics, internships helped me explore and refine my aspirations. Eager to go beyond courses and start contributing to science directly, I took an internship in Prof. Jeffrey Quilliam's group during the first summer of my undergraduate studies. The project revolved around the conception of an interface treating nuclear magnetic resonance data generated by experiments on frustrated Kagome crystals at high pressure. Throughout the summer, I developed a rigorous work ethic as I implemented Fourier analysis of spin echos, curve fitting tools, and efficient data storage. Along the way, interactions with experimentalists complemented my theorist view of physics with an appreciation of the work behind data.\\[0.2cm]

With the reality of experimentalists in mind, my theoretical research journey started in Prof. André-Marie Tremblay's group. My role was to produce and analyze data from two-particle self-consistent approach simulations of electron-doped cuprates. At the beginning of the project, I demonstrated efficient self-learning by rapidly gaining a working understanding of Matsubara frequencies and their role in the evaluation of many-body functions. Meticulously exploring simulation data, I devised a procedure to locate the quantum critical point of the electron-doped phase diagram by approximating temperature for antiferromagnetic pseudogap appearance. My analysis allowed me to qualify the effect of a simple implementation of disorder on the quantum critical point and on the Vilk criterion for hot spots appearence. These results helped orient further research that would lead to a publication in Physical Review B. Looking back, I see this internship as the origin of my current ambitions in the material sciences. It is the idea that my work could lead to new superconductivity-related technologies that made it so engaging for me.\\[0.2cm] 

The thrill of material sciences became even more vivid with my next internship project. This time, I was working with Prof. Ion Garate on an adaptation of the Van Roosbroeck system of partial differential equations to describe Weyl semimetals interacting with light. The goal of the project was to solve the equations to gain insight into the semi-classical role of the chiral anomaly. Independently, I found a way to decouple the equations further using an adaptation of Ampère's law and animated the evolution of charge densities through time. In the end, we identified the presence of photoinduced plasma oscillations, leading to a publication in Physical Review B. This internship introduced me to the intricacies of non-equilibrium phenomena and the challenging task of bringing a theoretical model closer to experiment.\\[0.2cm] 

As a second exposition to non-equilibrium phenomena, I collaborated with a student from Prof. Alexandre Blais's group on a small project about the mixed regular/chaotic dynamics of a driven superconducting cat qubit and its classical limit. The fact we were working with the sophisticated ideas of chaos theory to ultimately improve quantum gates was fulfilling and laid the basis of my current career ambitions. Following my objective to play a role in the second quantum revolution, I will participate in a winter research project on a new qubit implementation relying on the stimulated Unruh effect with Dr. Barbara Soda. \\[0.2cm]

%I am also interested in the quantum information aspects of quantum computing including quantum error correction.  

I am currently studying at Perimeter Institute as a Perimeter Scholars International master's student. The coursework portion of the program has shown me excellence in education through its world-class teachers and its exceptional students. With exposition to a broad range of advanced topics, I realized how challenging and stimulating studying theoretical physics truly is. I am eager to continue taking on this challenge with the diverse PhD courses offered at MIT. For the research portion of the program, I will work under the supervision of Prof. Yin-Chen He on a novel way to study emergent conformal symmetry associated to critical phenomena called the "Fuzzy Sphere" method. MIT provides a continuation of the excellence in education and research I am growing with at Perimeter Institute while giving me access to a bigger community working closer to technology. \\[0.2cm]

In light of my past exploration, I can say I have a strong desire to work on the uses of quantum information methods in many-body physics and non-equilibrium dynamics. I am fascinated by the multidisciplinary aspects of this research direction and its concrete implications for understanding and controlling quantum phenomena at the origin of new tachnologies. At MIT, I would be thrilled to join Prof. Soonwon Choi's group because of the diversity of his methods, their proximity to quantum technology, and their alignment with my interests. As a result of my project on chaotic-driven cat qubits, I am particularly curious about his work on ergodicity in time-dependent quantum systems. I am also curious about the criticality aspects of Rydberg quantum simulators. Following my interest in criticality, I also see myself joining Prof. Senthil Todadri's group to expand on my research about non-fermi liquids and quantum criticality in cuprates. A Ph.D. with either of these professors would build on my experience and satisfy my ambitions.\\[0.2cm]

In a general sense, I look forward to making meaningful connections with the great minds of MIT. I would be honored to take part in the cutting-edge research happening at MIT and I am convinced the institute's professors and students will allow me to reach my full potential in this century of great scientific opportunities.\\

Sincerely,

Pierre-Antoine Graham
%\includegraphics[scale = 0.4]{Signed.png}

\end{document}