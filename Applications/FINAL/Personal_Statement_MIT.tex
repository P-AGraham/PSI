\documentclass[12pt]{article}
\usepackage[T1]{fontenc}
\usepackage[utf8]{inputenc}
\usepackage[a4paper, total={6in, 10.5in}]{geometry}

\newcommand{\HRule}{\rule{\linewidth}{0.5mm}}
\newcommand{\Hrule}{\rule{\linewidth}{0.3mm}}

\makeatletter% since there's an at-sign (@) in the command name
\renewcommand{\@maketitle}{%
  \parindent=0pt% don't indent paragraphs in the title block
  \centering
  {\Large \bfseries\textsc{\@title}}
  \HRule\par%
  \textit{\@author \hfill \@date}
  \par
}
\makeatother% resets the meaning of the at-sign (@)

\title{Personal Statement}
\author{Pierre-Antoine Graham}
\date{MIT physics Ph.D.}

\begin{document}

\maketitle
\vspace{0.5cm}

Each time the frontier of physics moves, it takes more ingenuity to push it further. I deeply believe that we need all the minds humanity has to offer to get the best out of the challenging problems physics keeps revealing. As a member of the scientific community and the LGBTQ2+ community, I aim to make science more accessible, inclusive and modify how it is seen to make more people engage in it.\\[0.3cm]

My first effort to reach these goals took the form of outreach. After four high school participations at the \textit{Hydro-Québec} science fair, I worked as a summer camp scientific demonstrator for four years. My curiosity towards astrophysics then brought me to an astronomical observatory where I prepared and presented material about cosmology and space exploration. These opportunities to share my growing passion with the general public led me to value scientific communication deeply. This was reflected through my three participations in outreach events at \textit{Universite de Sherbrooke} and my organization of the 2021 edition of the diversity committee popularization event. My projects took the form of \textit{Manim} videos and a talk at \textit{Institut Quantique}. Lately, I have been contributing to the organization of a seminar series for students of my program to share their interests. I plan to carry my desire for excellence in scientific communication to the impressive outreach programs at MIT, notably to the Educational Studies Program.\\[0.3cm]

To make physics more accessible, I complemented my outreach experience by joining tutoring programs where I helped students from diverse backgrounds overcome their difficulties. Explaining physics concepts from high school to undergraduate level made me realize that teaching was as valuable for me as it was for the students I was helping. As a result of this experience, I look forward to rediscovering my undergraduate knowledge with a teacher assistantship throughout my Ph.D.\\[0.3cm]

At Perimeter Institute, I have experienced the full power of an international scientific community. My classmates are exceptional physicists from 17 different countries. Learning about their unique experiences and challenges made me more aware of the international state of the physics community. Every day, I grow from cultural exchange and collaborative problem solving where I give my all to embody the values of inclusivity and collaboration of Perimeter Institute. My peers selected me to receive the Marsland Family PSI award, highlighting my contribution to the learning environment of the program and alignment with the values of the institute.\\[0.3cm]

I am eager to bring my collaborative spirit to MIT and become a catalyst of ideas nurturing a collective passion for science at the heart of its international community. \\[0.3cm]
% add a finishing sentence with international community of MIT
% While the program showed me how welcoming and collaborative an environment can be, it also made me want my privileged reality to be shared by every physics institution. In particular,

Sincerely, \\

Pierre-Antoine Graham

\end{document}