\documentclass[12pt]{article}
\usepackage[T1]{fontenc}
\usepackage[utf8]{inputenc}
\usepackage[a4paper, total={6in, 10.5in}]{geometry}

\newcommand{\HRule}{\rule{\linewidth}{0.5mm}}
\newcommand{\Hrule}{\rule{\linewidth}{0.3mm}}

\makeatletter% since there's an at-sign (@) in the command name
\renewcommand{\@maketitle}{%
  \parindent=0pt% don't indent paragraphs in the title block
  \centering
  {\Large \bfseries\textsc{\@title}}
  \HRule\par%
  \textit{\@author \hfill \@date}
  \par
}
\makeatother% resets the meaning of the at-sign (@)

\title{Statement of Purpose}
\author{Pierre-Antoine Graham}
\date{Perimeter Institute PhD Residency Program Award}

\begin{document}

\maketitle
\vspace{0.2cm}

Following my previous experiences in research internships and courses—both at Université de Sherbrooke to the PSI program, I have decided to pursue my PhD researching in quantum matter and quantum information. At first, my passion for interdisciplinary methods spanning various disciplines in modern physics was challenging to balance with a particular choice of specialization. This led to my interest in the phenomena of strongly correlated systems relating to diverse tools from quantum field theory to general relativity. I am now determined to advance our understanding and control of quantum materials while  simultaneously employing theoretical sophistication and practical application. This brings me to apply to the Perimeter Institute PhD residency award because it provides an environment where I have strived as a theoretical physicist and person like never before with researchers that will lead me to the cutting-edge of theoretical research. Specifically, I want to join Prof. Yin-Chen He's group to produce new results about emergent conformal symmetry with the fuzzy sphere method which I will start working on this winter in the context of my PSI essay. I am also interested in working with Prof. Timothy Hsieh on the rich structure emerging from entanglement at the intersection of quantum matter and quantum information. \\

My theoretical research journey started in Prof. André-Marie Tremblay's group at Université de Sherbrooke. My role was to produce and analyze data from two-particle self-consistent approach simulations of electron-doped cuprates. At the beginning of the project, I demonstrated self-learning abilities by gaining a working understanding of Matsubara frequencies and their role in the evaluation of many-body functions. Through a detailed exploration of simulation data, I devised a procedure to locate the quantum critical point of the electron-doped phase diagram by approximating temperature for antiferromagnetic pseudogap appearance. My analysis allowed me to quantify the effect of a simple implementation of disorder on the quantum critical point. These results led to a publication in Physical Review B. Looking back, I see this internship as the origin of my current ambitions in quantum matter. It is the idea that my work could lead to new superconductivity-related technologies that made it so engaging for me. This project also introduced me to the mysteries of phase transitions. The statistical physics course from the PSI program then strengthened my grasp and curiosity for this subject by connecting it to renormalization, conformal field theory, and universality. For my PSI essay, I will study the emergent conformal symmetry of second-order phase transitions through the fuzzy sphere lens with Prof. Yin-Chen He. I would be thrilled to continue unfolding the potential I see in this method throughout my PhD. It is tied to topology, relativity, and quantum field theory while targeting deep questions on phase transitions appealing to my passion for interdisciplinary research.\\

In my second theoretical physics research internship, I further pursued my interest in understanding and controllingcontrol exotic quantum materials. I worked with Prof. Ion Garate at Université de Sherbrooke on an adaptation of the Van Roosbroeck system of partial differential equations to describe Weyl semimetals interacting with light. The goal of the project was to solve the equations to gain insight into the semi-classical role of the chiral anomaly. I found a way to decouple the equations further using an adaptation of Ampère's law and animated the evolution of charge densities through time. In the end, we identified the presence of photoinduced plasma oscillations, leading to a publication in Physical Review B. This internship brought me to the mysterious properties of topological materials which I investigated further with a  literature review on topological insulators. Under the supervision of Prof. Timothy Hsieh, I could learn more about topological materials, their phase transitions, and their entanglement structure.\\

Having established my interest for condensed matter, I started exploring quantum computing directions by collaborating with a student from Prof. Alexandre Blais's group on the mixed regular/chaotic dynamics of a driven superconducting cat qubit and its classical limit. The fact we were working with the sophisticated ideas of chaos theory to ultimately improve quantum gates was fulfilling and laid the basis of my ambition to improve control of quantum systems. Following this ambition, I will participate in the PSI winter school with a project on a new qubit implementation relying on the stimulated Unruh effect with Dr. Barbara Soda. The combination of these projects with my courses on photonics, quantum computing, and quantum algorithms has sparked my interest in quantum circuit models. I see myself pursuing a PhD in Prof. Timothy Hsieh's group to continue investigating these models with a focus on entanglement. I am also motivated to explore measurement-induced phase transitions.\\

Computational aspects of theoretical physics research align directly with my passion for programming challenges. This appreciation for computer science originated from the Project Euler recreational programming challenges which I have worked on extensively leading me to be part of the top 100 solvers of problem 850. My passion then grew with my project on electron-doped cuprates bringing me to use the Cedar supercomputer to produce and analyze data on a larger scale. Then my internship on Weyl semimetals and my project on quantum chaotic qubits gave me more independance to implementing various algorithms for simulating physical systems. I am excited to continue improving my computer science skills and make the most out of the computational resources of Perimeter Institute. I am drawn to a PhD in Prof. Yin-Chen He's group provides diverse scientific programming challenges which I can't wait to tackle with creativity and elegance.\\

At Perimeter Institute, I have experienced the full power of an international scientific community. Learning about the unique experiences and challenges of my PSI classmates made me more aware of the international state of the physics community. Every day, I grow from cultural exchange and collaborative problem solving where I am entirely committed to embodying the values of inclusivity and collaboration of Perimeter Institute. During my PhD, I plan to take part in outreach efforts nurturing a collective passion for science and rediscover my PSI knowledge through teacher assistantship. I am eager to continue collaborating with the great minds of Perimeter Institute and become a catalyst of ideas at the heart of its international community.\\

Sincerely,\\
Pierre-Antoine Graham
%\includegraphics[scale = 0.4]{Signed.png}

\end{document}