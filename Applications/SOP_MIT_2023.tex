\documentclass[12pt]{article}
\usepackage[T1]{fontenc}
\usepackage[utf8]{inputenc}
\usepackage[a4paper, total={6in, 10.5in}]{geometry}

\newcommand{\HRule}{\rule{\linewidth}{0.5mm}}
\newcommand{\Hrule}{\rule{\linewidth}{0.3mm}}

\makeatletter% since there's an at-sign (@) in the command name
\renewcommand{\@maketitle}{%
  \parindent=0pt% don't indent paragraphs in the title block
  \centering
  {\Large \bfseries\textsc{\@title}}
  \HRule\par%
  \textit{\@author \hfill \@date}
  \par
}
\makeatother% resets the meaning of the at-sign (@)

\title{Statement of Objectives}
\author{Pierre-Antoine Graham}
\date{MIT physics Ph.D.}

\begin{document}

\maketitle
\vspace{0.5cm}


Is it possible to work at the frontier of theory and experiment and still explore abstract mathematical ideas? I think we live in a golden age of physics where technology opens up to the craziest ideas and I plan to take advantage of this in my career. Initially driven by outreach contests, I focused on communicating abstract theoretical ideas ranging from special relativity to classical electromagnetism. Then I had the chance to collaborate with four different research groups in the context of four-month internships. Each research project helped me build confidence as a scientist and define my aspirations. While the first three increased my appreciation of experiments and technological prospects, the last one confirmed my interest in abstract ideas. When I picture the career ahead of me, I see a scientist striving to contribute to technology while trying to make unsuspected connections between different areas of physics. 
\vspace{0.4cm}

During my first undergraduate year, I joined Prof. Jeffrey Quilliam's group to program a Python graphical user interface. The interface was meant to treat nuclear magnetic resonance data obtained from experiments on frustrated Kagome Crystals at high pressure. I managed to reproduce all the key features needed in the interface including Fourier analysis of spin echos and curve fitting tools. Along the way, I was introduced to the challenges and methods of experimental condensed matter physics. In terms, this internship tainted my theorist view of physics with an appreciation of the work behind data.\vspace{0.4cm}

My second internship took place in Prof. André-Marie's group. I was tasked to treat data from two-particle self-consistent approach simulations of electron-doped cuprates. At the beginning of the project, I had little knowledge of many-body physics and I demonstrated my learning abilities by reaching a sufficient understanding early on. Jumping into simulation data, I devised a procedure to approximate the temperature at which the antiferromagnetic pseudogap appears. I then used this temperature to locate the quantum critical point of the electron-dopped phase diagram. My analysis allowed me to determine the effect of a simple implementation of disorder on the quantum critical point and helped orient further research that would lead to a publication in Physical Review B. Overall, the internship made me realize how thrilling a very active material sciences field can be. It is the idea that my work could lead directly or indirectly to new technologies that made it so engaging for me.\vspace{0.4cm} 

The thrill of material sciences became even more vivid with my third internship project. This time, I was working with Prof. Ion Garate on an adaptation of the Van Roosbroeck system of partial differential equations providing a semi-classical description of Weyl semimetals. The goal of the project was to solve the equations to gain insight into the semi-classical role of the chiral anomaly. Over the course of the internship, I produced animations of the numerical evolution of charge densities through time. They allowed Prof. Garate and me to get intuition about the general behavior of the system. While building this intuition, I constantly discussed the limitations of the model and the measurability of our predictions with Prof. Garate. Once I was convinced of the experimental prospects of our approach, I found a way to decouple the equations further using Ampère's law. Approximating their solutions allowed me to compare them with my numerical results. In the end, we identified the presence of photoinduced plasma oscillations leading to a paper currently submitted to Physical Review Letters. 
\vspace{0.4cm}   

\newpage 

Although my first three internships were all connected to quantum materials, my curiosity pushed me to join Prof. Valerio Faraoni's Group to work with modified theories of gravity. The prototypical idea of the project was to map scalar-tensor gravity to a general relativistic dissipative fluid allowing for the identification of the temperature of gravity. The first part of my work focused on a family of cosmological solutions which had ill-defined temperatures. With perturbation theory, I showed that the problematic solutions were either unphysical or unstable and explained why the temperature was ill-defined in the first place. These results led to a publication in Physical Review D. In the second half of the internship, I proposed an extension of the formalism to multi-scalar tensor gravity, by introducing the idea of an effective multi-fluid described in a paper currently submitted to European Physical Journal C. While exploring this extension, I accessed the core of the research and took part in the group discussions. These discussions made me a more confident scientist by showing me how to take my place in an active research group.  \vspace{0.4cm}   

My past research demonstrates that I am open-minded about the possible directions my career can take. However, my interests recently leaned towards holographic duality. I find this area very appealing because it is both abstract and connects to experiments in ways that have surprised me. While exploring many-body physics, I found out that the duality could be used to gain insight into strange metals and I could not be more thrilled. Furthermore, I came across the duality in the context of quark-gluon plasma during my third internship. Knowing that Prof. Hong Liu is at the forefront of these research avenues and works on many more applications of the duality, I would certainly satisfy my ambitions by joining his group. Participating in this research would allow me to combine my knowledge of gravity, dissipative fluids, and strongly correlated systems. I also see myself working with Prof. Daniel Harlow on the connections of the holographic principle with quantum error correction codes. Over the course of my career, I hope to find practical use of a far-reaching connection across physics and holographic duality looks like one of the best areas to do it. \vspace{0.4cm}

In a general sense, I look forward to making meaningful connections with the great minds of MIT. Collaboration and exchange are central to launching a stimulating career I will be proud of. I would be honored to take part in the cutting-edge research happening at MIT. Through my internships, the motto \textit{mens et manus} has gained a lot of meaning for me and I am convinced MIT's professors and students will allow me to reach my full potential in this century of great scientific opportunities.
\vspace{0.4cm}

Sincerely,

\vspace{0.4cm}
Pierre-Antoine Graham
%\includegraphics[scale = 0.4]{Signed.png}

\end{document}