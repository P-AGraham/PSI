\documentclass[12pt]{article}
\usepackage[T1]{fontenc}
\usepackage[utf8]{inputenc}
\usepackage[a4paper, total={6in, 10.5in}]{geometry}

\newcommand{\HRule}{\rule{\linewidth}{0.5mm}}
\newcommand{\Hrule}{\rule{\linewidth}{0.3mm}}

\makeatletter% since there's an at-sign (@) in the command name
\renewcommand{\@maketitle}{%
  \parindent=0pt% don't indent paragraphs in the title block
  \centering
  {\Large \bfseries\textsc{\@title}}
  \HRule\par%
  \textit{\@author \hfill \@date}
  \par
}
\makeatother% resets the meaning of the at-sign (@)

\title{Cover Letter}
\author{Pierre-Antoine Graham}
\date{Vienna physics PhD}

\begin{document} 

\maketitle
\vspace{0.1cm}
Dear Prof. Schuch,\\

My interest in the interaction of quantum many-body physics, quantum computing and quantum information brings me to apply for a PhD position in your group. From high-temperature superconductors to superconducting qubits, ideas from each of these disciplines have reached me as my journey through academia started. They sparked my desire to become a leader in the second quantum revolution making no compromise between concrete application and theoretical sophistication. The rigor and diversity of the methods used by your group would allow me to satisfy this ambition and produce a body of research I will be proud of.\\

In the light of a diverse exploration from quantum many-body physics to modified theories of gravity, I discovered myself a passion for exotic phases and control of quantum systems. With three condensed matter internships, I was introduced to spin liquids, topological materials and quantum phase transitions in high temperature superconductors. These experiences are the origin of my desire to tackle challenges in the simulations of strongly correlated systems and critical phenomena at quantum phase transition. This winter, I will delve deeper in these challenges by working with Prof. Yin-Chen He on a novel way to study emergent conformal symmetry associated to critical phenomena called the "Fuzzy Sphere" method. I find your research on the classification of exotic phases particulrly appealing because it provides a unifying perspective on my past research.\\ 

Following a project about the chaos assisted tunneling of a superconducting qubit, I became strongly interested in the control of quantum materials applied in new technologies. The fact I was working with the sophisticated ideas of chaos theory to ultimately improve quantum gates was fulfilling and laid the basis of my current career ambitions. Working in your group, I could explore elaborate mathematical ideas while advancing the models at the root of technology.\\

Throughout my research experiences, entanglement has always fascinated me with the multitude of complex phenomena arising form its deceiving simplicity. I am drawn to tensor network methods because their elegance and simplicity would allow me to push the understanding of complex quantum systems further with a more explicit view of their entanglement structure. They would also provide an algorithmic aspect to my research which would connect with my view of computer science as a dimension of theoretical physics research. I learned programming with increasingly difficult programming challenges and my appreciation of computer science is tied to finding algorithms that render possible seemingly imposible tasks.  \\

I am currently studying as a Perimeter scholars international student. I recieved the PSI award for collaboration, inclusion and excellence and would be thrilled to bring my problem solving skills, multidisciplinary background and collaborative spirit to your group. I feel it provides an environnement where I can reach my full potential in this century of great scientific opportunities.\\
%Along the way, interactions with experimentalists complemented my theorist view of physics with an appreciation of connections with the phyiscal world. 

% paragraphs on numerical methods 

Sincerely,

%% brainstorm 

%

% rigor
% multidisciplinary research
% many body, applications

% I am a fan of algorithms, numerical methods, I find programming hads a complete problem solving dimension to theoretical physics. My skills regarding this kind of problem solving 
% Representation theory : link with particle physics, ciritcal systems (look into that)
% Entanglement pbased probes  
% introduce my fascination for entanglement 
% I like solving problems that appear impossible without a clever trick and then the answer feels like magic 
% (simulation method, I have been exposed to the challenges of simulating highly interacting quantum many-body systems)
% my background is very diverse and I am drawn to the multidisciplinary aspect of the research you are conducting at Universitat Wien
% Topological 

% paragraph on projects that interest me (https://schuch.univie.ac.at/sequam/project-objectives/) because of my interest and exposition to exotic phases of matter kagome spin liquid, cuprates
% Theoretical Computer Science concepts

%As a first contact with research, I joined Prof. Jeffrey Quilliam's group to conceive an interface treating data from high pressure frustrated Kagome crystals experiments. With this project, I developed a rigorous work ethic and gained interest for exotic many-body systems. My curiosity then led my to work on signatures of quantum criticality in Prof. André-Marie Tremblay's group. I was tasked to produce and analyze data from a non-perturbative approach to simulations of electron-doped cuprates. With this analysis, I quantified the effect of disorder on the quantum critical point orienting further research that would lead to a publication in Physical Review B. Looking back, I see this internship as the origin of my desire to tackle challenges in the simulations of strongly correlated systems and critical phenomena at quantum phase transition. This winter, I will delve deeper in these challenges by working with Prof. Yin-Chen He on a novel way to study emergent conformal symmetry associated to critical phenomena called the "Fuzzy Sphere" method. Having established my interest in strongly correlated systems, I was introduced to topological materials by working with Prof. Ion Garate. We worked on a set of partial differential equations describing Weyl semimetals interacting with light. With my simplification of the equations, we identified the presence of photoinduced plasma oscillations caused by the chiral anomaly which lead to a publication in Physical Review B. Complementing my knowledge with a litterature review of topological insulators, I gained an appreciation of the intricacies of topological order. 

%I was introduced to quantum devices by working in Prof. Alexandre Blais's group on a small project chaos-assisted tunneling of a driven superconducting qubit. The fact we were working with the sophisticated ideas of chaos theory to ultimately improve quantum gates was fulfilling and laid the basis of my PhD ambitions. I want to bring elegant mathematical ideas to the elaboration of quantum devices. 

\vspace{0.4cm}
Pierre-Antoine Graham

\end{document}