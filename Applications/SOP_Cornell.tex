\documentclass[12pt]{article}
\usepackage[T1]{fontenc}
\usepackage[utf8]{inputenc}
\usepackage[a4paper, total={6in, 10.5in}]{geometry}

\newcommand{\HRule}{\rule{\linewidth}{0.5mm}}
\newcommand{\Hrule}{\rule{\linewidth}{0.3mm}}

\makeatletter% since there's an at-sign (@) in the command name
\renewcommand{\@maketitle}{%
  \parindent=0pt% don't indent paragraphs in the title block
  \centering
  {\Large \bfseries\textsc{\@title}}
  \HRule\par%
  \textit{\@author \hfill \@date}
  \par
}
\makeatother% resets the meaning of the at-sign (@)

\title{Statement of Purpose}
\author{Pierre-Antoine Graham}
\date{Cornell physics Ph.D.}

\begin{document}

\maketitle
\vspace{0.5cm}

Is it possible to work at the frontier of a field of theoretical physics and still borrow from other fields? I think we live in a golden age of physics where progress is made of the least expected combinations of disciplines and I plan to take advantage of this in my career. Initially driven by outreach contests, I focused on communicating abstract theoretical ideas ranging from special relativity to classical electromagnetism. Then I had the chance to collaborate with three different theoretical research groups in the context of four-month internships. Each research project helped me build confidence as a scientist and define my aspirations. While the first three increased my appreciation of condensed matter physics from cuprates to topological materials, the last one sparked my interest in gravitation. When I picture the career ahead of me, I see a scientist striving to combine methods from different areas of physics to push and blend the frontiers of theory further. Cornell's multidisciplinary environment feels like the best place to make this vision a reality. 
\vspace{0.4cm}

My first theoretical internship took place in Prof. André-Marie's group. I was tasked to treat data from two-particle self-consistent approach simulations of electron-doped cuprates. At the beginning of the project, I had little knowledge of many-body physics and I demonstrated my learning abilities by reaching a sufficient understanding early on. Jumping into simulation data, I devised a procedure to approximate the temperature at which the antiferromagnetic pseudogap appears. I then used this temperature to locate the quantum critical point of the electron-dopped phase diagram. My analysis allowed me to determine the effect of a simple implementation of disorder on the quantum critical point and helped orient further research that would lead to a publication in Physical Review B. Overall, the internship made me realize how thrilling a very active material sciences field can be. It is the idea that my work could lead directly or indirectly to new technologies and experiments that made it so engaging for me.\vspace{0.4cm} 

The thrill of material sciences became even more vivid with my next internship project. This time, I was working with Prof. Ion Garate on an adaptation of the Van Roosbroeck system of partial differential equations providing a semi-classical description of Weyl semimetals. The goal of the project was to solve the equations to gain insight into the semi-classical role of the chiral anomaly. Over the course of the internship, I produced animations of the numerical evolution of charge densities through time. They allowed Prof. Garate and me to get intuition about the general behavior of the system. While building this intuition, I constantly discussed the limitations of the model and the measurability of our predictions with Prof. Garate. Once I was convinced of the experimental prospects of our approach, I found a way to decouple the equations further using Ampère's law. Approximating their solutions allowed me to compare them with my numerical results. In the end, we identified the presence of photoinduced plasma oscillations leading to a paper currently submitted to Physical Review Letters. 
\vspace{0.4cm}   

\newpage 

Although my first internships were all connected to quantum materials, my curiosity pushed me to join Prof. Valerio Faraoni's Group to work with modified theories of gravity. The prototypical idea of the project was to study a notion of temperature associated with scalar-tensor gravity. The first part of my work focused on a family of cosmological solutions which had ill-defined temperatures. With perturbation theory, I showed that the problematic solutions were either unphysical or unstable and explained why the temperature was ill-defined in the first place. These results led to a publication in Physical Review D. In the second half of the internship, I proposed an extension of the formalism to multi-scalar tensor gravity, by introducing the idea of an effective multi-fluid described in a paper currently submitted to European Physical Journal C. While exploring this extension, I accessed the core of the research and took part in the group discussions. These discussions made me a more confident scientist by showing me how to take my place in an active research group.  \vspace{0.4cm}   
%full blown

Aside from my research experience, I worked for the growth of an inclusive and accessible scientific community. My outreach efforts started with participation to four editions of the Quebec science fair when I was in high school. Then, during college, I spent a summer working in an astronomical observatory to explain ideas from cosmology to the general public hoping to spark interest. Pursuing the same outreach goals in undergrad, I joined the physics diversity committee (DiPhUS) to help organize an outreach contest meant to bring all members of the physics department together and make complex ideas more accessible. I also joined the physics tutoring center of my physics department to help students from diverse backgrounds overcome their difficulties. Explaining physics concepts from college to undergraduate level made me realize that teaching was as valuable for me as it was for the students I was helping. I look forward to rediscovering my undergraduate knowledge with teacher assistantship throughout my Ph.D.  

\vspace{0.4cm}

My past research demonstrates that I am open-minded about the possible directions my career can take. However, my need for connections with experiments has drawn me to Prof. Saul Teukolsky's research. To me, the comparison of numerical simulations with actual LIGO data is an incredible achievement and I would be thrilled to combine my knowledge of gravity and numerical methods in this setting. I also see myself working in Prof. Eun-Ah Kim's group since my research on cuprates and topological materials aligns with her work.  

\vspace{0.4cm}

In a general sense, I look forward to making meaningful connections with the great minds of Cornell. The fact Cornell's environment is oriented towards collaboration as opposed to competition is very appealing to me because collaboration and exchange are central to launching a stimulating career I will be proud of. I would be honored to take part in the cutting-edge research happening at Cornell and I am convinced Cornell's professors and students will allow me to reach my full potential in this century of great scientific opportunities.\\
%\includegraphics[scale = 0.4]{Signed.png}
\vspace{0.4cm}
Sincerely,

\vspace{0.4cm}
Pierre-Antoine Graham

\end{document}