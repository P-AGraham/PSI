\documentclass[12pt]{article}
\usepackage[T1]{fontenc}
\usepackage[utf8]{inputenc}
\usepackage[a4paper, total={6in, 10.5in}]{geometry}

\newcommand{\HRule}{\rule{\linewidth}{0.5mm}}
\newcommand{\Hrule}{\rule{\linewidth}{0.3mm}}

\makeatletter% since there's an at-sign (@) in the command name
\renewcommand{\@maketitle}{%
  \parindent=0pt% don't indent paragraphs in the title block
  \centering
  {\Large \bfseries\textsc{\@title}}
  \HRule\par%
  \textit{\@author \hfill \@date}
  \par
}
\makeatother% resets the meaning of the at-sign (@)

\title{EDI Statement}
\author{Pierre-Antoine Graham}
\date{PSI Master program}

\begin{document}

\maketitle
\vspace{0.5cm}

Each time the frontier of physics moves, it takes more and more ingenuity to push it further. I deeply believe that we need all physicists humanity has to offer to get the best out of the tremendously challenging problems physics keeps revealing. Driven by this belief, I have participated in scientific outreach efforts since high school and joined math/physics tutoring centers to help students from diverse backgrounds overcome their difficulties. As a member of the scientific community and the LGBTQ2+ community, part of my goals is to make science more accessible, inclusive, and approa-chable. I had the chance to learn in an environment where I could be myself and thrive, but I am well aware that this is not the situation for many people in underrepresented groups. During my career, I will contribute to remove learning barriers by proudly representing my group and attracting diverse people to physics with targeted outreach efforts.\\

One learning barrier I am familiar with is isolation. Since physics was an unpopular subject in my high school, I was initially exploring it alone. To break isolation and foster my passion, I started explaining physics to people around me by participating four times in the Quebec science fair throughout high school. During CEGEP, my outreach efforts took the form of presentations about cosmology followed by stargazing with the general public. With these experiences, I saw and keep seeing the potential in every question regardless of the questioner's background. Simple questions from others have led me to insights I could not have obtained alone. Furthermore, explaining physics concepts from college to undergraduate level through tutoring made me realize that teaching was as valuable for me as it was for the students I was helping. I am convinced that I cannot satisfy my research ambitions without the help of many viewpoints.\\

At the beginning of undergrad, I was pleased to learn that the diversity and inclusion committee of my physics department (DiPhUS) had organized a populariza-tion contest and I decided to participate. I then joined the committee to contribute to the organization of the following edition of the event. I think such events help build connections between professors and students, break isolation, and make the department more welcoming. Speaking of welcoming environments, I was impressed by the diversity of the students during the PSI-Start summer school. I now see Perimeter Institute as a leader in the endeavor to make the physics community more diverse and inclusive. For the first time, I really could appreciate how vast the community is and deeply realize physics is bigger than myself.\\

%I am drawn to Perimeter institute's values because they provide a setting where I can continue taking advantage of the diversity of viewpoints and see what a worldwide collaboration has to offer. \\


Sincerely, \\

Pierre-Antoine Graham

\end{document}
