\documentclass[10pt, a4paper]{article}

%%%%%%%%%%%%%%
%  Packages  %
%%%%%%%%%%%%%%


\usepackage{page_format}
\usepackage{special}
\usepackage{hyperref}
\usepackage{tikz}
\usepackage{quantikz}
\usepackage{cprotect}
\usepackage{caption}
\usepackage{subcaption}

\input{math_func}

\usepackage{slashed}

% References
\usepackage{biblatex}
\addbibresource{ref.bib}
\usetikzlibrary{positioning}


%%%%%%%%%%%%
%  Colors  %
%%%%%%%%%%%%
% ! EDIT HERE !
\colorlet{chaptercolor}{red!70!black} % Foreground color.
\colorlet{chaptercolorback}{red!10!white} % Background color

%%%%%%%%%%%%%%
% Page titre %
%%%%%%%%%%%%%%%
\title{Homework 2} % Title of the assignement.
\author{\PA} % Your name(s).
\teacher{Eduardo Martín-Martínez, Bindiya Arora } % Your teacher's name.
\class{Quantum Information} % The class title.

\university{Perimeter Institute for Theoretical Physics} % University
\faculty{Perimeter Scholars International} % Faculty
%\departement{<Departement>} % Departement
\date{\today} % Date.


%%%%%%%%%%%%%%%%%%%%%%
% Begin the document %
%%%%%%%%%%%%%%%%%%%%%%
\begin{document}

% Make the title page.
\maketitlepage

% Make table of contents
\maketableofcontents

% Assignment starts here ----------------------------

\footnotesize{

\section{Back to basics: quantum circuits}
In what follows, we evaluate the matrix expressions representing a quantum circuit unitary acting on a sequence of qubit input. We work in the computational basis $\{\ket{0}, \ket{1}\}$ and use the notation $X, Y, Z$ for the Pauli gates in this basis. 
\begin{enumerate}
  \item[(a)] First we consider the conjugation of a \verb|CNOT| by two \verb|CNOT| with control and target qubit reversed:
  \begin{align*}
  \begin{quantikz}[baseline={([yshift=-.5ex]current bounding box.center)}]
    \lstick{$1$} & \targ{}  & \ctrl{1}  & \targ{}  & \qw\\
    \lstick{$2$} & \ctrl{-1}  & \targ{}  & \ctrl{-1} & \qw
  \end{quantikz}
  &= 
  \begin{pmatrix}
    1 & 0 & 0 & 0\\
    0 & 0 & 0 & 1\\
    0 & 0 & 1 & 0\\
    0 & 1 & 0 & 0
  \end{pmatrix}
  \begin{pmatrix}
    1 & 0 & 0 & 0\\
    0 & 1 & 0 & 0\\
    0 & 0 & 0 & 1\\
    0 & 0 & 1 & 0
  \end{pmatrix}
  \begin{pmatrix}
    1 & 0 & 0 & 0\\
    0 & 0 & 0 & 1\\
    0 & 0 & 1 & 0\\
    0 & 1 & 0 & 0
  \end{pmatrix}
  =
  \begin{pmatrix}
    1 & 0 & 0 & 0\\
    0 & 0 & 1 & 0\\
    0 & 1 & 0 & 0\\
    0 & 0 & 0 & 1
  \end{pmatrix}
  \end{align*}
  which exchanges the qubits ($\ket{00} \to \ket{00}, \ket{01} \to \ket{10},  \ket{10} \to \ket{01}, \ket{11} \to \ket{11}$) and constitutes a \verb|SWAP| gate. The matrix expression for the reversed \verb|CNOT| was obtained by writing its action on the computational basis which reads $\ket{00} \to \ket{00}, \ket{01} \to \ket{11},  \ket{10} \to \ket{10}, \ket{11} \to \ket{01}$. 
  \item[(b)] Then we calculate the matrix expression of the entanglement-generating circuit 
  \begin{align*}
    &\begin{quantikz}[baseline={([yshift=-.5ex]current bounding box.center)}]
      \lstick{$1$} &      &   & \targ{}  & \qw\\
      \lstick{$2$} & \gate{H}  & \gate{R_{\pi/4}}  & \ctrl{-1} & \qw
    \end{quantikz} = (1_1 \otimes \ket{0}\bra{0}_2 + X_1 \otimes \ket{1}\bra{1}_2)  R_{\pi/4, 2} \left( 1_1 \otimes \frac{1}{\sqrt{2}} (X_2 + Z_2) \right) \\
    &= (1_1 \otimes \ket{0}\bra{0}_2 + X_1 \otimes e^{i\pi/4} \ket{1}\bra{1}_2)\left( 1_1 \otimes \frac{1}{\sqrt{2}} (X_2 + Z_2) \right) \\
    &= \frac{1}{\sqrt{2}} \left(1_1 \otimes \ket{0}_2(\bra{0}_2 + \bra{1}_2) + X_1 \otimes e^{i\pi/4} \ket{1}_2(\bra{0}_2 - \bra{1}_2)\right) = \frac{1}{\sqrt{2}} \left(\begin{pmatrix}
      1 & 0\\ 
      0 & 1 
    \end{pmatrix} \otimes 
    \begin{pmatrix}
      1 & 1 \\
      0 & 0 
    \end{pmatrix}
       + \begin{pmatrix}
        0 & 1\\
        1 & 0
      \end{pmatrix} \otimes  \begin{pmatrix}
        0 & 0\\
        e^{i\pi/4} & -e^{i\pi/4}
      \end{pmatrix}\right)\\
      &= \frac{1}{\sqrt{2}} \begin{pmatrix}
        1 & 1 & 0 & 0 \\ 
        0 & 0 & 0 & 0 \\
        0 & 0 & 1 & 1 \\
        0 & 0 & 0 & 0 
      \end{pmatrix} +
      \frac{1}{\sqrt{2}}
      \begin{pmatrix}
        0 & 0 & 0 & 0 \\ 
        0 & 0 & e^{i\pi/4} & -e^{i\pi/4} \\
        0 & 0 & 0 & 0\\ 
        e^{i\pi/4} & -e^{i\pi/4} & 0 & 0 \\
      \end{pmatrix} = 
      \frac{1}{\sqrt{2}}
      \begin{pmatrix}
        1 & 1 & 0 & 0 \\ 
        0 & 0 & e^{i\pi/4} & -e^{i\pi/4} \\
        0 & 0 & 1 & 1\\
        e^{i\pi/4} & -e^{i\pi/4} &  0 & 0
      \end{pmatrix}.
  \end{align*}
  If we set the phases to $1$, we recover the Bell state mapping $\ket{00} \to (\ket{00} + \ket{11})/\sqrt{2}$, $\ket{01} \to (\ket{00} - \ket{11})/\sqrt{2}$, $\ket{10} \to (\ket{01} + \ket{10})/\sqrt{2}$ and $\ket{11} \to (-\ket{01} + \ket{10})/\sqrt{2}$. 
  \item[(c)]  Finally, we calculate the matrix expression associated with a three-qubit circuit as follows:
  \begin{align*}
    &\begin{quantikz}[baseline={([yshift=-.5ex]current bounding box.center)}]
      \lstick{$1$} &  \gate{H}         & \ctrl{1} &          & \ctrl{1} & \qw\\
      \lstick{$2$} &                   & \targ{1} & \gate{H} & \ctrl{1} & \qw\\
      \lstick{$3$} &  \gate{R_{\pi/4}} &          &          & \targ{1} & \qw
    \end{quantikz}
    = \verb!TOFFOLI!\  R_{\pi/4, 3} H_2 (\ket{0}\bra{0}_1 \otimes 1_2 + \ket{1}\bra{1}_1 \otimes X_2) H_1\\
    &= \verb!TOFFOLI! \ R_{\pi/4, 3}\frac{1}{\sqrt{2}}(\ket{0}_1 (\bra{0}_1 + \bra{1}_1) \otimes H_2 +  \ket{1}_1 (\bra{0}_1 - \bra{1}_1) \otimes H_2 X_2)\\
    &= \verb!TOFFOLI! \ R_{\pi/4, 3}\frac{1}{2}\left(
      \begin{pmatrix}
        1 & 1 \\
        0 & 0
      \end{pmatrix}
       \otimes \begin{pmatrix}
        1 & 1 \\
        1 & -1
      \end{pmatrix} + \begin{pmatrix}
        0 &  0 \\
        1 & -1
      \end{pmatrix} \otimes \begin{pmatrix}
        0 & 1 \\
        1 & 0
      \end{pmatrix}\begin{pmatrix}
        1 & 1 \\
        1 & -1
      \end{pmatrix}\right)\\
      & = \verb!TOFFOLI! \ R_{\pi/4, 3}\frac{1}{2}\left(
      \begin{pmatrix}
        1 & 1 & 1 & 1\\
        1 & - 1 & 1 & - 1\\
        0 & 0 & 0 & 0\\
        0 & 0 & 0 & 0
      \end{pmatrix}
      +
      \begin{pmatrix}
        0 & 0 & 0 & 0\\
        0 & 0 & 0 & 0\\
        1 & 1 & -1 & -1\\
        -1 & 1 & 1 & -1\\
      \end{pmatrix}\right)\\ &= \verb!TOFFOLI! \ R_{\pi/4, 3}\frac{1}{2}
        \begin{pmatrix}
          1 & 1 & 1 & 1\\
          1 & - 1 & 1 & -1\\
          1 & 1 & -1 & -1\\
          -1 & 1 & 1 & -1
        \end{pmatrix} = \frac{1}{2}
        \begin{pmatrix}
          1_3 & 0 & 0 & 0\\
          0 & 1_3 & 0 & 0\\
          0 & 0 & 1_3 & 0\\
          0 & 0 & 0 & X_3
        \end{pmatrix}
        R_{\pi/4, 3}
        \begin{pmatrix}
          1_3 & 1_3 & 1_3 & 1_3\\
          1_3 & - 1_3 & 1_3 & -1_3\\
          1_3 & 1_3 & -1_3 & -1_3\\
          -1_3 & 1_3 & 1_3 & -1_3
        \end{pmatrix}.
  \end{align*}


\end{enumerate}

\section{Quantum Adder}

\begin{enumerate}
  \item[(a)] The \verb|TOFFOLI| gate can be generalized to $n$ qubits by increasing the number of control qubit to $n-1$ conditioning a \verb|NOT| operation on qubit $n$. The circuit corresponding to this generalisation is presented in Fig. 1 (a).
  

  \begin{figure}[h!]
    \centering
    \begin{subfigure}{.5\textwidth}
      \centering
      \begin{quantikz}[baseline={([yshift=-.5ex]current bounding box.center)}]
        \lstick{$1$} & \ctrl{1}    & \\
        \lstick{$2$} & \ctrl{1}    & \\
        \lstick{$\cdots$} & \ctrl{1}     &\\
        \lstick{$n$} & \targ{}     &
      \end{quantikz}
      \cprotect\caption{$n$-qubit generalisation of the \verb|TOFFOLI| gate. \label{toffoli}}
    \end{subfigure}%
    \begin{subfigure}{.5\textwidth}
      \centering
      \begin{quantikz}[baseline={([yshift=-.5ex]current bounding box.center)}]
        \lstick{$a$} & \ctrl{1}  & \ctrl{1} & \rstick{$a$} \\
        \lstick{$b$} & \ctrl{1} & \targ{}  & \rstick{$a \ {\tt XOR}\  b$} \\
        \lstick{$c$} & \targ{}   &         & \rstick{$a \ {\tt AND}\  b$}
      \end{quantikz}
      \caption{Circuit for the addition of two single bit numbers $a$ and $b$. \label{adder}}
    \end{subfigure}
    \caption{Circuits for 2 (a) and 2 (b).}
    \end{figure}
  \item[(b)] Given two single digit binary numbers $a$ and $b$, we can calculate $a+b$ with a quantum circuit by encoding them in input numbers in qubit states $\ket{a}$ and $\ket{b}$ where the digit forms the $0, 1$ label of a computation basis element. In other words, the classical bit adder algorithm can be implemented with a quantum circuit. This algorithm requires qubits for inputs $a$ and $b$ and a qubit initialized to $\ket{0}$ that will eventually be updated to store the carry on of $a+b$ (if we add $1 + 1$ we get $10$ which is represented here by having $1$ stored in the carry on qubit, and $0$ stored in the output state for the $b$ qubit hilbert space. The state of $a$ is unchanged to allow for reversibility of calculation and its implementation as a sequence of unitary operations).  The Quantum circuit implementing the single is presented in Fig. 1 (b). On one hand the \verb|TOFFOLI| gate flips the carry on $c$ to $1$ iff both $a$ and $b$ are initialized to $1$. On the other hand the \verb|CNOT| gate replaces the value of $b$ by $a \ \verb|XOR|\ b$ storing the first digit of the adition output in $b$. 

  \item[(c)] If we consider adding $4$ binary numbers $a, b, c, d$ together, we need an additionnal carry on qubit to represent the result since $1+1+1+1 = 100$ requires $3$ qubits to describe its digits. In this case we name the carry on $c_1$ and $c_2$. The circuit performing the addition of $4$ single bit numbers together is presented in Fig. 2
  \begin{figure}[h!]
    \centering
      \begin{quantikz}[baseline={([yshift=-.5ex]current bounding box.center)}]
        \lstick{$a$}  & \ctrl{1}    & \ctrl{1}&          &          &         &          &      &\\
        \lstick{$b$}  & \ctrl{3}    & \targ{} & \ctrl{1} & \ctrl{1} & \ctrl{1}&          &      &   \\
        \lstick{$c$}&             &           & \ctrl{2} & \ctrl{2} & \targ{} & \ctrl{1} & \ctrl{1} &  \ctrl{1}  \\
        \lstick{$d$}  &             &         &          &          &         & \ctrl{1} &  \ctrl{1}  & \targ{}   \\
        \lstick{$c_1$}  & \targ{}    &        & \ctrl{1} &  \targ{} &         & \ctrl{1} &  \targ{}  &     \\
        \lstick{$c_2$}&             &         & \targ{}  &          &         & \targ{}  & &
      \end{quantikz}
      \caption{Circuit for the addition of four single bit numbers $a, b, c, d$. The output is stored in qubit $d$ (first digit), $c_1$, $c_2$ (last digit) \label{A}}
  \end{figure}
  The first two layers of this circuit add $a$ and $b$ in the way explained in (b). At the second layer the \verb|CNOT| operation shifts the last digit output to the $b$ qubit. The next group of operations uses the value stored in $b, c, c_1$ to add the updated $b$ and $c$ while taking the $c_1$ carry on into account. The first gate of the third layed flips $c_2$ iff $b, c, c_1$  are all one (in our case this does not happen because $c_1 = 0$ implies $b = 0$). The fourth and fifth layers are copies of the operations described in (b) acting on $b, c, c_1$. At the fifth layer, the \verb|CNOT| operation shifts the last digit output to the $c$ qubit. At layer six, a generalised \verb|TOFFOLI| is applied to possibly flip the $c_2$ carry on if $c, d, c_1$ are all $1$ (which is a real possibility in this case). The seventh layer updated the value of the $c_1$ carry on from the values of $c, d$ (if it was $1$ and gets a $1$ contribution from $c,d$, it is flipped to $0$. The requiered carry on to $c_2$ associated with this flip was allready done by the sixth layer). At the last layer the \verb|CNOT| operation shifts the last digit output to the $d$ qubit. The final output of the addition is stored in qubits $d, c_1, c_2$. 
\end{enumerate}

\newpage

\section{Grover's algorithm on IBM composer}


\section{Acknowledgement}

}

% References 

%-------------------------------------------------------


%%%%%%%%%%%%%%%%%%%%%%%%
% Terminer le document %
%%%%%%%%%%%%%%%%%%%%%%%%
\end{document}
