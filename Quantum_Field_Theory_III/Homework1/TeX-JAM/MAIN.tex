\documentclass[10pt, a4paper]{article}

%%%%%%%%%%%%%%
%  Packages  %
%%%%%%%%%%%%%%


\usepackage{page_format}
\usepackage{special}
\usepackage{hyperref}
\usepackage{tikz}
\usepackage[compat=1.1.0]{tikz-feynman}
%----------------------------------------------------------------------
%\usepackage{amssymb} % Mathematical fonts.
%\usepackage{amsfonts} % Mathematical fonts.
\usepackage[nice]{nicefrac} % Nicer fractions
\usepackage{braket} % Dirac Notation.
\usepackage{bbm} % More bold fonts.
%\usepackage{mathrsfs} % Mathematical fonts.
\usepackage{esint} % Integrals
\usepackage{cancel} % Allows to scratch expressions.
\usepackage{mathtools} % Tools for math formating.
\usepackage{slashed} % Allows to slash individual characters.
\usepackage{xargs} % Better handling of optional arguments for commands
%----------------------------------------------------------------------
%\usepackage{lmodern} % Fonts.
\usepackage{feyn} % Feynman Diagrams in mathmode

%%%%%%%%%%%%%%%%%%%%%%%%%%%
% Mathématiques et physique
%%%%%%%%%%%%%%%%%%%%%%%%%%%%
% SI Units -----------------------
% The package 'siunitx' causes unresolved crashes (as of 22/08/31)
\newcommand{\ampere}{\text{A}}
\newcommand{\bell}{\text{B}}
\newcommand{\celsius}{\degree\text{C}}
\newcommand{\coulomb}{\text{C}}
\newcommand{\degree}{\,^{\circ}}
\newcommand{\farad}{\text{F}}
\newcommand{\electro}{\text{e}}
\newcommand{\gram}{\text{g}}
\newcommand{\henry}{\text{H}}
\newcommand{\hertz}{\text{Hz}}
\newcommand{\hour}{\text{h}}
\newcommand{\joule}{\text{J}}
\newcommand{\kelvin}{\text{K}}
\newcommand{\meter}{\text{m}}
\newcommand{\minute}{\text{m}}
\newcommand{\mole}{\text{mol}}
\newcommand{\newton}{\text{N}}
\newcommand{\ohm}{\Omega}
\newcommand{\pascal}{\text{Pa}}
\newcommand{\rad}{\text{rad}}
\newcommand{\second}{\text{s}}
\newcommand{\tesla}{\text{T}}
\newcommand{\torr}{\text{Torr}}
\newcommand{\volt}{\text{V}}
\newcommand{\watt}{\text{W}}
%
\newcommand{\tera}{\text{T}}
\newcommand{\giga}{\text{G}}
\newcommand{\mega}{~\text{M}}
\newcommand{\kilo}{~\text{k}}
\newcommand{\deci}{\text{d}}
\newcommand{\centi}{\text{c}}
\newcommand{\milli}{\text{m}}
\newcommand{\micro}{\mu}
\newcommand{\nano}{\text{n}}
\newcommand{\pico}{\text{p}}
\newcommand{\femto}{\text{f}}
%
\newcommand{\units}[1]{\text{#1}}
\newcommand{\tothe}[1]{\textsuperscript{#1}}
%
\newcommand{\per}{\text{/}}
%
\newcommand{\Time}[3]{#1\hour~#2\minute~#3\second} % TODO Optional arguments.
\newcommand{\Angle}[3]{#1^{\circ}~#2'~#3''} % TODO Optional arguments.


% Better epsilon -----------------------
\let\oldepsilon\epsilon
\let\epsilon\varepsilon
\let\varepsilon\oldepsilon


% Better \bar -----------------------
\renewcommand{\bar}[1]{\mkern 1.5mu\overline{\mkern-1.5mu#1\mkern-1.5mu}\mkern 1.5mu}


% Équations -----------------------
\newcommand{\al}[1]{\begin{align} #1 \end{align}} % Numbered equation(s),
\newcommand{\eqn}[1]{\begin{align*} #1 \end{align*}} % Number-less equation(s),
\newcommand{\sys}[1]{\begin{dcases*} #1 \end{dcases*}} % System of equations.


% Exponents -----------------------
\newcommand{\Exp}[1]{\text{e}^{#1}}		% e^#
\newcommand{\E}[1]{\times 10^{#1}}		% X 10^#


% Delimiters -----------------------
\newcommand{\p}[1]{\left( #1 \right)}	% (#)
\newcommand{\cro}[1]{\left[ #1 \right]}	% [#]
\newcommand{\abs}[1]{\left| #1\right|}	% |#|
\newcommand{\avg}[1]{\left\langle #1 \right\rangle} % <#>
\newcommand{\acc}[1]{\left\lbrace #1 \right\rbrace} % {#}


% Vectors -----------------------
\newcommand{\ve}[1]{\mathbf{#1}} % Upright bold face.
\newcommand{\vu}[1]{\hat{\ve{#1}}} % Hat vector upright bold face
\newcommand{\tens}{\otimes} % Tensor product
\newcommand{\nablav}{\bm{\nabla}} % Bold gradient


% Trig. functions with automatic formating  -----------------------
\newcommandx{\Sin}[2][1={}]{\text{sin}^{#1}\!\p{#2}}
\newcommandx{\Cos}[2][1={}]{\text{cos}^{#1}\!\p{#2}}
\newcommandx{\Tan}[2][1={}]{\text{tan}^{#1}\!\p{#2}}
\newcommandx{\Csc}[2][1={}]{\text{csc}^{#1}\!\p{#2}}
\newcommandx{\Sec}[2][1={}]{\text{sec}^{#1}\!\p{#2}}
\newcommandx{\Cot}[2][1={}]{\text{cot}^{#1}\!\p{#2}}
\newcommandx{\Arcsin}[2][1={}]{\text{arcsin}^{#1}\!\p{#2}}
\newcommandx{\Arccos}[2][1={}]{\text{arccos}^{#1}\!\p{#2}}
\newcommandx{\Arctan}[2][1={}]{\text{arctan}^{#1}\!\p{#2}}
\newcommandx{\Sinh}[2][1={}]{\text{sinh}^{#1}\!\p{#2}}
\newcommandx{\Cosh}[2][1={}]{\text{cosh}^{#1}\!\p{#2}}
\newcommandx{\Tanh}[2][1={}]{\text{tanh}^{#1}\!\p{#2}}


% Matrices -----------------------
\newcommand{\mat}[1]{\begin{bmatrix} #1 \end{bmatrix}} % Matrices with hooks.
\newcommand{\pmat}[1]{\begin{pmatrix} #1 \end{pmatrix}} % Matrices with parentheses.
\newcommand{\deter}[1]{\abs{\begin{matrix} #1 \end{matrix}}} % Determinant.
\newcommandx{\mO}[2][1={}, 2={}]{ \def\temp{#2}\ifx\temp\empty\ve{O}_{#1}\else\ve{O}_{#1\times #2}\fi}% Zero matrix.
\newcommandx{\mI}[2][1={}, 2={}]{ \def\temp{#2}\ifx\temp\empty\ve{I}_{#1}\else\ve{O}_{#1\times #2}\fi}%  Identity matrix.
\newcommand{\Det}[1]{\text{det}\p{#1}} % det(#)
\newcommand{\Tr}[1]{\text{Tr}\p{#1}} % Tr(#)


% Derivatives -----------------------
\newcommand{\D}{\text{d}} % Differential 'd'.
\newcommandx{\dd}[3][1={},3={}]{\frac{\D^{#3}#1}{\D{#2}^{#3}}} % Total derivative according to #2, #1 is the function and #3 is the order.
\newcommand{\del}{\partial} % Partial 'd'.
\newcommandx{\ddp}[3][1={},3={}]{\frac{\del^{#3}#1}{\del{#2}^{#3}}} % Dérivée partielle selon #2, #1 est la fonction est #3 est l'ordre.
\newcommand{\eval}[1]{\left. {#1} \right|} % Bar on the right of expression.
\newcommand{\delbar}{\slashed{\del}} % Partial Inexact differential.
\newcommand{\dbar}{\dj}% Inexact differential.


% Integrals -----------------------
\newcommand{\intinf}{\int\displaylimits_{-\infty}^{\infty}} % From -00 to 00.
\newcommandx{\Int}[2][1={},2={}]{\int\displaylimits_{#1}^{#2}} % Faster bounded integrals.


% Complex numbers -----------------------
\renewcommand{\Re}[1]{\text{Re}\acc{#1}} % Re{#}
\renewcommand{\Im}[1]{\text{Im}\acc{#1}} % Im{#}


% Sets -----------------------
\newcommand{\N}{\mathbbm{N}} % Natural numbers.
\newcommand{\Z}{\mathbbm{Z}} % Integers.
\newcommand{\Q}{\mathbbm{Q}} % Rational numbers.
\newcommandx{\R}[1][1={}]{\mathbbm{R}^{#1}} % Real numbers.
\newcommandx{\C}[1][1={}]{\mathbbm{C}^{#1}} % Complex numbers.
\newcommandx{\F}[1][1={}]{\mathbbm{F}^{#1}} % Some field.
\newcommand{\M}[3]{\mathbb{M}_{#1\times#2}(#3)}	% Matrices.
\newcommand{\Po}[2]{\mathbb{P}_{#1}(#2)} % Polynomials.
\newcommand{\Lin}{\mathbb{L}} % Linear maps.


% Constants and physical symbols -----------------------
\newcommand{\eo}{\epsilon_0} % epsilon 0.
\renewcommand{\L}{\mathcal{L}} % Lagrangian.

\usepackage{slashed}

% References
\usepackage{biblatex}
\addbibresource{ref.bib}
\usetikzlibrary{positioning}


%%%%%%%%%%%%
%  Colors  %
%%%%%%%%%%%%
% ! EDIT HERE !
\colorlet{chaptercolor}{red!70!black} % Foreground color.
\colorlet{chaptercolorback}{red!10!white} % Background color

%%%%%%%%%%%%%%
% Page titre %
%%%%%%%%%%%%%%%
\title{Homework 1} % Title of the assignement.
\author{\PA} % Your name(s).
\teacher{Jaume Gomis and Mykola Semenyakin} % Your teacher's name.
\class{Quantum Field Theory III} % The class title.

\university{Perimeter Institute for Theoretical Physics} % University
\faculty{Perimeter Scholars International} % Faculty
%\departement{<Departement>} % Departement
\date{\today} % Date.


%%%%%%%%%%%%%%%%%%%%%%
% Begin the document %
%%%%%%%%%%%%%%%%%%%%%%
\begin{document}

% Make the title page.
\maketitlepage

% Make table of contents
\maketableofcontents

% Assignment starts here ----------------------------

\footnotesize{

\section{Conformal invariance of the Maxwell action for $D=4$}

\begin{enumerate}
  \item[(a)] Consider a classical abelian gauge field $A_\mu$ on $D=3+1$ dimensionnal Minkowski spacetime. Under an infinitesimal conformal transformation, spacetime undergoes the transformation $\tilde{x}^{\mu} = f(x) = x^{\mu} + \xi^{\mu}(x)$ where $\xi^\mu(x)$ is a smal deformation. We want to calculate the effect of this transformation on the gauge field $A_\mu$. The starting point is that we expect $A_\mu$ to transform as a tensor under the lorenz transformation subgroup of the conformal group. This implies that $A_\mu$ is a primary operator and we denote its scaling dimension $\Delta$. The transformed field $\tilde{A}_\mu$ at $\tilde{x}$ is related to the original field $A_\mu$ at $x$ by an internal rotation, scaling, and special conformal transformation. The rotation operation acts on the components $A_\mu$ through its spin $1$ representation which is the defnining representation of rotations. The scaling and special conformal transformation act together through the multiplication of $A_\mu$ by the jacobian factor $|\partial x/\partial \tilde{x}|_{x}^{\Delta/D}$. Finally, translations act trivially internally. This can be summarized with the relation $\tilde{A}_\mu(\tilde x) = |\partial x/\partial \tilde{x}|_{x}^{\Delta/D} R_{\mu}^{\nu} A_{\nu}(x)$ where $R_{\mu}^{\nu}$ is the rotation matrix associated with $\xi^\mu(x)$. With this in mind, we calculate the jacobian of the infinitesimal transformation to be 
  \begin{align*}
    \left|\frac{\partial x^\mu }{\partial \tilde{x}^\nu}\right|_{x} = \left|\frac{\partial \tilde{x}^\mu }{\partial x^\nu}\right|_{x}^{-1} = |\delta_\nu^\mu + \partial_\nu \xi^\mu|_{x}^{-1} \approx |e^{-\partial_\nu \xi^\mu}|_{x} = e^{-\text{Tr} \partial_\nu \xi^\mu(x)} = 1- \partial_\mu \xi^\mu(x) + O(\xi^2). 
  \end{align*}
  The rotation matrix $R_{\mu}^{\nu} (x)$ can be extracted by dividing the matrix $(\partial x/\partial \tilde{x})_{x}$ by its jacobian to extract the $SO(3)$ operation (we have a positive determinant transformation since it is infinitesimally close to identity). We have 
  \begin{align*}
    R_{\mu}^{\nu} (x) = \frac{1}{1- \partial_\sigma \xi^\sigma(x) + O(\xi^2)} \left(\frac{\partial x^\nu }{\partial \tilde{x}^\mu}\right)_{x} = (1 + \partial_\sigma \xi^\sigma(x) + O(\xi^2))(\delta_\nu^\mu + \partial_\mu \xi^\nu(x) + O(\xi^2))^{-1} = \delta_\nu^\mu(1 + \partial_\sigma \xi^\sigma(x))  - \partial_\mu \xi^\nu(x) +  O(\xi^2). 
  \end{align*} 
  For a spacial conformal transformation, we have $\xi^\mu = -2 x^\mu x_\lambda b^\lambda$ paramatrized by the transaltion vector $b^\lambda$ around $\infty$. For this vector, we have 
  \begin{align*}
    R_{\mu}^{\nu} (x) = \delta_\nu^\mu + \partial_\sigma \xi^\sigma(x)  - \partial_\mu \xi^\nu(x) +  O(\xi^2) = \delta_\nu^\mu 
  \end{align*}
  As expected for a special conformal transformation. With these results, we can write the effect of the infinitesimal transformation as 
  \begin{align*}
    \tilde{A}_\mu(\tilde{x}) &= (1-\partial_\rho \xi^\rho(f^{-1}(\tilde{x})) + O(\xi^2))^{\Delta/D} (A_\mu(f^{-1}(\tilde{x})) + A_\mu(f^{-1}(\tilde{x})) \partial_\sigma \xi^\sigma(f^{-1}(\tilde{x}))  - A_\nu(f^{-1}(\tilde{x})) \partial_\mu \xi^\nu(f^{-1}(\tilde{x})) +  O(\xi^2)) \\
    &=  \left(1-\frac{\Delta}{D}\partial_\rho \xi^\rho(f^{-1}(\tilde{x})) + O(\xi^2)\right) (A_\mu(f^{-1}(\tilde{x})) + A_\mu(f^{-1}(\tilde{x})) \partial_\sigma \xi^\sigma(f^{-1}(\tilde{x}))  - A_\nu(f^{-1}(\tilde{x})) \partial_\mu \xi^\nu(f^{-1}(\tilde{x})) +  O(\xi^2))\\
    &= A_\mu(f^{-1}(\tilde{x}))-A_\mu(f^{-1}(\tilde{x})) \frac{\Delta}{D}\partial_\sigma \xi^\sigma(f^{-1}(\tilde{x}))+A_\mu(f^{-1}(\tilde{x})) \partial_\sigma \xi^\sigma(f^{-1}(\tilde{x}))  - A_\nu(f^{-1}(\tilde{x})) \partial_\mu \xi^\nu(f^{-1}(\tilde{x})) + O(\xi^2). 
  \end{align*}
  Since $\xi(f^{-1}(\tilde{x}))$ is already first order in $\xi$, the only term contribution to its expansion around $\xi = 0$ at $O(\xi)$ is $\xi(\tilde{x})$. To go further, we expand $f^{-1}(\tilde{x})$ at first order in $\xi(\tilde{x})$ with the ansatz $f^{-1}(\tilde{x})^{\nu} = \tilde{x}^{\nu} + B_{\mu}^{\nu}(\tilde{x}) \xi^{\mu}(\tilde{x})$ (the first term of this ansatz is justified by noticing the transformation reduces to identity at $\xi = 0$).  From $f(f^{-1}(\tilde{x})) = \tilde{x}$, we find 
  \begin{align*}
    \tilde{x}^{\nu} = \tilde{x}^{\nu} + B_\mu^{\nu}(\tilde{x}) \xi^\mu(\tilde{x}) + \xi(\tilde{x}^{\nu} +  B_\mu^{\nu}(\tilde{x}) \xi^\mu(\tilde{x})) + O(\xi^2) \implies B_\mu^{\nu}(\tilde{x}) \xi^\mu(\tilde{x}) + \xi^{\nu}(\tilde{x}) = 0, \quad  \forall \xi(\tilde{x}) \implies B_\mu^{\nu}(\tilde{x}) = -\delta_{\mu}^{\nu}.
  \end{align*}
  Using this result, we can expand $A_\mu(f^{-1}(\tilde{x}))$ as 
  \begin{align*}
    A_\mu(f^{-1}(\tilde{x})) = A_\mu(\tilde{x}^\nu-\xi^{\nu}(\tilde{x}) + O(\xi^2)) = A_\mu(\tilde{x}) - \xi^{\nu}(\tilde{x}) \partial_\nu A_\mu(\tilde{x}) + O(\xi^2)
  \end{align*}
  Combining this expression with the internal transformation at first order in $\xi$, we get 
  \begin{align*}
    \tilde{A}_\mu(\tilde{x}) &= \left(1-\frac{\Delta}{D}\partial_\sigma \xi^\sigma(\tilde{x})+\partial_\sigma \xi^\sigma(\tilde{x})  - \partial_\mu \xi^\nu(\tilde{x})\right)(A_\mu(\tilde{x}) - \xi^{\nu}(\tilde{x}) \partial_\nu A_\mu(\tilde{x})) + O(\xi^2)\\
    &= A_\mu(\tilde{x})-A_\mu(\tilde{x}) \frac{\Delta-D}{D}\partial_\sigma \xi^\sigma(\tilde{x}) - A_\nu(\tilde{x})\partial_\mu \xi^\nu(\tilde{x}) - \xi^{\nu}(\tilde{x}) \partial_\nu A_\mu(\tilde{x})+ O(\xi^2)
  \end{align*}
  with $\xi(f^{-1}(\tilde{x})) = \xi(\tilde{x}) + O(\xi^2)$. Form this transformed gauge field, we calculate the transformation of gauge field strength $F_{\mu\nu} = \partial_\mu A_\nu - \partial_\nu A_\mu$ to $\tilde{F}_{\mu\nu}$. We start by writting the transformation law of the derivatives used to construct $F_{\mu\nu}$. The chain rule yields
  \begin{align*}
    \tilde{\partial}_{\mu} \equiv \frac{\partial }{\partial \tilde{x}^\mu} = \left(\frac{\partial f^{-1}(\tilde{x})^\nu}{\partial \tilde{x}^\mu}\right)_{\tilde{x}}\left(\frac{\partial }{\partial x^\nu}\right)_{\tilde{x}} = \left(\frac{\partial \tilde{x}^{\nu} - \xi^{\nu}(\tilde{x})}{\partial \tilde{x}^\mu}\right)_{\tilde{x}}\left(\frac{\partial }{\partial x^\nu}\right)_{\tilde{x}} = \left(\frac{\partial \xi^{\nu}(\tilde{x})}{\partial \tilde{x}^\mu}\right)_{\tilde{x}}\left(\frac{\partial }{\partial x^\nu}\right)_{\tilde{x}} + \left(\frac{\partial }{\partial x^\mu}\right)_{\tilde{x}} \equiv \partial_\mu \xi^{\nu}(\tilde{x}) \partial_\nu + \partial_\mu. 
  \end{align*}
  Now we can calculate the transformed transformed field strength to be 
  \begin{align*}
    \tilde{F}_{\mu\nu} &= \tilde{\partial}_\mu \tilde{A}_{\nu} - (\mu \leftrightarrow \nu) \\
    &= \left(\partial_\mu \xi^{\rho}(\tilde{x}) \partial_\rho + \partial_\mu\right)\left(A_\nu(\tilde{x})-A_\nu(\tilde{x}) \frac{\Delta-D}{D}\partial_\sigma \xi^\sigma(\tilde{x}) - A_\lambda(\tilde{x})\partial_\nu \xi^\lambda(\tilde{x}) - \xi^{\lambda}(\tilde{x}) \partial_\lambda A_\nu(\tilde{x})\right)- (\mu \leftrightarrow \nu)\\
    &= \partial_\mu A_\nu(\tilde{x}) + \partial_\mu \xi^{\lambda}(\tilde{x}) \partial_\lambda A_\nu(\tilde{x}) -\partial_\mu \left(A_\nu(\tilde{x}) \frac{\Delta-D}{D}\partial_\lambda \xi^\lambda(\tilde{x})\right) - \partial_\mu\left(A_\lambda(\tilde{x})\partial_\nu \xi^\lambda(\tilde{x})\right) - \partial_\mu\left(\xi^{\lambda}(\tilde{x}) \partial_\lambda A_\nu(\tilde{x})\right) - (\mu \leftrightarrow \nu)\\
    &= \partial_\mu A_\nu(\tilde{x}) -\partial_\mu \left(A_\nu(\tilde{x}) \frac{\Delta-D}{D}\partial_\lambda \xi^\lambda(\tilde{x})\right) - \partial_\mu A_\lambda(\tilde{x})\partial_\nu \xi^\lambda(\tilde{x}) - A_\lambda(\tilde{x})\partial_\mu\partial_\nu \xi^\lambda(\tilde{x}) - \xi^{\lambda}(\tilde{x}) \partial_\lambda \partial_\mu A_\nu(\tilde{x}) - (\mu \leftrightarrow \nu)\\
    &= \partial_\mu A_\nu(\tilde{x}) -\partial_\mu \left(A_\nu(\tilde{x}) \frac{\Delta-D}{D}\partial_\lambda \xi^\lambda(\tilde{x})\right) - \partial_\mu A_\lambda(\tilde{x})\partial_\nu \xi^\lambda(\tilde{x}) - \xi^{\lambda}(\tilde{x}) \partial_\lambda \partial_\mu A_\nu(\tilde{x}) - (\mu \leftrightarrow \nu)\\
    &= F_{\mu \nu}(\tilde{x}) - F_{\mu \nu}(\tilde{x}) \frac{\Delta-D}{D}\partial_\lambda \xi^\lambda(\tilde{x}) - A_{(\nu}(\tilde{x}) \frac{\Delta-D}{D}\partial_{\mu)} \partial_\lambda \xi^\lambda(\tilde{x}) - \partial_{(\mu} A_\lambda(\tilde{x})\partial_{\nu)} \xi^\lambda(\tilde{x}) - \xi^{\lambda}(\tilde{x}) \partial_\lambda F_{\mu\nu}(\tilde{x}) 
  \end{align*}
  \item[(b)] 
\end{enumerate}

\section{Axial anomaly}

\begin{enumerate}
  \item[(a)]
  \item[(b)] 
  \item[(c)]
  \item[(d)] 
\end{enumerate}

\section{OPE coefficients from three point functions}

\begin{enumerate}
  \item[(a)]
  \item[(b)] 
  \item[(c)]
  \item[(d)] 
\end{enumerate}


\section{Acknowledgement}


}

% References
\makereferences
%-------------------------------------------------------


%%%%%%%%%%%%%%%%%%%%%%%%
% Terminer le document %
%%%%%%%%%%%%%%%%%%%%%%%%
\end{document}
