\documentclass[10pt, a4paper]{article}

%%%%%%%%%%%%%%
%  Packages  %
%%%%%%%%%%%%%%


\usepackage{page_format}
\usepackage{special}
\usepackage{hyperref}
\usepackage{tikz}
\usepackage[compat=1.1.0]{tikz-feynman}
%----------------------------------------------------------------------
%\usepackage{amssymb} % Mathematical fonts.
%\usepackage{amsfonts} % Mathematical fonts.
\usepackage[nice]{nicefrac} % Nicer fractions
\usepackage{braket} % Dirac Notation.
\usepackage{bbm} % More bold fonts.
%\usepackage{mathrsfs} % Mathematical fonts.
\usepackage{esint} % Integrals
\usepackage{cancel} % Allows to scratch expressions.
\usepackage{mathtools} % Tools for math formating.
\usepackage{slashed} % Allows to slash individual characters.
\usepackage{xargs} % Better handling of optional arguments for commands
%----------------------------------------------------------------------
%\usepackage{lmodern} % Fonts.
\usepackage{feyn} % Feynman Diagrams in mathmode

%%%%%%%%%%%%%%%%%%%%%%%%%%%
% Mathématiques et physique
%%%%%%%%%%%%%%%%%%%%%%%%%%%%
% SI Units -----------------------
% The package 'siunitx' causes unresolved crashes (as of 22/08/31)
\newcommand{\ampere}{\text{A}}
\newcommand{\bell}{\text{B}}
\newcommand{\celsius}{\degree\text{C}}
\newcommand{\coulomb}{\text{C}}
\newcommand{\degree}{\,^{\circ}}
\newcommand{\farad}{\text{F}}
\newcommand{\electro}{\text{e}}
\newcommand{\gram}{\text{g}}
\newcommand{\henry}{\text{H}}
\newcommand{\hertz}{\text{Hz}}
\newcommand{\hour}{\text{h}}
\newcommand{\joule}{\text{J}}
\newcommand{\kelvin}{\text{K}}
\newcommand{\meter}{\text{m}}
\newcommand{\minute}{\text{m}}
\newcommand{\mole}{\text{mol}}
\newcommand{\newton}{\text{N}}
\newcommand{\ohm}{\Omega}
\newcommand{\pascal}{\text{Pa}}
\newcommand{\rad}{\text{rad}}
\newcommand{\second}{\text{s}}
\newcommand{\tesla}{\text{T}}
\newcommand{\torr}{\text{Torr}}
\newcommand{\volt}{\text{V}}
\newcommand{\watt}{\text{W}}
%
\newcommand{\tera}{\text{T}}
\newcommand{\giga}{\text{G}}
\newcommand{\mega}{~\text{M}}
\newcommand{\kilo}{~\text{k}}
\newcommand{\deci}{\text{d}}
\newcommand{\centi}{\text{c}}
\newcommand{\milli}{\text{m}}
\newcommand{\micro}{\mu}
\newcommand{\nano}{\text{n}}
\newcommand{\pico}{\text{p}}
\newcommand{\femto}{\text{f}}
%
\newcommand{\units}[1]{\text{#1}}
\newcommand{\tothe}[1]{\textsuperscript{#1}}
%
\newcommand{\per}{\text{/}}
%
\newcommand{\Time}[3]{#1\hour~#2\minute~#3\second} % TODO Optional arguments.
\newcommand{\Angle}[3]{#1^{\circ}~#2'~#3''} % TODO Optional arguments.


% Better epsilon -----------------------
\let\oldepsilon\epsilon
\let\epsilon\varepsilon
\let\varepsilon\oldepsilon


% Better \bar -----------------------
\renewcommand{\bar}[1]{\mkern 1.5mu\overline{\mkern-1.5mu#1\mkern-1.5mu}\mkern 1.5mu}


% Équations -----------------------
\newcommand{\al}[1]{\begin{align} #1 \end{align}} % Numbered equation(s),
\newcommand{\eqn}[1]{\begin{align*} #1 \end{align*}} % Number-less equation(s),
\newcommand{\sys}[1]{\begin{dcases*} #1 \end{dcases*}} % System of equations.


% Exponents -----------------------
\newcommand{\Exp}[1]{\text{e}^{#1}}		% e^#
\newcommand{\E}[1]{\times 10^{#1}}		% X 10^#


% Delimiters -----------------------
\newcommand{\p}[1]{\left( #1 \right)}	% (#)
\newcommand{\cro}[1]{\left[ #1 \right]}	% [#]
\newcommand{\abs}[1]{\left| #1\right|}	% |#|
\newcommand{\avg}[1]{\left\langle #1 \right\rangle} % <#>
\newcommand{\acc}[1]{\left\lbrace #1 \right\rbrace} % {#}


% Vectors -----------------------
\newcommand{\ve}[1]{\mathbf{#1}} % Upright bold face.
\newcommand{\vu}[1]{\hat{\ve{#1}}} % Hat vector upright bold face
\newcommand{\tens}{\otimes} % Tensor product
\newcommand{\nablav}{\bm{\nabla}} % Bold gradient


% Trig. functions with automatic formating  -----------------------
\newcommandx{\Sin}[2][1={}]{\text{sin}^{#1}\!\p{#2}}
\newcommandx{\Cos}[2][1={}]{\text{cos}^{#1}\!\p{#2}}
\newcommandx{\Tan}[2][1={}]{\text{tan}^{#1}\!\p{#2}}
\newcommandx{\Csc}[2][1={}]{\text{csc}^{#1}\!\p{#2}}
\newcommandx{\Sec}[2][1={}]{\text{sec}^{#1}\!\p{#2}}
\newcommandx{\Cot}[2][1={}]{\text{cot}^{#1}\!\p{#2}}
\newcommandx{\Arcsin}[2][1={}]{\text{arcsin}^{#1}\!\p{#2}}
\newcommandx{\Arccos}[2][1={}]{\text{arccos}^{#1}\!\p{#2}}
\newcommandx{\Arctan}[2][1={}]{\text{arctan}^{#1}\!\p{#2}}
\newcommandx{\Sinh}[2][1={}]{\text{sinh}^{#1}\!\p{#2}}
\newcommandx{\Cosh}[2][1={}]{\text{cosh}^{#1}\!\p{#2}}
\newcommandx{\Tanh}[2][1={}]{\text{tanh}^{#1}\!\p{#2}}


% Matrices -----------------------
\newcommand{\mat}[1]{\begin{bmatrix} #1 \end{bmatrix}} % Matrices with hooks.
\newcommand{\pmat}[1]{\begin{pmatrix} #1 \end{pmatrix}} % Matrices with parentheses.
\newcommand{\deter}[1]{\abs{\begin{matrix} #1 \end{matrix}}} % Determinant.
\newcommandx{\mO}[2][1={}, 2={}]{ \def\temp{#2}\ifx\temp\empty\ve{O}_{#1}\else\ve{O}_{#1\times #2}\fi}% Zero matrix.
\newcommandx{\mI}[2][1={}, 2={}]{ \def\temp{#2}\ifx\temp\empty\ve{I}_{#1}\else\ve{O}_{#1\times #2}\fi}%  Identity matrix.
\newcommand{\Det}[1]{\text{det}\p{#1}} % det(#)
\newcommand{\Tr}[1]{\text{Tr}\p{#1}} % Tr(#)


% Derivatives -----------------------
\newcommand{\D}{\text{d}} % Differential 'd'.
\newcommandx{\dd}[3][1={},3={}]{\frac{\D^{#3}#1}{\D{#2}^{#3}}} % Total derivative according to #2, #1 is the function and #3 is the order.
\newcommand{\del}{\partial} % Partial 'd'.
\newcommandx{\ddp}[3][1={},3={}]{\frac{\del^{#3}#1}{\del{#2}^{#3}}} % Dérivée partielle selon #2, #1 est la fonction est #3 est l'ordre.
\newcommand{\eval}[1]{\left. {#1} \right|} % Bar on the right of expression.
\newcommand{\delbar}{\slashed{\del}} % Partial Inexact differential.
\newcommand{\dbar}{\dj}% Inexact differential.


% Integrals -----------------------
\newcommand{\intinf}{\int\displaylimits_{-\infty}^{\infty}} % From -00 to 00.
\newcommandx{\Int}[2][1={},2={}]{\int\displaylimits_{#1}^{#2}} % Faster bounded integrals.


% Complex numbers -----------------------
\renewcommand{\Re}[1]{\text{Re}\acc{#1}} % Re{#}
\renewcommand{\Im}[1]{\text{Im}\acc{#1}} % Im{#}


% Sets -----------------------
\newcommand{\N}{\mathbbm{N}} % Natural numbers.
\newcommand{\Z}{\mathbbm{Z}} % Integers.
\newcommand{\Q}{\mathbbm{Q}} % Rational numbers.
\newcommandx{\R}[1][1={}]{\mathbbm{R}^{#1}} % Real numbers.
\newcommandx{\C}[1][1={}]{\mathbbm{C}^{#1}} % Complex numbers.
\newcommandx{\F}[1][1={}]{\mathbbm{F}^{#1}} % Some field.
\newcommand{\M}[3]{\mathbb{M}_{#1\times#2}(#3)}	% Matrices.
\newcommand{\Po}[2]{\mathbb{P}_{#1}(#2)} % Polynomials.
\newcommand{\Lin}{\mathbb{L}} % Linear maps.


% Constants and physical symbols -----------------------
\newcommand{\eo}{\epsilon_0} % epsilon 0.
\renewcommand{\L}{\mathcal{L}} % Lagrangian.

\usepackage{slashed}

% References
\usepackage{biblatex}
\addbibresource{ref.bib}
\usetikzlibrary{positioning}


%%%%%%%%%%%%
%  Colors  %
%%%%%%%%%%%%
% ! EDIT HERE !
\colorlet{chaptercolor}{red!70!black} % Foreground color.
\colorlet{chaptercolorback}{red!10!white} % Background color

%%%%%%%%%%%%%%
% Page titre %
%%%%%%%%%%%%%%%
\title{Homework 1} % Title of the assignement.
\author{\PA} % Your name(s).
\teacher{Jaume Gomis and Mykola Semenyakin} % Your teacher's name.
\class{Quantum Field Theory III} % The class title.

\university{Perimeter Institute for Theoretical Physics} % University
\faculty{Perimeter Scholars International} % Faculty
%\departement{<Departement>} % Departement
\date{\today} % Date.


%%%%%%%%%%%%%%%%%%%%%%
% Begin the document %
%%%%%%%%%%%%%%%%%%%%%%
\begin{document}

% Make the title page.
\maketitlepage

% Make table of contents
\maketableofcontents

% Assignment starts here ----------------------------

\footnotesize{

\section{Conformal invariance of the Maxwell action for $D=4$}

\begin{enumerate}
  \item[(a)] Consider a classical abelian gauge field $A_\mu$ on $D=4$ dimensionnal Minkowski spacetime. Under an infinitesimal conformal transformation, spacetime undergoes the transformation $\tilde{x}^{\mu} = f(x) = x^{\mu} + \xi^{\mu}(x)$ where $\xi^\mu(x)$ is a smal deformation. We want to calculate the effect of this transformation on the gauge field $A_\mu$. The starting point is that we expect $A_\mu$ to transform as a tensor under the Lorenz transformation subgroup of the conformal group. This implies that $A_\mu$ is a primary operator and we denote its scaling dimension $\Delta$. The transformed field $\tilde{A}_\mu$ at $\tilde{x}$ is related to the original field $A_\mu$ at $x$ by an internal rotation, scaling, and special conformal transformation. The rotation operation acts on the components $A_\mu$ through its spin $1$ representation which is the defining representation of rotations. The scaling and special conformal transformation act together through the multiplication of $A_\mu$ by the Jacobian factor $|\partial x/\partial \tilde{x}|_{x}^{\Delta/D}$. Finally, translations act trivially internally. This can be summarized with the relation $\tilde{A}_\mu(\tilde x) = |\partial x/\partial \tilde{x}|_{x}^{\Delta/D} R_{\mu}^{\nu} A_{\nu}(x)$ where $R_{\mu}^{\nu}$ is the matrix associated with the part of $\xi^\mu(x)$ that does not change the metric components (after the Weyl and diffeomorphism transformations). With this in mind, we calculate the jacobian of the infinitesimal transformation to be 
  \begin{align*}
    \left|\frac{\partial x^\mu }{\partial \tilde{x}^\nu}\right|_{x} = \left|\frac{\partial \tilde{x}^\mu }{\partial x^\nu}\right|_{x}^{-1} = |\delta_\nu^\mu + \partial_\nu \xi^\mu|_{x}^{-1} \approx |e^{-\partial_\nu \xi^\mu}|_{x} = e^{-\text{Tr} \partial_\nu \xi^\mu(x)} = 1- \partial_\mu \xi^\mu(x) + O(\xi^2). 
  \end{align*}
  The matrix $R_{\mu}^{\nu} (x)$ can be extracted by dividing the matrix $(\partial x/\partial \tilde{x})_{x}$ by a factor $\Omega(x)$ such that we extract the "metric component preserving" operation. To find this factor we consider the effect on the metric of $\Omega^{-1}(x)(\partial x/\partial \tilde{x})_{x}$. We can write the "metric component preserving" property as
  \begin{align*}
    \Omega^{-2}(x) \left(\frac{\partial x^{\mu}}{\partial \tilde{x}^\sigma}\right)_{x} \left(\frac{\partial x^\nu}{\partial \tilde{x}^{\rho}}\right)_{x} \eta_{\mu\nu} = \eta_{\sigma \rho}.
  \end{align*}
  Since $\Omega(x)$ is a factor, we can extract it by taking the determinant on both sides of the previous relation to get 
  \begin{align*}
    \det(\eta) \Omega(x)^{-2D} \left|\frac{\partial x^\mu }{\partial \tilde{x}^\nu}\right|_{x}^2 = \det(\eta) \iff \Omega(x) = \left|\frac{\partial x^\mu }{\partial \tilde{x}^\nu}\right|_{x}^{-\frac{1}{D}}.
  \end{align*}
  This result can be intuitively understood from the fact the Jacobian measures $D$-volume rescaling. Since we want metric components (associated with distances) to be preserved by the rescaled transformation, we need to divide by the $D$-root of the jacobian. The matrix $R_{\mu}^{\nu} (x)$ provided by the rescaling is given by 
  \begin{align*}
    R_{\mu}^{\nu} (x) = \frac{1}{(1- \partial_\sigma \xi^\sigma(x) + O(\xi^2))^{1/D}} \left(\frac{\partial x^\nu }{\partial \tilde{x}^\mu}\right)_{x} &= (1 + \partial_\sigma \xi^\sigma(x)/D + O(\xi^2))(\delta_\nu^\mu + \partial_\mu \xi^\nu(x) + O(\xi^2))^{-1}\\
    &= \delta_\nu^\mu(1 + \partial_\sigma \xi^\sigma(x)/D)  - \partial_\mu \xi^\nu(x) +  O(\xi^2). 
  \end{align*} 
  We note that $R_{\mu}^{\nu} (x)$ will represent a rotation if $\partial_\sigma \xi^\sigma(x) = 0$ (bring the conformal Killing equation to the normal Killing equation with a rotation isometry as its solution). If $\partial_\sigma \xi^\sigma(x) \neq 0$, the rescaled transformation contains a special conformal transformation. The special conformal transformation as a Weyl transformation does not preserve distances but can be combined with a diffeomorphism to preserve the initial components of the metric. With these results, we can write the effect of the infinitesimal transformation as 
  \begin{align*}
    \tilde{A}_\mu(\tilde{x}) &= (1-\partial_\rho \xi^\rho(f^{-1}(\tilde{x})) + O(\xi^2))^{\Delta/D} (A_\mu(f^{-1}(\tilde{x})) + A_\mu(f^{-1}(\tilde{x})) \partial_\sigma \xi^\sigma(f^{-1}(\tilde{x}))\frac{1}{D}  - A_\nu(f^{-1}(\tilde{x})) \partial_\mu \xi^\nu(f^{-1}(\tilde{x})) +  O(\xi^2)) \\
    &=  \left(1-\frac{\Delta}{D}\partial_\rho \xi^\rho(f^{-1}(\tilde{x})) + O(\xi^2)\right) (A_\mu(f^{-1}(\tilde{x})) + A_\mu(f^{-1}(\tilde{x})) \partial_\sigma \xi^\sigma(f^{-1}(\tilde{x}))\frac{1}{D} - A_\nu(f^{-1}(\tilde{x})) \partial_\mu \xi^\nu(f^{-1}(\tilde{x})) +  O(\xi^2))\\
    &= A_\mu(f^{-1}(\tilde{x}))-A_\mu(f^{-1}(\tilde{x})) \frac{\Delta}{D}\partial_\sigma \xi^\sigma(f^{-1}(\tilde{x}))+A_\mu(f^{-1}(\tilde{x})) \partial_\sigma \xi^\sigma(f^{-1}(\tilde{x})) \frac{1}{D}  - A_\nu(f^{-1}(\tilde{x})) \partial_\mu \xi^\nu(f^{-1}(\tilde{x})) + O(\xi^2). 
  \end{align*}
  Since $\xi(f^{-1}(\tilde{x}))$ is already first order in $\xi$, the only term contribution to its expansion around $\xi = 0$ at $O(\xi)$ is $\xi(\tilde{x})$. To go further, we expand $f^{-1}(\tilde{x})$ at first order in $\xi(\tilde{x})$ with the ansatz $f^{-1}(\tilde{x})^{\nu} = \tilde{x}^{\nu} + B_{\mu}^{\nu}(\tilde{x}) \xi^{\mu}(\tilde{x})$ (the first term of this ansatz is justified by noticing the transformation reduces to identity at $\xi = 0$).  From $f(f^{-1}(\tilde{x})) = \tilde{x}$, we find 
  \begin{align*}
    \tilde{x}^{\nu} = \tilde{x}^{\nu} + B_\mu^{\nu}(\tilde{x}) \xi^\mu(\tilde{x}) + \xi(\tilde{x}^{\nu} +  B_\mu^{\nu}(\tilde{x}) \xi^\mu(\tilde{x})) + O(\xi^2) \implies B_\mu^{\nu}(\tilde{x}) \xi^\mu(\tilde{x}) + \xi^{\nu}(\tilde{x}) = 0, \quad  \forall \xi(\tilde{x}) \implies B_\mu^{\nu}(\tilde{x}) = -\delta_{\mu}^{\nu}.
  \end{align*}
  Using this result, we can expand $A_\mu(f^{-1}(\tilde{x}))$ as 
  \begin{align*}
    A_\mu(f^{-1}(\tilde{x})) = A_\mu(\tilde{x}^\nu-\xi^{\nu}(\tilde{x}) + O(\xi^2)) = A_\mu(\tilde{x}) - \xi^{\nu}(\tilde{x}) \partial_\nu A_\mu(\tilde{x}) + O(\xi^2)
  \end{align*}
  Combining this expression with the internal transformation at first order in $\xi$, we get 
  \begin{align*}
    \tilde{A}_\mu(\tilde{x}) &= \left(1-\frac{\Delta}{D}\partial_\sigma \xi^\sigma(\tilde{x})+\partial_\sigma \xi^\sigma(\tilde{x})  - \partial_\mu \xi^\nu(\tilde{x})\right)(A_\mu(\tilde{x}) - \xi^{\nu}(\tilde{x}) \partial_\nu A_\mu(\tilde{x})) + O(\xi^2)\\
    &= A_\mu(\tilde{x})-A_\mu(\tilde{x}) \frac{\Delta-1}{D}\partial_\sigma \xi^\sigma(\tilde{x}) - A_\nu(\tilde{x})\partial_\mu \xi^\nu(\tilde{x}) - \xi^{\nu}(\tilde{x}) \partial_\nu A_\mu(\tilde{x})+ O(\xi^2)
  \end{align*}
  with $\xi(f^{-1}(\tilde{x})) = \xi(\tilde{x}) + O(\xi^2)$. This result can be simplified by using the conformal killing equation $\partial_\mu \xi_\nu+\partial_\nu \xi_\mu=2 \eta_{\mu \nu} \partial_\sigma  \xi^\sigma/D$ as follows:
  \begin{align*}
    \tilde{A}_\mu(\tilde{x}) &= A_\mu(\tilde{x})-A_\mu(\tilde{x}) \frac{\Delta-1}{D}\partial_\sigma \xi^\sigma(\tilde{x}) - A_\nu(\tilde{x})\left(\frac{1}{2}\partial_\mu \xi^\nu(\tilde{x}) + \frac{1}{2}\partial_\mu \xi^\nu(\tilde{x})\right) - \xi^{\nu}(\tilde{x}) \partial_\nu A_\mu(\tilde{x})+ O(\xi^2)\\
    &= A_\mu(\tilde{x})-A_\mu(\tilde{x}) \frac{\Delta-1}{D}\partial_\sigma \xi^\sigma(\tilde{x}) - A_\nu(\tilde{x})\left(\frac{1}{2}\partial_\mu \xi^\nu(\tilde{x}) - \frac{1}{2}\partial_\nu \xi^\mu(\tilde{x}) + \delta_{\mu}^{\nu} \partial_\sigma \xi^\sigma(\tilde{x}) \frac{1}{D}\right) - \xi^{\nu}(\tilde{x}) \partial_\nu A_\mu(\tilde{x})+ O(\xi^2)\\
    &= A_\mu(\tilde{x})-A_\mu(\tilde{x}) \frac{\Delta}{D}\partial_\sigma \xi^\sigma(\tilde{x}) - A_\nu(\tilde{x})\underbrace{\left(\frac{1}{2}\partial_\mu \xi^\nu(\tilde{x}) - \frac{1}{2}\partial^\nu \xi_\mu(\tilde{x})\right)}_{M_{\mu}{}^{\nu}} - \xi^{\nu}(\tilde{x}) \partial_\nu A_\mu(\tilde{x})+ O(\xi^2).
  \end{align*}
  From this transformed gauge field, we calculate the transformation of gauge field strength $F_{\mu\nu} = \partial_\mu A_\nu - \partial_\nu A_\mu$ to $\tilde{F}_{\mu\nu}$. We start by writting the transformation law of the derivatives used to construct $F_{\mu\nu}$. The chain rule yields
  \begin{align*}
    \tilde{\partial}_{\mu} \equiv \frac{\partial }{\partial \tilde{x}^\mu} = \left(\frac{\partial f^{-1}(\tilde{x})^\nu}{\partial \tilde{x}^\mu}\right)_{\tilde{x}}\left(\frac{\partial }{\partial x^\nu}\right)_{\tilde{x}} = \left(\frac{\partial \tilde{x}^{\nu} - \xi^{\nu}(\tilde{x})}{\partial \tilde{x}^\mu}\right)_{\tilde{x}}\left(\frac{\partial }{\partial x^\nu}\right)_{\tilde{x}} = \left(-\frac{\partial \xi^{\nu}(\tilde{x})}{\partial \tilde{x}^\mu}\right)_{\tilde{x}}\left(\frac{\partial }{\partial x^\nu}\right)_{\tilde{x}} + \left(\frac{\partial }{\partial x^\mu}\right)_{\tilde{x}} \equiv -\partial_\mu \xi^{\nu}(\tilde{x}) \partial_\nu + \partial_\mu. 
  \end{align*}
  where the subscripts indicate that a partial derivative with respect to $x^\mu$ should be precomposed with $x = f^{-1}(x^{\mu})$ to yield a function dependent on the left-hand side variable $\tilde{x}$. Now we can calculate the transformed field strength at first order in $\xi$ to be 
  \begin{align*}
    \tilde{F}_{\mu\nu} &= \tilde{\partial}_\mu \tilde{A}_{\nu} - (\mu \leftrightarrow \nu) \\
    &= \left(-\partial_\mu \xi^{\rho}(\tilde{x}) \partial_\rho + \partial_\mu\right)\left(A_\nu(\tilde{x})-A_\nu(\tilde{x}) \frac{\Delta}{D}\partial_\sigma \xi^\sigma(\tilde{x}) - A_\lambda(\tilde{x})M_{\nu}{}^{\lambda} - \xi^{\lambda}(\tilde{x}) \partial_\lambda A_\nu(\tilde{x})\right)- (\mu \leftrightarrow \nu)\\
    &= \partial_\mu A_\nu(\tilde{x}) - (\partial_\mu \xi^{\lambda}(\tilde{x})) \partial_\lambda A_\nu(\tilde{x}) -\partial_\mu \left(A_\nu(\tilde{x}) \frac{\Delta}{D}\partial_\lambda \xi^\lambda(\tilde{x})\right) - \partial_\mu\left(A_\lambda(\tilde{x})M_{\nu}{}^{\lambda}\right) - \partial_\mu\left(\xi^{\lambda}(\tilde{x}) \partial_\lambda A_\nu(\tilde{x})\right) - (\mu \leftrightarrow \nu)\\
    &= \partial_\mu A_\nu(\tilde{x}) -\partial_\mu \left(A_\nu(\tilde{x}) \frac{\Delta}{D}\partial_\lambda \xi^\lambda(\tilde{x})\right) - \partial_\mu A_\lambda(\tilde{x})\partial_\nu \xi^\lambda(\tilde{x}) - A_\lambda(\tilde{x})\partial_\mu\partial_\nu \xi^\lambda(\tilde{x}) - \xi^{\lambda}(\tilde{x}) \partial_\lambda \partial_\mu A_\nu(\tilde{x}) - 2(\partial_\mu \xi^{\lambda}(\tilde{x})) \partial_\lambda A_\nu(\tilde{x}) - (\mu \leftrightarrow \nu)\\
    &= \partial_\mu A_\nu(\tilde{x}) -\partial_\mu \left(A_\nu(\tilde{x}) \frac{\Delta}{D}\partial_\lambda \xi^\lambda(\tilde{x})\right) - (\partial_\mu A_\lambda(\tilde{x}))M_{\nu}{}^{\lambda} - A_\lambda(\tilde{x})\partial_\mu M_{\nu}{}^{\lambda} - \xi^{\lambda}(\tilde{x}) \partial_\lambda \partial_\mu A_\nu(\tilde{x}) - 2(\partial_\mu \xi^{\lambda}(\tilde{x})) \partial_\lambda A_\nu(\tilde{x}) - (\mu \leftrightarrow \nu)\\
    &= F_{\mu \nu}(\tilde{x}) - F_{\mu \nu}(\tilde{x}) \frac{\Delta}{D}\partial_\lambda \xi^\lambda(\tilde{x}) - A_{(\nu}(\tilde{x}) \frac{\Delta}{D}\partial_{\mu)} \partial_\lambda \xi^\lambda(\tilde{x})  - (\partial_{(\mu} A_\lambda(\tilde{x}))M_{\nu)}{}^{\lambda} - A_\lambda(\tilde{x})\partial_{(\mu} M_{\nu)}{}^{\lambda} - \xi^{\lambda}(\tilde{x}) \partial_\lambda F_{\mu\nu}(\tilde{x}) - 2(\partial_{(\mu} \xi^{\lambda}(\tilde{x})) \partial_\lambda A_{\nu)}(\tilde{x})\\
    &= F_{\mu \nu}(\tilde{x}) - F_{\mu \nu}(\tilde{x}) \frac{\Delta}{D}\partial_\lambda \xi^\lambda(\tilde{x}) - A_{(\nu}(\tilde{x}) \frac{\Delta}{D}\partial_{\mu)} \partial_\lambda \xi^\lambda(\tilde{x})  - (\partial_{(\mu} A_\lambda(\tilde{x}))M_{\nu)}{}^{\lambda} - \xi^{\lambda}(\tilde{x}) \partial_\lambda F_{\mu\nu}(\tilde{x}) - 2(\partial_{(\mu} \xi^{\lambda}(\tilde{x})) \partial_\lambda A_{\nu)}(\tilde{x})
  \end{align*}
  where we simplified further by expliciting 
  \begin{align*}
    & 2\partial_{(\mu} M_{\nu)}{}^{\lambda} = \partial_\mu\partial_\nu \xi^\lambda(\tilde{x}) - \partial_\mu\partial^\lambda \xi_\nu(\tilde{x}) - \partial_\nu \partial_\mu \xi^\lambda(\tilde{x}) - \partial_\nu \partial^\lambda \xi_\mu(\tilde{x}) = 0.
  \end{align*}
  We note that the transformation law of $F_{\mu\nu}$ involves $A_{\mu}$ homogeneously which is an example of mixing of CFT fields under the transformation of a descendant. 
  \item[(b)]   For a $D$-dimensionnal spacetime, the Maxwell action reads 
  \begin{align*}
    S = \int \text{d}^D x \sqrt{|g|} \frac{1}{4} F_{\mu\nu} F^{\mu \nu} = \int \text{d}^D x \sqrt{|g|} \ g^{\mu \sigma} g^{\nu \rho}\frac{1}{4} F_{\mu\nu} F_{\sigma \rho}.  
  \end{align*}
  where $g$ is the metric (which we suppose conformally flat). We aim to apply the results found in (a) to determine when this action gains conformal symmetry. Under a conformal transformation given by the killing vector $\xi^{\mu}(x)$ and the scaling $\Omega(x) = 1 + \partial_\mu \xi^\mu(x)/D + O(\xi^2)$ of the metric components, we have 
  \begin{align*}
    &g_{\nu \rho}(x) = \Omega(f(x))^{-2} \tilde{g}_{\nu \rho}(f(x)) = \Omega(\tilde{x})^{-2} \tilde{g}_{\nu \rho}(\tilde{x}) \quad \text{Defining property of a conformal transformation}\\
    &|g|(x) = \Omega(f(x))^{-2D} |\tilde{g}|(f(x)), \quad g^{\nu \rho}(x) = \Omega(f(x))^{+2} \tilde{g}^{\nu \rho}(f(x)) = \Omega(\tilde{x})^{2} \tilde{g}^{\nu \rho}(\tilde{x}), \quad  \text{d}^D x \sqrt{|g|} = \text{d}^D \tilde{x}\ \Omega(\tilde{x})^{-D} \sqrt{|\tilde{g}|(\tilde{x})}
  \end{align*}
  Without loss of generality, we take the target metric $\tilde{g}$ to be the Minkowski metric.
  Inverting the result found in (a) for the transformation of the gauge field, we write 
  \begin{align*}
    A_{\mu}(x) =  |\partial x/\partial \tilde{x}|_{\tilde{x}}^{-\Delta/D} (R^{-1})_{\mu}^{\nu} \tilde{A}_{\nu}(\tilde{x}) &= \tilde{A}_\mu(\tilde{x})+\tilde{A}_\mu(\tilde{x}) \frac{\Delta}{D}\partial_\sigma \xi^\sigma(\tilde{x})-\tilde{A}_\mu(\tilde{x}) \partial_\sigma \xi^\sigma(\tilde{x}) \frac{1}{D}  + \tilde{A}_\nu(\tilde{x}) \partial_\mu \xi^\nu(\tilde{x}) + O(\xi^2) \\
    &= \tilde{A}_\mu(\tilde{x})+\tilde{A}_\mu(\tilde{x}) \frac{\Delta}{D}\partial_\sigma \xi^\sigma(\tilde{x}) + \frac{1}{2}\tilde{A}_\nu(\tilde{x}) \left(\partial_\mu \xi^\nu(\tilde{x}) - \partial^\nu \xi_\mu(\tilde{x})\right) + O(\xi^2). 
  \end{align*}
  Then, with the derivative $(\partial_{\mu})_{\tilde{x}} = \tilde{\partial}_\mu \xi^{\nu}(\tilde{x}) \tilde{\partial}_\nu + \tilde{\partial}_\mu$, the field strength transforms as 
  \begin{align*}
    F_{\mu \nu} = \partial_\mu A_{\nu}(x) - (\mu \leftrightarrow \nu) &= \left( \tilde{\partial}_\mu \xi^{\lambda}(\tilde{x}) \tilde{\partial}_\lambda + \tilde{\partial}_\mu\right)\left(\tilde{A}_\nu(\tilde{x})+\tilde{A}_\nu(\tilde{x}) \frac{\Delta}{D}\tilde{\partial}_\sigma \xi^\sigma(\tilde{x}) + \tilde{A}_\lambda(\tilde{x}) M_\nu{}^{\lambda}\right) - (\mu \leftrightarrow \nu)\\
    &=  \tilde{\partial}_\mu\left(\tilde{A}_\nu(\tilde{x})+\tilde{A}_\nu(\tilde{x}) \frac{\Delta}{D}\tilde{\partial}_\sigma \xi^\sigma(\tilde{x}) + \tilde{A}_\lambda(\tilde{x}) M_\nu{}^{\lambda}\right) + \tilde{\partial}_\mu \xi^{\lambda}(\tilde{x}) \tilde{\partial}_\lambda \tilde{A}_\nu(\tilde{x}) - (\mu \leftrightarrow \nu)\\
    &= \tilde{F}_{\mu\nu}(\tilde{x})+\tilde{F}_{\mu\nu}(\tilde{x}) \frac{\Delta}{D}\partial_\sigma \xi^\sigma(\tilde{x}) +   \tilde{A}_{(\nu}(\tilde{x}) \frac{\Delta}{D}\tilde{\partial}_{\mu)} \tilde{\partial}_\sigma \xi^\sigma(\tilde{x})   + \tilde{\partial}_{(\mu}(\tilde{A}_{\lambda}(\tilde{x})) M_{\nu)}{}^{\lambda} + \tilde{\partial}_{(\mu} \xi^{\lambda}(\tilde{x}) \tilde{\partial}_\lambda \tilde{A}_{\nu)}(\tilde{x})\\
  \end{align*}
  The contravariant equivalent of this result is given by 
  \begin{align*}
    F^{\mu \nu} = g^{\mu \sigma} g^{\nu \rho} F_{\sigma \rho} &= \Omega(\tilde{x})^{4}  \tilde{g}^{\mu \sigma} \tilde{g}^{\nu \rho} F_{\sigma \rho} \\
    &=\Omega(\tilde{x})^{4}  \left( \tilde{F}^{\mu\nu}(\tilde{x})+\tilde{F}^{\mu\nu}(\tilde{x}) \frac{\Delta}{D}\tilde{\partial}_\sigma \xi^\sigma(\tilde{x}) +  \tilde{g}^{\mu \sigma} \tilde{g}^{\nu \rho}    \tilde{A}_{(\nu}(\tilde{x}) \frac{\Delta}{D}\tilde{\partial}_{\mu)} \tilde{\partial}_\sigma \xi^\sigma(\tilde{x})   + \tilde{g}^{\mu \sigma} \tilde{g}^{\nu \rho} \tilde{\partial}_{(\sigma}(\tilde{A}_{\lambda}(\tilde{x})) M_{\rho)}{}^{\lambda}+ \tilde{g}^{\mu \sigma} \tilde{g}^{\nu \rho} \tilde{\partial}_{(\sigma} \xi^{\lambda}(\tilde{x}) \tilde{\partial}_\lambda \tilde{A}_{\rho)}(\tilde{x})  \right)
  \end{align*}
  
  Next, we calculate 
  \begin{align*}
    F_{\mu\nu} F^{\mu\nu} &= \left(\tilde{F}_{\mu\nu}(\tilde{x})+\tilde{F}_{\mu\nu}(\tilde{x}) \frac{\Delta}{D}\tilde{\partial}_\sigma \xi^\sigma(\tilde{x}) + \tilde{\partial}_{(\mu}(\tilde{A}_{\lambda}(\tilde{x})) M_{\nu)}{}^{\lambda} + \tilde{\partial}_{(\mu} \xi^{\lambda}(\tilde{x}) \tilde{\partial}_\lambda \tilde{A}_{\nu)}(\tilde{x}) + O(\xi^2)\right)\\
    &\times \Omega(\tilde{x})^{4}  \left( \tilde{F}^{\mu\nu}(\tilde{x})+\tilde{F}^{\mu\nu}(\tilde{x}) \frac{\Delta}{D}\tilde{\partial}_\sigma \xi^\sigma(\tilde{x}) +   \tilde{A}_{(\nu}(\tilde{x}) \frac{\Delta}{D}\tilde{\partial}_{\mu)} \tilde{\partial}_\sigma \xi^\sigma(\tilde{x}) + \tilde{g}^{\mu \sigma} \tilde{g}^{\nu \rho} \tilde{\partial}_{(\sigma}(\tilde{A}_{\lambda}(\tilde{x})) M_{\rho)}{}^{\lambda}+ \tilde{g}^{\mu \sigma} \tilde{g}^{\nu \rho} \tilde{\partial}_{(\sigma} \xi^{\lambda}(\tilde{x}) \tilde{\partial}_\lambda \tilde{A}_{\rho)}(\tilde{x})  \right)\\
    &= \Omega(\tilde{x})^4\left(\tilde{F}_{\mu\nu}(\tilde{x}) \tilde{F}^{\mu \nu}(\tilde{x}) + \tilde{F}_{\mu\nu}(\tilde{x})\tilde{F}^{\mu \nu}(\tilde{x}) \frac{2\Delta}{D} \tilde{\partial}_\sigma \xi^\sigma(\tilde{x}) + 2 \tilde{F}^{\mu\nu}\tilde{\partial}_{(\mu}(\tilde{A}_{\lambda}(\tilde{x}) ) M_{\nu)}{}^{\lambda}  + 2 \tilde{F}^{\mu\nu} \tilde{\partial}_{(\mu} \xi^{\lambda}(\tilde{x}) \tilde{\partial}_\lambda \tilde{A}_{\nu)}(\tilde{x}) +  2\tilde{F}^{\mu\nu}   \tilde{A}_{(\nu}(\tilde{x}) \frac{\Delta}{D}\tilde{\partial}_{\mu)} \tilde{\partial}_\sigma \xi^\sigma(\tilde{x})\right)\\
    &=  \Omega(\tilde{x})^4\left(\tilde{F}_{\mu\nu}(\tilde{x}) \tilde{F}^{\mu \nu}(\tilde{x}) + \tilde{F}_{\mu\nu}(\tilde{x})\tilde{F}^{\mu \nu}(\tilde{x}) \frac{2\Delta}{D} \tilde{\partial}_\sigma \xi^\sigma(\tilde{x}) + 2 \tilde{F}^{\mu\nu}\tilde{\partial}_{(\mu}(\tilde{A}_{\lambda}(\tilde{x}) ) M_{\nu)}{}^{\lambda}  + 2 \tilde{F}^{\mu\nu} \tilde{\partial}_{(\mu} \xi^{\lambda}(\tilde{x}) \tilde{\partial}_\lambda \tilde{A}_{\nu)}(\tilde{x}) +  4\tilde{F}^{\mu\nu} \tilde{A}_\nu(\tilde{x}) \frac{\Delta}{D}\tilde{\partial}_\mu \tilde{\partial}_\sigma \xi^\sigma(\tilde{x})\right)
  \end{align*}
  \textcolor{blue}{I realized this calculation only applies a passive transformation to the action and should not not change its value without necessarly corresponding to a symmetry. I would have to redo this calculation by applying an active conformal transformation to each element of the action.}
  
\end{enumerate}


\newpage
\section{Axial anomaly}

\begin{enumerate}
  \item[(a)] We consider a $D=2$-dimensionnal fermion field $\psi$ with vector current $j_{\mu}^V = \bar{\psi} \gamma_{\mu} \psi$ where $\gamma_\mu$ are matrices forming a 2-dimensionnal clifford algebra. We are interested in the 2-point correlator of the vector current $\langle j_{\mu}^{V}(x_1) j_{\nu}^{V}(x_2) \rangle$. By translational symmetry, the 2-point function is forced to be a function of the relative coordinates $x = (x_1-x_2)/2$. Translating by $-X_{12} = -(x_1 + x_2)/2$, we can bring the midpoint of the $x_1, x_2$ segment to the origin without changing the value of the 2-point function. Explicitly, we have $\langle j_{\mu}^{V}(x_1) j_{\nu}^{V}(x_2) \rangle = \langle j_{\mu}^{V}(x) j_{\nu}^{V}(-x) \rangle$. This property allows us to expand the 2-point function with a Fourier transform with respect to $x$ as 
  \begin{align*}
    F[\langle j_{\mu}^{V}(x_1) j_{\nu}^{V}(x_2) \rangle](q)  = \frac{1}{(2\pi)^2}\int \text{d}^2x e^{-i q \cdot  x} \langle j_{\mu}^{V}(x) j_{\nu}^{V}(-x) \rangle &= \frac{1}{(2\pi)^2}\int \text{d}^2x e^{-i q \cdot  x} \langle  \int \text{d}^2 k\  e^{+i k \cdot x} j_{\mu}^{V}(k) \int \text{d}^2 p\  e^{-i p \cdot x} j_{\nu}^{V}(p) \rangle\\
    &=  \frac{1}{2\pi} \langle  \int \text{d}^2 k\ \text{d}^2 p\ \delta(-q + k - p)  j_{\mu}^{V}(k) \  j_{\nu}^{V}(p) \rangle\\
    &= \frac{1}{2\pi}\int \ \text{d}^2 p \ \langle \ j_{\mu}^{V}(q-p) \ j_{\nu}^{V}(p) \rangle
  \end{align*}
  where the fourier decomposition $j_{\rho}^{V}(x_i) = \frac{1}{2\pi} \int \text{d}^2 p\ e^{i p \cdot x_i} j_{\rho}^{V}(p)$ of the vector current was used. In what follows, we focus on the Fourier space 2-point functions $\langle \ j_{\mu}^{V}(-p) \ j_{\nu}^{V}(p) \rangle$ contribution to the $q = 0$ Fourier component of the spacetime 2-point function. Lorentz invariance requires that $\langle \ j_{\mu}^{V}(q-p) \ j_{\nu}^{V}(p) \rangle$ is a sum of tensors (it can be extracted from a Fourier transform linearly combining tensor so it is a tensor). Furthermore, it only depends on components $p_\mu$ of $p$. The only tensors with two indices built can be constructed by combining the Minkowski metric $\eta_{\mu\nu}$, the components  $p_\mu$, the norm $p^2$ and the matrices $\gamma^\mu$ (we only need to include a term $\gamma_\mu \gamma_nu$ since the anticommutator $\{\gamma_\mu, \gamma_\nu\} = 2 \eta_{\mu \nu}$ relates it to $\gamma_\nu \gamma_\mu$). The most general form for the Fourier space 2-point function consistent with Lorentz invariance reads 
  \begin{align*}
    \langle \ j_{\mu}^{V}(-p) \ j_{\nu}^{V}(p) \rangle = F_1(p^2)\epsilon_{\mu\nu} + F_2(p^2)\eta_{\mu\nu} + F_3(p^2) p_\mu p_\nu + F_{4}(p^2) \gamma_\mu \gamma_\nu + F_{5}(p^2)\gamma_\mu p_\nu + F_{6}(p^2) \gamma_\nu p_\mu
  \end{align*}
  where the functions $F_{i} : \mathbb{R} \to \mathbb{C}$ provide full generality and $\epsilon_{\mu\nu}$ is the 2-dimensionnal Levi-Civita tensor. Since the current operator follows a Bose statistic (they each contain an even number of fermion operators), we can exchange them without changing the value of the 2-point function. This property can be expressed as
  \begin{align*}
    \langle \ j_{\mu}^{V}(-p) \ j_{\nu}^{V}(p) \rangle = \langle \ j_{\nu}^{V}(p) \ j_{\mu}^{V}(-p) \rangle = -F_1(p^2)\epsilon_{\mu\nu} + F_2(p^2)\eta_{\mu\nu} + F_3(p^2) (-p_\nu) (-p_\mu) + F_{4}(p^2) \gamma_\nu \gamma_\mu - F_{5}(p^2)\gamma_\nu p_\mu - F_{6}(p^2) \gamma_\mu p_\nu.
  \end{align*}
  Subtracting this exchanged expression from the initial expression, we get the constraint  
  \begin{align*}
    &0 = 2 F_1(p^2)\epsilon_{\mu\nu} + F_{4}(p^2) (\gamma_\mu \gamma_\nu - \gamma_\nu \gamma_\mu) + (F_{6}(p^2) + F_{5}(p^2)) \gamma_\nu p_\mu + (F_{6}(p^2) + F_{5}(p^2)) \gamma_\mu p_\nu, \forall p\\
    &\implies F_{4}(p^2) = F_1(p^2) = 0,\quad F_{6}(p^2) = - F_{5}(p^2)
  \end{align*}

  The current we are interested in is conserved as a result of the global symmetry $\psi \to e^{i\theta}\psi$, $\theta \in \mathbb{R}$. We note that applying the infinitesimal version of this symmetry transformation to the current leads to a vanishing variation $\delta j^\nu (x) = 0$. The ward identity corresponding to this symmetry reads 
  \begin{align*}
    \langle \partial^\mu j_\mu (x_1) j_\nu (x_2) \rangle = \delta(x_1 - x_2)\langle \delta j_\nu(x_2)\rangle = 0. 
  \end{align*} 
  We then compute the Fourier transformation with respect to $x_1, x_2$ to get 
  \begin{align*}
    0 = F[\langle  \partial^\mu  j_{\mu}^{V}(x_1) j_{\nu}^{V}(x_2) \rangle](q, p) 
    &= \int \text{d}^2x_1 e^{-i q \cdot  x_1} \int \text{d}^2x_2 e^{-i p \cdot  x_2} \langle \partial^{\mu} j_{\mu}^{V}(x_1) j_{\nu}^{V}(x_2) \rangle\\
    &=  \langle (+iq^{\mu})  \int \text{d}^2x_1 e^{-i q \cdot  x_1} j_{\mu}^{V}(x_1)   \int \text{d}^2x_2 e^{-i p \cdot  x_2}  j_{\nu}^{V}(x_2) \rangle\quad \text{with integration by parts}\\
    &= iq^{\mu} \langle  j_{\mu}^{V}(q)   j_{\nu}^{V}(p) \rangle
  \end{align*}
  At $q = -p$, we find $p^\mu \langle j_{\mu}^{V}(-p) \ j_{\nu}^{V}(p) \rangle = 0$ wich implies 
  \begin{align*}
    0=p^\mu \langle \ j_{\mu}^{V}(-p) \ j_{\nu}^{V}(p) \rangle &= p^\mu  \left(F_2(p^2)\eta_{\mu\nu} + F_3(p^2) p_\mu p_\nu\right) + F_{5}(p^2)p^\mu\left(\gamma_\mu p_\nu - \gamma_\nu p_\mu\right) \\
    &= \left(F_2(p^2) + F_3(p^2) p^2 \right)p_\nu + F_{5}(p^2) \left(p^\mu \gamma_\mu p_\nu - \gamma_\nu p^2\right), \forall p\\
    &\implies F_2(p^2) = -F_3(p^2) p^2, \quad F_5(p^2) = 0.
  \end{align*}
  More explicitly, starting with the constraint $F_5(p^2)(p^\mu \gamma_\mu p_\nu - \gamma_\nu p^2) = 0 $, we can have either $F_5(p^2) = 0$ or $p^\mu \gamma_\mu p_\nu - \gamma_\nu p^2$. The latter case is associated with $0 = \gamma_0 + \gamma_1$ for $p^\mu = (1, 1)$ showing it can't hold for all $p$ and forcing $F_5(p^2)$. 
  The updated expression for the 2-momentum curent correlator is
  \begin{align*}
    \langle j_{\mu}^{V}(-p) \ j_{\nu}^{V}(p) \rangle =  F_3(p^2)\left(-p^2\eta_{\mu\nu} + p_\mu p_\nu\right). 
  \end{align*}
  \newpage
  The form of $F_3(p^2)$ can be made more precise by imposing scale invariance. To use scale invariance, we first relate scale invariance in position space to scale invariance in momentum space. Since the conserved current operators have no anomalous scaling dimension (leaving the scaling dimension $\Delta = D = 2$), we have 
  \begin{align*}
    \langle j_\mu (x_1) j_\nu (x_2 ) \rangle = \lambda^{2 + 2}\langle j_\mu (x_1 \lambda) j_\nu (x_2 \lambda) \rangle \implies \langle j_{\mu}^{V}(-p) \ j_{\nu}^{V}(p) \rangle &=  \lambda^{2 + 2} F[\langle  \partial^\mu  j_{\mu}^{V}(\lambda x_1) j_{\nu}^{V}(\lambda x_2) \rangle](-p, p)\\ &= \lambda^4 \langle \int \text{d}^2x_1 e^{-i q \cdot  x_1} j_{\mu}^{V}(\lambda x_1)   \int \text{d}^2x_2 e^{-i p \cdot  x_2}  j_{\nu}^{V}(\lambda x_2) \rangle\\
    &=  \lambda^4 \langle \frac{1}{\lambda^2}\int \text{d}^2x_1 e^{-i q \cdot  x_1/\lambda} j_{\mu}^{V}(x_1) \frac{1}{\lambda^2}\int \text{d}^2x_2 e^{-i p \cdot  x_2/\lambda}  j_{\nu}^{V}(x_2) \rangle\\
    &= \langle j_{\mu}^{V}(-p/\lambda) \ j_{\nu}^{V}(p/\lambda) \rangle.
  \end{align*} 
  Applying this constraint to the 2-momentum curent correlator yields
  \begin{align*}
    F_3(p^2)\left(-p^2\eta_{\mu\nu} + p_\mu p_\nu\right) = \frac{1}{\lambda^4} F_3(p^2/\lambda^2)\left(-p^2\eta_{\mu\nu} + p_\mu p_\nu\right) \implies F_3(p^2) = \frac{1}{\lambda^2}F_3(p^2/\lambda^2). 
  \end{align*}
  The previous implication suggests the powerlaw ansatz $F_3 = a p^b$ depending on constants $a, b$. Substituting this ansatz, we get $a p^b = a p^b \lambda^{-b-2} \implies b=-2$. The final form for the correlator is 
  \begin{align*}
    \langle j_{\mu}^{V}(-p) \ j_{\nu}^{V}(p) \rangle = \frac{a}{p^2}\left(-p^2\eta_{\mu\nu} + p_\mu p_\nu\right).
  \end{align*} 
  \item[(b)] We consider the axial current defined by $j^A_\mu = \epsilon_{\mu\nu} j^{V, \nu}$. In $D=2$, the non zero components of the levi-civita tensor are $\epsilon_{01} = -\epsilon_{10} = 1$. To determine if the axial current is classicaly conserved, we calculate its divergence as follows 
  \begin{align*}
    \partial^\mu j^A_\mu &= \partial^\mu \epsilon_{\mu\nu} j^{V, \nu} = \partial^\mu (\bar{\psi} \epsilon_{\mu\nu} \gamma^{\nu} \psi) = \partial^\mu (\bar{\psi} \eta_{\mu \sigma} \gamma^1 \gamma^0 \gamma^{\sigma} \psi) = \partial_\mu (\bar{\psi} \gamma^1 \gamma^0 \gamma^{\mu} \psi)\\
    &= (\partial_\mu \bar{\psi}) \gamma^1 \gamma^0 \gamma^{\mu} \psi +  \bar{\psi} \gamma^1 \gamma^0 \gamma^{\mu} (\partial_\mu \psi) = 0
  \end{align*}
  where we used the property $\epsilon_{\mu\nu} \gamma^{\nu} =\eta_{\mu \sigma} \gamma^1 \gamma^0 \gamma^{\sigma}$, the classical equation of motion for a free massless fermion field $0 = \gamma^1 \gamma^0 \gamma^{\mu} \partial_\mu \psi$ and its conjugate $0 =  \partial_\mu \psi^\dagger \gamma^0 \gamma^1 \gamma^0 \gamma^{\mu} \gamma^0 = \partial_\mu \bar{\psi} \gamma^1 \gamma^0 \gamma^{\mu} \gamma^0$. We explicitly check 
  \begin{align*}
    \gamma^{1} &= \epsilon_{01} \gamma^{1} =  \eta_{00} \gamma^1 \gamma^0 \gamma^{0} = (\eta_{00}\eta^{00}) \gamma^1 = \gamma^1,\\
    -\gamma^{0} &= \epsilon_{10} \gamma^{0} = \eta_{11} \gamma^1 \gamma^0 \gamma^{1} = -(\eta_{11} \eta^{11})\gamma^0 = -\gamma^0.
  \end{align*}
  We conclude that the classical equations of motion imply both the classical conservation of the vector current and the classical conservation of the axial current. 
  \item[(c)] To test if the classical conservation of the axial current survives quantum effects, we calculate 
  \begin{align*}
    p^\mu \langle j^A_\mu (-p) j^V_\nu (p) \rangle = \eta^{\sigma \rho}\epsilon_{\mu\sigma}p^\mu \langle j^V_\rho (-p) j^V_\nu (p) \rangle &= \eta^{\sigma \rho}\epsilon_{\mu\sigma}p^\mu \frac{a}{p^2}\left(-p^2\eta_{\rho\nu} + p_\rho p_\nu\right)\\
    &= \epsilon_{\mu\sigma} \frac{a}{p^2}\left(-p^2 p^\mu \delta_{\nu}^{\sigma} + p^\mu p^\sigma p_\nu \right)\\
    &= \frac{a}{p^2}(-p^2  \epsilon_{\mu\nu} p^\mu + \underbrace{\epsilon_{\mu\sigma} p^\mu p^\sigma}_{p^0 p^1 - p^1 p^0 = 0} p_\nu ) = -a\epsilon_{\mu\nu} p^\mu
  \end{align*}
  and Fourier transfrom the result to get 
  \begin{align*}
    \frac{1}{2\pi}\int \text{d}p\ e^{i p \cdot x}  p^\mu \langle j^A_\mu (-p) j^V_\nu (p) \rangle &= \frac{-i}{2\pi} \partial^\mu \langle \int \text{d}p \ e^{i p \cdot x} \int \text{d}x_1 \ e^{+ i p \cdot x_1} j_{\mu}^{A}(x_1)  \int \text{d}x_2\ e^{- i p \cdot x_2} j_{\nu}^{V}(x_2) \rangle \\
    &= -i \partial^\mu \langle \int \text{d}x_1 \text{d}x_2 \  j_{\mu}^{A}(x_1) j_{\nu}^{V}(x_2) \delta(x + x_1 - x_2)\rangle \\
    &= -i  \langle \int \text{d}x_1 \  \partial^\mu j_{\mu}^{A}(x_1) j_{\nu}^{V}(x + x_1)\rangle = -i   \langle  \  \partial^\mu j_{\mu}^{A}(0) j_{\nu}^{V}(x)\rangle \int \text{d}x_1
  \end{align*}
  and 
  \begin{align*}
    - \frac{1}{2\pi}\int \text{d}p\ e^{i p \cdot x} a\epsilon_{\mu\nu} p^\mu = a\ i\ \epsilon_{\mu\nu} \partial^\mu \delta(x)
  \end{align*}
  which shows that the divergence of the axial current has a non-vanishing correlator (can't be explained by contact terms since they would vanish through $\delta j_\nu(x) = 0$) with the vector current contradicting the Ward identity for the axial current. 
  
  \item[(d)] When a background gauge field $A^{\mu}$ is added, the axial anomaly affects the expectation value of the axial current divergence dirrectly. This expectation value is evaluated at first non-trivial order 
  \begin{align*}
    \left\langle\partial_\mu j^{A, \mu}(0)\right\rangle_{A^\mu} \equiv\left\langle\partial_\mu j^{A, \mu}(0) e^{\int d x A^\nu j_\nu^V}\right\rangle &\approx \left\langle\partial_\mu j^{A, \mu}(0) 1\right\rangle  +   \left\langle\partial_\mu j^{A, \mu}(0) \int d x A^\nu j_\nu^V\right\rangle \\
    &\approx \left\langle\partial_\mu j^{A, \mu}(0) \int d x A^\nu j_\nu^V\right\rangle 
  \end{align*}
\end{enumerate}

\section{OPE coefficients from three-point functions}

\begin{enumerate}
  \item[(a)] Given scalar CFT operators $O_{\Delta_1}$ and $O_{\Delta_2}$ with respective scaling dimensions $\Delta_1$ and $\Delta_2$, there is only one two-point function consistent with conformal symmetry given by 
  \begin{align*}
    \left\langle\mathcal{O}_{\Delta_1}(x_1) \mathcal{O}_{\Delta_2}(x_2)\right\rangle=\frac{\delta_{\Delta_1, \Delta_2}}{|x_{12}|^{2 \Delta_1}}
  \end{align*}
  where $x_{ij} = x_i - x_j$. Adding a third operator $O_{\Delta_3}$ with scaling dimension $\Delta_3$, the three-point function has residual freedom in the coefficients $C_{123}$ depending on the operators involved. The most general expression reads 
  \begin{align*}
    \left\langle\mathcal{O}_{\Delta_1}\left(x_1\right) \mathcal{O}_{\Delta_2}\left(x_2\right) \mathcal{O}_{\Delta_3}\left(x_3\right)\right\rangle=\frac{C_{123}}{|x_{12}|^{\Delta_1+\Delta_2-\Delta_3} |x_{23}|^{\Delta_2+\Delta_3-\Delta_1} |x_{31}|^{\Delta_3+\Delta_1-\Delta_2}}.
  \end{align*}
  Assumming\footnote{This assumption can be seen to hold with the state-operator correspondance of the radial quantization of the CFT. This state operator correspondance associates each CFT operator (primary and descendants) at a point $x$ to an eigensate of the dilation operator. Using these states, we can write a resolution of identity allowing to decompose any other state as a sum of states corresponding to primary and descendant CFT operators. This decomposition can then be brought back to operators using the state operator correspondance in the reverse direction.} that the operator algebra of the CFT is closed (multiplying operators produces an algebra operator which is in the same vector space) and has the primary and their descendants as a vector basis, we have that any product of CFT operators can be decomposed as a sum of primary and descendants operators. Explicitly we have the general operator product expansion
  \begin{align*}
    \mathcal{O}_{\Delta_1}\left(x_1\right) \mathcal{O}_{\Delta_2}\left(x_2\right) = \sum_{\Delta'} \sum_l \alpha^{\Delta', l}_{\Delta_1, \Delta_2} (x_{1}, x_{2}) \partial_{l} O_{\Delta'}(x_2)
  \end{align*}
  where $l$ is a string representing the sequence of components of $x_2$\footnote{This point is the point at which we take our asymptotic radial quandized states for usage of the state operator correspondance} with respect to which we differentciate to reach all descendants, $\alpha$ represents linear combinations of CFT operators with full generality. We sum over $\Delta'$ to include all conformal towers and we sum over $l$ to reach all the members of a given tower. We can make the depedance on $x_1$ and $x_2$ more precise by imposing the transformation properties of each operator. We first use the translation symmetry. Having fixed the expansion point $x_2$ (it will not be affected by a translation of the operators involved in the product), the uniqueness of the expansion imposes that we find the same coefficients if we translate $x_1$ and $x_2$ by the same amount $a$ (this gives us back the same operator because it consists in a \textit{representation} of the translation operation on the operator product). We conclude that translationnal symmetry forces $\alpha^{\Delta', l}_{\Delta_1, \Delta_2} (x_{1}, x_{2}) = \alpha^{\Delta', l}_{\Delta_1, \Delta_2} (x_{1} + a, x_{2} + a) =  \alpha^{\Delta', l}_{\Delta_1, \Delta_2} (x_{12})$. Then, for a scale transformation with scale factor $\lambda$, we have 
  \begin{align*}
    \mathcal{O}_{\Delta_1}\left(x_1\right)\mathcal{O}_{\Delta_2}\left(x_2\right) = \lambda^{\Delta_1 + \Delta_2} \mathcal{O}_{\Delta_1}\left(\lambda x_1\right)\mathcal{O}_{\Delta_2}\left(\lambda x_2\right) &\implies \sum_{\Delta'} \alpha^{\Delta', l}_{\Delta_1, \Delta_2} (x_{12}) \partial_{l} O_{\Delta'}(x_2) = \sum_{\Delta'} \sum_l \lambda^{\Delta_1 + \Delta_2} \alpha^{\Delta', l}_{\Delta_1, \Delta_2} (\lambda x_{12}) \partial_{l} O_{\Delta'}(\lambda x_2)\\
    &\implies \sum_{\Delta'} \alpha^{\Delta', l}_{\Delta_1, \Delta_2}(x_{12})\partial_{l} O_{\Delta'}(x_2) = \sum_{\Delta'} \sum_l \lambda^{\Delta_1 + \Delta_2-\Delta'} \alpha^{\Delta', l}_{\Delta_1, \Delta_2} (\lambda x_{12}) \lambda^{-|l|} \partial_{l} O_{\Delta'}(x_2)\\
    &\implies \alpha^{\Delta', l}_{\Delta_1, \Delta_2} (x_{12}) = \lambda^{\Delta_1 + \Delta_2-\Delta' - |l|} \alpha^{\Delta', l}_{\Delta_1, \Delta_2} (\lambda x_{12})
  \end{align*} 
  where we used the uniqueness of the expansion provided a point $x_2$ and denoted $|l|$ the number of elements in $l$. To apply the scale transfromation to the descendant operators we used 
  \begin{align*}
    \left.\frac{\partial }{\partial x^{l}} O_{\Delta'}(x)\right|_{x = \lambda x_2} = \left.\frac{\partial x/\lambda}{\partial x^{l}}  \frac{\partial }{\partial y^{l}} O_{\Delta'}\left(\lambda y\right)\right|_{y = x/\lambda, y = x_2} = \left.\lambda^{|l|} \lambda^{-\Delta'} \frac{\partial }{\partial y^{l}} O_{\Delta'}\left(y\right)\right|_{y = x_2} = \lambda^{-|l|} \lambda^{-\Delta'} \frac{\partial }{\partial y^{l}} \partial_{l} O_{\Delta'}(x_2). 
  \end{align*}

  \newpage

  We can also extract constraints using the rotational invariance of the product. Supposing the operators in the product are scalar primaries, we can write $\mathcal{O}_{\Delta_1}\left(x_1\right)\mathcal{O}_{\Delta_2}\left(x_2\right) = \mathcal{O}_{\Delta_1}\left(R(x_1)\right)\mathcal{O}_{\Delta_2}\left(R(x_2)\right)$ where $R$ is a rotation in $\mathbb{R}^D$. Substituting this relation in the expansion yields 
  \begin{align*}
    \mathcal{O}_{\Delta_1}\left(x_1\right)\mathcal{O}_{\Delta_2}\left(x_2\right) = \mathcal{O}_{\Delta_1}\left(R(x_1)\right)\mathcal{O}_{\Delta_2}\left(R(x_2)\right) &\implies \sum_{\Delta'} \sum_l \alpha^{\Delta', l}_{\Delta_1, \Delta_2} (x_{12}) \partial_{l} O_{\Delta'}(x_2) = \sum_{\Delta'} \sum_l \alpha^{\Delta', l}_{\Delta_1, \Delta_2} (R(x_{12})) \partial_{l} O_\Delta'(R(x_2))\\
    &\implies \sum_{\Delta'} \sum_l \alpha^{\Delta', l}_{\Delta_1, \Delta_2} (x_{12}) \partial_{l} O_\Delta'(x_2) = \sum_{\Delta'} \sum_l \sum_{l'} \alpha^{\Delta', l}_{\Delta_1, \Delta_2} (R(x_{12})) L(R^{-1})^{l'}_{l} \partial_{l'} O_{\Delta'}(x_2)\\
    &\implies \sum_{\Delta'} \sum_{l'} \alpha^{\Delta', l'}_{\Delta_1, \Delta_2} (x_{12}) \partial_{l'} O_{\Delta'}(x_2) = \sum_{\Delta'}  \sum_{l'} \left(\sum_l \alpha^{\Delta', l}_{\Delta_1, \Delta_2} (R(x_{12})) L(R^{-1})^{l'}_{l}\right) \partial_{l'} O_{\Delta'}(x_2)\\
    &\implies \alpha^{\Delta', l'}_{\Delta_1, \Delta_2} (x_{12}) = \sum_l \alpha^{\Delta', l}_{\Delta_1, \Delta_2} (R(x_{12})) L(R^{-1})^{l'}_{l}
  \end{align*} 
  which shows that $\alpha^{\Delta', l}_{\Delta_1, \Delta_2}$ transforms like a $(0, l)$ tensor under rotation. Since the only translationnaly invariant tensor accessible is $x_{12}$ we have to build $ \alpha^{\Delta', l}_{\Delta_1, \Delta_2}$ as a product of the components $x_{12}^{l_i}$ for $l_i\in l$ with some power of the scalar $|x_{12}|$ to account for scaling. In other words, we build a tensor \textit{density} representation from the tensor product of the spin $1$ vector representations (the only tensor building block available). As a result of the rotation constraint, we have $\alpha^{\Delta', l}_{\Delta_1, \Delta_2} = \gamma^{\Delta', l}_{\Delta_1, \Delta_2} |x_{12}|^\Delta (x_{12})^{l}$ (where $\gamma^{\Delta', l}_{\Delta_1, \Delta_2}$ is now a set of $x_{12}$ independant coefficients). We could include (but we do not for simplication) contractions of the derivatives with each other and contractions of the $x_{12}^\mu$ with each other which both consists in singlet contributions. Substituting this ansatz in the scaling constraint found above, we find 
  \begin{align*}
    \beta^{\Delta', l}_{\Delta_1, \Delta_2} |x_{12}|^\Delta (x_{12})^{l} = \lambda^{\Delta_1 + \Delta_2-\Delta' - |l|} \gamma^{\Delta', l}_{\Delta_1, \Delta_2} \lambda^{\Delta} |x_{12}|^\Delta \lambda^{|l|} (x_{12})^{l} \implies \Delta = -\Delta_1 - \Delta_2+\Delta'
  \end{align*}
  leading to the expansion 
  \begin{align*}
    \mathcal{O}_{\Delta_1}\left(x_1\right) \mathcal{O}_{\Delta_2}\left(x_2\right) = \sum_{\Delta'} \sum_l \gamma^{\Delta', l}_{\Delta_1, \Delta_2} |x_{12}|^{-\Delta_1 - \Delta_2+\Delta'} (x_{12})^{l} \partial_{l} O_{\Delta'}(x_2)
  \end{align*}
  with $l_i \geq 0,\ \forall l_i \in l$. 
  \item[(b)] We note that the multi-index $l$ is associated with a contraction of indices and it is therefore running over all multiplets. If $\gamma^{\Delta', l}_{\Delta_1, \Delta_2}$ distinguished between the ordering of the multiplet up to identical elements, it would spoil the Lorentz invariance of the contraction $(x_{12})^{l} \partial_{l} O_{\Delta'}(x_2)$. Without loss of generality, we rewrite the expansion by fixing $\gamma^{\Delta', l}_{\Delta_1, \Delta_2} = \beta^{\Delta', |l|}_{\Delta_1, \Delta_2}$ where $\beta^{\Delta', |l|}_{\Delta_1, \Delta_2}$ is a set of numbers depending only on the number of indices $|l|$ of the multi-index $l$.
  \item[(c)] From the OPE expansion derived in (a), we can evaluate three-point functions. We start by reducing the product $O_{\Delta_1}(x_1) O_{\Delta_2}(x_2)$ to a sum of operators. Then we use the linearity of the expectation value to break the correlator of this sum with $O_{\Delta_3}(x_3)$ into a sum of two-point functions which we can evaluate. Explicitly, we have
  \begin{align*}
    \left\langle\mathcal{O}_{\Delta_1}\left(x_1\right) \mathcal{O}_{\Delta_2}\left(x_2\right) \mathcal{O}_{\Delta_3}\left(x_3\right)\right\rangle &= \left\langle \sum_{\Delta'} \sum_l \beta^{\Delta', |l|}_{\Delta_1, \Delta_2} |x_{12}|^{-\Delta_1 - \Delta_2+\Delta'} (x_{12})^{l} \partial_{l} O_{\Delta'}(x_2) \mathcal{O}_{\Delta_3}\left(x_3\right)\right\rangle\\
    &=\sum_{\Delta'}  \sum_l \beta^{\Delta', |l|}_{\Delta_1, \Delta_2} |x_{12}|^{-\Delta_1 - \Delta_2+\Delta'} (x_{12})^{l} \partial_{l} \left\langle    O_{\Delta'}(x_2) \mathcal{O}_{\Delta_3}\left(x_3\right)\right\rangle\\
    &= \sum_{\Delta'} \delta_{\Delta', \Delta_3} \sum_l \beta^{\Delta', |l|}_{\Delta_1, \Delta_2} |x_{12}|^{-\Delta_1 - \Delta_2+\Delta'} (x_{12})^{l} \partial_{l} \frac{1}{|x_{23}|^{2 \Delta_3}}\\
    &= \frac{1}{|x_{12}|^{\Delta_1 + \Delta_2 - \Delta_3}}\sum_l \beta^{\Delta_3, |l|}_{\Delta_1, \Delta_2}  (x_{12})^{l} \frac{\partial}{\partial x^l_2} \frac{1}{|x_{23}|^{2 \Delta_3}}, \quad \text{supposing only one primary operator for each $\Delta_3$}. 
  \end{align*}
  Comparing this result with the first expression given for the three-point function, we get 
  \begin{align*}
    \frac{C_{123}}{|x_{12}|^{\Delta_1+\Delta_2-\Delta_3} |x_{23}|^{\Delta_2+\Delta_3-\Delta_1} |x_{31}|^{\Delta_3+\Delta_1-\Delta_2}}
    &= \frac{1}{|x_{12}|^{\Delta_1 + \Delta_2 - \Delta_3}}\sum_l \beta^{\Delta_3, l}_{\Delta_1, \Delta_2}  (x_{12})^{l} \partial_{l} \frac{1}{|x_{13}-x_{12}|^{2 \Delta_3}}. \\
    &\iff \frac{C_{123}}{|x_{23}|^{\Delta_2+\Delta_3-\Delta_1} |x_{31}|^{\Delta_3+\Delta_1-\Delta_2}} =  \sum_l \beta^{\Delta_3, l}_{\Delta_1, \Delta_2}  (x_{12})^{l} \partial_{l} \frac{1}{|x_{13}-x_{12}|^{2 \Delta_3}}
  \end{align*}


  \item[(d)] 
\end{enumerate}


\section{Acknowledgement}

Thanks to Thiago for a discussion about question 1 (b)

}

% References c
%\makereferences
%-------------------------------------------------------


%%%%%%%%%%%%%%%%%%%%%%%%
% Terminer le document %
%%%%%%%%%%%%%%%%%%%%%%%%
\end{document}
