%======================================================%
% Perimeter Scholars International Essay Template 2024 %
% Author: Agata Branczyk, January 2019                 %
% Last updated: 16 January 2024 by Giuseppe Sellaroli %
%======================================================%

\documentclass[12pt,twoside]{book}

%%%%%%%%%%%%%%%%%%%%%%%%%%%%%%%%%%%%%%%%%%%%%%%%%%%%%%%%%%%%%
%% EDIT THE FILE BELOW TO REFLECT YOUR ESSAY PROJECT       %%
%% (this will automatically populate the entire document)  %%
%%%%%%%%%%%%%%%%%%%%%%%%%%%%%%%%%%%%%%%%%%%%%%%%%%%%%%%%%%%%%

\newcommand{\essaytitle}{Your PSI Essay Title}
\newcommand{\shortessaytitle}{Your (short) PSI Essay Title} % Make this shorter if you see it spilling over within the header
\newcommand{\yourname}{Your name}
\newcommand{\yoursupervisor}{Your Supervisor's name/s}

%%%%%%%%%%%%%%%%%%%%%%%%%%%%%%%%%%
%% TEMPLATE FILES - DO NOT EDIT %%
%%%%%%%%%%%%%%%%%%%%%%%%%%%%%%%%%%
%%%%%%%%%%%%%%%%%%%%%%%%%%%
%% DO NOT EDIT THIS FILE %%
%%%%%%%%%%%%%%%%%%%%%%%%%%%

\usepackage[utf8]{inputenc}
\usepackage[T1]{fontenc}
\usepackage{lmodern}
\usepackage{fancyhdr}
\usepackage{lipsum}
\usepackage{graphicx}
\usepackage{titlesec}
\usepackage{appendix}
\usepackage[sectionbib]{chapterbib}
\usepackage[breakwords]{truncate}
\usepackage{lastpage}
\usepackage{amsmath}
\usepackage{mathrsfs}
\usepackage{amssymb}
\usepackage[font={small}]{caption}
\usepackage{cite}
\usepackage{physics}
\usepackage{graphicx,color}
\usepackage[dvipsnames]{xcolor}
\usepackage{hyperref}
\hypersetup{
	colorlinks=false,
	linkcolor=black,
	citecolor=black,
	urlcolor=black
}


%%%%%%%%%%%%%%%%%%%%%%%%%%%%%%%%%%%%%%
%% You may add your packages in the %%
%% main text, or in a separate file %%
%%%%%%%%%%%%%%%%%%%%%%%%%%%%%%%%%%%%%%
%%%%%%%%%%%%%%%%%%%%%%%%%%%
%% DO NOT EDIT THIS FILE %%
%%%%%%%%%%%%%%%%%%%%%%%%%%%

% Makes the sections have chapter numbering
\renewcommand*\thesection{\arabic{section}}

% Renames "Bibliography" to "References"
  \usepackage[nottoc,notlof,notlot]{tocbibind}
 \renewcommand{\bibname}{References}

% Formatting for the title on page 1
\titleformat{\chapter}[display]
  {\bfseries\large}{}{-10ex}
  {\titlerule\vspace{2ex}\filright\Huge}
  [\vspace{1ex}\titlerule]

 \renewcommand\sectionmark[1]{%
        \markright{\thesection\ ~~#1}}% gets rid of the dot in the section number in the header

% Headers, footers, and page numbers
\fancypagestyle{header} {
\fancyhf{}
    \renewcommand{\headrulewidth}{1pt}%

    \fancyhead[CE]{\S \emph{\rightmark}} % makes the section title header
    \fancyhead[CO]{\emph{\expandafter\MakeUppercase\expandafter{\shortessaytitle}}} % makes the essay title header
    \fancyhead[LE,RO]{\textbf{\thepage}} % makes the page numbers header
}

\fancypagestyle{plain}{%
  \renewcommand{\headrulewidth}{0pt}%
  \fancyhf{}%
  \fancyhead[LE,RO]{\textbf{\thepage}}
}

% Sets the margins
\setlength{\oddsidemargin}{1.4cm}
\setlength{\evensidemargin}{1.4cm}


\begin{document}
%%%%%%%%%%%%%%%%%%%%%%%%%%%%%%
%% TITLE PAGE - DO NOT EDIT %%
%%%%%%%%%%%%%%%%%%%%%%%%%%%%%%
%\setcounter{page}{1} % sets the page number for the book compilation (ignore this)
\input{./style_files/PSI_front_matter_essay}

%%%%%%%%%%%%%%%%%%%%%%%%%%%%%%%%
%% ESSAY ABSTRACT - EDIT THIS %%
%%%%%%%%%%%%%%%%%%%%%%%%%%%%%%%%
\begin{quote}
The abstract is an important component of your essay. It is likely the first substantive description of your work read by an external examiner. You should view it as an opportunity to set accurate expectations. The abstract is a summary of the whole thesis. It presents all the major elements of your work in a highly condensed form. An abstract is not merely an introduction in the sense of a preface, preamble, or advance organizer that prepares the reader for the thesis. It must also be capable of substituting for the whole thesis when there is insufficient time and space for the full text.
\end{quote}

%%%%%%%%%%%%%%%%%%%%%%%%%%%%%%%%%%%%%%%%%%%%
%% YOUR ESSAY CHAPTERS - EDIT THESE FILES %%
%%%%%%%%%%%%%%%%%%%%%%%%%%%%%%%%%%%%%%%%%%%

\section*{Statement of original research}

Chapters 1 and 2 of this essay contain literature review; Chapter 3 is based on original code that reproduces the results in Ref. [17]; Chapter 4 reproduces the results in Ref. [8] and also describes original work, whose results are summarized in Chapter 5. 

% comment out the following line if you want don't want to include a climate impact statement
\section*{Climate impact}

%%%%%%%%%%%%%%%%%%%%%%%%%%%%%%%%%%%%%%%%%%%%%%%%%%%%%%%%%%%%%%%%%%%%%%%%%%%%
%% Edit this file with information bout the climate impact of your essay. %%
%% The following is just a template, you are free to use a completely     %%
%% different format.                                                      %%
%%%%%%%%%%%%%%%%%%%%%%%%%%%%%%%%%%%%%%%%%%%%%%%%%%%%%%%%%%%%%%%%%%%%%%%%%%%%

\begin{center}
\begin{tabular}[b]{l c}
\hline
\textbf{Numerical simulations} & \\
\hline
Total Kernel Hours [$\mathrm{h}$]& 8260\\
Thermal Design Power Per Kernel [$\mathrm{W}$]& 5.75\\
Total Energy Consumption Simulations [$\mathrm{kWh}$] & 82\\
Average Emission Of CO$_2$ In Germany [$\mathrm{kg/kWh}$]& 0.56\\
Total CO$_2$-Emission For Numerical Simulations [$\mathrm{kg}$] & 45\\
Were The Emissions Offset? & \textbf{Yes}\\
\hline
\textbf{Transport} & \\
\hline
Total CO$_2$-Emission For Transport [$\mathrm{kg}$] & 1050\\
Were The Emissions Offset? & \textbf{Yes}\\
\hline
Total CO$_2$-Emission [$\mathrm{kg}$] & 1095\\
\hline
\hline
\end{tabular}
  \captionof{table}{Example of a CO$_2$-table that can be included towards the end of a scientific publication.
  Please also consider referencing
  \href{https://scientific-conduct.github.io}{scientific-conduct.github.io} to
  enhance visibility in your community.}
\end{center}

\pagestyle{header} %% Don't change this

\input{./chapters/chapter1}
\input{./chapters/chapter2}
\input{./chapters/chapter3}
\input{./chapters/chapter4}
% You can add/remove chapters

\input{./acknowledgements}

%%%%%%%%%%%%%%%%%%%%%%%%%%%%%%%%%%%%%%%
%% THE REFERENCES - EDIT THESE FILES %%
%%%%%%%%%%%%%%%%%%%%%%%%%%%%%%%%%%%%%%%
\bibliographystyle{style_files/utphys}
\bibliography{references}

%%%%%%%%%%%%%%%%%%%%%%%%%%%%%%%%%%%%%%%
%% THE APPENDICES - EDIT THESE FILES %%
%%%%%%%%%%%%%%%%%%%%%%%%%%%%%%%%%%%%%%%
\begin{subappendices} % DO NOT EDIT THIS LINE
\renewcommand\thesection{\Alph{section}} % DO NOT EDIT THIS LINE

\input{./chapters/appendix1}
\input{./chapters/appendix2}
% You can add/remove appendices

\end{subappendices} % DO NOT EDIT THIS LINE

% \cleardoublepage % makes sure there is an even number of pages for the book compilation (ignore this)

\end{document}