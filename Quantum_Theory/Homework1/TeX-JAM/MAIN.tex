\documentclass[10pt, a4paper]{article}

%%%%%%%%%%%%%%
%  Packages  %
%%%%%%%%%%%%%%


\usepackage{page_format}
\usepackage{special}
\usepackage{hyperref}
%----------------------------------------------------------------------
%\usepackage{amssymb} % Mathematical fonts.
%\usepackage{amsfonts} % Mathematical fonts.
\usepackage[nice]{nicefrac} % Nicer fractions
\usepackage{braket} % Dirac Notation.
\usepackage{bbm} % More bold fonts.
%\usepackage{mathrsfs} % Mathematical fonts.
\usepackage{esint} % Integrals
\usepackage{cancel} % Allows to scratch expressions.
\usepackage{mathtools} % Tools for math formating.
\usepackage{slashed} % Allows to slash individual characters.
\usepackage{xargs} % Better handling of optional arguments for commands
%----------------------------------------------------------------------
%\usepackage{lmodern} % Fonts.
\usepackage{feyn} % Feynman Diagrams in mathmode

%%%%%%%%%%%%%%%%%%%%%%%%%%%
% Mathématiques et physique
%%%%%%%%%%%%%%%%%%%%%%%%%%%%
% SI Units -----------------------
% The package 'siunitx' causes unresolved crashes (as of 22/08/31)
\newcommand{\ampere}{\text{A}}
\newcommand{\bell}{\text{B}}
\newcommand{\celsius}{\degree\text{C}}
\newcommand{\coulomb}{\text{C}}
\newcommand{\degree}{\,^{\circ}}
\newcommand{\farad}{\text{F}}
\newcommand{\electro}{\text{e}}
\newcommand{\gram}{\text{g}}
\newcommand{\henry}{\text{H}}
\newcommand{\hertz}{\text{Hz}}
\newcommand{\hour}{\text{h}}
\newcommand{\joule}{\text{J}}
\newcommand{\kelvin}{\text{K}}
\newcommand{\meter}{\text{m}}
\newcommand{\minute}{\text{m}}
\newcommand{\mole}{\text{mol}}
\newcommand{\newton}{\text{N}}
\newcommand{\ohm}{\Omega}
\newcommand{\pascal}{\text{Pa}}
\newcommand{\rad}{\text{rad}}
\newcommand{\second}{\text{s}}
\newcommand{\tesla}{\text{T}}
\newcommand{\torr}{\text{Torr}}
\newcommand{\volt}{\text{V}}
\newcommand{\watt}{\text{W}}
%
\newcommand{\tera}{\text{T}}
\newcommand{\giga}{\text{G}}
\newcommand{\mega}{~\text{M}}
\newcommand{\kilo}{~\text{k}}
\newcommand{\deci}{\text{d}}
\newcommand{\centi}{\text{c}}
\newcommand{\milli}{\text{m}}
\newcommand{\micro}{\mu}
\newcommand{\nano}{\text{n}}
\newcommand{\pico}{\text{p}}
\newcommand{\femto}{\text{f}}
%
\newcommand{\units}[1]{\text{#1}}
\newcommand{\tothe}[1]{\textsuperscript{#1}}
%
\newcommand{\per}{\text{/}}
%
\newcommand{\Time}[3]{#1\hour~#2\minute~#3\second} % TODO Optional arguments.
\newcommand{\Angle}[3]{#1^{\circ}~#2'~#3''} % TODO Optional arguments.


% Better epsilon -----------------------
\let\oldepsilon\epsilon
\let\epsilon\varepsilon
\let\varepsilon\oldepsilon


% Better \bar -----------------------
\renewcommand{\bar}[1]{\mkern 1.5mu\overline{\mkern-1.5mu#1\mkern-1.5mu}\mkern 1.5mu}


% Équations -----------------------
\newcommand{\al}[1]{\begin{align} #1 \end{align}} % Numbered equation(s),
\newcommand{\eqn}[1]{\begin{align*} #1 \end{align*}} % Number-less equation(s),
\newcommand{\sys}[1]{\begin{dcases*} #1 \end{dcases*}} % System of equations.


% Exponents -----------------------
\newcommand{\Exp}[1]{\text{e}^{#1}}		% e^#
\newcommand{\E}[1]{\times 10^{#1}}		% X 10^#


% Delimiters -----------------------
\newcommand{\p}[1]{\left( #1 \right)}	% (#)
\newcommand{\cro}[1]{\left[ #1 \right]}	% [#]
\newcommand{\abs}[1]{\left| #1\right|}	% |#|
\newcommand{\avg}[1]{\left\langle #1 \right\rangle} % <#>
\newcommand{\acc}[1]{\left\lbrace #1 \right\rbrace} % {#}


% Vectors -----------------------
\newcommand{\ve}[1]{\mathbf{#1}} % Upright bold face.
\newcommand{\vu}[1]{\hat{\ve{#1}}} % Hat vector upright bold face
\newcommand{\tens}{\otimes} % Tensor product
\newcommand{\nablav}{\bm{\nabla}} % Bold gradient


% Trig. functions with automatic formating  -----------------------
\newcommandx{\Sin}[2][1={}]{\text{sin}^{#1}\!\p{#2}}
\newcommandx{\Cos}[2][1={}]{\text{cos}^{#1}\!\p{#2}}
\newcommandx{\Tan}[2][1={}]{\text{tan}^{#1}\!\p{#2}}
\newcommandx{\Csc}[2][1={}]{\text{csc}^{#1}\!\p{#2}}
\newcommandx{\Sec}[2][1={}]{\text{sec}^{#1}\!\p{#2}}
\newcommandx{\Cot}[2][1={}]{\text{cot}^{#1}\!\p{#2}}
\newcommandx{\Arcsin}[2][1={}]{\text{arcsin}^{#1}\!\p{#2}}
\newcommandx{\Arccos}[2][1={}]{\text{arccos}^{#1}\!\p{#2}}
\newcommandx{\Arctan}[2][1={}]{\text{arctan}^{#1}\!\p{#2}}
\newcommandx{\Sinh}[2][1={}]{\text{sinh}^{#1}\!\p{#2}}
\newcommandx{\Cosh}[2][1={}]{\text{cosh}^{#1}\!\p{#2}}
\newcommandx{\Tanh}[2][1={}]{\text{tanh}^{#1}\!\p{#2}}


% Matrices -----------------------
\newcommand{\mat}[1]{\begin{bmatrix} #1 \end{bmatrix}} % Matrices with hooks.
\newcommand{\pmat}[1]{\begin{pmatrix} #1 \end{pmatrix}} % Matrices with parentheses.
\newcommand{\deter}[1]{\abs{\begin{matrix} #1 \end{matrix}}} % Determinant.
\newcommandx{\mO}[2][1={}, 2={}]{ \def\temp{#2}\ifx\temp\empty\ve{O}_{#1}\else\ve{O}_{#1\times #2}\fi}% Zero matrix.
\newcommandx{\mI}[2][1={}, 2={}]{ \def\temp{#2}\ifx\temp\empty\ve{I}_{#1}\else\ve{O}_{#1\times #2}\fi}%  Identity matrix.
\newcommand{\Det}[1]{\text{det}\p{#1}} % det(#)
\newcommand{\Tr}[1]{\text{Tr}\p{#1}} % Tr(#)


% Derivatives -----------------------
\newcommand{\D}{\text{d}} % Differential 'd'.
\newcommandx{\dd}[3][1={},3={}]{\frac{\D^{#3}#1}{\D{#2}^{#3}}} % Total derivative according to #2, #1 is the function and #3 is the order.
\newcommand{\del}{\partial} % Partial 'd'.
\newcommandx{\ddp}[3][1={},3={}]{\frac{\del^{#3}#1}{\del{#2}^{#3}}} % Dérivée partielle selon #2, #1 est la fonction est #3 est l'ordre.
\newcommand{\eval}[1]{\left. {#1} \right|} % Bar on the right of expression.
\newcommand{\delbar}{\slashed{\del}} % Partial Inexact differential.
\newcommand{\dbar}{\dj}% Inexact differential.


% Integrals -----------------------
\newcommand{\intinf}{\int\displaylimits_{-\infty}^{\infty}} % From -00 to 00.
\newcommandx{\Int}[2][1={},2={}]{\int\displaylimits_{#1}^{#2}} % Faster bounded integrals.


% Complex numbers -----------------------
\renewcommand{\Re}[1]{\text{Re}\acc{#1}} % Re{#}
\renewcommand{\Im}[1]{\text{Im}\acc{#1}} % Im{#}


% Sets -----------------------
\newcommand{\N}{\mathbbm{N}} % Natural numbers.
\newcommand{\Z}{\mathbbm{Z}} % Integers.
\newcommand{\Q}{\mathbbm{Q}} % Rational numbers.
\newcommandx{\R}[1][1={}]{\mathbbm{R}^{#1}} % Real numbers.
\newcommandx{\C}[1][1={}]{\mathbbm{C}^{#1}} % Complex numbers.
\newcommandx{\F}[1][1={}]{\mathbbm{F}^{#1}} % Some field.
\newcommand{\M}[3]{\mathbb{M}_{#1\times#2}(#3)}	% Matrices.
\newcommand{\Po}[2]{\mathbb{P}_{#1}(#2)} % Polynomials.
\newcommand{\Lin}{\mathbb{L}} % Linear maps.


% Constants and physical symbols -----------------------
\newcommand{\eo}{\epsilon_0} % epsilon 0.
\renewcommand{\L}{\mathcal{L}} % Lagrangian.

% References
\usepackage{biblatex}
\addbibresource{ref.bib}


%%%%%%%%%%%%
%  Colors  %
%%%%%%%%%%%%
% ! EDIT HERE !
\colorlet{chaptercolor}{red!70!black} % Foreground color.
\colorlet{chaptercolorback}{red!10!white} % Background color


%%%%%%%%%%%%%%
% Page titre %
%%%%%%%%%%%%%%
\title{Homework 1} % Title of the assignement.
\author{\PA} % Your name(s).
\teacher{Bindiya Arora} % Your teacher's name.
\class{Quantum Mechanics} % The class title.

\university{Perimeter Institute for Theoretical Physics} % University
\faculty{Perimeter Scholars International} % Faculty
%\departement{<Departement>} % Departement
\date{\today} % Date.


%%%%%%%%%%%%%%%%%%%%%%
% Begin the document %
%%%%%%%%%%%%%%%%%%%%%%
\begin{document}

% Make the title page.
\maketitlepage

% Make table of contents
\maketableofcontents

% Assignment starts here ----------------------------
\section{Quantum revivals}
Consider a one-dimensionnal quantum harmonic oscillator with mass $m$, frequency $\omega$, momentum operator $p$ and position operator $x$. The Hamiltonian governing the evolution of $x$ and $p$ in the Heisenberg picture is
\begin{align*}
    H=\frac{p^2(t)}{2 m}+\frac{1}{2} m \omega^2 x^2(t).
\end{align*}
\subsection{Operator time dependance}
In the Schrodinger picture, the time dependence of $x$, and $p$ is given by 
\begin{align*}
    &\dfrac{dx}{dt} =\dfrac{1}{i\hbar}  [x, H] = \dfrac{1}{i\hbar} \left([x, \frac{p^2(t)}{2 m}+\frac{1}{2} m \omega^2 x^2(t)]\right) = \dfrac{1}{2i\hbar m} \left([x, p(t)]p + p[x, p(t)]\right) = \dfrac{2}{2 i\hbar m} [x, p]\dfrac{p}{m} = \dfrac{p}{m}  \\
    &\dfrac{dp}{dt} =\dfrac{1}{i\hbar}  [p, H] = \dfrac{1}{i\hbar} \left([p, \frac{p^2(t)}{2 m}+ x^2(t)]\right) = \dfrac{m \omega^2}{2i\hbar} \left([p, \frac{1}{2} m \omega^2 x(t)]x + x[p,  x(t)]\right) = \dfrac{2 m \omega^2}{2i\hbar} [p, x] = - m \omega^2 x
\end{align*}
because $[x, p] = -[p, x] = i\hbar \bf{1}$ is a multiple of the identity and commutes with $x$ and $p$. To solve for the time evolution of $x$ and $p$, we first differentiate the first equation to get 
\begin{align*}
    \dfrac{d^2 x}{dt^2} = \dfrac{1}{m} \dfrac{dp}{dt} =  -\omega^2 x. 
\end{align*}
The solution of this second-order operator differential equation can be found componentwise because all components are decoupled from each other (the initial conditions will ensure $x$ is hermitian). For each component $\bra{x'} x(t) \ket{x"}$ in the eigenbasis of $x(0)$ We get a scalar harmonic oscillator equation
\begin{align*}
    \dfrac{d^2}{dt^2} \bra{x'} x(t) \ket{x"} = -\omega^2 \bra{x'} x(t) \ket{x"} \iff \bra{x'} x(t) \ket{x"} = A(x', x") \cos(\omega t) +  B(x', x") \dfrac{\sin(\omega t)}{\omega}
\end{align*} 
with $A, B$ determined by the initial conditions $x(t) = x(0)$. Evaluating the solution and its derivatives at $t = 0$ we have 
\begin{align*}
    \bra{x'} x(0) \ket{x"} = A(x', x"), \quad \text{and} \quad
    \bra{x'} \dfrac{dx}{dt}(0) \ket{x"} = \dfrac{1}{m}\bra{x'} p(0) \ket{x"}  = B(x', x"). 
\end{align*}
The functions $A$ and $B$ are therefore components of the operators $x(0)$ and $p(0)/(m)$ (initial position and initial velocity respectively) leading to the explicit solution of the initial value problem $x(t) = x(0) \cos(\omega t) +  (p(0)/m) \dfrac{\sin(\omega t)}{\omega}$. To obtain $p(t)$ we use the expression found for the time derivative of $x$ to find 
\begin{align*}
    p(t) = m \dfrac{dx}{dt} =  -m\omega x(0) \sin(\omega t) +  p(0) \cos(\omega t).  
\end{align*}

\subsection{Correlation function}
The position time-correlation function evaluated on the ground state $\ket{0}$ of the harmonic oscillator is given by 
\begin{align*}
    C(t) = \bra{0}x(0)x(t)\ket{0} &= \bra{0} \int \text{d}x' \ket{x'}\bra{x'} x(0)(x(0) \cos(\omega t) +  (p(0)/m) \dfrac{\sin(\omega t)}{\omega})\ket{0}\\
    &= \int \text{d}x'  \left(x'^2 \vert\psi_0(x')\vert^2 \cos(\omega t) +  \dfrac{i\hbar}{m} \dfrac{\sin(\omega t)}{\omega} \psi_0 \dfrac{d}{dx'} (x' \psi_0^*) \right) \\
    &= \cos(\omega t) \int \text{d}x'  \left(x'^2 \vert\psi_0(x')\vert^2  \right) + \dfrac{i\hbar}{m} \dfrac{\sin(\omega t)}{\omega} \int \text{d}x'   \vert\psi_0(x')\vert^2 + \dfrac{i \hbar}{m} \dfrac{\sin(\omega t)}{\omega} \int \text{d}x'  \left( \psi_0 x' \dfrac{d}{dx'} \psi_0^*  \right)\\
    &= \cos(\omega t) \int \text{d}x'  \left(x'^2 \vert\psi_0(x')\vert^2  \right) + \dfrac{i\hbar}{m} \dfrac{\sin(\omega t)}{\omega} + \dfrac{i \hbar}{m} \dfrac{\sin(\omega t)}{\omega} \int \text{d}x'  \left( \psi_0 x' \dfrac{d}{dx'} \psi_0^*  \right)
\end{align*}
using the wavefunction $\psi_0(x') = \langle x'\vert 0\rangle $, $\bra{x'} x(0) = \bra{x'} x' $ and $ \bra{x'}p(0)\ket{0} = \dfrac{d}{dx'} \psi_0$. To evaluate the first integral, we use the explicit expression $$
\psi_0(x') = \left(\frac{m \omega}{\pi \hbar}\right)^{\frac{1}{4}} \exp \left(-\frac{m \omega}{2 \hbar} x'^2\right) \implies \vert \psi_0(x')\vert^2 = \left(\frac{m \omega}{\pi \hbar}\right)^{\frac{1}{2}} \exp \left(-\frac{m \omega}{ \hbar} x'^2\right)$$
to get 
\begin{align*}
    \left(\frac{m \omega}{\pi \hbar}\right)^{\frac{1}{2}} \cos(\omega t) \int \text{d}x' x'^2\exp \left(-\frac{m \omega}{ \hbar} x'^2\right) &= \left(\frac{m \omega}{\pi \hbar}\right)^{\frac{1}{2}} \cos(\omega t) \dfrac{-\hbar}{\omega}\dfrac{d}{dm}\int \text{d}x' \exp \left(-\frac{m \omega}{ \hbar} x'^2\right) \\&= \left(\frac{m \omega}{\pi \hbar}\right)^{\frac{1}{2}} \cos(\omega t) \dfrac{-\hbar}{\omega}\dfrac{d}{dm} \left(\dfrac{\pi \hbar}{m \omega}\right)^{\frac12}\\
    &= \left(\frac{m \omega}{\pi \hbar}\right)^{\frac{1}{2}} \cos(\omega t) \dfrac{-\hbar}{2 m\omega}\left(\dfrac{\pi \hbar}{m \omega}\right)^{\frac12} = -\dfrac{\hbar}{2 m\omega}\cos(\omega t). 
\end{align*}
The last integral reads 
\begin{align*}
    \dfrac{i \hbar}{m} \dfrac{\sin(\omega t)}{\omega} \int \text{d}x'  \left( \psi_0 x' \dfrac{d}{dx'} \psi_0^*  \right) &= \frac{-m \omega}{\hbar}\dfrac{i \hbar}{m} \left(\frac{m \omega}{\pi \hbar}\right)^{\frac{1}{2}}  \dfrac{\sin(\omega t)}{\omega} \int \text{d}x'  \left(x'^2  \exp \left(-\frac{m \omega}{\hbar} x'^2\right)\right) \\
    &= \frac{-m \omega}{\hbar}\dfrac{i \hbar}{m} \left(\frac{m \omega}{\pi \hbar}\right)^{\frac{1}{2}}  \dfrac{\sin(\omega t)}{\omega} \dfrac{-\hbar}{2 m\omega}\left(\dfrac{\pi \hbar}{m \omega}\right)^{\frac12} = i \sin(\omega t) \dfrac{-\hbar}{2 m\omega}. 
\end{align*}
Combining all terms, we get 
$$
C(t) = -\dfrac{\hbar}{2 m\omega}\cos(\omega t) +   \dfrac{i \hbar}{m} \dfrac{\sin(\omega t)}{\omega} + i \dfrac{-i\hbar}{2 m\omega} \sin(\omega t)  = - \dfrac{\hbar}{2 m\omega} e^{-i \omega t}.
$$
% luke integration trick insight on time reversal of correlation function through conjugation as a result of time translation symmetry.


\section{Composite Spin}
The Hilbert space $\mathcal{H}$ of two particles of spin $1/2$ with hilbert spaces $\mathcal{H}_1, \mathcal{H}_2$ is given by the tensor product $\mathcal{H}_1 \otimes \mathcal{H}_2$. We are interested in the matrix representation of the total spin component operators. In the tensor product basis $\{\ket{11}, \ket{01}, \ket{10}, \ket{00}\}$, they are expressed as  
\begin{align*}
    &\sigma_x := \sigma_x^{(1)} \otimes 1^{(2)} + 1^{(1)} \otimes \sigma_x^{(2)}\\
    &\sigma_y := \sigma_y^{(1)} \otimes 1^{(2)} + 1^{(1)} \otimes \sigma_y^{(2)}\\
    &\sigma_z := \sigma_z^{(1)} \otimes 1^{(2)} + 1^{(1)} \otimes \sigma_z^{(2)}
\end{align*}
where $1^{(i)}$ and $\sigma_{x, y, z}^{(i)}$ are respectively the identity matrix and the Pauli matrices in the ${\ket{1}, \ket{0}}$ basis of $\mathcal{H}_i$. The Pauli matrices are given by 
\begin{align*}
    \begin{aligned}
        & \sigma_{x}=\begin{pmatrix}
        0 & 1 \\
        1 & 0
        \end{pmatrix}, \quad 
        & \sigma_{y}=
        \begin{pmatrix}
        0 & -i \\
        i & 0
        \end{pmatrix}, \quad \text{and} \quad 
        & \sigma_{z}=\begin{pmatrix}
        1 & 0 \\
        0 & -1
        \end{pmatrix}
        \end{aligned}
\end{align*}

The tensor product operation leads to the following $\sigma_{x, y, z}$ matrices : 
\begin{align*}
    &\sigma_x = \begin{pmatrix}1\cdot\sigma_x^{(1)} & 0\cdot\sigma_x^{(1)}\\0\cdot\sigma_x^{(1)} & 1\cdot\sigma_x^{(1)}\end{pmatrix} + \begin{pmatrix}(\sigma_{x})_{11}\cdot 1^{(1)} & (\sigma_{x})_{10}\cdot 1^{(1)}\\(\sigma_{x})_{01}\cdot 1^{(1)}& (\sigma_{x})_{00}\cdot 1^{(1)}\end{pmatrix} = 
    \begin{pmatrix}
        0 & 1 & 0 & 0\\
        1 & 0 & 0 & 0\\
        0 & 0 & 0 & 1\\
        0 & 0 & 1 & 0
    \end{pmatrix} + 
    \begin{pmatrix}
        0 & 0 & 1 & 0\\
        0 & 0 & 0 & 1\\
        1 & 0 & 0 & 0\\
        0 & 1 & 0 & 0
    \end{pmatrix}
    =
    \begin{pmatrix}
        0 & 1 & 1 & 0\\
        1 & 0 & 0 & 1\\
        1 & 0 & 0 & 1\\
        0 & 1 & 1 & 0
    \end{pmatrix}
    \quad 
    \begin{array}{ll}
        \ket{11} \\ \ket{01}\\ \ket{10}\\ \ket{00}
    \end{array}
    \\
    &\sigma_y = \begin{pmatrix}1\cdot\sigma_y^{(1)} & 0\cdot\sigma_y^{(1)}\\0\cdot\sigma_y^{(1)} & 1\cdot\sigma_y^{(1)}\end{pmatrix} + \begin{pmatrix}(\sigma_{y})_{11}\cdot 1^{(1)} & (\sigma_{y})_{10}\cdot 1^{(1)}\\(\sigma_{y})_{01}\cdot 1^{(1)}& (\sigma_{y})_{00}\cdot 1^{(1)}\end{pmatrix} =  
    \begin{pmatrix}
        0 & -i & 0 & 0\\
        i & 0 & 0 & 0\\
        0 & 0 & 0 & -i\\
        0 & 0 & i & 0
    \end{pmatrix} + 
    \begin{pmatrix}
        0 & 0 & -i & 0\\
        0 & 0 & 0 & -i\\
        i & 0 & 0 & 0\\
        0 & i & 0 & 0
    \end{pmatrix}
    =
    \begin{pmatrix}
        0 & -i & -i & 0\\
        i & 0 & 0 & -i\\
        i & 0 & 0 & -i\\
        0 & i & i & 0
    \end{pmatrix}\\
    &\sigma_z = \begin{pmatrix}1\cdot\sigma_z^{(1)} & 0\cdot\sigma_z^{(1)}\\0\cdot\sigma_y^{(1)} & 1\cdot\sigma_z^{(1)}\end{pmatrix} + \begin{pmatrix}(\sigma_{z})_{11}\cdot 1^{(1)} & (\sigma_{z})_{10}\cdot 1^{(1)}\\(\sigma_{z})_{01}\cdot 1^{(1)}& (\sigma_{z})_{00}\cdot 1^{(1)}\end{pmatrix}=
    \begin{pmatrix}
        1 & 0 & 0 & 0\\
        0 & -1 & 0 & 0\\
        0 & 0 & 1 & 0\\
        0 & 0 & 0 & -1
    \end{pmatrix}
    +
    \begin{pmatrix}
        1 & 0 & 0 & 0\\
        0 & 1 & 0 & 0\\
        0 & 0 & -1 & 0\\
        0 & 0 & 0 & -1
    \end{pmatrix}
    =
    \begin{pmatrix}
        2 & 0 & 0 & 0\\
        0 & 0 & 0 & 0\\
        0 & 0 & 0 & 0\\
        0 & 0 & 0 & -2
    \end{pmatrix}
\end{align*}
\newpage
% cite wikipedia for both questions 
\section{Free Path Integral}
{\footnotesize
The lagrangian of a free one-dimensional particle with mass $m$ described by a generalised coordinate $q$, is $L=\frac12 m \dot{q}$. To use the path integral formalism, we need to discretize the trajectory $q(t)$ in $N$ steps. Each step is associated with an independent variable $q_n$ corresponding to the coordinate of the particle at time $n T/N$ where $T$ is the final time at which we wish to observe the particle. The time interval for a step is $\Delta t = T/N$ and we have $\dot{q} = \frac{q_{n+1}-q_{n}}{\Delta t}$. Going further, the action integral is replaced by a discrete sum expressed as 
$$
S = \sum_{n=0}^{N-1} \frac12 m \left(\frac{q_{n+1}-q_{n}}{\Delta t}\right)^2 \Delta t.
$$
The path integral representation of the amplitude $A$ for the particle to scatter from $q_0$ to $q_{N}$ is given in the discretized picture by 
\begin{align*}
A &= \left(\dfrac{im}{2\hbar \pi \Delta t}\right)^{N/2}\left(\prod_{n = 1}^{N-1}\int_{-\infty}^{\infty} \text{d} q_n \right) \exp\left(\frac{i}{\hbar}\sum_{n=0}^{N-1} \frac12 m \left(\frac{q_{n+1}-q_{n}}{\Delta t}\right)^2  \Delta t\right)
\end{align*}
To compute it, we consider the sequence
\begin{align*}
    S(r) &= \left(\dfrac{im}{2 \hbar \pi \Delta t}\right)^{N/2}\left(\prod_{n = 1}^{N-r}\int_{-\infty}^{\infty} \text{d} q_n \right) \exp\left(\frac{i}{\hbar}\sum_{n=0}^{N-r-1} \frac12 m \left(\frac{q_{n+1}-q_{n}}{\Delta t}\right)^2  \Delta t + \frac{i}{\hbar}\frac12 m \left(\frac{q_{N}-q_{N-r}}{\Delta t r}\right)^2  \Delta tr\right)\\
    &= \left(\dfrac{im}{2\hbar \pi \Delta t}\right)^{N/2}\left(\prod_{n = 1}^{N-r-1}\int_{-\infty}^{\infty} \text{d} q_n \right) \exp\left(\frac{i}{\hbar}\sum_{n=0}^{N-r-2} \frac12 m \left(\frac{q_{n+1}-q_{n}}{\Delta t}\right)^2  \Delta t\right)  \int_{-\infty}^{\infty} \text{d} q_{N-r} \exp\left(\frac12 m \left(\frac{q_{N}-q_{N-r}}{\Delta tr}\right)^2  \Delta tr + \frac12 m \left(\frac{q_{N-r}-q_{N-r-1}}{\Delta t}\right)^2  \Delta t\right)\\
    &= \left(\dfrac{im}{2\hbar \pi \Delta t}\right)^{N/2}\left(\prod_{n = 1}^{N-(r+1)}\int_{-\infty}^{\infty} \text{d} q_n \right) \exp\left(\frac{i}{\hbar}\sum_{n=0}^{N-(r+1)-1} \frac12 m \left(\frac{q_{n+1}-q_{n}}{\Delta t}\right)^2  \Delta t + \frac{m i}{2\hbar} \left(\dfrac{q_{N}-q_{N-(r+1)}}{(r+1)\Delta t}\right)^2 (r+1)\Delta t\right)  \left(\dfrac{2\hbar\pi \Delta t r}{m i (r+1)}\right)^{1/2}\\
    &= \left(\dfrac{im}{2\hbar \pi \Delta t}\right)^{N/2}\left(\dfrac{2\hbar\pi \Delta t r}{m i (r+1)}\right)^{1/2} S(r+1)
\end{align*}

where we used

\begin{align*}
    &\int_{-\infty}^{\infty} \text{d} q_{N-r} \exp\left(\frac{im}{2\hbar} \left(\frac{q_{N}-q_{N-r}}{r\Delta t}\right)^2  r\Delta t + \frac{im}{2\hbar} \left(\frac{q_{N-r}-q_{N-r-1}}{\Delta t}\right)^2  \Delta t\right)\\
    =& \int_{-\infty}^{\infty} \text{d} q_{N-r} \exp\left(\frac{im}{2\hbar\Delta t} \left(\left(\dfrac{r+1}{r}\right)q_{N-r}^2 - 2\left(\dfrac{q_{M}}{r} + q_{N-r-1}\right)q_{N-r}\right) \right)\exp\left(\frac{im}{2\hbar\Delta t} \left(q_{N-r-1}^2 + \dfrac{q_{N}^2}{r}\right)\right)\\
    =& \int_{-\infty}^{\infty} \text{d} q_{N-r} \exp\left(\frac{im}{2\hbar\Delta t} \left(\dfrac{r+1}{r}\right) \left(q_{N-r}^2 - \left(\dfrac{2}{r+1}\right)(q_{N} +  rq_{N-r-1})q_{N-r} +  \left(\dfrac{q_{N} + rq_{N-r-1}}{r+1}\right)^2\right) \right)\exp\left(\frac{im}{2\hbar\Delta t} \left(q_{N-r-1}^2 + \dfrac{q_{N}^2}{r} -  \left(\dfrac{r+1}{r}\right)\left(\dfrac{q_{N} + rq_{N-r-1}}{r+1}\right)^2\right)\right)\\
    =& \left(\dfrac{2\hbar\pi \Delta t r}{m i (r+1)}\right)^{1/2}\exp\left(\frac{im}{2\hbar\Delta t (r+1)} \left((r+1)q_{N-r-1}^2 + \dfrac{q_{N}^2(r+1)}{r} -\left(\dfrac{q_{N}^2}{r} + r q_{N-r-1}^2  + 2q_{N}q_{N-r-1}\right)\right)\right)\\
    =& \left(\dfrac{2\hbar\pi \Delta t r}{m i (r+1)}\right)^{1/2}\exp\left(\frac{im}{2\hbar\Delta t (r+1)} \left((r+1)q_{N-r-1}^2 + q_{N}^2 - r q_{N-r-1}^2 - 2q_{N}q_{N-r-1}\right)\right)\\
    =& \left(\dfrac{2\hbar\pi \Delta t r}{m i (r+1)}\right)^{1/2}\exp\left(\frac{m i}{2\hbar} \left(\dfrac{q_{N}-q_{N-r-1}}{(r+1)\Delta t}\right)^2 (r+1)\Delta t\right) 
\end{align*}
Comparing $S$ with $A$ we see $A = S(1)$ and we also note that the maximal value for $r$ is provided by $N-r = 1 \iff N-1 = r$ which corresponds to 
\begin{align*}
    S(N-1)&=\left(\dfrac{im}{2\hbar\pi \Delta t}\right)^{N/2}\left(\prod_{n = 1}^{1}\int_{-\infty}^{\infty} \text{d} q_n \right) \exp\left(\frac{i}{\hbar}\frac12 m \left(\frac{q_{0+1}-q_{0}}{\Delta t}\right)^2  \Delta t + \frac{i}{\hbar}\frac12 m \left(\frac{q_{N}-q_{1}}{\Delta t (N-1)}\right)^2  \Delta t(N-1)\right) \\
    &= \left(\dfrac{im}{2\hbar\pi \Delta t}\right)^{N/2} \left(\dfrac{\hbar\pi \Delta t (N-1)}{m i (N)}\right)^{1/2}\exp\left(\frac{i}{\hbar}\frac12 m \left(\frac{q_{N}-q_{0}}{N\Delta t}\right)^2  N\Delta t\right).
\end{align*}
 Unpacking the telescopic expression for $S(0)$ we have 
\begin{align*}
    S(1) &= \left(\dfrac{2\hbar\pi \Delta t (1)}{m i (1+1)}\right)^{1/2} S(1) = \left(\dfrac{2\hbar\pi \Delta t 2}{m i (2+1)}\right)^{1/2} S(2) = \left(\dfrac{2\hbar\pi \Delta t (1)}{m i (1+1)}\right)^{1/2}\left(\dfrac{2\hbar\pi \Delta t (2)}{m i (2+1)}\right)^{1/2} S(3) = S(N-1)\prod_{r=1}^{N-2} \left(\dfrac{2\hbar\pi \Delta t (r)}{m i (r+1)}\right)^{1/2} \\&=  \left(\dfrac{im}{2\pi \Delta t}\right)^{N/2} \left(\dfrac{2\hbar\pi \Delta t (N-1)}{m i (N)}\right)^{1/2}\exp\left(\frac{i}{\hbar}\frac12 m \left(\frac{q_{N}-q_{0}}{N\Delta t}\right)^2  N\Delta t\right) \left(\dfrac{2\hbar\pi \Delta t}{m i}\right)^{(N-2)/2} \left(\dfrac{(1)}{(N-1)}\right)^{1/2}\\
    &=  \left(\dfrac{im}{2\hbar \pi \Delta t}\right)^{N/2}\exp\left(\frac{i}{\hbar}\frac12 m \left(\frac{q_{N}-q_{0}}{N\Delta t}\right)^2  N\Delta t\right) \left(\dfrac{2\hbar\pi \Delta t}{m i}\right)^{(N-1)/2} \left(\dfrac{1}{N}\right)^{1/2} = \exp\left(\frac{mi}{2\hbar T} \frac{(q_{N}-q_{0})^2}{2}\right) \left(\dfrac{m i}{2\hbar\pi T}\right)^{1/2}
\end{align*} 
This result coincides with the quantum mechanical free particle propagator $K(q_0, q_1, 0, T) = \bra{q_1} e^{-i H_{\text{free} T/\hbar}} \ket{q_0}$ (evolving with $H_{\text{free}}$) which is the imaginary time heat kernel (the free Schrodinger equation is a wick rotated heat equation so it makes sense that its propagator can be deduced by a wick rotation of the heat kernel). Furthermore, for fixed $q_0$ this propagator is the diffusing wave function of a free wave packet initially localized at $q_0$.
}

\section{Mach-Zehnder Interferometer}
The Mach-Zehnder interferometer is a sequence consisting of a beam splitter followed by a phase element on one of the split paths finally ending with a second beam splitter. In a quantum computer, we can represent the two paths with the orthogonal states $\ket{0}$ and $\ket{1}$ of a qubit. The first beam splitter is represented by the action of a Hadamard gate on $\ket{0}$ and the phase is represented by a phase gate leaving $\ket{0}$ unchanged while multiplying $\ket{1}$ by a phase. The final beam splitter is again represented with a Hadamard gate.  The probabilities to find the qubit in state $\ket{0}$ or $\ket{1}$ will oscillate as the phase shift applied increases (we get an interference pattern). To go further, we can implement the effect of a probe on the qubit. The probe is represented by a second qubit. After the first Hadamard gate, the state of the first qubit represents the path: it should activate the probe (flip its state from $\ket{0}$ to $\ket{1}$ state) only if the path is $\ket{1}$. The quantum computation gate that does this is the CNOT gate. It entangles the path with the probe state and no interference pattern is observed. 
\href{https://quantum-computing.ibm.com/composer/files/new?initial=N4IgdghgtgpiBcIAKBLGAndMC0BBMALgPYpgwD6A4uhABbTlIDKAkgEwAMbAzCADQgAjhADOUBCADySAKIA5AIq4mAWQAEbAHQcA3AB0wpAMYAbAK4ATGGr1CYJlACMAjJuO39YA4KwBzNYIA2mwAup5GfmpGgQAsYQa0AYEc8WAADgAUaSgA9GwAlEkpnolBxQb8IFYiEShpBChEYBIgAL5AA}{Quantum circuit for Mach-Zehnder} 



\section{Acknowledgement}
Thanks to Luke for pointing out mistakes in my calculation for the first question and for a discussion about the time reversal effect of conjugating a correlation function.

% References
\makereferences
%-------------------------------------------------------


%%%%%%%%%%%%%%%%%%%%%%%%
% Terminer le document %
%%%%%%%%%%%%%%%%%%%%%%%%
\end{document}