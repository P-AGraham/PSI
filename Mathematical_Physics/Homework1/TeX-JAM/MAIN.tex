\documentclass[10pt, a4paper]{article}

%%%%%%%%%%%%%%
%  Packages  %
%%%%%%%%%%%%%%


\usepackage{page_format}
\usepackage{special}
\usepackage{hyperref}
\usepackage{tikz}
\usepackage[compat=1.1.0]{tikz-feynman}
%----------------------------------------------------------------------
%\usepackage{amssymb} % Mathematical fonts.
%\usepackage{amsfonts} % Mathematical fonts.
\usepackage[nice]{nicefrac} % Nicer fractions
\usepackage{braket} % Dirac Notation.
\usepackage{bbm} % More bold fonts.
%\usepackage{mathrsfs} % Mathematical fonts.
\usepackage{esint} % Integrals
\usepackage{cancel} % Allows to scratch expressions.
\usepackage{mathtools} % Tools for math formating.
\usepackage{slashed} % Allows to slash individual characters.
\usepackage{xargs} % Better handling of optional arguments for commands
%----------------------------------------------------------------------
%\usepackage{lmodern} % Fonts.
\usepackage{feyn} % Feynman Diagrams in mathmode

%%%%%%%%%%%%%%%%%%%%%%%%%%%
% Mathématiques et physique
%%%%%%%%%%%%%%%%%%%%%%%%%%%%
% SI Units -----------------------
% The package 'siunitx' causes unresolved crashes (as of 22/08/31)
\newcommand{\ampere}{\text{A}}
\newcommand{\bell}{\text{B}}
\newcommand{\celsius}{\degree\text{C}}
\newcommand{\coulomb}{\text{C}}
\newcommand{\degree}{\,^{\circ}}
\newcommand{\farad}{\text{F}}
\newcommand{\electro}{\text{e}}
\newcommand{\gram}{\text{g}}
\newcommand{\henry}{\text{H}}
\newcommand{\hertz}{\text{Hz}}
\newcommand{\hour}{\text{h}}
\newcommand{\joule}{\text{J}}
\newcommand{\kelvin}{\text{K}}
\newcommand{\meter}{\text{m}}
\newcommand{\minute}{\text{m}}
\newcommand{\mole}{\text{mol}}
\newcommand{\newton}{\text{N}}
\newcommand{\ohm}{\Omega}
\newcommand{\pascal}{\text{Pa}}
\newcommand{\rad}{\text{rad}}
\newcommand{\second}{\text{s}}
\newcommand{\tesla}{\text{T}}
\newcommand{\torr}{\text{Torr}}
\newcommand{\volt}{\text{V}}
\newcommand{\watt}{\text{W}}
%
\newcommand{\tera}{\text{T}}
\newcommand{\giga}{\text{G}}
\newcommand{\mega}{~\text{M}}
\newcommand{\kilo}{~\text{k}}
\newcommand{\deci}{\text{d}}
\newcommand{\centi}{\text{c}}
\newcommand{\milli}{\text{m}}
\newcommand{\micro}{\mu}
\newcommand{\nano}{\text{n}}
\newcommand{\pico}{\text{p}}
\newcommand{\femto}{\text{f}}
%
\newcommand{\units}[1]{\text{#1}}
\newcommand{\tothe}[1]{\textsuperscript{#1}}
%
\newcommand{\per}{\text{/}}
%
\newcommand{\Time}[3]{#1\hour~#2\minute~#3\second} % TODO Optional arguments.
\newcommand{\Angle}[3]{#1^{\circ}~#2'~#3''} % TODO Optional arguments.


% Better epsilon -----------------------
\let\oldepsilon\epsilon
\let\epsilon\varepsilon
\let\varepsilon\oldepsilon


% Better \bar -----------------------
\renewcommand{\bar}[1]{\mkern 1.5mu\overline{\mkern-1.5mu#1\mkern-1.5mu}\mkern 1.5mu}


% Équations -----------------------
\newcommand{\al}[1]{\begin{align} #1 \end{align}} % Numbered equation(s),
\newcommand{\eqn}[1]{\begin{align*} #1 \end{align*}} % Number-less equation(s),
\newcommand{\sys}[1]{\begin{dcases*} #1 \end{dcases*}} % System of equations.


% Exponents -----------------------
\newcommand{\Exp}[1]{\text{e}^{#1}}		% e^#
\newcommand{\E}[1]{\times 10^{#1}}		% X 10^#


% Delimiters -----------------------
\newcommand{\p}[1]{\left( #1 \right)}	% (#)
\newcommand{\cro}[1]{\left[ #1 \right]}	% [#]
\newcommand{\abs}[1]{\left| #1\right|}	% |#|
\newcommand{\avg}[1]{\left\langle #1 \right\rangle} % <#>
\newcommand{\acc}[1]{\left\lbrace #1 \right\rbrace} % {#}


% Vectors -----------------------
\newcommand{\ve}[1]{\mathbf{#1}} % Upright bold face.
\newcommand{\vu}[1]{\hat{\ve{#1}}} % Hat vector upright bold face
\newcommand{\tens}{\otimes} % Tensor product
\newcommand{\nablav}{\bm{\nabla}} % Bold gradient


% Trig. functions with automatic formating  -----------------------
\newcommandx{\Sin}[2][1={}]{\text{sin}^{#1}\!\p{#2}}
\newcommandx{\Cos}[2][1={}]{\text{cos}^{#1}\!\p{#2}}
\newcommandx{\Tan}[2][1={}]{\text{tan}^{#1}\!\p{#2}}
\newcommandx{\Csc}[2][1={}]{\text{csc}^{#1}\!\p{#2}}
\newcommandx{\Sec}[2][1={}]{\text{sec}^{#1}\!\p{#2}}
\newcommandx{\Cot}[2][1={}]{\text{cot}^{#1}\!\p{#2}}
\newcommandx{\Arcsin}[2][1={}]{\text{arcsin}^{#1}\!\p{#2}}
\newcommandx{\Arccos}[2][1={}]{\text{arccos}^{#1}\!\p{#2}}
\newcommandx{\Arctan}[2][1={}]{\text{arctan}^{#1}\!\p{#2}}
\newcommandx{\Sinh}[2][1={}]{\text{sinh}^{#1}\!\p{#2}}
\newcommandx{\Cosh}[2][1={}]{\text{cosh}^{#1}\!\p{#2}}
\newcommandx{\Tanh}[2][1={}]{\text{tanh}^{#1}\!\p{#2}}


% Matrices -----------------------
\newcommand{\mat}[1]{\begin{bmatrix} #1 \end{bmatrix}} % Matrices with hooks.
\newcommand{\pmat}[1]{\begin{pmatrix} #1 \end{pmatrix}} % Matrices with parentheses.
\newcommand{\deter}[1]{\abs{\begin{matrix} #1 \end{matrix}}} % Determinant.
\newcommandx{\mO}[2][1={}, 2={}]{ \def\temp{#2}\ifx\temp\empty\ve{O}_{#1}\else\ve{O}_{#1\times #2}\fi}% Zero matrix.
\newcommandx{\mI}[2][1={}, 2={}]{ \def\temp{#2}\ifx\temp\empty\ve{I}_{#1}\else\ve{O}_{#1\times #2}\fi}%  Identity matrix.
\newcommand{\Det}[1]{\text{det}\p{#1}} % det(#)
\newcommand{\Tr}[1]{\text{Tr}\p{#1}} % Tr(#)


% Derivatives -----------------------
\newcommand{\D}{\text{d}} % Differential 'd'.
\newcommandx{\dd}[3][1={},3={}]{\frac{\D^{#3}#1}{\D{#2}^{#3}}} % Total derivative according to #2, #1 is the function and #3 is the order.
\newcommand{\del}{\partial} % Partial 'd'.
\newcommandx{\ddp}[3][1={},3={}]{\frac{\del^{#3}#1}{\del{#2}^{#3}}} % Dérivée partielle selon #2, #1 est la fonction est #3 est l'ordre.
\newcommand{\eval}[1]{\left. {#1} \right|} % Bar on the right of expression.
\newcommand{\delbar}{\slashed{\del}} % Partial Inexact differential.
\newcommand{\dbar}{\dj}% Inexact differential.


% Integrals -----------------------
\newcommand{\intinf}{\int\displaylimits_{-\infty}^{\infty}} % From -00 to 00.
\newcommandx{\Int}[2][1={},2={}]{\int\displaylimits_{#1}^{#2}} % Faster bounded integrals.


% Complex numbers -----------------------
\renewcommand{\Re}[1]{\text{Re}\acc{#1}} % Re{#}
\renewcommand{\Im}[1]{\text{Im}\acc{#1}} % Im{#}


% Sets -----------------------
\newcommand{\N}{\mathbbm{N}} % Natural numbers.
\newcommand{\Z}{\mathbbm{Z}} % Integers.
\newcommand{\Q}{\mathbbm{Q}} % Rational numbers.
\newcommandx{\R}[1][1={}]{\mathbbm{R}^{#1}} % Real numbers.
\newcommandx{\C}[1][1={}]{\mathbbm{C}^{#1}} % Complex numbers.
\newcommandx{\F}[1][1={}]{\mathbbm{F}^{#1}} % Some field.
\newcommand{\M}[3]{\mathbb{M}_{#1\times#2}(#3)}	% Matrices.
\newcommand{\Po}[2]{\mathbb{P}_{#1}(#2)} % Polynomials.
\newcommand{\Lin}{\mathbb{L}} % Linear maps.


% Constants and physical symbols -----------------------
\newcommand{\eo}{\epsilon_0} % epsilon 0.
\renewcommand{\L}{\mathcal{L}} % Lagrangian.

\usepackage{slashed}

% References
\usepackage{biblatex}
\addbibresource{ref.bib}


%%%%%%%%%%%%
%  Colors  %
%%%%%%%%%%%%
% ! EDIT HERE !
\colorlet{chaptercolor}{red!70!black} % Foreground color.
\colorlet{chaptercolorback}{red!10!white} % Background color

%%%%%%%%%%%%%%
% Page titre %
%%%%%%%%%%%%%%%
\title{Homework 1} % Title of the assignement.
\author{\PA} % Your name(s).
\teacher{Giuseppe Sellaroli} % Your teacher's name.
\class{Mathematical Physics} % The class title.

\university{Perimeter Institute for Theoretical Physics} % University
\faculty{Perimeter Scholars International} % Faculty
%\departement{<Departement>} % Departement
\date{\today} % Date.


%%%%%%%%%%%%%%%%%%%%%%
% Begin the document %
%%%%%%%%%%%%%%%%%%%%%%
\begin{document}

% Make the title page.
\maketitlepage

% Make table of contents
\maketableofcontents

% Assignment starts here ----------------------------

\footnotesize{

\section{Hodge star operator and vector calculus}
We are interested in the spaces $\Omega^k(M)$ of $k$-forms over a smooth manifold $M$ of dimension $n$ equiped with a pseudo-Riemanian metric tensor $g$ represented as $g = \gamma_{ij} \text{d}x^i \otimes \text{d}x^j$ in the frame field induced by the coordinate maps $x_i$ over a open subset $U \subset M$. The Hodge star operator constitutes a linear map $\star : \Omega^k(M) \to \Omega^{n-k}(M)$. In a local frame given by $\text{d}x^i$, its action is specified by 
\begin{align*}
  \star 1 = \sqrt{|\det{\gamma}|} \text{d}x^1 \wedge \cdots \wedge \text{d}x^n, \quad \text{d}x^{i_{1}} \wedge \cdots \wedge \text{d}x^{i_k} = \frac{1}{(n-k)!} \sqrt{\det \gamma} (\gamma^{-1})^{i_1 j_1}  \cdots (\gamma^{-1})^{i_1 j_1} \varepsilon_{j_1 \cdots j_n}  \text{d}x^{j_{k+1}} \wedge \cdots \wedge \text{d}x^{j_n} \quad \& \quad \star (f \alpha) = f \star \alpha  
\end{align*}
where $\varepsilon$ is the levi-civita symbol, $f\in \Omega^0(M)$ and $\alpha \in \Omega^k(M)$.
\begin{enumerate}
  \item[(a)] For now, we treat a cartesian frame field $\text{d}x^1 = \text{d}x, \text{d}x^2 = \text{d}y, \text{d}x^3 = \text{d}z$ in three dimensionnal euclidean space by replacing $\gamma_{ij}$ by $\delta_{ij}$. The Hodge dual of each frame field calculated as follows 
  \begin{align*}
    &\star \text{d}x = \frac{\sqrt{|1|}}{(3-1)!}(\delta^{-1})^{11}\epsilon_{123} \text{d}x^2 \wedge \text{d}x^3 + \frac{\sqrt{|1|}}{(3-1)!}(\delta^{-1})^{11}\epsilon_{132} \text{d}x^3 \wedge \text{d}x^2 = \frac{1}{2}(+1) \text{d}x^2 \wedge \text{d}x^3 + \frac{1}{2} (-1)(-1) \text{d}x^2 \wedge \text{d}x^3 = \text{d}x^2 \wedge \text{d}x^3\\
    &\star \text{d}y = \frac{\sqrt{|1|}}{(3-1)!}(\delta^{-1})^{22}\epsilon_{231} \text{d}x^3 \wedge \text{d}x^1 + \frac{\sqrt{|1|}}{(3-1)!}(\delta^{-1})^{22}\epsilon_{213} \text{d}x^1 \wedge \text{d}x^3 = \frac{1}{2}(+1) \text{d}x^3 \wedge \text{d}x^1 + \frac{1}{2} (-1)(-1) \text{d}x^3 \wedge \text{d}x^1 = \text{d}x^3 \wedge \text{d}x^1\\
    &\star \text{d}z = \frac{\sqrt{|1|}}{(3-1)!}(\delta^{-1})^{33}\epsilon_{312} \text{d}x^1 \wedge \text{d}x^2 + \frac{\sqrt{|1|}}{(3-1)!}(\delta^{-1})^{33}\epsilon_{321} \text{d}x^2 \wedge \text{d}x^1 = \frac{1}{2}(+1) \text{d}x^1 \wedge \text{d}x^2 + \frac{1}{2} (-1)(-1) \text{d}x^1 \wedge \text{d}x^2 = \text{d}x^1 \wedge \text{d}x^2
  \end{align*}
  \item[(b)] Using the metric we can associate a vector $\alpha^\sharp$ to each one-from $\alpha$ such that $g(\alpha^\sharp, v) = v(\alpha)$ for all vector fields $v$. Consider the expansions $\alpha = \alpha_x \text{d}x + \alpha_y \text{d}y + \alpha_z \text{d}z$,  $\alpha^\sharp = \alpha^\sharp_x \frac{\partial}{\partial x} + \alpha^\sharp_y \frac{\partial}{\partial y} + \alpha^\sharp_z \frac{\partial}{\partial z}$ and $v = v_x \frac{\partial}{\partial x} + v_y \frac{\partial}{\partial y} + v_z \frac{\partial}{\partial z}$. If the components of $g$ are $\delta_{ij}$, then $g(\alpha^\sharp, v) = \alpha^\sharp_x v_x + \alpha^\sharp_y v_y + \alpha^\sharp_z v_z$ and $\alpha(v) = \alpha_x v_x + \alpha_y v_y + \alpha_z v_z$. Since these two expressions are equal for all $v$ by definition of $\sharp$, we conclude $\alpha^\sharp_x = \alpha_x,\ \alpha^\sharp_y = \alpha_y,\ \alpha^\sharp_z = \alpha_z$ for the cartesian euclidean metric. We have a correspondance between one-forms and vectors which can be used to recover vector calculus from diferential form operations. 
  \begin{enumerate}
    \item[$\bullet$] Suppose $f$ is a smooth function then $\text{d}f = \partial_x f \text{d}x + \partial_y f \text{d}y + \partial_z f \text{d}z$ which maps to the vector $\text{d}f^\sharp = \partial_x f \frac{\partial}{\partial x} + \partial_y f \frac{\partial}{\partial y} + \partial_z f \frac{\partial}{\partial z}$ which corresponds to the usual notion of a gradient $\nabla f$. 
    
    \item[$\bullet$] For a one-form $\alpha = \alpha_x \text{d}x + \alpha_y \text{d}y + \alpha_z \text{d}z$, the exterior derivative reads 
    \begin{align*}
      \text{d}\alpha &= (\partial_x \alpha_x \text{d}x + \partial_y \alpha_x \text{d}y + \partial_z \alpha_x \text{d}z)\wedge \text{d}x +  (\partial_x \alpha_y \text{d}x + \partial_y \alpha_y \text{d}y + \partial_z \alpha_y \text{d}z)\wedge \text{d}y  +  (\partial_x \alpha_z \text{d}x + \partial_y \alpha_z \text{d}y + \partial_z \alpha_z \text{d}z)\wedge \text{d}z \\
      &= \partial_y \alpha_x \text{d}y \wedge \text{d}x + \partial_z \alpha_x \text{d}z \wedge \text{d}x + \partial_x \alpha_y \text{d}x \wedge \text{d}y + \partial_z \alpha_y \text{d}z \wedge \text{d}y + \partial_x \alpha_z \text{d}x \wedge \text{d}z + \partial_y \alpha_z \text{d}y \wedge \text{d}z\\
      &= (\partial_x \alpha_y -\partial_y \alpha_x) \text{d}x \wedge \text{d}y + (\partial_z \alpha_x - \partial_x \alpha_z) \text{d}z \wedge \text{d}x + (\partial_y \alpha_z -  \partial_z \alpha_y) \text{d}y \wedge \text{d}z.
    \end{align*}
    From the property $\star \star \alpha = (-1)^{1(3-1)} \text{sgn}(\text{det}(\delta_{ij})) = \alpha$ (for one-forms), applying $\star$ to the results found in (a) should bring us back to the expression on which $\star$ was applied to obtain them. Then it follows that 
    \begin{align*}
      \star \text{d}\alpha &= (\partial_x \alpha_y -\partial_y \alpha_x) \star\text{d}x \wedge \text{d}y + (\partial_z \alpha_x - \partial_x \alpha_z) \star\text{d}z \wedge \text{d}x + (\partial_y \alpha_z -  \partial_z \alpha_y) \star\text{d}y \wedge \text{d}z \quad \text{linearity of $\star$ and $\star(\text{coeff}\ \alpha) = \text{coeff}\star \alpha$}\\
      &= (\partial_x \alpha_y -\partial_y \alpha_x) \text{d}z + (\partial_z \alpha_x - \partial_x \alpha_z) \text{d}y + (\partial_y \alpha_z -  \partial_z \alpha_y) \text{d}x.
    \end{align*} 
    Finally, using $\sharp$ this result is mapped to the vector with components resulting from the vector product $\nabla \times \alpha^{\sharp}$. Indeed, we can write $(\star \text{d}\alpha)^{\sharp}  = \nabla \times \alpha^{\sharp}$.
    \item[$\bullet$] Now consider again a one-form $\alpha = \alpha_x \text{d}x + \alpha_y \text{d}y + \alpha_z \text{d}z$. This time, we start by applying $\star$ followed by an exterior derivative to obtain 
    \begin{align*}
      &\text{d}\star \alpha =  \text{d}(\alpha_x \text{d}y \wedge \text{d}z + \alpha_y \text{d}z \wedge \text{d}y + \alpha_z \text{d}x \wedge \text{d}y)\\
      &= (\partial_x \alpha_x \text{d}x + \partial_y \alpha_x \text{d}y + \partial_z \alpha_x \text{d}z) \wedge \text{d}y \wedge \text{d}z + (\partial_x \alpha_y \text{d}x + \partial_y \alpha_y \text{d}y + \partial_z \alpha_y \text{d}z) \wedge \text{d}z \wedge \text{d}x +  (\partial_x \alpha_z \text{d}x + \partial_y \alpha_z \text{d}y + \partial_z \alpha_z \text{d}z) \wedge \text{d}x \wedge \text{d}y\\
      &= \partial_x \alpha_x \text{d}x \wedge \text{d}y \wedge \text{d}z + \partial_y \alpha_y \text{d}y \wedge \text{d}z \wedge \text{d}x + \partial_z \alpha_z \text{d}z \wedge \text{d}x \wedge \text{d}y = (\partial_x \alpha_x + \partial_y \alpha_y + \partial_z \alpha_z)\text{d}x \wedge \text{d}y \wedge \text{d}z.
    \end{align*}
    Now using the fact $\star$ is an involution in our space (see previous $\bullet$), we have $1 = \star\star 1= \star \sqrt{\text{det}(\delta_{ij})} \text{d}x \wedge \text{d}y \wedge \text{d}z = \star \text{d}x \wedge \text{d}y \wedge \text{d}z$ leading to 
    \begin{align*}
      \star \text{d}\star \alpha = \star (\partial_x \alpha_x + \partial_y \alpha_y + \partial_z \alpha_z)\text{d}x \wedge \text{d}y \wedge \text{d}z = \partial_x \alpha_x + \partial_y \alpha_y + \partial_z \alpha_z
    \end{align*}
    which is directly the scalar result obtained when taking the divergence $\nabla \cdot \alpha^{\sharp}$. 
    \item[$\bullet$] Finally, consider two one forms $\alpha = \alpha_x \text{d}x + \alpha_y \text{d}y + \alpha_z \text{d}z$ and $\beta = \beta_x \text{d}x + \beta_y \text{d}y + \beta_z \text{d}z$. The $\star$ of their $\wedge$ product reads 
    \begin{align*}
      \star (\alpha \wedge \beta) &= \star((\alpha_x \text{d}x + \alpha_y \text{d}y + \alpha_z \text{d}z) \wedge (\beta_x \text{d}x + \beta_y \text{d}y + \beta_z \text{d}z))\\
      &= \star(\alpha_x \beta_y \text{d}x\wedge\text{d}y + \alpha_x \beta_z \text{d}x\wedge\text{d}z + \alpha_y \beta_x \text{d}y\wedge\text{d}x + \alpha_y \beta_z \text{d}y\wedge\text{d}z + \alpha_z \beta_x \text{d}z\wedge\text{d}x + \alpha_z\beta_y \text{d}z\wedge\text{d}y)\\
      &= (\alpha_x \beta_y - \alpha_y \beta_y) \star\text{d}x\wedge\text{d}y + (\alpha_z \beta_x - \alpha_x \beta_z) \star\text{d}z\wedge\text{d}x + (\alpha_y \beta_z - \alpha_z \beta_y)  \star\text{d}y\wedge\text{d}z \\
      &= (\alpha_x \beta_y - \alpha_y \beta_y) \text{d}z + (\alpha_z \beta_x - \alpha_x \beta_z) \text{d}y + (\alpha_y \beta_z - \alpha_z \beta_y) \text{d}x
    \end{align*}
    which matched the component of a vector product and gives the relation $(\star (\alpha \wedge \beta))^\sharp = \alpha^\sharp \times \beta^\sharp$.
  \end{enumerate}
\end{enumerate}  

\section{Maxwell's equations}
To express Maxwell's equations with differential forms, we use the following representations of the electric field $E$, magnetic field $B$, current density $J$
\begin{align*}
  & B=B_x \mathrm{d} y \wedge \mathrm{d} z+B_y \mathrm{d} z \wedge \mathrm{d} x+B_z \mathrm{d} x \wedge \mathrm{d} y, \quad E=E_x \mathrm{d} x+E_y \mathrm{d} y+E_z \mathrm{d} z, \quad J=J_x \mathrm{d} x+J_y \mathrm{d} y+J_z \mathrm{d} z. 
\end{align*}
and we take the charge density $\rho$ to be a zero-form. 
\begin{enumerate}
  \item[(a)] The representation of $B$ as a two-form is motivated by the fact magnetic field components behave as pseudo-vectors under full inversion $(x, y, z) \mapsto (-x, -y, -z)$. To verify that the two-form representation is consistent with this property, we perform a change basis $\{\text{d}x^i\}_{i=1}^3$ to the basis $\{\text{d}\tilde{x}^i\}_{i=1}^3$ generated by fully inverted spatial coordinates. We have $\text{d}\tilde{x}^i = \text{d}(-x) = -\text{d}x^i$ and $\text{d}\tilde{x}^i \wedge \text{d}\tilde{x}^j = (-1)^2 \text{d}x^i \wedge \text{d}x^j$. This implies that at the level of fully spatial two-forms, the full inversion of space leaves the components invariant as expected for a magnetic field.
  \item[(b)] To connect with the usual vector form of Maxwell's equations, we notice that the usual electric $\tilde{E}$, magnetic $\tilde{B}$ and current density $\tilde{J}$ vector fields are related to the forms given above by $\tilde{E} = E^\sharp, \ \tilde{B} = (\star B)^\sharp = (B_x \text{d}x + B_y \text{d}y + B_z \text{d}z)^\sharp, \ \quad \tilde{J} = J^\sharp$.
  Using results derived from problem 1, (b) we can use the original set of Maxwell equations to write  
  \begin{align*}
    &0 = \nabla \cdot \tilde{B} = \nabla \cdot (\star B)^\sharp = \star \text{d} \star (\star B) = \star \text{d} B \iff \text{d} B = 0 \quad \text{($\star$ is an involution)}\\
    &\mu_0 \rho =  \nabla \cdot \tilde{E} = \nabla \cdot E^\sharp = \star \text{d} \star E\\
    &0 = \frac{\partial \tilde{B}}{\partial t} + \nabla \times \tilde{E} = \frac{\partial (\star B)^\sharp}{\partial t} + (\star \text{d}E)^{\sharp} \iff 0 = \frac{\partial \star B}{\partial t} + (\star \text{d}E) \iff 0 = \frac{\partial B}{\partial t} + \text{d}E \quad \text{($\star$ is an involution, $\sharp$ is invertible)}\\
    &\mu_0 \tilde{J} = \mu_0 J^\sharp = -\frac{\partial \tilde{E}}{\partial t} + \nabla \times \tilde{B} = -\frac{\partial E^\sharp}{\partial t} + \nabla \times (\star B)^\sharp = -\frac{\partial E^\sharp}{\partial t} + (\star \text{d} \star B)^{\sharp} \iff \mu_0 J = -\frac{\partial E^\sharp}{\partial t} + (\star \text{d} \star B)^{\sharp} \quad \text{($\sharp$ is invertible)}
  \end{align*}
  where $\mu_0$ is the magnetic permeability (the equations are written in $c = 1$ units)
  \item[(c)] By Poincaré's lemma, every closed form on $\mathbb{R}^n$ is exact. Maxwell's equations tell us that $B$ is a closed two-form implying $B$ is also exact. Then, there exists a one-form $A$ such that $B = \text{d}A$. Inserting this result in Faraday's law, we get 
  \begin{align*}
    0 = \frac{\partial B}{\partial t} + \text{d}E = \frac{\partial \text{d}A}{\partial t} + \text{d}E = \text{d}\left(\frac{\partial A}{\partial t} + E\right). 
  \end{align*}
  Since $-\frac{\partial A}{\partial t} + E$ is a closed one-form it is also exact and there must exist zero-form $-\phi$ such that $-\text{d}\phi = \frac{\partial A}{\partial t} + E \iff E = -\text{d}\phi -\frac{\partial A}{\partial t}$.
  \item[(d)] We now go back to minkowski space (with $-+++$ signature) with coordinates $(x^0, x^1, x^2, x^3) =(x, y, z, t)$ associated to the one-form frame field $\{\text{d}x^\mu\}_{\mu=0}^3$. We define the two-form $F = B + E \wedge \text{d}t$ and combine the current and charge densities into a single one-form $J = -\rho \star \mathrm{d} t+J_x \star \mathrm{d} x+J_y \star \mathrm{d} y+J_z \star \mathrm{d} z$. To write Maxwell's equations in terms of these new objects, we need to determine the effect of $\star$ and its relation to vector calculus in Minkowski space. We start by calculating 
  \begin{align*}
    \star \text{d}t &= \frac{\sqrt{|-1|}}{(4-1)!}(\gamma^{-1})^{00}\epsilon_{0123} \text{d}x^1\wedge \text{d}x^2 \wedge \text{d}x^3 + \frac{\sqrt{|-1|}}{(4-1)!}(\gamma^{-1})^{00}\epsilon_{0132} \text{d}x^1 \wedge \text{d}x^3 \wedge \text{d}x^2\\
    &+ \frac{\sqrt{|-1|}}{(4-1)!}(\gamma^{-1})^{00}\epsilon_{0213} \text{d}x^2\wedge \text{d}x^1 \wedge \text{d}x^3 + \frac{\sqrt{|-1|}}{(4-1)!}(\gamma^{-1})^{00}\epsilon_{0231} \text{d}x^2 \wedge \text{d}x^3 \wedge \text{d}x^1\\
    &+ \frac{\sqrt{|-1|}}{(4-1)!}(\gamma^{-1})^{00}\epsilon_{0312} \text{d}x^3\wedge \text{d}x^1 \wedge \text{d}x^2 + \frac{\sqrt{|-1|}}{(4-1)!}(\gamma^{-1})^{00}\epsilon_{0321} \text{d}x^3 \wedge \text{d}x^2 \wedge \text{d}x^1\\
    &= (\gamma^{-1})^{00}\text{d}x^1 \wedge \text{d}x^3 \wedge \text{d}x^2 = -\text{d}x \wedge \text{d}y \wedge \text{d}z
  \end{align*}
  For the next calculations, we use the fact $\star$ acting on a $\wedge$ product of basis forms will yield the $\wedge$ product of the basis forms absent of the original product with a sign. The order of the resulting product is such that concatenating it with the original product on the left will produce an even permutation of $txyz$. To fully determine the sign factor, we add multiply by $-1$ if one of the one-forms in the original product is $\text{d}t$. With this in mind, we can write 
  \begin{align*}
    &\star \text{d}x = -\text{d}t \wedge \text{d}y \wedge \text{d}z \quad (xtyz\ \text{odd}), \quad \star \text{d}y = \text{d}t \wedge \text{d}x \wedge \text{d}z \quad (ytxz\ \text{even}), \quad \star \text{d}z = -\text{d}t \wedge \text{d}x \wedge \text{d}y\quad (ztxy\ \text{odd})\\
    &\star (\text{d}x \wedge \text{d}y) = \text{d}t \wedge \text{d}z \quad (xytz\ \text{even}), \quad \star (\text{d}y \wedge \text{d}z) = \text{d}t \wedge \text{d}x \quad (yztx\ \text{even}), \quad \star (\text{d}z \wedge \text{d}x) = \text{d}t \wedge \text{d}y \quad (zxty\ \text{even})\\
    &\star (\text{d}t \wedge \text{d}x) = (-1)\text{d}y \wedge \text{d}z \quad (txyz\ \text{even}), \quad \star (\text{d}t \wedge \text{d}y) = (-1)\text{d}z \wedge \text{d}x \quad (tyzx\ \text{even})\quad \star (\text{d}t \wedge \text{d}z) = (-1)\text{d}x \wedge \text{d}y \quad (tzxy\ \text{even})\\
    &\star 1 = \sqrt{|\text{det}(\gamma)|}\text{d}t \wedge \text{d}x \wedge \text{d}y \wedge \text{d}z = \text{d}t \wedge \text{d}x \wedge \text{d}y \wedge \text{d}z.
  \end{align*}
  These relations are completed with $\star \star \alpha = (-1)^{k(4-k)} (-1) \alpha$.
  \item[(e)] If we express $E$ and $B$ with the potentials $A = A_x \text{d}x + A_y \text{d}y + A_z \text{d}z$ and $\phi = -A_t$, $F$ becomes 
  \begin{align*}
    F &= \text{d}A +  \left(-\mathrm{d} \phi-\frac{\partial A}{\partial t}\right)\wedge \text{d}t \\
    &= (\partial_x A_y -\partial_y A_x) \text{d}x \wedge \text{d}y + (\partial_z A_x - \partial_x A_z) \text{d}z \wedge \text{d}x + (\partial_y A_z -  \partial_z A_y) \text{d}y \wedge \text{d}z \\
    &+ \partial_x A_t \text{d}x\wedge \text{d}t + \partial_y A_t \text{d}y\wedge \text{d}t + \partial_z A_t \text{d}z\wedge \text{d}t - (\partial_t A_x \text{d}x + \partial_t A_y \text{d}y + \partial_t A_z \text{d}z)\wedge \text{d} t\\
    &= (\partial_x A_y -\partial_y A_x) \text{d}x \wedge \text{d}y + (\partial_z A_x - \partial_x A_z) \text{d}z \wedge \text{d}x + (\partial_y A_z -  \partial_z A_y) \text{d}y \wedge \text{d}z 
    (\partial_t A_x - \partial_x A_t) \text{d}t\wedge \text{d}x + (\partial_t A_y - \partial_y A_t) \text{d}t\wedge \text{d}y  + (\partial_t A_z - \partial_z A_t) \text{d}t\wedge \text{d}z
  \end{align*}
  We see the equality of coefficients of $\text{d}x^{\mu} \wedge \text{d}x^{\nu}$ with components of the covariant Faraday tensor $F_{\mu\nu} = \partial_\mu A_\nu - \partial_\nu A_\mu$.
  \item[(f)] The Faraday tensor formulation of Maxwell's equations reads $\partial_\mu F^{\mu \nu} = \mu_0 J^\nu$ and $\partial_{(\mu} F_{\nu \sigma)} = 0$. We first notice that
  \begin{align*}
    \partial_{(\mu} F_{\nu \sigma)} = 0 \iff dF = \partial_{\sigma} F_{\mu\nu} \text{d}x^{\sigma} \wedge \text{d}x^{\mu} \wedge \text{d}x^{\nu}/2  = 6\partial_{(\sigma} F_{\mu\nu)} \text{d}x^{\sigma} \wedge \text{d}x^{\mu} \wedge \text{d}x^{\nu}/2 = 0.
  \end{align*}
  Then we can explicitly verify that 
  \begin{align*}
    \star F_{\mu \nu} \text{d}x^{\mu}\wedge\text{d}x^{\nu} &= F_{01} \star \text{d}x^{0}\wedge\text{d}x^{1} + F_{02} \star \text{d}x^{0}\wedge\text{d}x^{2} + F_{03} \star \text{d}x^{0}\wedge\text{d}x^{3} + F_{12} \star \text{d}x^{1}\wedge\text{d}x^{2} + F_{23} \star \text{d}x^{2}\wedge\text{d}x^{3} + F_{31} \star \text{d}x^{3}\wedge\text{d}x^{1}\\
    &=  -F_{01}  \text{d}x^{2}\wedge\text{d}x^{3} -F_{02}  \text{d}x^{3}\wedge\text{d}x^{1} - F_{03}  \text{d}x^{1}\wedge\text{d}x^{2} + F_{12} \text{d}x^{0}\wedge\text{d}x^{3} + F_{23}  \text{d}x^{0}\wedge\text{d}x^{1} + F_{31} \text{d}x^{0}\wedge\text{d}x^{2}
  \end{align*}
  and apply an exterior derivative to find 
  \begin{align*}
   \text{d} \star F &= -\partial_0 F_{01}  \text{d}x^{0} \wedge \text{d}x^{2}\wedge\text{d}x^{3} + \partial_0 F_{02}  \text{d}x^{0} \wedge \text{d}x^{1}\wedge\text{d}x^{3} - \partial_0 F_{03} \text{d}x^{0} \wedge \text{d}x^{1}\wedge\text{d}x^{2} - \partial_1 F_{12}  \text{d}x^{0} \wedge\text{d}x^{1}\wedge\text{d}x^{3} + \partial_2 F_{23}   \text{d}x^{0} \wedge\text{d}x^{1}\wedge\text{d}x^{2} + \partial_3 F_{31}  \text{d}x^{0} \wedge \text{d}x^{2}\wedge\text{d}x^{3}\\
  &+\partial_1 F_{10}  \text{d}x^{1} \wedge \text{d}x^{2}\wedge\text{d}x^{3} + \partial_2 F_{20}  \text{d}x^{1} \wedge \text{d}x^{2}\wedge\text{d}x^{3} + \partial_3 F_{30}  \text{d}x^{1} \wedge \text{d}x^{2}\wedge\text{d}x^{3} + \partial_2 F_{21}  \text{d}x^{0} \wedge\text{d}x^{2}\wedge\text{d}x^{3} - \partial_3 F_{32}   \text{d}x^{0} \wedge\text{d}x^{1}\wedge\text{d}x^{3} + \partial_1 F_{13}  \text{d}x^{0} \wedge \text{d}x^{1}\wedge\text{d}x^{2}\\
  &= (\partial_1 F_{10} + \partial_2 F_{20} + \partial_3 F_{30})\text{d}x^{1} \wedge \text{d}x^{2}\wedge\text{d}x^{3} + \cdots = -(\mu_0 J^0) \star \text{d}t  + \cdots = (\mu_0 J_0) \star \text{d}t  + \cdots
  \end{align*}
  We see that the $-1$ factors introduced by $\star$ essentially raise the indices of $F_{\mu\nu}$ and the exterior derivative contracts them with a partial derivative. We recover a three-form ($J$ is represented as a three-form).
  \item[(g)] The result found in (e) can be restated as $F = (\partial_\mu A_\nu - \partial_\nu A_\mu) \text{d}x^{\mu} \wedge \text{d}x^{\nu}/2 = \partial_\mu A_\nu \text{d}x^{\mu} \wedge \text{d}x^{\nu} = \text{d}A$ by definition of the exterior derivative. 
\end{enumerate}


\section{Acknowledgement}
I worked on this assignment on my own.


}

% References
\makereferences
%-------------------------------------------------------


%%%%%%%%%%%%%%%%%%%%%%%%
% Terminer le document %
%%%%%%%%%%%%%%%%%%%%%%%%
\end{document}
