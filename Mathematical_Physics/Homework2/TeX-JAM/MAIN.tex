\documentclass[10pt, a4paper]{article}

%%%%%%%%%%%%%%
%  Packages  %
%%%%%%%%%%%%%%


\usepackage{page_format}
\usepackage{special}
\usepackage{hyperref}
\usepackage{tikz}
\usepackage[compat=1.1.0]{tikz-feynman}
%----------------------------------------------------------------------
%\usepackage{amssymb} % Mathematical fonts.
%\usepackage{amsfonts} % Mathematical fonts.
\usepackage[nice]{nicefrac} % Nicer fractions
\usepackage{braket} % Dirac Notation.
\usepackage{bbm} % More bold fonts.
%\usepackage{mathrsfs} % Mathematical fonts.
\usepackage{esint} % Integrals
\usepackage{cancel} % Allows to scratch expressions.
\usepackage{mathtools} % Tools for math formating.
\usepackage{slashed} % Allows to slash individual characters.
\usepackage{xargs} % Better handling of optional arguments for commands
%----------------------------------------------------------------------
%\usepackage{lmodern} % Fonts.
\usepackage{feyn} % Feynman Diagrams in mathmode

%%%%%%%%%%%%%%%%%%%%%%%%%%%
% Mathématiques et physique
%%%%%%%%%%%%%%%%%%%%%%%%%%%%
% SI Units -----------------------
% The package 'siunitx' causes unresolved crashes (as of 22/08/31)
\newcommand{\ampere}{\text{A}}
\newcommand{\bell}{\text{B}}
\newcommand{\celsius}{\degree\text{C}}
\newcommand{\coulomb}{\text{C}}
\newcommand{\degree}{\,^{\circ}}
\newcommand{\farad}{\text{F}}
\newcommand{\electro}{\text{e}}
\newcommand{\gram}{\text{g}}
\newcommand{\henry}{\text{H}}
\newcommand{\hertz}{\text{Hz}}
\newcommand{\hour}{\text{h}}
\newcommand{\joule}{\text{J}}
\newcommand{\kelvin}{\text{K}}
\newcommand{\meter}{\text{m}}
\newcommand{\minute}{\text{m}}
\newcommand{\mole}{\text{mol}}
\newcommand{\newton}{\text{N}}
\newcommand{\ohm}{\Omega}
\newcommand{\pascal}{\text{Pa}}
\newcommand{\rad}{\text{rad}}
\newcommand{\second}{\text{s}}
\newcommand{\tesla}{\text{T}}
\newcommand{\torr}{\text{Torr}}
\newcommand{\volt}{\text{V}}
\newcommand{\watt}{\text{W}}
%
\newcommand{\tera}{\text{T}}
\newcommand{\giga}{\text{G}}
\newcommand{\mega}{~\text{M}}
\newcommand{\kilo}{~\text{k}}
\newcommand{\deci}{\text{d}}
\newcommand{\centi}{\text{c}}
\newcommand{\milli}{\text{m}}
\newcommand{\micro}{\mu}
\newcommand{\nano}{\text{n}}
\newcommand{\pico}{\text{p}}
\newcommand{\femto}{\text{f}}
%
\newcommand{\units}[1]{\text{#1}}
\newcommand{\tothe}[1]{\textsuperscript{#1}}
%
\newcommand{\per}{\text{/}}
%
\newcommand{\Time}[3]{#1\hour~#2\minute~#3\second} % TODO Optional arguments.
\newcommand{\Angle}[3]{#1^{\circ}~#2'~#3''} % TODO Optional arguments.


% Better epsilon -----------------------
\let\oldepsilon\epsilon
\let\epsilon\varepsilon
\let\varepsilon\oldepsilon


% Better \bar -----------------------
\renewcommand{\bar}[1]{\mkern 1.5mu\overline{\mkern-1.5mu#1\mkern-1.5mu}\mkern 1.5mu}


% Équations -----------------------
\newcommand{\al}[1]{\begin{align} #1 \end{align}} % Numbered equation(s),
\newcommand{\eqn}[1]{\begin{align*} #1 \end{align*}} % Number-less equation(s),
\newcommand{\sys}[1]{\begin{dcases*} #1 \end{dcases*}} % System of equations.


% Exponents -----------------------
\newcommand{\Exp}[1]{\text{e}^{#1}}		% e^#
\newcommand{\E}[1]{\times 10^{#1}}		% X 10^#


% Delimiters -----------------------
\newcommand{\p}[1]{\left( #1 \right)}	% (#)
\newcommand{\cro}[1]{\left[ #1 \right]}	% [#]
\newcommand{\abs}[1]{\left| #1\right|}	% |#|
\newcommand{\avg}[1]{\left\langle #1 \right\rangle} % <#>
\newcommand{\acc}[1]{\left\lbrace #1 \right\rbrace} % {#}


% Vectors -----------------------
\newcommand{\ve}[1]{\mathbf{#1}} % Upright bold face.
\newcommand{\vu}[1]{\hat{\ve{#1}}} % Hat vector upright bold face
\newcommand{\tens}{\otimes} % Tensor product
\newcommand{\nablav}{\bm{\nabla}} % Bold gradient


% Trig. functions with automatic formating  -----------------------
\newcommandx{\Sin}[2][1={}]{\text{sin}^{#1}\!\p{#2}}
\newcommandx{\Cos}[2][1={}]{\text{cos}^{#1}\!\p{#2}}
\newcommandx{\Tan}[2][1={}]{\text{tan}^{#1}\!\p{#2}}
\newcommandx{\Csc}[2][1={}]{\text{csc}^{#1}\!\p{#2}}
\newcommandx{\Sec}[2][1={}]{\text{sec}^{#1}\!\p{#2}}
\newcommandx{\Cot}[2][1={}]{\text{cot}^{#1}\!\p{#2}}
\newcommandx{\Arcsin}[2][1={}]{\text{arcsin}^{#1}\!\p{#2}}
\newcommandx{\Arccos}[2][1={}]{\text{arccos}^{#1}\!\p{#2}}
\newcommandx{\Arctan}[2][1={}]{\text{arctan}^{#1}\!\p{#2}}
\newcommandx{\Sinh}[2][1={}]{\text{sinh}^{#1}\!\p{#2}}
\newcommandx{\Cosh}[2][1={}]{\text{cosh}^{#1}\!\p{#2}}
\newcommandx{\Tanh}[2][1={}]{\text{tanh}^{#1}\!\p{#2}}


% Matrices -----------------------
\newcommand{\mat}[1]{\begin{bmatrix} #1 \end{bmatrix}} % Matrices with hooks.
\newcommand{\pmat}[1]{\begin{pmatrix} #1 \end{pmatrix}} % Matrices with parentheses.
\newcommand{\deter}[1]{\abs{\begin{matrix} #1 \end{matrix}}} % Determinant.
\newcommandx{\mO}[2][1={}, 2={}]{ \def\temp{#2}\ifx\temp\empty\ve{O}_{#1}\else\ve{O}_{#1\times #2}\fi}% Zero matrix.
\newcommandx{\mI}[2][1={}, 2={}]{ \def\temp{#2}\ifx\temp\empty\ve{I}_{#1}\else\ve{O}_{#1\times #2}\fi}%  Identity matrix.
\newcommand{\Det}[1]{\text{det}\p{#1}} % det(#)
\newcommand{\Tr}[1]{\text{Tr}\p{#1}} % Tr(#)


% Derivatives -----------------------
\newcommand{\D}{\text{d}} % Differential 'd'.
\newcommandx{\dd}[3][1={},3={}]{\frac{\D^{#3}#1}{\D{#2}^{#3}}} % Total derivative according to #2, #1 is the function and #3 is the order.
\newcommand{\del}{\partial} % Partial 'd'.
\newcommandx{\ddp}[3][1={},3={}]{\frac{\del^{#3}#1}{\del{#2}^{#3}}} % Dérivée partielle selon #2, #1 est la fonction est #3 est l'ordre.
\newcommand{\eval}[1]{\left. {#1} \right|} % Bar on the right of expression.
\newcommand{\delbar}{\slashed{\del}} % Partial Inexact differential.
\newcommand{\dbar}{\dj}% Inexact differential.


% Integrals -----------------------
\newcommand{\intinf}{\int\displaylimits_{-\infty}^{\infty}} % From -00 to 00.
\newcommandx{\Int}[2][1={},2={}]{\int\displaylimits_{#1}^{#2}} % Faster bounded integrals.


% Complex numbers -----------------------
\renewcommand{\Re}[1]{\text{Re}\acc{#1}} % Re{#}
\renewcommand{\Im}[1]{\text{Im}\acc{#1}} % Im{#}


% Sets -----------------------
\newcommand{\N}{\mathbbm{N}} % Natural numbers.
\newcommand{\Z}{\mathbbm{Z}} % Integers.
\newcommand{\Q}{\mathbbm{Q}} % Rational numbers.
\newcommandx{\R}[1][1={}]{\mathbbm{R}^{#1}} % Real numbers.
\newcommandx{\C}[1][1={}]{\mathbbm{C}^{#1}} % Complex numbers.
\newcommandx{\F}[1][1={}]{\mathbbm{F}^{#1}} % Some field.
\newcommand{\M}[3]{\mathbb{M}_{#1\times#2}(#3)}	% Matrices.
\newcommand{\Po}[2]{\mathbb{P}_{#1}(#2)} % Polynomials.
\newcommand{\Lin}{\mathbb{L}} % Linear maps.


% Constants and physical symbols -----------------------
\newcommand{\eo}{\epsilon_0} % epsilon 0.
\renewcommand{\L}{\mathcal{L}} % Lagrangian.

\usepackage{slashed}

% References
\usepackage{biblatex}
\addbibresource{ref.bib}


%%%%%%%%%%%%
%  Colors  %
%%%%%%%%%%%%
% ! EDIT HERE !
\colorlet{chaptercolor}{red!70!black} % Foreground color.
\colorlet{chaptercolorback}{red!10!white} % Background color

%%%%%%%%%%%%%%
% Page titre %
%%%%%%%%%%%%%%%
\title{Homework 2} % Title of the assignement.
\author{\PA} % Your name(s).
\teacher{Giuseppe Sellaroli} % Your teacher's name.
\class{Mathematical Physics} % The class title.

\university{Perimeter Institute for Theoretical Physics} % University
\faculty{Perimeter Scholars International} % Faculty
%\departement{<Departement>} % Departement
\date{\today} % Date.


%%%%%%%%%%%%%%%%%%%%%%
% Begin the document %
%%%%%%%%%%%%%%%%%%%%%%
\begin{document}

% Make the title page.
\maketitlepage

% Make table of contents
\maketableofcontents

% Assignment starts here ----------------------------

\footnotesize{

\section{Dynamics on the tangent bundle}

\begin{enumerate}
  \item[(a)] We are interested in the description of the dynamics of a set of particles with the language of vector bundles. Our starting point is to take the allowed positions $\mathbf{q}$ to constitute a smooth $n$-manifold $Q$. At each point, $\mathbf{q}$, the tangent space $T_{\mathbf{q}}Q$ is the vector space of directional derivatives $\mathbf{v}$ along trajectories going through $\mathbf{q}$. These derivatives are identified with the velocities allowed at $\mathbf{q}$. The complete description of dynamics is provided by the tangent bundle $TQ$ containing the pairs $(\mathbb{q}, \mathbb{v})$ describing all instantaneous configurations of the system.\medskip
  
  
  To use the usual analysis of dynamics we use coordinate charts on $Q$ given by the coordinate functions $\{q^{i}\}_{i = 1}^{n}$. A coordinate chart on $TQ$ can be constructed by appending the components of vectors in the coordinate basis induced by $q^{i}$ at $\mathbf{q}$ to the coordinates produced by $q^{i}$. The maps $\{v^i\}_{i = 1}^{n}$ returning the the vector components at $\mathbf{q}$ can be expressed with the dual coordinate basis $\text{d}q^{i}_\mathbf{q}$ through the relation $v^{i}(\mathbf{q}, \mathbf{v}) = \text{d}q^{i}_\mathbf{q} (\mathbf{v})$.\medskip
  
  The dynamics of the system is represented by a Lagrangian smooth function $L: TQ \to \mathbb{R}$. The legender transform associate to $L$ is the map between $TQ$ and the cotangent bundle $T^\star Q$ given by $\mathbf{F}L : (\mathbf{q}, \mathbf{v}) \mapsto (\mathbf{q}, DL_\mathbf{q}(\mathbf{v}))$ where $DL_\mathbf{q} : \mathbf{v} \in T_{\mathbf{q}} Q \mapsto \frac{\partial \hat{L}}{\partial v^i} (\hat{q}, \hat{v}) \text{d}q^{i}_{\mathbf{q}} \in T_\mathbf{q}^\star Q$ (with the coordinate representation $\hat{L} = L \circ ((q^i)^{-1}, (v^i)^{-1})$ and $\hat{q}^i = q^{i}(\mathbf{q}, \mathbf{v})$ and $\hat{v}^i = v^{i}(\mathbf{q}, \mathbf{v})$).\medskip 
  
  %$1$-form $\theta = p_i \text{d}q^i \in T^\star T^\star Q$ associated with    the form $\theta_L \in T^\star T^\star Q$
  Since the Legendre transform provides a smooth map between $TQ$ and $T^\star Q$, we can use it to pull back the canonical symplectic structure on $T^\star Q$ and bring it to $TQ$. This structure is provided by the symplectic potential $1$-form $\theta = p_i \text{d}q^i \in T^\star T^\star Q$ where $p_i$ are coordinate functions forming a chart $T^\star Q$ when combined with $q^i$. More precisely, the $p_i$ functions give the components of covectors $\mathbf{p}$ at point $\mathbf{q}$ trough the relation $p_{i}(\mathbf{q}, \mathbf{p}) = \left.\frac{\partial}{\partial q^i}\right|_\mathbf{q} (\mathbf{p})$.\medskip
  
  The pullback $\theta_L = \mathbf{F}L^{\star}(\theta) \in T^{\star} TQ$ of $\theta$ is both linear and commutes with exterior derivatives. Using these properties we can calculate $\theta_L$ by first calculating the pullback of $q^i$ as functions over $TQ$ and then taking the exterior derivative. At $(\mathbf{q}, \mathbf{v}) \in TQ$, we have 
  \begin{align*}
    \mathbf{F}L^{\star} q^i (\mathbf{q}, \mathbf{v}) = q^i \circ \mathbf{F}L (\mathbf{q}, \mathbf{v}) =  q^i(\mathbf{q}, DL_\mathbf{q}(\mathbf{v})) = q^i(\mathbf{q}, \mathbf{p})
  \end{align*}
  and applying an exterior derivatives leads to $\mathbf{F}L^{\star} \text{d} q^i_{\mathbf{q}, \mathbf{p}} = \text{d}(\mathbf{F}L^{\star} q^i) = \text{d}q^i_{\mathbf{q}, \mathbf{v}}$. We note that while $\text{d}q^i\in T^{\star}Q$ can be evaluated at $\mathbf{q}$, the new $\text{d}q^i$ obtained here is constructed from a function over the bundle $TQ$ and is therefore evaluated at $\mathbf{q}, \mathbf{v}$. Then we evaluate the pullback of the functions $p_{i}$ at $(\mathbf{q}, \mathbf{v}) \in TQ$ to be
  \begin{align*}
    \mathbf{F}L^{\star} p_i(\mathbf{q}, \mathbf{v}) = p_i \circ \mathbf{F}L (\mathbf{q}, \mathbf{v}) =  p_i (\mathbf{q}, DL_\mathbf{q}(\mathbf{v})) =  \left.\frac{\partial}{\partial q^i}\right|_\mathbf{q} DL_\mathbf{q}(\mathbf{v}) = \frac{\partial \hat{L}}{\partial v^j} (\hat{q}, \hat{v}) \left.\frac{\partial}{\partial q^i}\right|_\mathbf{q} \text{d}q^{j}_{\mathbf{q}} = \frac{\partial \hat{L}}{\partial v^i} (\hat{q}, \hat{v}).
  \end{align*}
  Combining these results with the linearity of the pullback, we get $\theta_L(\mathbf{q}, \mathbf{v}) = \frac{\partial \hat{L}}{\partial v^i} (\hat{q}, \hat{v}) \text{d}q^{i}_{\mathbf{q}, \mathbf{v}}$. 
  \item[(b)] Using again the commutation of pullback and exterior derivative, we obtain the pullback at $(\mathbf{q}, \mathbf{v})$ of the symplectic form $\omega = -\text{d}\theta$ by $\mathbf{F}L$ as follows:
  \begin{align*}
    \omega_L(\mathbf{q}, \mathbf{v}) &= (\mathbf{F}L^{\star}\omega) (\mathbf{q}, \mathbf{v}) = -(\mathbf{F}L^{\star}\text{d}\theta) (\mathbf{q}, \mathbf{v}) =  -\text{d} (\mathbf{F}L^{\star}\theta) (\mathbf{q}, \mathbf{v}) = -\text{d} \left(\frac{\partial \hat{L}}{\partial v^i} (\hat{q}, \hat{v}) \text{d}q^{i}_{\mathbf{q}, \mathbf{v}}\right)\\
    &= \underbrace{-\frac{\partial \hat{L}}{\partial v^j \partial v^i} (\hat{q}, \hat{v})}_B  \text{d}v^{j}_{\mathbf{q}, \mathbf{v}} \wedge \text{d}q^{i}_{\mathbf{q}, \mathbf{v}} + \underbrace{\frac{1}{2}\left(\frac{\partial \hat{L}}{\partial q^i \partial v^j} (\hat{q}, \hat{v}) - \frac{\partial \hat{L}}{\partial q^j \partial v^i} (\hat{q}, \hat{v})\right)}_{A}\text{d}q^{j}_{\mathbf{q}, \mathbf{v}} \wedge \text{d}q^{i}_{\mathbf{q}, \mathbf{v}}.
  \end{align*}
  \item[(c)] This $2$-form is a section on $T^\star TQ$ and we now determine under which condition on $L$ it becomes a symplectic $2$-form. In a local basis $\text{d}x^{j}_{\mathbf{q}, \mathbf{v}} \wedge \text{d}x^{i}_{\mathbf{q}, \mathbf{v}}$ with $\{x^i\}_{i = 1}^{2n} = \{q^1 \cdots q^n, v^1 \cdots v^n\}$, a symplectif $2$-form must be given by $\omega_{i, j}\text{d}x^{j}_{\mathbf{q}, \mathbf{v}} \wedge \text{d}x^{i}_{\mathbf{q}, \mathbf{v}}$ with $\omega_{j, i}$ having non-vanishing determinant as a matrix. Here we have the matrix 
  \begin{align*}
    [\omega_{i, j}] = 
    \begin{pmatrix}
      A & B \\
      -B & 0
    \end{pmatrix}\implies \det [\omega_{i, j}] = -\det \begin{pmatrix}
      B & A\\
      0 & -B
    \end{pmatrix} = -\det \left[\frac{\partial \hat{L}}{\partial v^j \partial v^i}\right]^2.
  \end{align*}
  As long as the determinant of $\left[\frac{\partial \hat{L}}{\partial v^j \partial v^i}\right]^2$ does not vanish the $2$-form considered will be non-degenerate. Since $\omega_L$ was computed by taking an exterior derivative of potential, it is exact forcing it to be closed and symplectic if $\left[\frac{\partial \hat{L}}{\partial v^j \partial v^i}\right]$ is regular. 
  \item[(d)] Now supposing $\left[\frac{\partial \hat{L}}{\partial v^j \partial v^i}\right]$ is regular, we have built a symplectic $1$-form $\omega_L$ on $\Omega_2(TQ)$. In order to use it to describe dynamics we need a Lagrangian vector field of which the integral curves are the trajectories of the set of particles on $Q$. This vector field is defined trough the energy function $E : (\mathbf{q}, \mathbf{v}) \mapsto (DL_{\mathbf{q}}(\mathbf{v}))(\mathbf{v}) - L(\mathbf{q}, \mathbf{v})$. To get this energy as a function of coordinate $\hat{q}, \hat{v}$ we use the coordinate function $q^{i}, v^{i}$ (regrouped in a chart map $\phi$ with $\phi^{-1}$ which return a pair $\phi^{-1}_{\mathbf{q}}(\hat{q}, \hat{v}) = \mathbf{q}$ and $\phi^{-1}_{\mathbf{v}}(\hat{q}, \hat{v}) = \mathbf{v}\in T_{\mathbf{q}}Q$) to write 
  \begin{align*}
    \hat{E}(\hat{q}, \hat{v}) = (E \circ \phi^{-1})(q^{i}(\mathbf{q}, \mathbf{v}), v^{i}(\mathbf{q}, \mathbf{v})) = \left(\frac{\partial \hat{L}}{\partial v^i} (\hat{q}, \hat{v}) \text{d}q^{i}_{\mathbf{q}}\right)(\mathbf{v}) - L\circ \phi^{-1} (q^{i}(\mathbf{q}), v^i(\mathbf{v})) =  \frac{\partial \hat{L}}{\partial v^i} (\hat{q}, \hat{v}) \hat{v}^i - \hat{L}(\hat{q}, \hat{v})
  \end{align*} 
  where we used $DL_{\mathbf{q}}(\mathbf{v}) = DL_{\phi^{-1}_{\mathbf{q}}} \phi^{-1}_{\mathbf{v}}\circ (\hat{q}, \hat{v}) = \frac{\partial \hat{L}}{\partial v^i} (\hat{q}, \hat{v}) \text{d}q^{i}_{\mathbf{q}}$ and applied it to $\mathbf{v}$. By definition, the action of $DL_{\mathbf{q}}(\mathbf{v})$ on $\mathbf{v}$ extracts the $v^i$ component of $\mathbf{v}$ in the coordinate basis. Strictly speaking, to properly precompose with $\phi^{-1}$, we should have considered $\text{d}q^{i}_{\phi_{\mathbf{q}^{-1}}} \circ \phi_\mathbf{v}^{-1}(\hat{q}, \hat{v})$  where $\text{d}q^{i}_{\phi_{\mathbf{q}^{-1}}}\circ \phi_\mathbf{v}^{-1} = \text{d}\hat{q}^i$ is the pullback by the coordinate chart on $Q$ of the $1$-form basis (indeed, we can interpret $\phi_v$ as a pushforward of vectors on $TQ \to T\mathbb{R}^n$ since it maps the tangent vector to a curve to the tangent vector of the image of the curve by $\phi_\mathbf{q}$ by construction).
  \item[(e)] From the energy function and symplectic form $\omega_L$, we can define the Lagrangian vector field $X_E$ (section over $TTQ$) by the relation $\omega_L(X_E, \bullet) = \text{d}E$. To use this definition, we work with the decomposition $X_E = X_E^{i}\left.\frac{\partial}{\partial x^i}\right|_{\mathbf{x}} = X_{E, q}^{i} \left.\frac{\partial}{\partial q^i}\right|_{\mathbf{q}, \mathbf{v}} + X_{E, v}^{i} \left.\frac{\partial}{\partial v^i}\right|_{\mathbf{q}, \mathbf{v}}$ in the coordinate basis of $TTQ$. We also evaluate the coordinate representation of the exterior derivative of $E$ to obtain 
  \begin{align*}
    \text{d}\hat{E} &= \frac{\partial^2 \hat{L}}{\partial v^i \partial q^j}(\hat{q}, \hat{v}) \hat{v}^i \text{d}\hat{q}^j + \frac{\partial^2 \hat{L}}{\partial v^i \partial v^j}(\hat{q}, \hat{v}) \hat{v}^i \text{d}\hat{v}^j + \frac{\partial \hat{L}}{\partial v^i}(\hat{q}, \hat{v}) \text{d}\hat{v}^i - \frac{\partial \hat{L}}{\partial v^j}(\hat{q}, \hat{v}) \text{d}\hat{v}^j - \frac{\partial \hat{L}}{\partial q^j}(\hat{q}, \hat{v}) \text{d}\hat{q}^j\\
    &= \frac{\partial^2 \hat{L}}{\partial v^i \partial q^j}(\hat{q}, \hat{v}) \hat{v}^i \text{d}\hat{q}^j + \frac{\partial^2 \hat{L}}{\partial v^i \partial v^j}(\hat{q}, \hat{v}) \hat{v}^i \text{d}\hat{v}^j - \frac{\partial \hat{L}}{\partial q^j}(\hat{q}, \hat{v}) \text{d}\hat{q}^j.
  \end{align*}
  We note that the form on $T^\star TQ$ is obteined from this coordinate representation by replacing the basis $\text{d}\hat{v}^j, \text{d}\hat{q}^j$ by its pullback $\text{d}q^{i}_{\mathbf{q}, \mathbf{v}}, \text{d}v^{i}_{\mathbf{q}, \mathbf{v}}$ to $T^\star TQ$ by the chart coordinate maps. 
  % goblin monkey 
  Applying our symplectic form to $X_E$ in its first entry, we find 
  \begin{align*}
    \omega_L(X_E, \bullet) &= \left(X_{E, q}^{k} \left.\frac{\partial}{\partial q^k}\right|_{\mathbf{q}, \mathbf{v}} + X_{E, v}^{k} \left.\frac{\partial}{\partial v^k}\right|_{\mathbf{q}, \mathbf{v}}\right)\left(-\frac{\partial^2 \hat{L}}{\partial v^j \partial v^i} (\hat{q}, \hat{v})  \text{d}v^{j}_{\mathbf{q}, \mathbf{v}} \wedge \text{d}q^{i}_{\mathbf{q}, \mathbf{v}} +\frac{1}{2}\left(\frac{\partial^2 \hat{L}}{\partial q^i \partial v^j} (\hat{q}, \hat{v}) - \frac{\partial^2 \hat{L}}{\partial q^j \partial v^i} (\hat{q}, \hat{v})\right)\text{d}q^{j}_{\mathbf{q}, \mathbf{v}} \wedge \text{d}q^{i}_{\mathbf{q}, \mathbf{v}}\right)\\
    &= -X_{E,v}^{i} \frac{\partial^2 \hat{L}}{\partial v^j \partial v^i} (\hat{q}, \hat{v}) \text{d}q^{j}_{\mathbf{q}, \mathbf{v}}+X_{E,q}^{i} \frac{\partial^2 \hat{L}}{\partial v^j \partial v^i} (\hat{q}, \hat{v}) \text{d}v^{j}_{\mathbf{q}, \mathbf{v}}\\
    &+ X_{E,q}^{j} \frac{1}{2}\left(\frac{\partial^2 \hat{L}}{\partial q^i \partial v^j} (\hat{q}, \hat{v}) - \frac{\partial^2 \hat{L}}{\partial q^j \partial v^i} (\hat{q}, \hat{v})\right) \text{d}q^{i}_{\mathbf{q}, \mathbf{v}} - X_{E,q}^{i} \frac{1}{2}\left(\frac{\partial \hat{L}}{\partial q^j \partial v^i} (\hat{q}, \hat{v}) - \frac{\partial^2 \hat{L}}{\partial q^i \partial v^j} (\hat{q}, \hat{v})\right) \text{d}q^{j}_{\mathbf{q}, \mathbf{v}}\quad \text{(reindex and cancel)} \\
    &= -X_{E,v}^{i} \frac{\partial^2 \hat{L}}{\partial v^j \partial v^i} (\hat{q}, \hat{v}) \text{d}q^{j}_{\mathbf{q}, \mathbf{v}}+X_{E,q}^{i} \frac{\partial^2 \hat{L}}{\partial v^j \partial v^i} (\hat{q}, \hat{v}) \text{d}v^{j}_{\mathbf{q}, \mathbf{v}}
  \end{align*}
  Comparing this result with the expression for $\text{d}E$, linear independence leads to the relations
  \begin{align*}
    &\frac{\partial^2 \hat{L}}{\partial v^i \partial v^j}(\hat{q}, \hat{v}) \hat{v}^i = X_{E,q}^{i} \frac{\partial^2 \hat{L}}{\partial v^j \partial v^i} (\hat{q}, \hat{v}) \implies X_{E,q}^{i} = \hat{v}_i \quad \text{$[]^{-1}$ exists because $\frac{\partial^2 \hat{L}}{\partial v^j \partial v^i} (\hat{q}, \hat{v})$ is regular} \\
    &-X_{E,v}^{i} \frac{\partial^2 \hat{L}}{\partial v^j \partial v^i} (\hat{q}, \hat{v})  = \hat{v}^i \frac{\partial^2 \hat{L}}{\partial q^i \partial v^j} (\hat{q}, \hat{v})  - \frac{\partial \hat{L}}{\partial q^j}(\hat{q}, \hat{v}) \quad \text{can solve with $[]^{-1}$ exists because $\frac{\partial^2 \hat{L}}{\partial v^j \partial v^i} (\hat{q}, \hat{v})$ is regular.} 
  \end{align*}
  \item[(f)] We now consider curve $\gamma : U \subset \mathbb{R} \to TQ$ given in the coordinate chart by $\hat{q}(t)$ and $\hat{v}(t)$. This curve represents a trajectory if it is an integral curve of $X_E$: composing the tangent vector to the curve $\frac{\text{d}\gamma}{\text{d}t}$ at the point $\gamma(t)$ with the coordinate functions should return the components of $X_E$ associated to the point. An integral curve has to satisfy  
  \begin{align*}
    &\frac{\text{d}\gamma}{\text{d}t} (q^{i}) = \frac{\text{d}}{\text{d}t} \hat{q}^{i}(t) = X_E(q^{i}) = X_{E, q}^i = \hat{v}^i(t)\\
    &\frac{\text{d}\gamma}{\text{d}t} (v^{i}) = \frac{\text{d}}{\text{d}t}\hat{v}^{i}(t) = X_E(v^{i}) = X_{E, v}^i \implies -\frac{\text{d}}{\text{d}t}\hat{v}^{i}(t)\left(\frac{\partial^2 \hat{L}}{\partial v^j \partial v^i} (\hat{q}, \hat{v})\right)= \left( \hat{v}^i \frac{\partial^2 \hat{L}}{\partial q^i \partial v^j} (\hat{q}, \hat{v}) - \frac{\partial \hat{L}}{\partial q^j}(\hat{q}, \hat{v})\right)
  \end{align*}
  The second equation can be cast in the usual form of the Euler-Lagrange equations with Leibniz's rule in the following way 
  \begin{align*}
    \frac{\partial \hat{L}}{\partial q^j}(\hat{q}, \hat{v}) =  \frac{\partial^2 \hat{L}}{\partial v^j \partial v^i} (\hat{q}, \hat{v}) \frac{\text{d}}{\text{d}t}\hat{v}^{i}(t) + \hat{v}^i \frac{\partial^2 \hat{L}}{\partial q^i \partial v^j} (\hat{q}, \hat{v}) = \frac{\text{d}}{\text{d}t} \frac{\partial \hat{L}}{\partial v^j}(\hat{q}, \hat{v}).
  \end{align*} 

\end{enumerate}


\section{Acknowledgement}
I worked on this assignment on my own.
% cite the notes for integral curves x


}

% References
\makereferences
%-------------------------------------------------------


%%%%%%%%%%%%%%%%%%%%%%%%
% Terminer le document %
%%%%%%%%%%%%%%%%%%%%%%%%
\end{document}
