\documentclass[10pt, a4paper]{article}

%%%%%%%%%%%%%%
%  Packages  %
%%%%%%%%%%%%%%


\usepackage{page_format}
\usepackage{special}
\usepackage{hyperref}
\usepackage{tikz}
\usepackage[compat=1.1.0]{tikz-feynman}
%----------------------------------------------------------------------
%\usepackage{amssymb} % Mathematical fonts.
%\usepackage{amsfonts} % Mathematical fonts.
\usepackage[nice]{nicefrac} % Nicer fractions
\usepackage{braket} % Dirac Notation.
\usepackage{bbm} % More bold fonts.
%\usepackage{mathrsfs} % Mathematical fonts.
\usepackage{esint} % Integrals
\usepackage{cancel} % Allows to scratch expressions.
\usepackage{mathtools} % Tools for math formating.
\usepackage{slashed} % Allows to slash individual characters.
\usepackage{xargs} % Better handling of optional arguments for commands
%----------------------------------------------------------------------
%\usepackage{lmodern} % Fonts.
\usepackage{feyn} % Feynman Diagrams in mathmode

%%%%%%%%%%%%%%%%%%%%%%%%%%%
% Mathématiques et physique
%%%%%%%%%%%%%%%%%%%%%%%%%%%%
% SI Units -----------------------
% The package 'siunitx' causes unresolved crashes (as of 22/08/31)
\newcommand{\ampere}{\text{A}}
\newcommand{\bell}{\text{B}}
\newcommand{\celsius}{\degree\text{C}}
\newcommand{\coulomb}{\text{C}}
\newcommand{\degree}{\,^{\circ}}
\newcommand{\farad}{\text{F}}
\newcommand{\electro}{\text{e}}
\newcommand{\gram}{\text{g}}
\newcommand{\henry}{\text{H}}
\newcommand{\hertz}{\text{Hz}}
\newcommand{\hour}{\text{h}}
\newcommand{\joule}{\text{J}}
\newcommand{\kelvin}{\text{K}}
\newcommand{\meter}{\text{m}}
\newcommand{\minute}{\text{m}}
\newcommand{\mole}{\text{mol}}
\newcommand{\newton}{\text{N}}
\newcommand{\ohm}{\Omega}
\newcommand{\pascal}{\text{Pa}}
\newcommand{\rad}{\text{rad}}
\newcommand{\second}{\text{s}}
\newcommand{\tesla}{\text{T}}
\newcommand{\torr}{\text{Torr}}
\newcommand{\volt}{\text{V}}
\newcommand{\watt}{\text{W}}
%
\newcommand{\tera}{\text{T}}
\newcommand{\giga}{\text{G}}
\newcommand{\mega}{~\text{M}}
\newcommand{\kilo}{~\text{k}}
\newcommand{\deci}{\text{d}}
\newcommand{\centi}{\text{c}}
\newcommand{\milli}{\text{m}}
\newcommand{\micro}{\mu}
\newcommand{\nano}{\text{n}}
\newcommand{\pico}{\text{p}}
\newcommand{\femto}{\text{f}}
%
\newcommand{\units}[1]{\text{#1}}
\newcommand{\tothe}[1]{\textsuperscript{#1}}
%
\newcommand{\per}{\text{/}}
%
\newcommand{\Time}[3]{#1\hour~#2\minute~#3\second} % TODO Optional arguments.
\newcommand{\Angle}[3]{#1^{\circ}~#2'~#3''} % TODO Optional arguments.


% Better epsilon -----------------------
\let\oldepsilon\epsilon
\let\epsilon\varepsilon
\let\varepsilon\oldepsilon


% Better \bar -----------------------
\renewcommand{\bar}[1]{\mkern 1.5mu\overline{\mkern-1.5mu#1\mkern-1.5mu}\mkern 1.5mu}


% Équations -----------------------
\newcommand{\al}[1]{\begin{align} #1 \end{align}} % Numbered equation(s),
\newcommand{\eqn}[1]{\begin{align*} #1 \end{align*}} % Number-less equation(s),
\newcommand{\sys}[1]{\begin{dcases*} #1 \end{dcases*}} % System of equations.


% Exponents -----------------------
\newcommand{\Exp}[1]{\text{e}^{#1}}		% e^#
\newcommand{\E}[1]{\times 10^{#1}}		% X 10^#


% Delimiters -----------------------
\newcommand{\p}[1]{\left( #1 \right)}	% (#)
\newcommand{\cro}[1]{\left[ #1 \right]}	% [#]
\newcommand{\abs}[1]{\left| #1\right|}	% |#|
\newcommand{\avg}[1]{\left\langle #1 \right\rangle} % <#>
\newcommand{\acc}[1]{\left\lbrace #1 \right\rbrace} % {#}


% Vectors -----------------------
\newcommand{\ve}[1]{\mathbf{#1}} % Upright bold face.
\newcommand{\vu}[1]{\hat{\ve{#1}}} % Hat vector upright bold face
\newcommand{\tens}{\otimes} % Tensor product
\newcommand{\nablav}{\bm{\nabla}} % Bold gradient


% Trig. functions with automatic formating  -----------------------
\newcommandx{\Sin}[2][1={}]{\text{sin}^{#1}\!\p{#2}}
\newcommandx{\Cos}[2][1={}]{\text{cos}^{#1}\!\p{#2}}
\newcommandx{\Tan}[2][1={}]{\text{tan}^{#1}\!\p{#2}}
\newcommandx{\Csc}[2][1={}]{\text{csc}^{#1}\!\p{#2}}
\newcommandx{\Sec}[2][1={}]{\text{sec}^{#1}\!\p{#2}}
\newcommandx{\Cot}[2][1={}]{\text{cot}^{#1}\!\p{#2}}
\newcommandx{\Arcsin}[2][1={}]{\text{arcsin}^{#1}\!\p{#2}}
\newcommandx{\Arccos}[2][1={}]{\text{arccos}^{#1}\!\p{#2}}
\newcommandx{\Arctan}[2][1={}]{\text{arctan}^{#1}\!\p{#2}}
\newcommandx{\Sinh}[2][1={}]{\text{sinh}^{#1}\!\p{#2}}
\newcommandx{\Cosh}[2][1={}]{\text{cosh}^{#1}\!\p{#2}}
\newcommandx{\Tanh}[2][1={}]{\text{tanh}^{#1}\!\p{#2}}


% Matrices -----------------------
\newcommand{\mat}[1]{\begin{bmatrix} #1 \end{bmatrix}} % Matrices with hooks.
\newcommand{\pmat}[1]{\begin{pmatrix} #1 \end{pmatrix}} % Matrices with parentheses.
\newcommand{\deter}[1]{\abs{\begin{matrix} #1 \end{matrix}}} % Determinant.
\newcommandx{\mO}[2][1={}, 2={}]{ \def\temp{#2}\ifx\temp\empty\ve{O}_{#1}\else\ve{O}_{#1\times #2}\fi}% Zero matrix.
\newcommandx{\mI}[2][1={}, 2={}]{ \def\temp{#2}\ifx\temp\empty\ve{I}_{#1}\else\ve{O}_{#1\times #2}\fi}%  Identity matrix.
\newcommand{\Det}[1]{\text{det}\p{#1}} % det(#)
\newcommand{\Tr}[1]{\text{Tr}\p{#1}} % Tr(#)


% Derivatives -----------------------
\newcommand{\D}{\text{d}} % Differential 'd'.
\newcommandx{\dd}[3][1={},3={}]{\frac{\D^{#3}#1}{\D{#2}^{#3}}} % Total derivative according to #2, #1 is the function and #3 is the order.
\newcommand{\del}{\partial} % Partial 'd'.
\newcommandx{\ddp}[3][1={},3={}]{\frac{\del^{#3}#1}{\del{#2}^{#3}}} % Dérivée partielle selon #2, #1 est la fonction est #3 est l'ordre.
\newcommand{\eval}[1]{\left. {#1} \right|} % Bar on the right of expression.
\newcommand{\delbar}{\slashed{\del}} % Partial Inexact differential.
\newcommand{\dbar}{\dj}% Inexact differential.


% Integrals -----------------------
\newcommand{\intinf}{\int\displaylimits_{-\infty}^{\infty}} % From -00 to 00.
\newcommandx{\Int}[2][1={},2={}]{\int\displaylimits_{#1}^{#2}} % Faster bounded integrals.


% Complex numbers -----------------------
\renewcommand{\Re}[1]{\text{Re}\acc{#1}} % Re{#}
\renewcommand{\Im}[1]{\text{Im}\acc{#1}} % Im{#}


% Sets -----------------------
\newcommand{\N}{\mathbbm{N}} % Natural numbers.
\newcommand{\Z}{\mathbbm{Z}} % Integers.
\newcommand{\Q}{\mathbbm{Q}} % Rational numbers.
\newcommandx{\R}[1][1={}]{\mathbbm{R}^{#1}} % Real numbers.
\newcommandx{\C}[1][1={}]{\mathbbm{C}^{#1}} % Complex numbers.
\newcommandx{\F}[1][1={}]{\mathbbm{F}^{#1}} % Some field.
\newcommand{\M}[3]{\mathbb{M}_{#1\times#2}(#3)}	% Matrices.
\newcommand{\Po}[2]{\mathbb{P}_{#1}(#2)} % Polynomials.
\newcommand{\Lin}{\mathbb{L}} % Linear maps.


% Constants and physical symbols -----------------------
\newcommand{\eo}{\epsilon_0} % epsilon 0.
\renewcommand{\L}{\mathcal{L}} % Lagrangian.

\usepackage{slashed}

% References
\usepackage{biblatex}
\addbibresource{ref.bib}


%%%%%%%%%%%%
%  Colors  %
%%%%%%%%%%%%
% ! EDIT HERE !
\colorlet{chaptercolor}{red!70!black} % Foreground color.
\colorlet{chaptercolorback}{red!10!white} % Background color

%%%%%%%%%%%%%%
% Page titre %
%%%%%%%%%%%%%%%
\title{Homework 2: Monopoles} % Title of the assignement.
\author{\PA} % Your name(s).
\teacher{Ruth Gregory} % Your teacher's name.
\class{Gravitational Physics} % The class title.

\university{Perimeter Institute for Theoretical Physics} % University
\faculty{Perimeter Scholars International} % Faculty 
%\departement{<Departement>} % Departement
\date{\today} % Date.


%%%%%%%%%%%%%%%%%%%%%%
% Begin the document %
%%%%%%%%%%%%%%%%%%%%%%
\begin{document}

% Make the title page.
\maketitlepage

% Make table of contents
\maketableofcontents

% Assignment starts here ----------------------------

\footnotesize{

\section{Dirac}

\begin{enumerate}

  \item[(a)] We are interested in the relation between the global properties of a manifold $M$ and the structure of diffrential forms taking values on its cotangent bundle $T^\star M$ at each point of $M$.

  \textbf{Poincaré's lemma on $M = \mathbb{R}$:} Let $\omega$ be a $p$-form ($p \in \{0, 1\}$) constructed from the cotengent space $T^\star M$ of $M$. Then $\text{d}\omega = 0$ ($\omega$ is closed) implies $\omega = \text{d}\lambda$ ($\omega$ is closed) where $\lambda$ is a $(p-1)$-form ($0$-form).

  \textbf{Proof:} On $\mathbb{R}$, we can use the identity map as a global coordinate chart. The induced basis on 1-forms is $\{\text{d}x\}$ (a smooth frame field) and any 1-forms can be written as $\omega = g \text{d}x$ with $ g \in C^{\infty}(\mathbb{R})$. Suppose now that $\omega$ is closed: we have $0 = \text{d}\omega = \partial_x g \text{d}x \wedge \text{d}x = 0, \forall g \in C^{\infty}(\mathbb{R})$ ($\omega$ being a 1-form is not restrictive, but would be for $\mathbb{R}^n$ with $n>1$). Then we take the $0$-form $\lambda = G$ where $G$ is any primitive of $g$ ($G(x)$ exists because $g$ is smooth) and apply an exterior derivative to get $\text{d}\lambda = g \text{d}x$. Because there are no $(0-1)$-forms there is no need to check the lemma for $0$-forms. 

  \textbf{Counterexample:} Consider the circle smooth manifold $\mathbb{S}^1 \subset \mathbb{R}^2$ (embeded as $\{x^2 + y^2 = 1|(x, y)\in\mathbb{R}^2\}$ for simplicity). It takes at least two charts to cover this manifold and, although on individual charts all closed 1-forms are exact (charts make the manifold look like $\mathbb{R}$ locally), this property is lost globally. Choose the chart map $\theta = \arctan_2$ sending points $(x, y)$ on the circle to their angle with the $x$ axis excluding the point $(1, 0)$ so that the domain is open. With this chart we have the coordinate induced one form frame field $\text{d}\theta$ which we use to construct the closed form $\omega = \text{d}\theta$. On $(0, 2\pi)$, this form is exact since we have a $0$-form $\lambda =  F \in C^{\infty}((0, 2\pi))$ such that $\omega = \text{d}\lambda = \partial_\theta F \text{d}\theta = \text{d}\theta$ forcing $F = \theta + c$, $c \in \mathbb{R}$ since $F$ has to be a primitive of $1$ in the variable $\theta$. The function $F$ is smooth on the chart, but can never be extended to s smooth function over $\mathbb{S}^1$ globally. Indeed, $0$ and $2\pi$ being identified, a continuous function on $\mathbb{S}^1$ should be consistant at the excluded point $(0, 1)$ and this would requiere  $\lim_{\theta \to 0+} (\theta + c) = \lim_{\theta \to 2\pi^-} (\theta + c)$ which is impossible. Therefore there is a closed form on $\mathbb{S}^1$ that is not exact. 
  \item[(b)] Let $F^{(2)}$ be a 2-form on the 2-sphere $\mathbb{S}^2$. Suppose $F^{(2)}$ is globally exact implying there is a 1-form $\omega$ such that $F^{(2)} = \text{d}\omega$. Then we can use Stokes theorem in combination with the fact $\mathbb{S}^2$ has no boundary to write $g= \frac{1}{4\pi} \int_{\mathbb{S}^2} F^{(2)} = \frac{1}{4\pi} \int_{\partial \mathbb{S}^2} \text{d}\omega = 0$. 
  \item[(c)] Now working in Minkowski space $\{\eta,\ \mathbb{R}^{1, 3}\}$ with mostly $+$ signature in the coordinate chart $(t, r, \theta, \phi)$ (this order for the variables provides the notion of positive orientation of a basis) built from spherical coordinates on $\mathbb{R}^3$, we have the 2-form $F^{(4)} = Q \sin(\theta)\ \text{d}\theta \wedge \text{d}\phi$ with $Q\in \mathbb{R}$. We want to determine if $F^{(4)}$ satisfies Maxwell's equations $\text{d} F^{(4)}=0, \quad \text{d} \star F^{(4)}=0$. We have $\text{d} F^{(4)} =  Q \cos(\theta) \ \text{d}\theta \wedge \text{d}\theta \wedge \text{d}\phi = 0$ (This is true at everypoint of the spherical coordinate chart, but fails at the singularity of the monopole located at the origin). To evaluate the Hodge dual of $F^{(4)}$, we first calculate 
  \begin{align*}
  \star\ \text{d}\theta \wedge \text{d}\phi &= \sqrt{|r^4 \sin^2 \theta|}\frac{1}{2!}\frac{1}{2!}\varepsilon^{\theta \phi}{}_{r t} \text{d} r \wedge \text{d} t + \frac{1}{2!}\frac{1}{2!}\varepsilon^{\theta \phi}{}_{t r} \text{d} t \wedge \text{d} r - \frac{1}{2!}\frac{1}{2!}\varepsilon^{\phi \theta}{}_{r t} \text{d} r \wedge \text{d} t - \frac{1}{2!}\frac{1}{2!}\varepsilon^{\phi \theta}{}_{t r} \text{d} t \wedge \text{d} r \\
  &= r^2 |\sin \theta|\eta^{\theta \theta}\eta^{\phi \phi}\left(\frac{1}{2!}\frac{1}{2!}\varepsilon_{\theta \phi r t} \text{d} r \wedge \text{d} t + \frac{1}{2!}\frac{1}{2!}\varepsilon_{\theta \phi t r} \text{d} t \wedge \text{d} r - \frac{1}{2!}\frac{1}{2!}\varepsilon_{\phi \theta r t} \text{d} r \wedge \text{d} t - \frac{1}{2!}\frac{1}{2!}\varepsilon_{\phi \theta t r} \text{d} t \wedge \text{d} r\right)\\
  &= r^2 |\sin \theta|\frac{1}{r^4 \sin^2 \theta}\frac{1}{2!}\frac{1}{2!}\left((-1)\ \text{d} r \wedge \text{d} t + (+1)\ \text{d} t \wedge \text{d} r - (+1)\ \text{d} r \wedge \text{d} t - (-1)\ \text{d} t \wedge \text{d} r\right) = \text{d} t \wedge \text{d} r
  \end{align*}
  and it follows that $\text{d} \star F^{(4)} = \text{d} (Q/r^2\ \left(\text{d} t \wedge \text{d} r\right)) =  -Q/r^3\ \left( \text{d} r \wedge \text{d} t \wedge \text{d} r\right) = 0$ where the absolute value was ignored because $\theta \in (0, 2\pi)$ making $\sin(\theta) > 0$. 
  \item[(d)] We can convert the form $F^{(4)}$ to cartesian coordinates with the relations 
  \begin{align*}
      &\phi = \arctan_2\left(y, x\right),\quad
      \theta = \arctan_2\left(z, \sqrt{x^2 + y^2}\right)
      \implies
    \text{d}\phi = \dfrac{-y \text{d}x + x\text{d}y}{x^2 + y^2}, \quad 
    \text{d}\theta = \dfrac{\sqrt{x^2 + y^2}\text{d}z - (x\text{d}x + y\text{d}y)\frac{z}{\sqrt{x^2 + y^2}}}{r^2}
  \end{align*} 
  leading to 
  \begin{align*}
    F^{(4)} &= Q \sin(\theta)\ \text{d}\theta \wedge \text{d}\phi = Q \frac{\sqrt{x^2+y^2}}{r}\ \left(\dfrac{\sqrt{x^2 + y^2}\text{d}z - (x\text{d}x + y\text{d}y)\frac{z}{\sqrt{x^2 + y^2}}}{r^2}\right) \wedge \left(\dfrac{-y \text{d}x + x\text{d}y}{x^2 + y^2}\right)\\
    &= Q \frac{1}{r^3}\ \left(\text{d}z \wedge (-y \text{d}x + x\text{d}y) - (x^2\text{d}x \wedge \text{d} y - y^2\text{d}y \wedge \text{d}x)\frac{z}{x^2 + y^2}\right) = Q \frac{1}{r^3}\ \left(-y\text{d}z \wedge \text{d}x - x\text{d}y \wedge \text{d}z - z\text{d}x \wedge \text{d} y\right). 
  \end{align*}
  We note the electric field components (associated to $\text{d}x^{i} \wedge \text{d}t$) all vanish and we only have a magnetic field (associated to $\text{d}x^i \wedge \text{d}x^j$). The magnetic field has the same from has an electric monopole (inverse square law multiplies by a unit "vector").
  \item[(e)] Since the monopole field is static, we drop the time direction by mapping $F^{(4)}$ to the two-form $F^{(3)}$ in the cotangent bundle over $\mathbb{R}^3$ on a fixed time slice. Going further we can map $F^{(3)}$ on the cotangent bundle over $\mathbb{S}^2$ (embeded in $\mathbb{R}^3$ as a sphere of radius $1$) to get the two-form $F^{(2)}$. To characterize the two-form $F^{(2)}$, we evaluate the integral given in (b) as 
  \begin{align*}
  g = \frac{Q}{4\pi}\int_{\mathbb{S}^2} \sin(\theta) \ \text{d} \theta \wedge \text{d}\phi = \frac{Q}{4\pi}\int_{\mathbb{S}^2} \sin(\theta) \ \text{d} \theta (e_\theta) \wedge \text{d}\phi (e_\phi) = Q
  \end{align*}
  with $e_{\phi}, e_\theta$ the dual vector basis to $\text{d} \phi, \text{d} \theta$. More formally, this integration on $V\subset\mathbb{S}^2$ is brought to an integral in $U\subset\mathbb{R}^2$ on the pullback of $F^{(2)}$ by a diffeomorphism mapping $U$ to $V$. A convenient choice of diffeomorphism is the coordinate chart already used to write $F^{(2)}$. Under this diffeomrophism, $\text{d}\theta$ and $\text{d}\phi$ are mapped to the exterior derivatives of the coordinate fucntions $\theta, \phi$ over $U$ (the exterior derivative of the projection map on each axis which are aslo named $\text{d}\theta$ and $\text{d}\phi$). This allows us to use regular borns of integration where $\theta$ and $\phi$ range from $0$ to $\pi$ and $0$ to $2\pi$ respectively and use the coordinate representation of the two-form components. Since \textbf{exact} $\implies$ \textbf{vanishing of $g$} as shown in (b), we have \textbf{non vanishing of $g$} $\implies$ \textbf{not exact} and $F^{(2)}$ is not exact. One could say that $g= \frac{1}{4\pi} \int_{\mathbb{S}^2 = \partial \text{Ball}} F^{(2)} = \frac{1}{4\pi} \int_{\text{Ball}} \text{d}F^{(2)} = 0$ forming a contradiction with $F^{(2)}$ not being exact. The solution to this problem can be seen with result (c) where $F^{(2)}$ is shown to be ill-defined at the origin. Therfore we need to puncture $\mathbb{R}^3$ by removing the origin from the domain of definition of $F^{(3)}$ creating a second boundary restoring the result $0 = \frac{1}{4\pi} \int_{\partial \text{Ball} + \text{puncture}} F^{(2)}$. Normally the set added to the boundary would be of zero measure, but comparing with the usual treatement of electric monopoles, we get that a dirac delta at the puncture point will change the value of $g$ from $0$ to $Q$. 
  \item[(f)] Stereographic projections provide maps from $U_{+} = \mathbb{S}^2-\text{North pole}$ (projecting from the north pole) and $U_{-} = \mathbb{S}^2-\text{South pole}$ (projecting to the south pole) to all of $\mathbb{R}^2$. Expressed in the cartesian coordinates of the embeding space of $\mathbb{S}^2$ in $\mathbb{R}^3$, the associated coordinate maps $\varphi_{\pm}$ are 
  \begin{align*}
    \varphi_\pm : (x, y, z) \mapsto (u_\pm, v_\pm) = \left(\frac{x}{1\mp z}, \frac{y}{1\mp z}\right).
  \end{align*} 
  This expression can be obtained by looking at a cut of the sphere in a $zw$-plane containing the $z$ axis. In this plane, we look for the intersection $u_\pm, v_\pm$ of a line passing trough the relevant pole and the point $x, y, z$ with the $xy$-plane. In the section plane, the line is given by points of coordinates $w_l, z_l$ such that $z_l = 1 - \frac{1+z}{w}w_l$ (North pole) or $z_l = -1 + \frac{1-z}{w} w_l$ (South pole). The intersection with the $xy$-plane is given by $u_\pm =\frac{w}{1\mp z} \frac{x}{w}$ and $v_\pm = \frac{w}{1\mp z} \frac{y}{w}$ ($w$ coordinate projected on the $x$ and $y$ axis respectively).

  A candidate for a potential $A$ such that $F^{(2)} = \text{d}A$ is $A = -\cos(\theta)\text{d}\phi$. On our unit sphere $z = \cos(\theta)$ and in the stereographic projection charts $x = u (1\mp z)$, $y = v (1\mp z)$ (omiting the sign subscript on $u, v$) and 
  \begin{align*}
    &u_\pm^2 + v_\pm^2 = (1-z^2)/(1\mp z)^2 = (1 \pm z)/(1 \mp z) \implies u_\pm^2 + v_\pm^2 \mp z(u_\pm^2 + v_\pm^2)  = 1 \pm z \implies  z_\pm = \pm \frac{1 - u_\pm^2 - v_\pm^2}{1 + u_\pm^2 + v_\pm^2}\\
    &\phi = \arctan_2(y, x) = \arctan_2(v_\pm, u_\pm) \implies \text{d}\phi = -\frac{v_\pm}{u_\pm^2+v_\pm^2}\text{d}u_\pm + \frac{u_\pm}{u^2_\pm+v^2_\pm}\text{d}v_\pm
  \end{align*}
  The expression for the potential in these charts is 
  \begin{align*}
    A_\pm = \mp Q\frac{1 - u_\pm^2 - v_\pm^2}{1 + u_\pm^2 + v_\pm^2} \left(-\frac{v_\pm}{u_\pm^2+v_\pm^2}\text{d}u_\pm + \frac{u_\pm}{u_\pm^2+v_\pm^2}\text{d}v_{\pm} \right). 
  \end{align*}
  We notice that is is ill defined at $u_\pm, v_\pm = 0$ and add the exact one-form $\pm Q\text{d}\phi$ (This form is only exact on $U_{+} \cap U_{-}$ as it derives from the angle function $\phi$ fails to be smooth at both poles of the sphere. Adding this form to the initial $A_\pm$ is a bit like adding two functions that are not well defined at the same point and then realising that the result can be analyticaly continuated to a smooth function. In our case the combination of forms has a smooth extension to one pole) to it (without affecting $F^{(2)}$) to get 
  \begin{align*}
    A_\pm &= \left(\pm Q \mp Q  \frac{1 - u_\pm^2 - v_\pm^2}{1 + u_\pm^2 + v_\pm^2}\right) \left(-\frac{v_\pm}{u_\pm^2+v_\pm^2}\text{d}u_\pm + \frac{u_\pm}{u_\pm^2+v_\pm^2}\text{d}v_{\pm}\right)\\
    &=  
    \pm Q\left(- 2 \frac{u_\pm^2 + v_\pm^2}{1 + u_\pm^2 + v_\pm^2}\right) \left(-\frac{v_\pm}{u_\pm^2+v_\pm^2}\text{d}u_\pm + \frac{u_\pm}{u_\pm^2+v_\pm^2}\text{d}v_{\pm} \right) = \mp 2Q\frac{1}{1 + u_\pm^2 + v_\pm^2}\left(-v_\pm\text{d}u_\pm + u_\pm\text{d}v_{\pm} \right) 
  \end{align*}
  which is well defined at the pole of interest (mapped to $u, v = 0, 0$) and \textit{provides and extension of $A$} defined everywhere on the $\mathbb{R}^2$ plane. Before adding $\pm Q\text{d}\phi$, $A_{\pm}$ as an invariant object was independant of $\pm$ and the symbolic difference were only caused by the difference of the charts used to express $A_\pm$. The added one-form depends on the chart and makes $A$ well behaved on different manifolds: The sphere with a puncture north and the sphere with a puncture south. What about the sphere with both punctures? If we want to define gauge transfomations of $A$ (see (g)) to extract physical observables as gauge invariant effects, we need to work on a manifold where both $A_+$ and $A_-$ are defined simultaneously : we work on a doubly punctured sphere. When the full dimensionnality of euclidean space is restored the puncture (at the unregulated pole) on concentric spheres of different radius becomes a one-dimensionnal "puncture" which constitute the dirac string. To bring $A^{(2)}$ on $\mathbb{S}^2$ back to $\mathbb{R}^3$, we need  to invert the pullback of the embeding map $\mathbb{R}^3 \hookrightarrow \mathbb{S}^2$ applied to bring forms on $\mathbb{R}^3$ to $\mathbb{S}^2$. Since this pullback has $\text{d}r \mapsto 0$ and preserves the angular forms, we have $A^{(3)} = \pm Q\text{d}\phi + [\cdots]\text{d}r$. However we also have $\text{d}A^{(3)} = F^{(3)} \propto \text{d}\phi \wedge \text{d}\theta$ which force the coefficient of $\text{d}r$ to depend on $r$ only so that the exterior of the assciated term vanishes. For simplicity, we set $[\cdots] = 0$. FInally, we notice that while the singularity of $F^{(3)}$ is of dimension $0$ and is located at $r=0$ (see (d)), the singularity of $A^{(3)}$ is one dimensionnal and located on a half of the $z$ axis extending from the $F$ singularity in the direction of unregulated poles. 
  
  \item[(g)] On $\mathbb{S}^2-\{\text{North, South pole}\}$ ($U_{+} \cap U_{-}$), the two realisations of $A$ can be compared and we have $A_- - Q\text{d}\phi = A_+ + Q\text{d}\phi$. This relation suggests that $A_-$ and $A_+$ are connected by a gauge transformation of the general form $A_- = A_+ + 1/(ie) \gamma^{-1} \text{d}\gamma$ for a $U(1)$-valued $\gamma : \mathbb{S}^2 \to U(1)$ complex field with electric charge $e$. Setting this gauge transformation term equal to the $A_--A_+ = 2Q\text{d}\phi$, we get the differential equation set 
  \begin{align*}
    \frac{1}{ie}\gamma^{-1} \text{d}\gamma = (ie\gamma^{-1}\partial_r \gamma \text{d}r + ie\gamma^{-1}\partial_\phi \gamma \text{d}\phi + ie\gamma^{-1}\partial_\theta \gamma \text{d}\theta)  = 2 Q \text{d}\phi \iff ie\gamma^{-1}\partial_r \gamma = 0, \quad ie\gamma^{-1}\partial_\theta \gamma = 0, \quad ie\gamma^{-1}\partial_\phi \gamma = 2Q. 
  \end{align*}
  The first two differential equation allow us to take $\gamma$ independant of $r, \theta$. Then, the last equation is solved by $C e^{i Qe\phi}$ where $C\in U(1)$ and taken to be $0$. In the end, we found a gauge transformation function $\gamma$ linking the two realisations of $A$. For this function to be smooth on $\mathbb{R}^3-\text{String}$, we need to make sure that its value at $\phi \to 0^+$ and at $\phi \to 2\pi^-$ agree. This corresponds to the condition $e^{0} = e^{i Qe(2\pi)}\implies (2\pi) = Qe (2\pi n),\ n \in \mathbb{N} \implies Qe = n$.  

  \item[(h)] Knowing that $Qe = n$, we go back to the gauge transformation function $\gamma = e^{i Qe \phi} =  e^{i n \phi}$. Assuming the gauge group is $(\mathbb{R}, +)$ would lead to a transformation $\text{d}\lambda$ generated by a smooth function $\lambda$ over $U_{+} \cap U_{-}$. From the above calculations $\lambda = Qe\phi + c$ and is not a smooth function because of the jump from $\phi \to 0^+$ to $\phi \to 2\pi^-$ associated to arbitrarely close points on the manifold. However $\gamma$ can be extended smoothly smooth on $U_{+} \cap U_{-}$ as a $U(1)$ valued function since the complex exponential smoothly identifies the two points values in the jump. We note that the gauge term should be closed (to ensure gauge invariance), but not exact since it originates from the extension of a dicontinuous function on the manifold (analogous to the counter example given in (a), $U_{+} \cap U_{-}$ is homeomorphic to the cylinder and Poincaré's lemma does not apply). 
  
  \item[(i)] From the resul found in (d), we see that $F$ is rotationnaly invariant and takes the same form in all spherical coordinate system (with arbitrary $z$ axis). Suppose we take the $y$ axis as our new $z$ axis. The new angle coordinate function $\phi'$ (set to $0$ on the $x$ axis) is given in the initial coordinates by  $\phi' = \arctan_2(z, x) = \arctan_2(r \sin(\theta)\cos(\phi), r \cos(\theta)) = \arctan_2(\sin(\theta)\cos(\phi), \cos(\theta))$. We can use function to generate a smooth $U(1)$ gauge transformation $\sim e^{i Qe \phi'}$ wich will end up adding $\pm Q\text{d}\phi'$ to the singular $A$ (at both poles) found in (f). This singular $A$ was constructed from $F^(2)$ and takes the same form $A = -\cos\theta^{(')}\text{d}\phi^{(')}$ in all spherical coordinate systems. Starting with $A_\pm$ in a given spherical coordinate system we can get $A_{\pm}'$ in a different coordinate system by first removing the added $\pm Q\text{d}\phi$ and then adding the new $\pm Q \text{d}\phi'$. The total transformation looks like $A_\pm' = A_\pm \mp Q\text{d}\phi \pm Q\text{d}\phi' = \pm Q\text{d}(\phi'-\phi)$. Written like this, the transformation looks like a gauge transformation associated to the gauge function $\gamma = e^{i Qe (\phi'-\phi)} = e^{i Qe (\phi')} e^{i Qe (-\phi)}$ (the total transformation belongs to single valued $U(1)$ functions because it is the local group multiplication of two such functions). We note that at each step of the procedure, the domain of maximal extension of $A$ on $+$ and $-$ charts is different because the poles are different. This gauge transformation effectively rotates the Dirac string from the $z$ axis to the $y$ axis. Question: can a $r$ dependant modify the shape of the string in an arbitrary way?
\end{enumerate}

\section{Taub-NUT, or the gravitomagnetic monopole}
\begin{enumerate}
  \item[(a)] Since linearized gravity behaves in similar ways to electromagnetism, we are interested in gravitomagnetic monopoles. In what follows, we consider the vacuum solution to Einstein's equations given by the metric
  \begin{align*}
    \text{d} s^2=f\left(\text{d} t+2 A_\sigma\right)^2-\frac{\text{d} r^2}{f}-\left(r^2+n^2\right)\left(\text{d} \theta^2+\sin ^2 \theta \text{d} \phi^2\right), \quad f=\frac{r^2-2 m r-n^2}{r^2+n^2}, \quad A_\sigma=n(\cos \theta+\sigma) \text{d} \phi
  \end{align*}
  where $A_{\sigma}$ is the gravitomagnetic potential, $n$ is the monopole charge, $m$ is a mass parameter and $\sigma$ is an aditionnal parameter associated to coordinate singularities. We connect $m$ to a mass interpretation more directly by setting $n=0$ to get 
  \begin{align*}
    f=\frac{r^2-2 m r}{r^2} = 1-\frac{2m}{r}, \quad A_\sigma=0, \quad \text{d} s^2= \left(1-\frac{2m}{r}\right) d t^2-\left(1-\frac{2m}{r}\right)^{-1}\text{d} r^2 - r^2\left(\text{d} \theta^2+\sin ^2 \theta \text{d} \phi^2\right)
  \end{align*}
  which is the Schwartzschild solution with mass $m$.
  \item[(b)] To study the effect of $n$ and compare it to the effect of the magnetic monopole charge, we restrict out analysis to surfaces of fixed $r, t$. Pulling back the metric on one of these surfaces yields 
  \begin{align*}
    \text{d} s^2=4 f n^2 (\cos \theta+\sigma)^2-\left(r^2+n^2\right)\left(\text{d} \theta^2+\sin ^2 \theta \text{d} \phi^2\right).
  \end{align*}
  Like before the $\text{d}\phi$ circles around the poles and this circle behavior makes the value of the gravitomagnetic potential ill defined there. To fix the problem, we can set $\sigma = \pm 1$ ($+1$ for the north pole and $-1$ for the south pole) to make the coefficient of $\text{d}\phi$ vanish at one of the poles while keeping the behavior ill defined at the other pole. Restoring the $r$ direction, the accumulation of the singularities at the unregulated poles produces a Misner string singularity on half of the $z$ axis. For $\sigma = 0$ both poles have ill defined behavior and the string extends over the entire $z$ axis. We note that restoring the $t$ direction makes the string look like a two-dimensionnal singularity originating on the worldline of the gravitomagnetic monopole located at the origin of our spherical coordinate system. 
  \item[(c)] We consider a change of coordinate $t \to t_\sigma = t + 2n \sigma \phi$ which leads the metric to become
  \begin{align*}
    \text{d} s^2 &= f\left(\text{d} t + 2 n\cos \theta \text{d}\phi + 2n\sigma \text{d}\phi\right)^2-\frac{\text{d} r^2}{f}-\left(r^2+n^2\right)\left(\text{d} \theta^2+\sin ^2 \theta \text{d} \phi^2\right)\\ &=  f\left(\text{d} t_\sigma - 2n \sigma \text{d}\phi + 2 n\cos \theta \text{d}\phi + 2n\sigma \text{d}\phi \right)^2-\frac{\text{d} r^2}{f}-\left(r^2+n^2\right)\left(\text{d} \theta^2+\sin ^2 \theta \text{d} \phi^2\right)\\
    &= f\left(\text{d} t_\sigma + 2 n\cos \theta \text{d}\phi\right)^2-\frac{\text{d} r^2}{f}-\left(r^2+n^2\right)\left(\text{d} \theta^2+\sin ^2 \theta \text{d} \phi^2\right). 
  \end{align*}
  This transformation links metric with different value of $\sigma$ and can be interpreted as a gauge transformation analogous to the gauge transformation treated in the Dirac string analysis. Going from the $\sigma = +1$ chart to the $\sigma = -1$ chart can be done using the time shift $t_{-1} \to t_{+1} = t_{-1} + 4n \phi$. The new time depending on angular coordinates only changes smoothly around $\phi = 0^+$ and $\phi = 2\pi^-$ if we identify times seperated by $4n (2\pi)$. This transformation leads to a shift in the position of the string from the upper half of the $z$ axis to the lower half of the $z$ axis. Since it is only a coordinate (gauge) transformation, we conclude that the dirac string should be unobservable. Since the time coordinate is forced to be periodic by the angular shift making the dirac string unobservable, the associated spacetime has one extra compactified time dimension. This also hapened with the Kaluza-Klein spacetime where one extra dimension is compactified leading to the emergence of electromagnetism on regular $3+1$ dimensionnal spacetime. The difference with this spacetime and the monopole spacetime treated here is the fact the non-compact manifold has a remaining time-like direction. Our manifold only has non compact spacial directions which removes dynamics. Since the magnetic monopole described above is described by a static field, we can expect that the structure of the emergent electromagnetic theory $A_{\pm} = n(\pm 1 +\cos(\theta))\text{d}\phi$ (the gravitomagnetic potential becomes the electromagnetic potential of the emergent theory) matches the structure of a normal magnetic monopole in $\mathbb{R}^3$ up to the space curvature effects of general ralativity.  
  \item[(d)] We finally calculate the Komar angular momentum of the spacetime. It involves the killing vector $k = \partial_{\phi}$ and its corresponding form 
  \begin{align*}
  k^{\flat} &= g(\partial_\phi, \bullet) = f\left(\text{d} t^2(\partial_\phi)+4 n(\pm 1 + \cos(\theta))\text{d}\phi(\partial_\phi) \text{d}t  +4n^2(\pm 1 + \cos(\theta))^2\text{d}\phi^2(\partial_\phi)\right)-\frac{\text{d} r^2}{f}-\left(r^2+n^2\right)\left(\text{d} \theta^2(\partial_\phi)+\sin ^2 \theta \text{d} \phi^2(\partial_\phi)\right)\\
  &= 4 f n(\pm 1 + \cos(\theta)) \text{d}t  +4 f n^2(\pm 1 + \cos(\theta))^2\text{d}\phi-\left(r^2+n^2\right) \sin^2 \theta\ \text{d} \phi. 
  \end{align*}
  We also need the $k^{\flat}$ for a $m=0$ space time which is given by 
  \begin{align*}
    k^{\flat}_{m=0} = 4 \frac{r^2-n^2}{r^2 + n^2} n(\pm 1 + \cos(\theta)) \text{d}t  +4 \frac{r^2-n^2}{r^2 + n^2} n^2(\pm 1 + \cos(\theta))^2\text{d}\phi-\left(r^2+n^2\right) \sin^2 \theta\ \text{d} \phi. 
  \end{align*}
  and the difference
  \begin{align*}
    k^{\flat} - k^{\flat}_{m=0} &= 4 \left(\frac{r^2-2 m r-n^2}{r^2+n^2}-\frac{r^2-n^2}{r^2 + n^2}\right) n(\pm 1 + \cos(\theta)) \text{d}t  +4 \left(\frac{r^2-2 m r-n^2}{r^2+n^2}-\frac{r^2-n^2}{r^2 + n^2}\right) n^2(\pm 1 + \cos(\theta))^2\text{d}\phi\\
    &= \frac{-8 nm r}{r^2+n^2} (\pm 1 + \cos(\theta)) \text{d}t  + \frac{-8 n^2 m r}{r^2+n^2} (\pm 1 + \cos(\theta))^2\text{d}\phi. 
  \end{align*}
  Then the exterior derivative reads 
  \begin{align*}
    \text{d}(k^{\flat} - k^{\flat}_{m=0}) &= \left(\frac{-8 nm (r^2+n^2)}{(r^2+n^2)^2} + \frac{16 nm r^2}{(r^2+n^2)^2}\right) (\pm 1 + \cos(\theta)) \text{d}t\wedge \text{d}r  + \left(\frac{-8 n^2m (r^2+n^2)}{(r^2+n^2)^2} + \frac{16 n^2m r^2}{(r^2+n^2)^2}\right) (\pm 1 + \cos(\theta)) (\pm 1 + \cos(\theta))^2\text{d}\phi \wedge \text{d} r\\
    &+\frac{-8 nm r}{r^2+n^2} (-\sin(\theta)) \text{d}t\wedge\text{d}\theta  - 2\frac{-8 n^2 m r}{r^2+n^2} \sin(\theta)(\pm 1 + \cos(\theta))\text{d}\phi\wedge\text{d}\theta. 
  \end{align*}
  In preparation for the calculation of the Hodge dual of this form, we write 
  \begin{align*}
    &\star \text{d}t\wedge \text{d}r = \sqrt{r^4 (r^2 + n^2)^2 \sin^2 \theta }g^{\phi\phi}g^{\theta\theta}\text{d}\phi \wedge \text{d}\theta,\\
    &\star \text{d}\phi \wedge \text{d}r = \sqrt{r^4 (r^2 + n^2)^2 \sin^2 \theta }g^{tt}g^{\theta\theta}\text{d}t \wedge \text{d}\theta,\\
    &\star \text{d}t \wedge \text{d}\theta = \sqrt{r^4 (r^2 + n^2)^2 \sin^2 \theta }g^{rr}g^{\phi\phi}\text{d}r \wedge \text{d}\phi,\\
    &\star \text{d}\phi \wedge \text{d}\theta = \sqrt{r^4 (r^2 + n^2)^2 \sin^2 \theta }g^{rr}g^{tt}\text{d}r \wedge \text{d}t.
  \end{align*}
\end{enumerate}

% Thiago 
% Otavio 


}

% References
\makereferences
%-------------------------------------------------------


%%%%%%%%%%%%%%%%%%%%%%%%
% Terminer le document %
%%%%%%%%%%%%%%%%%%%%%%%%
\end{document}
