\documentclass[10pt, a4paper]{article}

%%%%%%%%%%%%%%
%  Packages  %
%%%%%%%%%%%%%%


\usepackage{page_format}
\usepackage{special}
\usepackage{hyperref}
\usepackage{tikz}
\usepackage[compat=1.1.0]{tikz-feynman}
%----------------------------------------------------------------------
%\usepackage{amssymb} % Mathematical fonts.
%\usepackage{amsfonts} % Mathematical fonts.
\usepackage[nice]{nicefrac} % Nicer fractions
\usepackage{braket} % Dirac Notation.
\usepackage{bbm} % More bold fonts.
%\usepackage{mathrsfs} % Mathematical fonts.
\usepackage{esint} % Integrals
\usepackage{cancel} % Allows to scratch expressions.
\usepackage{mathtools} % Tools for math formating.
\usepackage{slashed} % Allows to slash individual characters.
\usepackage{xargs} % Better handling of optional arguments for commands
%----------------------------------------------------------------------
%\usepackage{lmodern} % Fonts.
\usepackage{feyn} % Feynman Diagrams in mathmode

%%%%%%%%%%%%%%%%%%%%%%%%%%%
% Mathématiques et physique
%%%%%%%%%%%%%%%%%%%%%%%%%%%%
% SI Units -----------------------
% The package 'siunitx' causes unresolved crashes (as of 22/08/31)
\newcommand{\ampere}{\text{A}}
\newcommand{\bell}{\text{B}}
\newcommand{\celsius}{\degree\text{C}}
\newcommand{\coulomb}{\text{C}}
\newcommand{\degree}{\,^{\circ}}
\newcommand{\farad}{\text{F}}
\newcommand{\electro}{\text{e}}
\newcommand{\gram}{\text{g}}
\newcommand{\henry}{\text{H}}
\newcommand{\hertz}{\text{Hz}}
\newcommand{\hour}{\text{h}}
\newcommand{\joule}{\text{J}}
\newcommand{\kelvin}{\text{K}}
\newcommand{\meter}{\text{m}}
\newcommand{\minute}{\text{m}}
\newcommand{\mole}{\text{mol}}
\newcommand{\newton}{\text{N}}
\newcommand{\ohm}{\Omega}
\newcommand{\pascal}{\text{Pa}}
\newcommand{\rad}{\text{rad}}
\newcommand{\second}{\text{s}}
\newcommand{\tesla}{\text{T}}
\newcommand{\torr}{\text{Torr}}
\newcommand{\volt}{\text{V}}
\newcommand{\watt}{\text{W}}
%
\newcommand{\tera}{\text{T}}
\newcommand{\giga}{\text{G}}
\newcommand{\mega}{~\text{M}}
\newcommand{\kilo}{~\text{k}}
\newcommand{\deci}{\text{d}}
\newcommand{\centi}{\text{c}}
\newcommand{\milli}{\text{m}}
\newcommand{\micro}{\mu}
\newcommand{\nano}{\text{n}}
\newcommand{\pico}{\text{p}}
\newcommand{\femto}{\text{f}}
%
\newcommand{\units}[1]{\text{#1}}
\newcommand{\tothe}[1]{\textsuperscript{#1}}
%
\newcommand{\per}{\text{/}}
%
\newcommand{\Time}[3]{#1\hour~#2\minute~#3\second} % TODO Optional arguments.
\newcommand{\Angle}[3]{#1^{\circ}~#2'~#3''} % TODO Optional arguments.


% Better epsilon -----------------------
\let\oldepsilon\epsilon
\let\epsilon\varepsilon
\let\varepsilon\oldepsilon


% Better \bar -----------------------
\renewcommand{\bar}[1]{\mkern 1.5mu\overline{\mkern-1.5mu#1\mkern-1.5mu}\mkern 1.5mu}


% Équations -----------------------
\newcommand{\al}[1]{\begin{align} #1 \end{align}} % Numbered equation(s),
\newcommand{\eqn}[1]{\begin{align*} #1 \end{align*}} % Number-less equation(s),
\newcommand{\sys}[1]{\begin{dcases*} #1 \end{dcases*}} % System of equations.


% Exponents -----------------------
\newcommand{\Exp}[1]{\text{e}^{#1}}		% e^#
\newcommand{\E}[1]{\times 10^{#1}}		% X 10^#


% Delimiters -----------------------
\newcommand{\p}[1]{\left( #1 \right)}	% (#)
\newcommand{\cro}[1]{\left[ #1 \right]}	% [#]
\newcommand{\abs}[1]{\left| #1\right|}	% |#|
\newcommand{\avg}[1]{\left\langle #1 \right\rangle} % <#>
\newcommand{\acc}[1]{\left\lbrace #1 \right\rbrace} % {#}


% Vectors -----------------------
\newcommand{\ve}[1]{\mathbf{#1}} % Upright bold face.
\newcommand{\vu}[1]{\hat{\ve{#1}}} % Hat vector upright bold face
\newcommand{\tens}{\otimes} % Tensor product
\newcommand{\nablav}{\bm{\nabla}} % Bold gradient


% Trig. functions with automatic formating  -----------------------
\newcommandx{\Sin}[2][1={}]{\text{sin}^{#1}\!\p{#2}}
\newcommandx{\Cos}[2][1={}]{\text{cos}^{#1}\!\p{#2}}
\newcommandx{\Tan}[2][1={}]{\text{tan}^{#1}\!\p{#2}}
\newcommandx{\Csc}[2][1={}]{\text{csc}^{#1}\!\p{#2}}
\newcommandx{\Sec}[2][1={}]{\text{sec}^{#1}\!\p{#2}}
\newcommandx{\Cot}[2][1={}]{\text{cot}^{#1}\!\p{#2}}
\newcommandx{\Arcsin}[2][1={}]{\text{arcsin}^{#1}\!\p{#2}}
\newcommandx{\Arccos}[2][1={}]{\text{arccos}^{#1}\!\p{#2}}
\newcommandx{\Arctan}[2][1={}]{\text{arctan}^{#1}\!\p{#2}}
\newcommandx{\Sinh}[2][1={}]{\text{sinh}^{#1}\!\p{#2}}
\newcommandx{\Cosh}[2][1={}]{\text{cosh}^{#1}\!\p{#2}}
\newcommandx{\Tanh}[2][1={}]{\text{tanh}^{#1}\!\p{#2}}


% Matrices -----------------------
\newcommand{\mat}[1]{\begin{bmatrix} #1 \end{bmatrix}} % Matrices with hooks.
\newcommand{\pmat}[1]{\begin{pmatrix} #1 \end{pmatrix}} % Matrices with parentheses.
\newcommand{\deter}[1]{\abs{\begin{matrix} #1 \end{matrix}}} % Determinant.
\newcommandx{\mO}[2][1={}, 2={}]{ \def\temp{#2}\ifx\temp\empty\ve{O}_{#1}\else\ve{O}_{#1\times #2}\fi}% Zero matrix.
\newcommandx{\mI}[2][1={}, 2={}]{ \def\temp{#2}\ifx\temp\empty\ve{I}_{#1}\else\ve{O}_{#1\times #2}\fi}%  Identity matrix.
\newcommand{\Det}[1]{\text{det}\p{#1}} % det(#)
\newcommand{\Tr}[1]{\text{Tr}\p{#1}} % Tr(#)


% Derivatives -----------------------
\newcommand{\D}{\text{d}} % Differential 'd'.
\newcommandx{\dd}[3][1={},3={}]{\frac{\D^{#3}#1}{\D{#2}^{#3}}} % Total derivative according to #2, #1 is the function and #3 is the order.
\newcommand{\del}{\partial} % Partial 'd'.
\newcommandx{\ddp}[3][1={},3={}]{\frac{\del^{#3}#1}{\del{#2}^{#3}}} % Dérivée partielle selon #2, #1 est la fonction est #3 est l'ordre.
\newcommand{\eval}[1]{\left. {#1} \right|} % Bar on the right of expression.
\newcommand{\delbar}{\slashed{\del}} % Partial Inexact differential.
\newcommand{\dbar}{\dj}% Inexact differential.


% Integrals -----------------------
\newcommand{\intinf}{\int\displaylimits_{-\infty}^{\infty}} % From -00 to 00.
\newcommandx{\Int}[2][1={},2={}]{\int\displaylimits_{#1}^{#2}} % Faster bounded integrals.


% Complex numbers -----------------------
\renewcommand{\Re}[1]{\text{Re}\acc{#1}} % Re{#}
\renewcommand{\Im}[1]{\text{Im}\acc{#1}} % Im{#}


% Sets -----------------------
\newcommand{\N}{\mathbbm{N}} % Natural numbers.
\newcommand{\Z}{\mathbbm{Z}} % Integers.
\newcommand{\Q}{\mathbbm{Q}} % Rational numbers.
\newcommandx{\R}[1][1={}]{\mathbbm{R}^{#1}} % Real numbers.
\newcommandx{\C}[1][1={}]{\mathbbm{C}^{#1}} % Complex numbers.
\newcommandx{\F}[1][1={}]{\mathbbm{F}^{#1}} % Some field.
\newcommand{\M}[3]{\mathbb{M}_{#1\times#2}(#3)}	% Matrices.
\newcommand{\Po}[2]{\mathbb{P}_{#1}(#2)} % Polynomials.
\newcommand{\Lin}{\mathbb{L}} % Linear maps.


% Constants and physical symbols -----------------------
\newcommand{\eo}{\epsilon_0} % epsilon 0.
\renewcommand{\L}{\mathcal{L}} % Lagrangian.

\usepackage{slashed}

% References
\usepackage{biblatex}
\addbibresource{ref.bib}


%%%%%%%%%%%%
%  Colors  %
%%%%%%%%%%%%
% ! EDIT HERE !
\colorlet{chaptercolor}{red!70!black} % Foreground color.
\colorlet{chaptercolorback}{red!10!white} % Background color

%%%%%%%%%%%%%%
% Page titre %
%%%%%%%%%%%%%%%
\title{Homework 1} % Title of the assignement.
\author{\PA} % Your name(s).
\teacher{Ruth Gregory} % Your teacher's name.
\class{Gravitational Physics} % The class title.

\university{Perimeter Institute for Theoretical Physics} % University
\faculty{Perimeter Scholars International} % Faculty
%\departement{<Departement>} % Departement
\date{\today} % Date.


%%%%%%%%%%%%%%%%%%%%%%
% Begin the document %
%%%%%%%%%%%%%%%%%%%%%%
\begin{document}

% Make the title page.
\maketitlepage

% Make table of contents
\maketableofcontents

% Assignment starts here ----------------------------

\footnotesize{

\section{Cartan in a FLRW universe}
\begin{enumerate}
  \item[(a)] The Friedmann-Lemaitre-Robinson-Walker (FLRW) metric two-form describes a spacetime with spacelike foliation in homogeneous and isotropic hypersurfaces. In a coordinate chart with coordinates $x^\mu = \{t, \theta, \phi, r\}$ making the isotropy and foliation manifest, this metric reads 
  \vspace{-0.4cm}
  \begin{align*}
    g_{\mu \nu} \underline{d} x^\mu \otimes \underline{d} x^\nu \equiv  \underline{d} t \otimes \underline{d} t -a^2(t)\left(\frac{\underline{d} r \otimes \underline{d} r }{1-k r^2}+r^2\left(\underline{d} \theta \otimes \underline{d} \theta +\sin ^2 \theta \underline{d} \phi \otimes \underline{d} \phi \right) \right)
  \end{align*}
  where $\{\underline{d}x^\mu\}_{\mu = 0}^3 = \{\underline{d}t, \underline{d}\theta, \underline{d}\phi, \underline{d}r\}$ are the coordinate on-forms dual to the vector basis $\underline{e}_{a} = \{\partial_{t}, \partial_{\theta}, \partial_{\phi}, \partial_{r}\}$,  $a(t)> 0$ is the scale factor and $k = 0, -1, 1$ gives the sign of the curvature of the spacelike hypersurfaces (respectively flat, Anti-de Sitter, de Sitter). In what follows, the tensor products are implicit. At every point in our chart, we define an orthonormal basis of one-forms $\underline{\omega}^a = c_\mu^a \underline{d}x^\mu$ such that $g_{\mu \nu} \underline{d} x^\mu \underline{d} x^\nu = \eta_{ab} \underline{\omega}^a \underline{\omega}^b$ where $\eta_{ab}$ is the Minkowski metric components with signature $(+, -, -, -)$. We can write 
  \begin{align*}
    g_{\mu \nu} \underline{d} x^\mu \underline{d} x^\nu &=  \underline{d} t \underline{d} t -\left(\frac{a(t)\underline{d} r}{\sqrt{1-k r^2}}\right)\left(\frac{a(t)\underline{d} r}{\sqrt{1-k r^2}}\right)-\left(a(t) r\underline{d} \theta\right)\left(a(t) r\underline{d} \theta\right) -(a(t) r\sin \theta \underline{d} \phi) (a(t) r \sin \theta\underline{d} \phi)\\
     &= \underline{\omega}^0 \underline{\omega}^0 - \underline{\omega}^1 \underline{\omega}^1 - \underline{\omega}^2 \underline{\omega}^2 - \underline{\omega}^3 \underline{\omega}^3 
  \end{align*}
  where $\{\underline{\omega}^a\}_{a=0}^{3} = \{\underline{d} t, \ a(t) r\underline{d} \theta, \ a(t) r\sin \theta \underline{d} \phi, \ \frac{a(t)}{\sqrt{1-k r^2}} \underline{d} r\}$ is shown to satisfy the orthonormality condition. We note that the resulting choice of basis is unique up to a local lorentz transformation (which preserves orthonormality). 
  \item[(b)] To calculate the connection one-forms $\underline{\theta}^a{}_b$, we use the orthonormal basis found in (a) and Cartan's first structure equation for vanishing torsion to get 
  \begin{align*}
    \underline{\theta}^a{ }_b \wedge \underline{\omega}^b = -\underline{d \omega}^a &= 
    \begin{cases}
     -\partial_{\mu}(1)\ \underline{d} x^\mu \wedge \underline{d} t\\
     -\partial_{\mu}(a(t) r)\ \underline{d} x^\mu \wedge \underline{d} \theta\\
     -\partial_{\mu}(a(t) r\sin \theta)\ \underline{d} x^\mu \wedge \underline{d} \phi\\
     -\partial_{\mu}\left(\frac{a(t)}{\sqrt{1-k r^2}}\right)\ \underline{d} x^\mu \wedge \underline{d} r
    \end{cases}=
    \begin{cases}
      0\\
     -a'(t) r\underline{d} t \wedge \underline{d} \theta - a(t)\underline{d} r \wedge \underline{d} \theta\\
     -a'(t) r\sin \theta \underline{d}t \wedge \underline{d} \phi -  a(t) \sin \theta \underline{d}r \wedge \underline{d} \phi - a(t) r\cos \theta \underline{d}\theta \wedge \underline{d} \phi\\
     -\frac{a'(t)}{\sqrt{1-k r^2}}\underline{d} t \wedge \underline{d} r - [\cdots]\underline{d} r \wedge \underline{d} r 
    \end{cases}
    \\
    &=
    \begin{cases}
      0\\
     \frac{a'(t)}{a(t)} \underline{\omega}^1 \wedge \underline{\omega}^0 + \frac{1}{a(t)r}\sqrt{1-k r^2}\underline{\omega}^1 \wedge \underline{\omega}^3\\
     \frac{a'(t)}{a(t)} \underline{\omega}^2 \wedge \underline{\omega}^0 +  \frac{1}{a(t)r}\sqrt{1-k r^2} \underline{\omega}^2 \wedge \underline{\omega}^3 + \frac{1}{a(t)r}\cot \theta \underline{\omega}^2 \wedge \underline{ \omega}^1\\
     \frac{a'(t)}{a(t)}\underline{\omega}^3 \wedge \underline{\omega}^0
    \end{cases} 
    = \begin{cases}
      \underline{\theta}^0{ }_b \wedge \underline{\omega}^b \\
      \underline{\theta}^1{ }_b \wedge \underline{\omega}^b \\
      \underline{\theta}^2{ }_b \wedge \underline{\omega}^b \\
      \underline{\theta}^3{ }_b \wedge \underline{\omega}^b 
    \end{cases}
  \end{align*}
  Since the $\wedge$ product with $\underline{\omega}^b$ maps $\underline{\omega}^{c\neq b}$ to linearly independant two-forms, we can read the coefficients of $\underline{\omega}^{c\neq b}$ preceeding the $\wedge$ product in the previous expressions. We have
  \begin{align*}
    \begin{cases}
    \underline{\theta}^0{ }_{1} =  [\cdots] \underline{\omega}^1, \quad \underline{\theta}^0{ }_{2} =  [\cdots] \underline{\omega}^2, \quad \underline{\theta}^0{ }_{3} =  [\cdots] \underline{\omega}^3\\
    \underline{\theta}^1{ }_0 = \frac{a'(t)}{a(t)}\underline{\omega}^1 + [\cdots] \underline{\omega}^0,\quad \underline{\theta}^1{ }_{2} = [\cdots] \underline{\omega}^2, \quad \underline{\theta}^1{ }_3 = \frac{1}{a(t)r}\sqrt{1-k r^2}\underline{\omega}^1 + [\cdots] \underline{\omega}^3\\
    \underline{\theta}^2{ }_0 = \frac{a'(t)}{a(t)}\underline{\omega}^2 + [\cdots]\underline{\omega}^0,\quad \underline{\theta}^2{ }_3 = \frac{1}{a(t)r}\sqrt{1-k r^2}\underline{\omega}^2 + [\cdots]\underline{\omega}^3,\quad \underline{\theta}^2{ }_1 = \frac{1}{a(t)r}\cot \theta \underline{\omega}^2 + [\cdots]\underline{\omega}^1\\
    \underline{\theta}^3{ }_0 = \frac{a'(t)}{a(t)}\underline{\omega}^3 + [\cdots]\underline{\omega}^0,\quad \underline{\theta}^3{ }_1 = [\cdots]\underline{\omega}^1, \quad \underline{\theta}^3{ }_2 = [\cdots]\underline{\omega}^2
    \end{cases}
  \end{align*} 
  where $[\cdots]$ terms represent the terms mapped to $0$ by the $\wedge$ product from which information about $\underline{\theta}^a{ }_b$ was read. From the first line we can also read $\underline{\theta}^0{ }_{1, 2, 3} = [\cdots] \underline{\omega}^{1, 2, 3}$
  
  To fully determine the one-forms components from these relations, we invoke the relation $\underline{\theta}_{ab} + \underline{\theta}_{ba} = \underline{d}g_{ab}$ where $\underline{\theta}_{ba} = g_{bc}\underline{\theta}^{c}{}_a$. Recalling that in our orthonormal basis $g_{ab} = \eta_{ab}$, we get the antisymmetry relation $\underline{\theta}_{ab} + \underline{\theta}_{ba}$. It follows that $\underline{\theta}^{a}{}_{a} = 0, \ \forall a$ and we can use it to determine $[...]$. Expliciting the relation between $\underline{\theta}^{b}{}_a$ and  $\underline{\theta}^{a}{}_b$ yields
  \begin{align*}
    \begin{cases}
      b \ \text{spacelike} \implies 
      \underline{\theta}^{b}{}_a = \eta^{bc}\underline{\theta}_{ca} = (-1) \underline{\theta}_{ba} = \underline{\theta}_{ab} \implies 
      \begin{cases}
        a \ \text{spacelike} \implies \underline{\theta}^{b}{}_a = -\underline{\theta}^{a}{}_b\\
        a \ \text{timelike} \implies \underline{\theta}^{b}{}_a =   \underline{\theta}^{a}{}_b
      \end{cases}
        \\
      b \ \text{timelike} \implies 
      \underline{\theta}^{b}{}_a = \eta^{bc}\underline{\theta}_{ca} = \underline{\theta}_{ba} = -\underline{\theta}_{ab}  \implies 
      \begin{cases}
        a \ \text{spacelike} \implies \underline{\theta}^{b}{}_a = \underline{\theta}^{a}{}_b\\
        a \ \text{timelike} \implies \underline{\theta}^{b}{}_a = -\underline{\theta}^{a}{}_b \quad \text{never happens ($a \neq b$)}
      \end{cases}
    \end{cases}
  \end{align*}
  Comparing $\underline{\theta}^{a}{}_b$ with $\underline{\theta}^{b}{}_a$, we finally see 
  \begin{align*}
    &[\cdots] \underline{\omega}^1 = \underline{\theta}^0{ }_{1} = \underline{\theta}^1{ }_{0} = \frac{a'(t)}{a(t)}\underline{\omega}^1 + [\cdots] \underline{\omega}^0 \iff \underline{\theta}^1{ }_{0} = \frac{a'(t)}{a(t)}\underline{\omega}^1, \quad \underline{\theta}^0{ }_{1} = \frac{a'(t)}{a(t)}\underline{\omega}^1\\
    &[\cdots] \underline{\omega}^2 = \underline{\theta}^0{ }_{2} = \underline{\theta}^2{ }_{0} = \frac{a'(t)}{a(t)}\underline{\omega}^2 + [\cdots] \underline{\omega}^0 \iff \underline{\theta}^2{ }_{0} = \frac{a'(t)}{a(t)}\underline{\omega}^2, \quad \underline{\theta}^0{ }_{2} = \frac{a'(t)}{a(t)}\underline{\omega}^2\\
    &[\cdots] \underline{\omega}^3 = \underline{\theta}^0{ }_{3} = \underline{\theta}^3{ }_{0} = \frac{a'(t)}{a(t)}\underline{\omega}^3 + [\cdots] \underline{\omega}^0 \iff \underline{\theta}^3{ }_{0} = \frac{a'(t)}{a(t)}\underline{\omega}^3, \quad \underline{\theta}^0{ }_{3} = \frac{a'(t)}{a(t)}\underline{\omega}^3\\
    &[\cdots] \underline{\omega}^2 = \underline{\theta}^1{ }_{2} = -\underline{\theta}^2{ }_{1} = -\frac{1}{a(t)r}\cot \theta \underline{\omega}^2 - [\cdots]\underline{\omega}^1 \iff \underline{\theta}^1{ }_{2} = -\frac{1}{a(t)r}\cot \theta \underline{\omega}^2, \quad \underline{\theta}^2{ }_{1} = \frac{1}{a(t)r}\cot \theta \underline{\omega}^2\\
    &[\cdots] \underline{\omega}^2 = \underline{\theta}^3{ }_{2} = -\underline{\theta}^2{ }_{3} = -\frac{1}{a(t)r}\sqrt{1-k r^2}\underline{\omega}^2 - [\cdots]\underline{\omega}^3 \iff \underline{\theta}^3{ }_{2} = -\frac{1}{a(t)r}\sqrt{1-k r^2}\underline{\omega}^2, \quad \underline{\theta}^3{ }_{2} = \frac{1}{a(t)r}\sqrt{1-k r^2}\underline{\omega}^2\\
    &[\cdots] \underline{\omega}^1 = \underline{\theta}^3{ }_{1} = -\underline{\theta}^1{ }_{3} = -\frac{1}{a(t)r}\sqrt{1-k r^2}\underline{\omega}^1 - [\cdots]\underline{\omega}^3 \iff \underline{\theta}^3{ }_{1} = -\frac{1}{a(t)r}\sqrt{1-k r^2}\underline{\omega}^1, \quad \underline{\theta}^1{ }_{3} = \frac{1}{a(t)r}\sqrt{1-k r^2}\underline{\omega}^1
  \end{align*}
  
  \item[(c)] The curvature two-forms are computed from the connection one-forms calculated above with the relation 
  \begin{align*}
    \underline{R}_b^a=\underline{d \theta}_b^a+\underline{\theta}_c^a \wedge \underline{\theta}_b^c
  \end{align*}
  \item[(d)]    
\end{enumerate}

\section{Acknowledgement}

Thanks to Luke for help reviewing and understanding the concepts used in this assignement


}

% References
\makereferences
%-------------------------------------------------------


%%%%%%%%%%%%%%%%%%%%%%%%
% Terminer le document %
%%%%%%%%%%%%%%%%%%%%%%%%
\end{document}
