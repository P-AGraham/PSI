\documentclass[10pt, a4paper]{article}

%%%%%%%%%%%%%%
%  Packages  %
%%%%%%%%%%%%%%


\usepackage{page_format}
\usepackage{special}
\usepackage{hyperref}
\usepackage{tikz}
\usepackage[compat=1.1.0]{tikz-feynman}
\input{math_func}

\usepackage{slashed}

% References
\usepackage{biblatex}
\addbibresource{ref.bib}


%%%%%%%%%%%%
%  Colors  %
%%%%%%%%%%%%
% ! EDIT HERE !
\colorlet{chaptercolor}{red!70!black} % Foreground color.
\colorlet{chaptercolorback}{red!10!white} % Background color

%%%%%%%%%%%%%%
% Page titre %
%%%%%%%%%%%%%%%
\title{Homework 1} % Title of the assignement.
\author{\PA} % Your name(s).
\teacher{Ruth Gregory} % Your teacher's name.
\class{Gravitational Physics} % The class title.

\university{Perimeter Institute for Theoretical Physics} % University
\faculty{Perimeter Scholars International} % Faculty
%\departement{<Departement>} % Departement
\date{\today} % Date.


%%%%%%%%%%%%%%%%%%%%%%
% Begin the document %
%%%%%%%%%%%%%%%%%%%%%%
\begin{document}

% Make the title page.
\maketitlepage

% Make table of contents
\maketableofcontents

% Assignment starts here ----------------------------



\section{Cartan in a FLRW universe}
\begin{enumerate}
  \item[(a)] The Friedmann-Lemaitre-Robinson-Walker (FLRW) metric two-form describes a spacetime with spacelike foliation in homogeneous and isotropic hypersurfaces. In a coordinate chart with coordinates $x^\mu = \{t, \theta, \phi, r\}$ making the isotropy and foliation manifest, this metric reads 
  \begin{align*}
    g_{\mu \nu} \underline{d} x^\mu \otimes \underline{d} x^\nu \equiv  \underline{d} t \otimes \underline{d} t -a^2(t)\left(\frac{\underline{d} r \otimes \underline{d} r }{1-k r^2}+r^2\left(\underline{d} \theta \otimes \underline{d} \theta +\sin ^2 \theta \underline{d} \phi \otimes \underline{d} \phi \right) \right)
  \end{align*}
  where $\{\underline{d}x^\mu\}_{\mu = 0}^3 = \{\underline{d}t, \underline{d}\theta, \underline{d}\phi, \underline{d}r\}$ are the coordinate on-forms dual to the vector basis $\underline{e}_{a} = \{\partial_{t}, \partial_{\theta}, \partial_{\phi}, \partial_{r}\}$,  $a(t)> 0$ is the scale factor and $k = 0, -1, 1$ gives the sign of the curvature of the spacelike hypersurfaces (respectively flat, Anti-de Sitter, de Sitter). In what follows, the tensor products are implicit. At every point in our chart, we define an orthonormal basis of one-forms $\underline{\omega}^a = c_\mu^a \underline{d}x^\mu$ such that $g_{\mu \nu} \underline{d} x^\mu \underline{d} x^\nu = \eta_{ab} \underline{\omega}^a \underline{\omega}^b$ where $\eta_{ab}$ is the Minkowski metric components with signature $(+, -, -, -)$. We can write 
  \begin{align*}
    &g_{\mu \nu} \underline{d} x^\mu \underline{d} x^\nu\\
     &=  \underline{d} t \underline{d} t -\left(\frac{a(t)\underline{d} r}{\sqrt{1-k r^2}}\right)\left(\frac{a(t)\underline{d} r}{\sqrt{1-k r^2}}\right)-\left(a(t) r\underline{d} \theta\right)\left(a(t) r\underline{d} \theta\right) -(a(t) r\sin \theta \underline{d} \phi) (a(t) r \sin \theta\underline{d} \phi)\\
     &= \underline{\omega}^0 \underline{\omega}^0 - \underline{\omega}^1 \underline{\omega}^1 - \underline{\omega}^2 \underline{\omega}^2 - \underline{\omega}^3 \underline{\omega}^3 
  \end{align*}
  where $\{\underline{\omega}^a\}_{a=0}^{3} = \{\underline{d} t, \ a(t) r\underline{d} \theta, \ a(t) r\sin \theta \underline{d} \phi, \ \frac{a(t)}{\sqrt{1-k r^2}} \underline{d} r\}$ is shown to satisfy the orthonormality condition. We note that the resulting choice of basis is unique up to a local lorentz transformation (which preserves orthonormality). 
  \item[(b)] To calculate the connection one-forms $\underline{\theta}^a{}_b$, we use the orthonormal basis found in (a) and Cartan's first structure equation for vanishing torsion to get 
  \begin{align*}
    \underline{\theta}^a{ }_b \wedge \underline{\omega}^b = -\underline{d \omega}^a &= 
    \begin{cases}
     -\partial_{\mu}(1)\ \underline{d} x^\mu \wedge \underline{d} t\\
     -\partial_{\mu}(a(t) r)\ \underline{d} x^\mu \wedge \underline{d} \theta\\
     -\partial_{\mu}(a(t) r\sin \theta)\ \underline{d} x^\mu \wedge \underline{d} \phi\\
     -\partial_{\mu}\left(\frac{a(t)}{\sqrt{1-k r^2}}\right)\ \underline{d} x^\mu \wedge \underline{d} r
    \end{cases}\\
    &=
    \begin{cases}
      0\\
     -a'(t) r\underline{d} t \wedge \underline{d} \theta - a(t)\underline{d} r \wedge \underline{d} \theta\\
     -a'(t) r\sin \theta \underline{d}t \wedge \underline{d} \phi -  a(t) \sin \theta \underline{d}r \wedge \underline{d} \phi - a(t) r\cos \theta \underline{d}\theta \wedge \underline{d} \phi\\
     -\frac{a'(t)}{\sqrt{1-k r^2}}\underline{d} t \wedge \underline{d} r - [\cdots]\underline{d} r \wedge \underline{d} r 
    \end{cases}
    \\
    &= 
    \begin{cases}
      0\\
     \frac{a'(t)}{a(t)} \underline{\omega}^1 \wedge \underline{\omega}^0 + \frac{1}{a(t)r}\sqrt{1-k r^2}\underline{\omega}^1 \wedge \underline{\omega}^3\\
     \frac{a'(t)}{a(t)} \underline{\omega}^2 \wedge \underline{\omega}^0 +  \frac{1}{a(t)r}\sqrt{1-k r^2} \underline{\omega}^2 \wedge \underline{\omega}^3 + \frac{1}{a(t)r}\cot \theta \underline{\omega}^2 \wedge \underline{ \omega}^1\\
     \frac{a'(t)}{a(t)}\underline{\omega}^0 \wedge \underline{\omega}^3
    \end{cases} 
    = \begin{cases}
      \underline{\theta}^0{ }_b \wedge \underline{\omega}^b \\
      \underline{\theta}^1{ }_b \wedge \underline{\omega}^b \\
      \underline{\theta}^2{ }_b \wedge \underline{\omega}^b \\
      \underline{\theta}^3{ }_b \wedge \underline{\omega}^b 
    \end{cases}
  \end{align*}
  Since the $\wedge$ product with $\underline{\omega}^b$ maps $\underline{\omega}^{c\neq b}$ to linearly independant two-forms, we can read the coefficients of $\underline{\omega}^{c\neq b}$ preceeding the $\wedge$ product in the previous expressions. We have
  \begin{align*}
    \begin{cases}
    \underline{\theta}^1{ }_0 = \frac{a'(t)}{a(t)}\underline{\omega}^1 + [\cdots] \underline{\omega}^0,\quad \underline{\theta}^1{ }_3 = \frac{1}{a(t)r}\sqrt{1-k r^2}\underline{\omega}^1 + [\cdots] \underline{\omega}^3\\
    \underline{\theta}^2{ }_0 = \frac{a'(t)}{a(t)}\underline{\omega}^2,\quad \underline{\theta}^2{ }_3 = \frac{1}{a(t)r}\sqrt{1-k r^2}\underline{\omega}^2,\quad \underline{\theta}^2{ }_1 = \frac{1}{a(t)r}\cot \theta \underline{\omega}^2\\
    \underline{\theta}^3{ }_0 = \frac{a'(t)}{a(t)}\underline{\omega}^3
    \end{cases}
  \end{align*} 
  where $[\cdots]$ terms represent the terms mapped to $0$ by the $\wedge$ product from which information about $\underline{\theta}^a{ }_b$ was read. 
  
  To extract the connection one-forms components from these relations, we invoke the antisymmetry relation $g_{ca} \underline{\theta}^c{ }_b + g_{ca} \underline{\theta}_b{ }^c = \underline{d}g_{ab}$. Recalling that in our orthonormal basis $g_{ab} = \eta_{ab}$,  we have 
  \begin{align*}
    \eta_{ca}\underline{\theta}^c{ }_b + \eta_{ca} \underline{\theta}_b{ }^c = 0 \iff  \eta_{ca}\eta^{ad}\underline{\theta}^c{ }_b + \eta_{ca}\eta^{ad} \underline{\theta}_b{ }^c = \underline{\theta}^d{ }_b + \underline{\theta}_b{ }^d = 0
  \end{align*}
  
  \item[(c)]
  \item[(d)]    
\end{enumerate}

\section{Acknowledgement}

Thanks to Luke for help reviewing and understanding the concepts used in this assignement




% References
\makereferences
%-------------------------------------------------------


%%%%%%%%%%%%%%%%%%%%%%%%
% Terminer le document %
%%%%%%%%%%%%%%%%%%%%%%%%
\end{document}
