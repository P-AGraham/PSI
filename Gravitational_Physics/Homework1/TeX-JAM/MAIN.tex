\documentclass[10pt, a4paper]{article}

%%%%%%%%%%%%%%
%  Packages  %
%%%%%%%%%%%%%%


\usepackage{page_format}
\usepackage{special}
\usepackage{hyperref}
\usepackage{tikz}
\usepackage[compat=1.1.0]{tikz-feynman}
\input{math_func}

\usepackage{slashed}

% References
\usepackage{biblatex}
\addbibresource{ref.bib}


%%%%%%%%%%%%
%  Colors  %
%%%%%%%%%%%%
% ! EDIT HERE !
\colorlet{chaptercolor}{red!70!black} % Foreground color.
\colorlet{chaptercolorback}{red!10!white} % Background color

%%%%%%%%%%%%%%
% Page titre %
%%%%%%%%%%%%%%%
\title{Homework 1} % Title of the assignement.
\author{\PA} % Your name(s).
\teacher{Ruth Gregory} % Your teacher's name.
\class{Gravitational Physics} % The class title.

\university{Perimeter Institute for Theoretical Physics} % University
\faculty{Perimeter Scholars International} % Faculty
%\departement{<Departement>} % Departement
\date{\today} % Date.


%%%%%%%%%%%%%%%%%%%%%%
% Begin the document %
%%%%%%%%%%%%%%%%%%%%%%
\begin{document}

% Make the title page.
\maketitlepage

% Make table of contents
\maketableofcontents

% Assignment starts here ----------------------------

\footnotesize{

\section{Cartan in a FLRW universe}
\begin{enumerate}
  \item[(a)] The Friedmann-Lemaitre-Robinson-Walker (FLRW) metric two-form describes a spacetime with spacelike foliation in homogeneous and isotropic hypersurfaces. In a coordinate chart with coordinates $x^\mu = \{t, \theta, \phi, r\}$ making the isotropy and foliation manifest, this metric reads 
  \vspace{-0.4cm}
  \begin{align*}
    g_{\mu \nu} \underline{d} x^\mu \otimes \underline{d} x^\nu \equiv  \underline{d} t \otimes \underline{d} t -a^2(t)\left(\frac{\underline{d} r \otimes \underline{d} r }{1-k r^2}+r^2\left(\underline{d} \theta \otimes \underline{d} \theta +\sin ^2 \theta \underline{d} \phi \otimes \underline{d} \phi \right) \right)
  \end{align*}
  where $\{\underline{d}x^\mu\}_{\mu = 0}^3 = \{\underline{d}t, \underline{d}\theta, \underline{d}\phi, \underline{d}r\}$ are the coordinate on-forms dual to the vector basis $\underline{e}_{a} = \{\partial_{t}, \partial_{\theta}, \partial_{\phi}, \partial_{r}\}$,  $a(t)> 0$ is the scale factor and $k = 0, -1, 1$ gives the sign of the curvature of the spacelike hypersurfaces (respectively flat, Anti-de Sitter, de Sitter). In what follows, the tensor products are implicit. At every point in our chart, we define an orthonormal basis of one-forms $\underline{\omega}^a = c_\mu^a \underline{d}x^\mu$ such that $g_{\mu \nu} \underline{d} x^\mu \underline{d} x^\nu = \eta_{ab} \underline{\omega}^a \underline{\omega}^b$ where $\eta_{ab}$ is the Minkowski metric components with signature $(+, -, -, -)$. We can write 
  \begin{align*}
    g_{\mu \nu} \underline{d} x^\mu \underline{d} x^\nu &=  \underline{d} t \underline{d} t -\left(\frac{a(t)\underline{d} r}{\sqrt{1-k r^2}}\right)\left(\frac{a(t)\underline{d} r}{\sqrt{1-k r^2}}\right)-\left(a(t) r\underline{d} \theta\right)\left(a(t) r\underline{d} \theta\right) -(a(t) r\sin \theta \underline{d} \phi) (a(t) r \sin \theta\underline{d} \phi)\\
     &= \underline{\omega}^0 \underline{\omega}^0 - \underline{\omega}^1 \underline{\omega}^1 - \underline{\omega}^2 \underline{\omega}^2 - \underline{\omega}^3 \underline{\omega}^3 
  \end{align*}
  where $\{\underline{\omega}^a\}_{a=0}^{3} = \{\underline{d} t, \ a(t) r\underline{d} \theta, \ a(t) r\sin \theta \underline{d} \phi, \ \frac{a(t)}{\sqrt{1-k r^2}} \underline{d} r\}$ is shown to satisfy the orthonormality condition. We note that the resulting choice of basis is unique up to a local Lorentz transformation (which preserves orthonormality). 
  \item[(b)] To calculate the connection one-forms $\underline{\theta}^a{}_b$, we use the orthonormal basis found in (a) and Cartan's first structure equation for vanishing torsion to get 
  \begin{align*}
    \underline{\theta}^a{ }_b \wedge \underline{\omega}^b = -\underline{d \omega}^a &= 
    \begin{cases}
     -\partial_{\mu}(1)\ \underline{d} x^\mu \wedge \underline{d} t\\
     -\partial_{\mu}(a(t) r)\ \underline{d} x^\mu \wedge \underline{d} \theta\\
     -\partial_{\mu}(a(t) r\sin \theta)\ \underline{d} x^\mu \wedge \underline{d} \phi\\
     -\partial_{\mu}\left(\frac{a(t)}{\sqrt{1-k r^2}}\right)\ \underline{d} x^\mu \wedge \underline{d} r
    \end{cases}=
    \begin{cases}
      0\\
     -a'(t) r\underline{d} t \wedge \underline{d} \theta - a(t)\underline{d} r \wedge \underline{d} \theta\\
     -a'(t) r\sin \theta \underline{d}t \wedge \underline{d} \phi -  a(t) \sin \theta \underline{d}r \wedge \underline{d} \phi - a(t) r\cos \theta \underline{d}\theta \wedge \underline{d} \phi\\
     -\frac{a'(t)}{\sqrt{1-k r^2}}\underline{d} t \wedge \underline{d} r - [\cdots]\underline{d} r \wedge \underline{d} r 
    \end{cases}
    \\
    &=
    \begin{cases}
      0\\
     \frac{a'(t)}{a(t)} \underline{\omega}^1 \wedge \underline{\omega}^0 + \frac{1}{a(t)r}\sqrt{1-k r^2}\underline{\omega}^1 \wedge \underline{\omega}^3\\
     \frac{a'(t)}{a(t)} \underline{\omega}^2 \wedge \underline{\omega}^0 +  \frac{1}{a(t)r}\sqrt{1-k r^2} \underline{\omega}^2 \wedge \underline{\omega}^3 + \frac{1}{a(t)r}\cot \theta \underline{\omega}^2 \wedge \underline{ \omega}^1\\
     \frac{a'(t)}{a(t)}\underline{\omega}^3 \wedge \underline{\omega}^0
    \end{cases} 
    = \begin{cases}
      \underline{\theta}^0{ }_b \wedge \underline{\omega}^b \\
      \underline{\theta}^1{ }_b \wedge \underline{\omega}^b \\
      \underline{\theta}^2{ }_b \wedge \underline{\omega}^b \\
      \underline{\theta}^3{ }_b \wedge \underline{\omega}^b 
    \end{cases}
  \end{align*}
  Since the $\wedge$ product with $\underline{\omega}^b$ maps $\underline{\omega}^{c\neq b}$ to linearly independent two-forms, we can read the coefficients of $\underline{\omega}^{c\neq b}$ preceding the $\wedge$ product in the previous expressions. We have
  \begin{align*}
    \begin{cases}
    \underline{\theta}^0{ }_{1} =  [\cdots] \underline{\omega}^1, \quad \underline{\theta}^0{ }_{2} =  [\cdots] \underline{\omega}^2, \quad \underline{\theta}^0{ }_{3} =  [\cdots] \underline{\omega}^3\\
    \underline{\theta}^1{ }_0 = \frac{a'(t)}{a(t)}\underline{\omega}^1 + [\cdots] \underline{\omega}^0,\quad \underline{\theta}^1{ }_{2} = [\cdots] \underline{\omega}^2, \quad \underline{\theta}^1{ }_3 = \frac{1}{a(t)r}\sqrt{1-k r^2}\underline{\omega}^1 + [\cdots] \underline{\omega}^3\\
    \underline{\theta}^2{ }_0 = \frac{a'(t)}{a(t)}\underline{\omega}^2 + [\cdots]\underline{\omega}^0,\quad \underline{\theta}^2{ }_3 = \frac{1}{a(t)r}\sqrt{1-k r^2}\underline{\omega}^2 + [\cdots]\underline{\omega}^3,\quad \underline{\theta}^2{ }_1 = \frac{1}{a(t)r}\cot \theta \underline{\omega}^2 + [\cdots]\underline{\omega}^1\\
    \underline{\theta}^3{ }_0 = \frac{a'(t)}{a(t)}\underline{\omega}^3 + [\cdots]\underline{\omega}^0,\quad \underline{\theta}^3{ }_1 = [\cdots]\underline{\omega}^1, \quad \underline{\theta}^3{ }_2 = [\cdots]\underline{\omega}^2
    \end{cases}
  \end{align*} 
  where $[\cdots]$ terms represent the terms mapped to $0$ by the $\wedge$ product from which information about $\underline{\theta}^a{ }_b$ was read. From the first line we can also read $\underline{\theta}^0{ }_{1, 2, 3} = [\cdots] \underline{\omega}^{1, 2, 3}$
  
  To fully determine the one-forms components from these relations, we invoke the relation $\underline{\theta}_{ab} + \underline{\theta}_{ba} = \underline{d}g_{ab}$ where $\underline{\theta}_{ba} = g_{bc}\underline{\theta}^{c}{}_a$. Recalling that in our orthonormal basis $g_{ab} = \eta_{ab}$, we get the antisymmetry relation $\underline{\theta}_{ab} + \underline{\theta}_{ba}$. It follows that $\underline{\theta}^{a}{}_{a} = 0, \ \forall a$ and we can use it to determine $[...]$. Making the relation between $\underline{\theta}^{b}{}_a$ and  $\underline{\theta}^{a}{}_b$ more explicit yields
  \begin{align*}
    \begin{cases}
      b \ \text{spacelike} \implies 
      \underline{\theta}^{b}{}_a = \eta^{bc}\underline{\theta}_{ca} = (-1) \underline{\theta}_{ba} = \underline{\theta}_{ab} \implies 
      \begin{cases}
        a \ \text{spacelike} \implies \underline{\theta}^{b}{}_a = -\underline{\theta}^{a}{}_b\\
        a \ \text{timelike} \implies \underline{\theta}^{b}{}_a =   \underline{\theta}^{a}{}_b
      \end{cases}
        \\
      b \ \text{timelike} \implies 
      \underline{\theta}^{b}{}_a = \eta^{bc}\underline{\theta}_{ca} = \underline{\theta}_{ba} = -\underline{\theta}_{ab}  \implies 
      \begin{cases}
        a \ \text{spacelike} \implies \underline{\theta}^{b}{}_a = \underline{\theta}^{a}{}_b\\
        a \ \text{timelike} \implies \underline{\theta}^{b}{}_a = -\underline{\theta}^{a}{}_b \quad \text{never happens ($a \neq b$)}
      \end{cases}
    \end{cases}
  \end{align*}
  Comparing $\underline{\theta}^{a}{}_b$ with $\underline{\theta}^{b}{}_a$, we finally see 
  \begin{align*}
    &[\cdots] \underline{\omega}^1 = \underline{\theta}^0{ }_{1} = \underline{\theta}^1{ }_{0} = \frac{a'(t)}{a(t)}\underline{\omega}^1 + [\cdots] \underline{\omega}^0 \iff \underline{\theta}^1{ }_{0} = \frac{a'(t)}{a(t)}\underline{\omega}^1, \quad \underline{\theta}^0{ }_{1} = \frac{a'(t)}{a(t)}\underline{\omega}^1\\
    &[\cdots] \underline{\omega}^2 = \underline{\theta}^0{ }_{2} = \underline{\theta}^2{ }_{0} = \frac{a'(t)}{a(t)}\underline{\omega}^2 + [\cdots] \underline{\omega}^0 \iff \underline{\theta}^2{ }_{0} = \frac{a'(t)}{a(t)}\underline{\omega}^2, \quad \underline{\theta}^0{ }_{2} = \frac{a'(t)}{a(t)}\underline{\omega}^2\\
    &[\cdots] \underline{\omega}^3 = \underline{\theta}^0{ }_{3} = \underline{\theta}^3{ }_{0} = \frac{a'(t)}{a(t)}\underline{\omega}^3 + [\cdots] \underline{\omega}^0 \iff \underline{\theta}^3{ }_{0} = \frac{a'(t)}{a(t)}\underline{\omega}^3, \quad \underline{\theta}^0{ }_{3} = \frac{a'(t)}{a(t)}\underline{\omega}^3\\
    &[\cdots] \underline{\omega}^2 = \underline{\theta}^1{ }_{2} = -\underline{\theta}^2{ }_{1} = -\frac{1}{a(t)r}\cot \theta \underline{\omega}^2 - [\cdots]\underline{\omega}^1 \iff \underline{\theta}^1{ }_{2} = -\frac{1}{a(t)r}\cot \theta \underline{\omega}^2, \quad \underline{\theta}^2{ }_{1} = \frac{1}{a(t)r}\cot \theta \underline{\omega}^2\\
    &[\cdots] \underline{\omega}^2 = \underline{\theta}^3{ }_{2} = -\underline{\theta}^2{ }_{3} = -\frac{1}{a(t)r}\sqrt{1-k r^2}\underline{\omega}^2 - [\cdots]\underline{\omega}^3 \iff \underline{\theta}^3{ }_{2} = -\frac{1}{a(t)r}\sqrt{1-k r^2}\underline{\omega}^2, \quad \underline{\theta}^3{ }_{2} = \frac{1}{a(t)r}\sqrt{1-k r^2}\underline{\omega}^2\\
    &[\cdots] \underline{\omega}^1 = \underline{\theta}^3{ }_{1} = -\underline{\theta}^1{ }_{3} = -\frac{1}{a(t)r}\sqrt{1-k r^2}\underline{\omega}^1 - [\cdots]\underline{\omega}^3 \iff \underline{\theta}^3{ }_{1} = -\frac{1}{a(t)r}\sqrt{1-k r^2}\underline{\omega}^1, \quad \underline{\theta}^1{ }_{3} = \frac{1}{a(t)r}\sqrt{1-k r^2}\underline{\omega}^1
  \end{align*}
  
  \item[(c)] The curvature two-forms are obtained from the connection one-forms calculated above with the relation $\underline{R}^a{}_b=\underline{d \theta}^a{}_b+\underline{\theta}^a{}_c \wedge \underline{\theta}^c{}_b$. Using $H = a'(t)/a(t)$, $A = \frac{1}{a(t)r}\sqrt{1-k r^2}$ and $B = \frac{1}{a(t)r}\cot \theta$ the connection one form can be organised as 
  \begin{align*}
    [\underline{\theta}^{a}{}_{b}] = 
    \begin{pmatrix}
      0 & H\underline{\omega}^1 & H\underline{\omega}^2 & H\underline{\omega}^3\\
      H\underline{\omega}^1 & 0 & B\underline{\omega}^2 & A\underline{\omega}^1\\
      H\underline{\omega}^2 & -B \underline{\omega}^2 & 0 & A\underline{\omega}^2\\
      H\underline{\omega}^3 & -A\underline{\omega}^1 & -A\underline{\omega}^2 & 0
    \end{pmatrix} 
  \end{align*}
  and the second term in the curvature two-forms can be expressed as a matrix multiplication where the elementwise multiplication is a $\wedge$. We have 
  \begin{align*}
    &[\underline{\theta}^a{}_c \wedge \underline{\theta}^c{}_b]\\
    &= \begin{pmatrix}
      0 & H\underline{\omega}^1 & H\underline{\omega}^2 & H\underline{\omega}^3\\
      H\underline{\omega}^1 & 0 & B\underline{\omega}^2 & A\underline{\omega}^1\\
      H\underline{\omega}^2 & -B \underline{\omega}^2 & 0 & A\underline{\omega}^2\\
      H\underline{\omega}^3 & -A\underline{\omega}^1 & -A\underline{\omega}^2 & 0
    \end{pmatrix} \wedge \begin{pmatrix}
      0 & H\underline{\omega}^1 & H\underline{\omega}^2 & H\underline{\omega}^3\\
      H\underline{\omega}^1 & 0 & B\underline{\omega}^2 & A\underline{\omega}^1\\
      H\underline{\omega}^2 & -B \underline{\omega}^2 & 0 & A\underline{\omega}^2\\
      H\underline{\omega}^3 & -A\underline{\omega}^1 & -A\underline{\omega}^2 & 0
    \end{pmatrix} \\
    &=
    \begin{pmatrix}
      0 & H\underline{\omega}^2 \wedge(-B \underline{\omega}^2) + H\underline{\omega}^3 \wedge (-A\underline{\omega}^1)  & H\underline{\omega}^1 \wedge(B \underline{\omega}^2) + H\underline{\omega}^3 \wedge(-A \underline{\omega}^2) & 0\\
      A\underline{\omega}^1 \wedge(H\underline{\omega}^3) & 0 & H\underline{\omega}^1 \wedge(H\underline{\omega}^2) + A\underline{\omega}^1 \wedge(-A \underline{\omega}^2) & H \underline{\omega}^1\wedge(H \underline{\omega}^3)\\
      -B \underline{\omega}^2\wedge(H\underline{\omega}^1) + A\underline{\omega}^2 \wedge(H \underline{\omega}^3) & H \underline{\omega}^2\wedge(H \underline{\omega}^1) + A\underline{\omega}^2\wedge(-A \underline{\omega}^1) & 0 & H\underline{\omega}^2\wedge(H \underline{\omega}^3) -B \underline{\omega}^2 \wedge(A \underline{\omega}^1)\\
      0 & H\underline{\omega}^3\wedge(H \underline{\omega}^1) & H \underline{\omega}^3\wedge(H\underline{\omega}^2) - A\underline{\omega}^1\wedge(B\underline{\omega}^2) & 0
    \end{pmatrix}\\
    &=
    \begin{pmatrix}
      0 & (HA)\underline{\omega}^1 \wedge \underline{\omega}^3  & (HB)\underline{\omega}^1 \wedge\underline{\omega}^2 - (HA)\underline{\omega}^3 \wedge\underline{\omega}^2 & 0\\
      (HA)\underline{\omega}^1 \wedge \underline{\omega}^3 & 0 & (H^2-A^2)\underline{\omega}^1 \wedge \underline{\omega}^2 & (H^2) \underline{\omega}^1\wedge \underline{\omega}^3\\
      (HB) \underline{\omega}^1\wedge \underline{\omega}^2 - (HA)\underline{\omega}^3 \wedge\underline{\omega}^2 & -(H^2-A^2) \underline{\omega}^1\wedge\underline{\omega}^2 & 0 & (H^2)\underline{\omega}^2\wedge \underline{\omega}^3 - (AB) \underline{\omega}^2 \wedge \underline{\omega}^1\\
      0 & -(H^2)\underline{\omega}^1 \wedge \underline{\omega}^3 & -((H^2) \underline{\omega}^2\wedge \underline{\omega}^3 - (AB)\underline{\omega}^2\wedge\underline{\omega}^1) & 0
    \end{pmatrix}
  \end{align*}
  Then, the first term in the curvature two-forms reads
  \begin{align*}
    &[\underline{d \theta}^a{}_b] = \begin{pmatrix}
      0 & H'\underline{d}t \wedge \underline{\omega}^1 + H \underline{d\omega}^1 & H'\underline{d}t \wedge \underline{\omega}^2 + H\underline{d\omega}^2  &H'\underline{d}t \wedge \underline{\omega}^3 + H\underline{d\omega}^3\\
      +[\cdots] & 0 & (\partial_r B \underline{d}r + \partial_\theta B \underline{d}\theta + \partial_t B \underline{d}t) \wedge \underline{\omega}^2 + B \underline{d\omega}^2 & (\partial_r A \underline{d}r + \partial_t A \underline{d}t) \wedge \underline{\omega}^1 + A \underline{d\omega}^1\\
      +[\cdots] & -[\cdots] & 0 & (\partial_r A \underline{d}r  + \partial_t A \underline{d}t) \wedge \underline{\omega}^2 + A \underline{d\omega}^2\\
      +[\cdots] & -[\cdots] & -[\cdots] & 0
    \end{pmatrix}\\
    &
    \begin{cases}
      H'\underline{d}t \wedge \underline{\omega}^1 + H \underline{d\omega}^1 = H'\underline{\omega}^0 \wedge \underline{\omega}^1 + H^2\underline{\omega}^0 \wedge \underline{\omega}^1 + (HA)\underline{\omega}^3 \wedge \underline{\omega}^1\\
      %
      H'\underline{d}t \wedge \underline{\omega}^2 + H\underline{d\omega}^2 = H'\underline{\omega}^0 \wedge \underline{\omega}^2 + H^2\underline{\omega}^0 \wedge \underline{\omega}^2 + (HA)\underline{\omega}^3 \wedge \underline{\omega}^2 + (HB) \underline{\omega}^1 \wedge \underline{\omega}^2\\ 
      %
      H'\underline{d}t \wedge \underline{\omega}^3 + H\underline{d\omega}^3 = H'\underline{\omega}^0 \wedge \underline{\omega}^3 + H^2 \underline{\omega}^0 \wedge \underline{\omega}^3\\
      %
      (\partial_r B \underline{d}r + \partial_\theta B \underline{d}\theta + \partial_t B \underline{d}t) \wedge \underline{\omega}^2 + B \underline{d\omega}^2 = -(AB) \underline{\omega}^3\wedge \underline{\omega}^2 - \frac{\csc^2(\theta)}{r^2 a(t)^2} \underline{\omega}^1\wedge \underline{\omega}^2  -(BH)  \underline{\omega}^0 \wedge \underline{\omega}^2 + (BH) \underline{\omega}^0 \wedge \underline{\omega}^2 +  (AB) \underline{\omega}^3 \wedge \underline{\omega}^2 + B^2 \underline{\omega}^2 \wedge \underline{ \omega}^1\\
      %
      (\partial_r A \underline{d}r + \partial_t A \underline{d}t) \wedge \underline{\omega}^1 + A \underline{d\omega}^1 = (-A^2 + \frac{k}{a(t)^2}) \underline{\omega}^3 \wedge \underline{\omega}^1 - (HA) \underline{\omega}^0 \wedge \underline{\omega}^1 + (HA)\underline{\omega}^0 \wedge \underline{\omega}^1 + A^2\underline{\omega}^3 \wedge \underline{\omega}^1\\
      %
      (\partial_r A \underline{d}r  + \partial_t A \underline{d}t) \wedge \underline{\omega}^2 + A \underline{d\omega}^2 = (-A^2 + \frac{k}{a(t)^2}) \underline{\omega}^3 \wedge \underline{\omega}^2 - (HA) \underline{\omega}^0 \wedge \underline{\omega}^2 + (HA) \underline{\omega}^0 \wedge \underline{\omega}^2 + A^2 \underline{\omega}^3 \wedge \underline{\omega}^2 + (AB) \underline{\omega}^1 \wedge \underline{\omega}^2
    \end{cases}\\
    &=\begin{cases}
      (H'+H^2)\underline{\omega}^0 \wedge \underline{\omega}^1 + (HA)\underline{\omega}^3 \wedge \underline{\omega}^1\\
      %
      (H' + H^2)\underline{\omega}^0 \wedge \underline{\omega}^2 + (HA)\underline{\omega}^3 \wedge \underline{\omega}^2 + (HB) \underline{\omega}^1 \wedge \underline{\omega}^2\\ 
      %
      (H' + H^2) \underline{\omega}^0 \wedge \underline{\omega}^3\\
      %
      -\left(\frac{\csc^2(\theta)}{r^2 a(t)^2} + B^2 \right)\underline{\omega}^1 \wedge \underline{\omega}^2\\
      %
      \frac{k}{a(t)^2} \underline{\omega}^3 \wedge \underline{\omega}^1\\
      %
      \frac{k}{a(t)^2} \underline{\omega}^3 \wedge \underline{\omega}^2 + (AB) \underline{\omega}^1 \wedge \underline{\omega}^2
    \end{cases}
  \end{align*}
  Summing the two terms leads to 
  \begin{align*}
    [\underline{R}^a{}_b] =
    \begin{pmatrix}
      0 & (H'+H^2)\underline{\omega}^0 \wedge \underline{\omega}^1  & (H' + H^2)\underline{\omega}^0 \wedge \underline{\omega}^2 + (2HB) \underline{\omega}^1 \wedge \underline{\omega}^2 & (H' + H^2) \underline{\omega}^0 \wedge \underline{\omega}^3\\
      %
      +[\cdots] & 0 & (H^2-A^2 -B^2 -\frac{\csc^2(\theta)}{r^2 a(t)^2})\underline{\omega}^1 \wedge \underline{\omega}^2 & (H^2-\frac{k}{a(t)^2}) \underline{\omega}^1\wedge \underline{\omega}^3\\
      %
      +[\cdots] & -[\cdots] & 0 & (H^2-\frac{k}{a(t)^2})\underline{\omega}^2\wedge \underline{\omega}^3\\
      %
      +[\cdots] & -[\cdots] & -[\cdots] & 0
    \end{pmatrix} 
  \end{align*}
  \newpage
  \item[(d)] From each curvature two-form found above, we can extract the components of the Riemann tensor with $\underline{R}^{a}{ }_{b}=\frac{1}{2} R^{a}{ }_{bcd} \underline{\omega}^{c} \wedge \underline{\omega}^{d}$. To make these components more transparent we use the new notation $0 \to \hat{t}, 1 \to \hat{\theta}, 2 \to \hat{\phi}, 3 \to \hat{r}$ and the only non-vanishing components of the Riemann tensor (up to symmetry property of indices) are 
  \begin{align*}
    2R^{\hat{t}}_{\hat{\phi} \hat{t} \hat{\phi}} =  2R^{\hat{t}}_{\hat{\theta} \hat{t} \hat{\theta}} = 2R^{\hat{t}}_{\hat{r} \hat{t} \hat{r}} = H' + H^2, \quad R^{\hat{t}}_{\hat{\phi} \hat{\theta} \hat{\phi}} = HB, \quad  2R^{\hat{\theta}}_{\hat{\phi} \hat{\theta} \hat{\phi}} = H^2-A^2 -B^2 -\frac{\csc^2(\theta)}{r^2 a(t)^2},\quad  2R^{\hat{\theta}}_{\hat{r} \hat{\theta}\hat{r}} = 2R^{\hat{\phi}}_{\hat{r} \hat{\phi}\hat{r}} = H^2-\frac{k}{a(t)^2}.
  \end{align*}
\end{enumerate}

\section{Acknowledgement}

Thanks to Luke for help reviewing and understanding the concepts used in this assignment

Thanks to Thomas for a discussion about the subtleties of "reading" connection one-forms from Cartan's structure equation

Thanks to Thomas for help verifying my answers for (b)


}

% References
\makereferences
%-------------------------------------------------------


%%%%%%%%%%%%%%%%%%%%%%%%
% Terminer le document %
%%%%%%%%%%%%%%%%%%%%%%%%
\end{document}
