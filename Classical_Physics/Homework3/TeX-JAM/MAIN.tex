\documentclass[10pt, a4paper]{article}

%%%%%%%%%%%%%%
%  Packages  %
%%%%%%%%%%%%%%


\usepackage{page_format}
\usepackage{special}
%----------------------------------------------------------------------
%\usepackage{amssymb} % Mathematical fonts.
%\usepackage{amsfonts} % Mathematical fonts.
\usepackage[nice]{nicefrac} % Nicer fractions
\usepackage{braket} % Dirac Notation.
\usepackage{bbm} % More bold fonts.
%\usepackage{mathrsfs} % Mathematical fonts.
\usepackage{esint} % Integrals
\usepackage{cancel} % Allows to scratch expressions.
\usepackage{mathtools} % Tools for math formating.
\usepackage{slashed} % Allows to slash individual characters.
\usepackage{xargs} % Better handling of optional arguments for commands
%----------------------------------------------------------------------
%\usepackage{lmodern} % Fonts.
\usepackage{feyn} % Feynman Diagrams in mathmode

%%%%%%%%%%%%%%%%%%%%%%%%%%%
% Mathématiques et physique
%%%%%%%%%%%%%%%%%%%%%%%%%%%%
% SI Units -----------------------
% The package 'siunitx' causes unresolved crashes (as of 22/08/31)
\newcommand{\ampere}{\text{A}}
\newcommand{\bell}{\text{B}}
\newcommand{\celsius}{\degree\text{C}}
\newcommand{\coulomb}{\text{C}}
\newcommand{\degree}{\,^{\circ}}
\newcommand{\farad}{\text{F}}
\newcommand{\electro}{\text{e}}
\newcommand{\gram}{\text{g}}
\newcommand{\henry}{\text{H}}
\newcommand{\hertz}{\text{Hz}}
\newcommand{\hour}{\text{h}}
\newcommand{\joule}{\text{J}}
\newcommand{\kelvin}{\text{K}}
\newcommand{\meter}{\text{m}}
\newcommand{\minute}{\text{m}}
\newcommand{\mole}{\text{mol}}
\newcommand{\newton}{\text{N}}
\newcommand{\ohm}{\Omega}
\newcommand{\pascal}{\text{Pa}}
\newcommand{\rad}{\text{rad}}
\newcommand{\second}{\text{s}}
\newcommand{\tesla}{\text{T}}
\newcommand{\torr}{\text{Torr}}
\newcommand{\volt}{\text{V}}
\newcommand{\watt}{\text{W}}
%
\newcommand{\tera}{\text{T}}
\newcommand{\giga}{\text{G}}
\newcommand{\mega}{~\text{M}}
\newcommand{\kilo}{~\text{k}}
\newcommand{\deci}{\text{d}}
\newcommand{\centi}{\text{c}}
\newcommand{\milli}{\text{m}}
\newcommand{\micro}{\mu}
\newcommand{\nano}{\text{n}}
\newcommand{\pico}{\text{p}}
\newcommand{\femto}{\text{f}}
%
\newcommand{\units}[1]{\text{#1}}
\newcommand{\tothe}[1]{\textsuperscript{#1}}
%
\newcommand{\per}{\text{/}}
%
\newcommand{\Time}[3]{#1\hour~#2\minute~#3\second} % TODO Optional arguments.
\newcommand{\Angle}[3]{#1^{\circ}~#2'~#3''} % TODO Optional arguments.


% Better epsilon -----------------------
\let\oldepsilon\epsilon
\let\epsilon\varepsilon
\let\varepsilon\oldepsilon


% Better \bar -----------------------
\renewcommand{\bar}[1]{\mkern 1.5mu\overline{\mkern-1.5mu#1\mkern-1.5mu}\mkern 1.5mu}


% Équations -----------------------
\newcommand{\al}[1]{\begin{align} #1 \end{align}} % Numbered equation(s),
\newcommand{\eqn}[1]{\begin{align*} #1 \end{align*}} % Number-less equation(s),
\newcommand{\sys}[1]{\begin{dcases*} #1 \end{dcases*}} % System of equations.


% Exponents -----------------------
\newcommand{\Exp}[1]{\text{e}^{#1}}		% e^#
\newcommand{\E}[1]{\times 10^{#1}}		% X 10^#


% Delimiters -----------------------
\newcommand{\p}[1]{\left( #1 \right)}	% (#)
\newcommand{\cro}[1]{\left[ #1 \right]}	% [#]
\newcommand{\abs}[1]{\left| #1\right|}	% |#|
\newcommand{\avg}[1]{\left\langle #1 \right\rangle} % <#>
\newcommand{\acc}[1]{\left\lbrace #1 \right\rbrace} % {#}


% Vectors -----------------------
\newcommand{\ve}[1]{\mathbf{#1}} % Upright bold face.
\newcommand{\vu}[1]{\hat{\ve{#1}}} % Hat vector upright bold face
\newcommand{\tens}{\otimes} % Tensor product
\newcommand{\nablav}{\bm{\nabla}} % Bold gradient


% Trig. functions with automatic formating  -----------------------
\newcommandx{\Sin}[2][1={}]{\text{sin}^{#1}\!\p{#2}}
\newcommandx{\Cos}[2][1={}]{\text{cos}^{#1}\!\p{#2}}
\newcommandx{\Tan}[2][1={}]{\text{tan}^{#1}\!\p{#2}}
\newcommandx{\Csc}[2][1={}]{\text{csc}^{#1}\!\p{#2}}
\newcommandx{\Sec}[2][1={}]{\text{sec}^{#1}\!\p{#2}}
\newcommandx{\Cot}[2][1={}]{\text{cot}^{#1}\!\p{#2}}
\newcommandx{\Arcsin}[2][1={}]{\text{arcsin}^{#1}\!\p{#2}}
\newcommandx{\Arccos}[2][1={}]{\text{arccos}^{#1}\!\p{#2}}
\newcommandx{\Arctan}[2][1={}]{\text{arctan}^{#1}\!\p{#2}}
\newcommandx{\Sinh}[2][1={}]{\text{sinh}^{#1}\!\p{#2}}
\newcommandx{\Cosh}[2][1={}]{\text{cosh}^{#1}\!\p{#2}}
\newcommandx{\Tanh}[2][1={}]{\text{tanh}^{#1}\!\p{#2}}


% Matrices -----------------------
\newcommand{\mat}[1]{\begin{bmatrix} #1 \end{bmatrix}} % Matrices with hooks.
\newcommand{\pmat}[1]{\begin{pmatrix} #1 \end{pmatrix}} % Matrices with parentheses.
\newcommand{\deter}[1]{\abs{\begin{matrix} #1 \end{matrix}}} % Determinant.
\newcommandx{\mO}[2][1={}, 2={}]{ \def\temp{#2}\ifx\temp\empty\ve{O}_{#1}\else\ve{O}_{#1\times #2}\fi}% Zero matrix.
\newcommandx{\mI}[2][1={}, 2={}]{ \def\temp{#2}\ifx\temp\empty\ve{I}_{#1}\else\ve{O}_{#1\times #2}\fi}%  Identity matrix.
\newcommand{\Det}[1]{\text{det}\p{#1}} % det(#)
\newcommand{\Tr}[1]{\text{Tr}\p{#1}} % Tr(#)


% Derivatives -----------------------
\newcommand{\D}{\text{d}} % Differential 'd'.
\newcommandx{\dd}[3][1={},3={}]{\frac{\D^{#3}#1}{\D{#2}^{#3}}} % Total derivative according to #2, #1 is the function and #3 is the order.
\newcommand{\del}{\partial} % Partial 'd'.
\newcommandx{\ddp}[3][1={},3={}]{\frac{\del^{#3}#1}{\del{#2}^{#3}}} % Dérivée partielle selon #2, #1 est la fonction est #3 est l'ordre.
\newcommand{\eval}[1]{\left. {#1} \right|} % Bar on the right of expression.
\newcommand{\delbar}{\slashed{\del}} % Partial Inexact differential.
\newcommand{\dbar}{\dj}% Inexact differential.


% Integrals -----------------------
\newcommand{\intinf}{\int\displaylimits_{-\infty}^{\infty}} % From -00 to 00.
\newcommandx{\Int}[2][1={},2={}]{\int\displaylimits_{#1}^{#2}} % Faster bounded integrals.


% Complex numbers -----------------------
\renewcommand{\Re}[1]{\text{Re}\acc{#1}} % Re{#}
\renewcommand{\Im}[1]{\text{Im}\acc{#1}} % Im{#}


% Sets -----------------------
\newcommand{\N}{\mathbbm{N}} % Natural numbers.
\newcommand{\Z}{\mathbbm{Z}} % Integers.
\newcommand{\Q}{\mathbbm{Q}} % Rational numbers.
\newcommandx{\R}[1][1={}]{\mathbbm{R}^{#1}} % Real numbers.
\newcommandx{\C}[1][1={}]{\mathbbm{C}^{#1}} % Complex numbers.
\newcommandx{\F}[1][1={}]{\mathbbm{F}^{#1}} % Some field.
\newcommand{\M}[3]{\mathbb{M}_{#1\times#2}(#3)}	% Matrices.
\newcommand{\Po}[2]{\mathbb{P}_{#1}(#2)} % Polynomials.
\newcommand{\Lin}{\mathbb{L}} % Linear maps.


% Constants and physical symbols -----------------------
\newcommand{\eo}{\epsilon_0} % epsilon 0.
\renewcommand{\L}{\mathcal{L}} % Lagrangian.

% References
\usepackage{biblatex}
\addbibresource{ref.bib}


%%%%%%%%%%%%
%  Colors  %
%%%%%%%%%%%%
% ! EDIT HERE !
\colorlet{chaptercolor}{red!70!black} % Foreground color.
\colorlet{chaptercolorback}{red!10!white} % Background color


%%%%%%%%%%%%%%
% Page titre %
%%%%%%%%%%%%%%
\title{Homework 3} % Title of the assignement.
\author{\PA} % Your name(s).
\teacher{Aldo Riello} % Your teacher's name.
\class{Classical Physics} % The class title.

\university{Perimeter Institute for Theoretical Physics} % University
\faculty{Perimeter Scholars International} % Faculty
%\departement{<Departement>} % Departement
\date{\today} % Date.


%%%%%%%%%%%%%%%%%%%%%%t
% Begin the document %
%%%%%%%%%%%%%%%%%%%%%%
\begin{document}

% Make the title page.
\maketitlepage

% Make table of contents
\maketableofcontents

\footnotesize{
% Assignment starts here ----------------------------
\section{Planar electromagnetic waves}
\subsection{Maxwell equations for the four-potential}
The components of the contravariant four potential are $A^{\mu} = (\varphi, \mathbf{A})$ ($A_{\mu} = (-\varphi, \mathbf{A})$ for the covariant components) where $\varphi$ is the electric potential and $\mathbf{A}$ is the magnetic potential vector. The sources generating each component of $A^{\mu}$ can be grouped in a current four vector $j^{\mu}=(\rho, \mathbf{j})$ ($j_{\mu} = (-\rho, \mathbf{j})$ for the covariant components) where $\rho$ is the charge density and  $\mathbf{j}$  is the current observed in the reference frame where we solve for $A^{\mu}$. In the lorentz gauge $0=\nabla_{\mu} A^{\mu}$, the Maxwell equations for $A^{\mu}$ with sources $j^{\mu}$ read $\Box A^{\mu} = -4\pi j^{\mu}$ ($\Box A_{\mu} = -4\pi j_{\mu}$ for the covariant components). 

\subsection{Plane wave Ansatz}
We now solve the Maxwell equations in the Lorentz gauge, by introducing the plane wave ansatz $A_\mu(t, \mathbf{x}) = a_\mu \exp \left(i k_\mu x^\mu\right)$ where $k^{\mu}= (\mu, \mathbf{k})$ is the four wave vector and $a^{\mu}$ is the four amplitude. On one hand, substituting this ansatz in the Lorentz gauge condition, we get 
\begin{align*}
    0 = \nabla_{\mu} A^{\mu} = \nabla_{\mu} \left(a^\mu \exp \left(i k_\nu x^\nu\right)\right) = a^\mu i\delta_\mu^\nu k_{\nu} \exp \left(i k_\nu x^\nu\right) = (a^{\mu} k_{\mu}) \exp \left(i k_\nu x^\nu\right) \iff  a^{\mu} k_{\mu} = 0.
\end{align*}
On the other hand, substituting the ansatz in the vacuum Maxwell equations ($ j_{\mu}$) yields 
\begin{align*}
    0 =  \nabla^{\mu}\nabla_{\mu} A_{\nu} =  i\delta_\mu^\rho k_{\rho} \nabla^{\mu} (\exp \left(i k^\rho x_\rho\right)) = -k^{\mu} k_{\mu} \exp \left(i k^\rho x_\rho\right) \iff k^{\mu} k_{\mu} = 0
\end{align*}
so the four wave vector is light-like in the vacuum. 

\subsection{Electric and Magnetic fields}
In terms of $A_\mu$, the electric and magnetic fields $\mathbf{E}, \mathbf{B}$ can be written as 
\begin{align*}
    &\mathbf{E} = \nabla_j A_0 -\nabla_0 \mathbf{A} = a_0\nabla_j\exp \left(i k_\mu x^\mu\right) - \mathbf{a} \nabla_0  \exp \left(i k_\mu x^\mu\right) = (i a_0 \mathbf{k} - i\mathbf{a} k_0)  \exp \left(i k_\mu x^\mu\right),\\
    &\mathbf{B} =  \epsilon_i^{jk} \nabla_j A_{k} = i\epsilon_i^{\ jk} k_j a_k\exp \left(i k_\mu x^\mu\right) = i\mathbf{k} \times \mathbf{a}\exp \left(i k_\mu x^\mu\right) 
\end{align*} with $\mathbf{A}$, $\mathbf{a}$ and $\mathbf{k}$ are respectively the spatial components of $A_{\mu}$, $a_{\mu}$ and $k_{\mu}$. We consider the projection of $\mathbf{E}, \mathbf{B}$ along $\mathbf{k}$. We define  $\mathbf{n} := \mathbf{k}/k$ to write the projections 
\begin{align*}
    &\mathbf{n} \cdot \mathbf{E} = \mathbf{k}/k \cdot \mathbf{E}= (i a_0 \mathbf{k}^2 - i\mathbf{k} \cdot \mathbf{a} k_0)  \exp \left(i k_\mu x^\mu\right)/k =  (i a_0 (k_0^2) - i(k_0 a_0) k_0)  \exp \left(i k_\mu x^\mu\right)/k = 0, \\
    &\mathbf{n} \cdot \mathbf{B} = \mathbf{k} \cdot \left(i\mathbf{k} \times \mathbf{a}\exp \left(i k_\mu x^\mu\right)\right)/k = 0.
\end{align*}
Furthermore, we can relate $\mathbf{E}$ and $\mathbf{B}$ in the following way:
\begin{align*}
    \mathbf{k} \times \mathbf{B}/k_0 &= i \mathbf{k} \times (\mathbf{k} \times \mathbf{a})\exp \left(i k_\mu x^\mu\right)\\ &=  i\left((\mathbf{k} \cdot \mathbf{a})\mathbf{k} - (\mathbf{k} \cdot \mathbf{k})\mathbf{a}\right)\exp \left(i k_\mu x^\mu\right)/k_0 \\
    &=  i\left((k_0 a_0)\mathbf{k} - (k_0^2)\mathbf{a}\right)\exp \left(i k_\mu x^\mu\right)/k_0\\
    &= i\left(a_0\mathbf{k} - k_0\mathbf{a}\right)\exp \left(i k_\mu x^\mu\right) = \mathbf{E}.
\end{align*}
Since $k_0^2-\mathbf{k}^2 = 0$ and $\mathbf{n} = \mathbf{k}/\sqrt{\mathbf{k}^2}$, $\mathbf{k} \times \mathbf{B}/k_0 = \mathbf{n} \times \mathbf{B} = \mathbf{E}$. The conclusion of these calculations is that $\mathbf{E}, \mathbf{B}$ are orthogonal to each oother and to the direction of propagation of the wave given by $\mathbf{k}$. To analyse the phase difference between $\mathbf{E
}$ and $\mathbf{B}$,  we notice that the global phase in $\mathbf{E}$ is the phase of the complex quantity $a_0\mathbf{k} - k_0\mathbf{a}$ and that the global phase in $\mathbf{B}$ is the phase in $\mathbf{a}$. 

\subsection{Linearly polarized waves}

In what follows, we set $A^0 = -\varphi = 0$, $a^0 = 0$ which corresponds to having a $0$ electric potential everywhere. The time derivative of the spatial components of four potential is
\begin{align*} 
    \dot{\mathbf{A}} = i k_0 \mathbf{a} \exp \left(i k_\mu x^\mu\right).
\end{align*}
It can be used to express $\mathbf{E}, \mathbf{B}$ when the $a^0 = 0$. Indeed
\begin{align*}
    \mathbf{E} =  -\mathbf{\dot{A}} =  (i (0) \mathbf{k} - i\mathbf{a} k_0)  \exp \left(i k_\mu x^\mu\right),\quad \mathbf{B} =  \mathbf{n} \times \dot{\mathbf{A}}
\end{align*}
\subsection{Poynting vector}
The energy-momentum transport associated to the electromagnetic field is described by the Poynting vector $\mathbf{S} = \mathbf{E} \times \mathbf{B}$. Here, we want to relate $\mathbf{S}$ to the electromagnetic energy density $\epsilon = (\mathbf{E}^2 +  \mathbf{B}^2)/2$. To do so, we differenciate $\epsilon$ with respect to time to get 
\begin{align*}
    \dfrac{\partial \epsilon}{\partial t} = \mathbf{E} \cdot \dfrac{\partial \mathbf{E}}{\partial t} + \mathbf{B} \cdot \dfrac{\partial \mathbf{B}}{\partial t} = \mathbf{E} \cdot \left(\nabla \times \mathbf{B}\right) - \mathbf{B} \cdot \left(\nabla \times \mathbf{E}\right) = - \nabla  \cdot \left(\mathbf{E} \times \mathbf{B}\right) \iff 0 = \dfrac{\partial \epsilon}{\partial t} + \nabla  \cdot \mathbf{S}
\end{align*}
where we have used the Faraday and Vacuum Ampere laws to express the partial derivatives. A continuity equation is found and we interpret $\mathbf{S}$ as the energy current density. 

\subsection{Asymptotic Power}
Following the analogy with the charge continuity equation, we can write an integral form of the energy continuity equation. We choose a spherical volume $V$ surrounded by a sphere surface $\partial V$ at radius $R$ with outgoing normal $\mathbf{n}_P$. Integrating he continuity equation for $\epsilon$ and $\mathbf{S}$, we get 
\begin{align*}
   0 = \int_V \text{d}^3 r \left(\dfrac{\partial \epsilon}{\partial t} + \nabla  \cdot \mathbf{S}\right) = \dfrac{\partial }{\partial t} \left(\int_V \text{d}^3 r \epsilon\right) + \int_V \text{d}^3 r \ \nabla  \cdot \mathbf{S} = \dfrac{d E}{d t}  + R^2\int_{\partial V} \sin(\theta) \text{d}\phi \text{d}\theta (\mathbf{n}_P \cdot \mathbf{S})
\end{align*}
Where $E$ represents the total electromagnetic energy in $V$. If $R$ is big enough compared to the caracteristic size of the emitting system, only radiation directed to infinity goes trough it and $\frac{dE}{dt}$ represents the total radiation power of the system. 

\subsection{Poynting vector for planar waves}
For planar waves, we have the following poynting vector:
\begin{align*}
    \mathbf{S} &= \mathbf{E}\times\mathbf{B} = -\mathbf{B} \times (\mathbf{n} \times \mathbf{B}) =- (\mathbf{B} \cdot \mathbf{n}) \mathbf{B} + (\mathbf{B} \cdot \mathbf{B}) \mathbf{n} = \dfrac{\mathbf{B}^2 + \mathbf{E}^2}{2} \mathbf{n} = \epsilon \mathbf{n} 
\end{align*}
where we used $\mathbf{E} = \mathbf{n} \times \mathbf{B}$, $0 = \mathbf{n} \cdot \mathbf{B}$ and $\mathbf{B}^2 = (\mathbf{n} \times \mathbf{B})^2 =  \mathbf{E}^2$. 
% thanks to Maita 
\section{Radiation of an isolated system}
\subsection{Lienard–Wiechert potential with isolated sources}
The Lienard–Wiechert potential provides an expression for the four-potential generated by a charge moving on a world line. 

Supposing the charges are moving slowly compared to the speed of light, the three potential $\mathbf{A}$ contribution at time $t$ and position $\mathbf{r}$ of a point charge charge $q$ with three-velocity $\mathbf{v}$ and three-position $\mathbf{r}'$ at time $t_R = t-|\mathbf{r} - \mathbf{r}'|$ reads: 
\begin{align*}
   \mathbf{A} =  \dfrac{q \mathbf{v}(t_R)}{|\mathbf{r} - \mathbf{r}'| - \mathbf{v}(t_R) \cdot (\mathbf{r} - \mathbf{r}')} \approx  \dfrac{q \mathbf{v}(t_R)}{|\mathbf{r} - \mathbf{r}'|} + O(|\mathbf{v}|^2)
\end{align*}
Here we are interested in the integrated potential generated by a continum of charges described by charge density $\rho(t, \mathbf{r})$ and a three-curent $\mathbf{j}(t, \mathbf{r})$ at time $t$ and cartesian three-position $\mathbf{r}$. In the limit of small velocities, the previous expression can be formulated in the charge continuum by replacing $q \mathbf{v}(t_R)$ by  the integral expression of the magnetic potential is given by $\mathbf{j}(t_R, \mathbf{r}')$ and integrating over a space-slice to combine the contribution of all sources. We have 
\begin{align*}
    \mathbf{A}(t, \mathbf{r}) = \int \text{d}^3r' \dfrac{\mathbf{j}(t_R, \mathbf{r}')}{|\mathbf{r} - \mathbf{r}'|}.
\end{align*}
If the observation point $\mathbf{r}$ of the three-potential is far from the sources, we can write 
\begin{align*}
    \mathbf{A}(t, \mathbf{r}) &= \int_{\mathbf{j}(t_R, \mathbf{r}') \sim 0,\ |\mathbf{r}|\sim |\mathbf{r}'|} \text{d}^3r' \dfrac{\mathbf{j}(t_R, \mathbf{r}')}{|\mathbf{r} - \mathbf{r}'|} + \int_{\mathbf{j}(t_R, \mathbf{r}') \not\sim 0,\ |\mathbf{r}|\gg |\mathbf{r}'|} \text{d}^3r' \dfrac{\mathbf{j}(t_R, \mathbf{r}')}{|\mathbf{r} - \mathbf{r}'|} \\&\approx \dfrac{1}{|\mathbf{r}|}\int_{\mathbf{j}(t_R, \mathbf{r}') \sim 0,\ |\mathbf{r}|\sim |\mathbf{r}'|} \text{d}^3r' \underbrace{\mathbf{j}(t_R, \mathbf{r}')}_{\sim 0}+ \dfrac{1}{|\mathbf{r}|}\int_{\mathbf{j}(t_R, \mathbf{r}') \not\sim 0,\ |\mathbf{r}|\gg |\mathbf{r}'|} \text{d}^3r' \mathbf{j}(t_R, \mathbf{r}') =  \dfrac{1}{|\mathbf{r}|} \int \text{d}^3r' \mathbf{j}(t_R, \mathbf{r}')\\
    &\approx \dfrac{1}{|\mathbf{r}|} \int \text{d}^3r' \mathbf{j}(t', \mathbf{r}') + \dfrac{1}{|\mathbf{r}|} \dfrac{\partial}{\partial t'}\int \text{d}^3r' \left(\mathbf{r}' \cdot \dfrac{\mathbf{r}}{|\mathbf{r}|}\right) \mathbf{j}(t', \mathbf{r}')
\end{align*}
where we have used the expansions 
\begin{align*}
    &|\mathbf{r}-\mathbf{r}'| = |\mathbf{r}|\left(\left.|\mathbf{r}-\mathbf{r}'|\right|_{\mathbf{r}' = 0}\dfrac{1}{|\mathbf{r}|} + \dfrac{\mathbf{r}'}{|\mathbf{r}|}\cdot \left.\dfrac{\partial}{\partial \mathbf{r}'} |\mathbf{r}-\mathbf{r}'|\right|_{\mathbf{r}' = 0} + O(|\mathbf{r}'|^2/|\mathbf{r}|^2)\right) = |\mathbf{r}|\left(1 - \dfrac{\mathbf{r}'}{|\mathbf{r}|}\cdot\dfrac{\mathbf{r}}{|\mathbf{r}|} + O(|\mathbf{r}'|^2/|\mathbf{r}|^2)\right),\\
    &\dfrac{1}{|\mathbf{r}-\mathbf{r}'|} =  \dfrac{1}{\mathbf{|\mathbf{r}|}} \dfrac{1}{1 - \mathbf{r}'\cdot\dfrac{\mathbf{r}}{|\mathbf{r}|^2} + O(|\mathbf{r}'|^2/|\mathbf{r}|^2)} = \dfrac{1}{\mathbf{|\mathbf{r}|}} \left(1 + \mathbf{r}'\cdot\dfrac{\mathbf{r}}{|\mathbf{r}|^2} + O(|\mathbf{r}'|^2/|\mathbf{r}|^2)\right)  \approx  \dfrac{1}{|\mathbf{r}|},\ |\mathbf{r}| \gg |\mathbf{r}|', \\
    &\mathbf{j}(t_R, \mathbf{r}') = \mathbf{j}\left(t-|\mathbf{r}| + \mathbf{r}'\cdot\dfrac{\mathbf{r}}{|\mathbf{r}|} + O(|\mathbf{r'}|^2), \mathbf{r}'\right)= \mathbf{j}(t', \mathbf{r}') + \mathbf{r}' \cdot \dfrac{\mathbf{r}}{|\mathbf{r}|} \left.\dfrac{\partial \mathbf{j}}{\partial t'} \right|_{(t', \mathbf{r'})} + O(|\mathbf{r'}|^2)\ \text{with}\ t' = t-|\mathbf{r}|.
\end{align*}

% Can we obtain the same result by writing the coulomb potential in the comoving rest frame of the particle and then boost the four potential to the frame where the particle is moving ?
\subsection{Dipole moment}
Using the divergence theorem we can write 
\begin{align*}
    A_i(t, \mathbf{r}) &\approx \dfrac{1}{|\mathbf{r}|} \int \text{d}^3r' \left[\nabla(r_i') \cdot \mathbf{j}(t', \mathbf{r}')\right] = \dfrac{1}{|\mathbf{r}|} \int \text{d}^3r' \left[\nabla \cdot(r_i'  \mathbf{j}(t', \mathbf{r}'))\right]-\dfrac{1}{|\mathbf{r}|} \int \text{d}^3r' \left[r_i' \nabla \cdot \mathbf{j}(t', \mathbf{r}')\right]
\end{align*}
where we indexed the compenents of relevant three-vectors by $i$. We used the fact $\nabla(r_i')$ is the unit vector in the $i$ direction to project $\mathbf{j}$ to its $i$ component and extract the contribution to the $i$ component of $\mathbf{A}$. Using the divergence theorem for the first integral of the right hand side can be converted to an integral on a surface $S$ with unit outgoing normal $\mathbf{n}_S$ enclosing the three-volume of the current density system (our system is infinite so we imagine pushing $S$ to infinity). Since we are trying to approximate $\mathbf{A}$ far from regions with significant $\mathbf{j}$, we take $\mathbf{j} \sim 0$ on $S$ to write 
\begin{align*}
    A_i(t, \mathbf{r}) &\approx \dfrac{1}{|\mathbf{r}|} \int_S \text{d}S [\mathbf{n}_S \cdot (r_i'  \underbrace{\mathbf{j}(t', \mathbf{r}')}_{\sim 0})]-\dfrac{1}{|\mathbf{r}|} \int \text{d}^3r' \left[r_i' \nabla \cdot \mathbf{j}(t', \mathbf{r}')\right] = -\dfrac{1}{|\mathbf{r}|} \int \text{d}^3r' \left[r_i' \nabla \cdot \mathbf{j}(t', \mathbf{r}')\right] 
\end{align*}
Using the continuity equation $-\nabla \cdot \mathbf{j}(t', \mathbf{r}') = \frac{\partial \rho(t', \mathbf{r}')}{\partial t}$, we relate the current density to the charge density to obtain
\begin{align*}
    A_i(t, \mathbf{r}) &\approx +\dfrac{1}{|\mathbf{r}|} \dfrac{\partial}{\partial t'}\int \text{d}^3r' \left[r_i' \rho(t', \mathbf{r}')\right] \iff \mathbf{A}(t, \mathbf{r}) \approx \dfrac{1}{|\mathbf{r}|} \dfrac{\partial}{\partial t'}\int \text{d}^3r' \left[\mathbf{r}' \rho(t', \mathbf{r}')\right]= \dfrac{1}{|\mathbf{r}|}\dfrac{\partial\mathbf{d}(t')}{\partial t'} := \dfrac{\dot{\mathbf{d}}}{|\mathbf{r}|}
\end{align*}
where we have introduced the time dependant electric dipole moment of the charge distribution $\mathbf{d}(t') := \int \text{d}^3r' \left[\mathbf{r}' \rho(t', \mathbf{r}')\right]$.


% div (j) r = 
\subsection{Asymptotic power}
The electromagnetic field far from sources will solve the vacuum Maxwell equations and our considerations of the previous question apply.  The total power $P$ emitted trough the spherical boundary (with outgoing unit vector $\mathbf{n}_P = \mathbf{r}(\phi, \theta)/R$) of ball with radius $R$ enclosing the system is obtained with the result of item F of the previous question. It reads 
\begin{align*}
    P = R^2 \int_{0}^{\pi} \text{d}\theta\ \sin(\theta)  \int_0^{2\pi} \text{d}\phi \ \mathbf{n}_P \cdot \mathbf{S} =  2\pi \int_{0}^{\pi} \text{d}\theta\ \sin(\theta)   \left(\left(1-\cos^2(\theta)\right)\dfrac{|\mathbf{\ddot{d}}|^2 R^2}{R^2} + \left[\sim \dfrac{R^2}{R^3}\right]\right) = 4 \pi \dfrac{2|\mathbf{\ddot{d}}|^2}{3}. 
\end{align*}
where $\theta$, $\phi$ are respectively the polar angle and longitudinal angles of a parametrisation of the boundary sphere. The poynting vector was computed assuming the electric potential decreases faster than $1/|\mathbf{r}|$ so that it doesn't contribute to the power at infinity (a decrease at rate $1/|\mathbf{r}|$ would beconsistant with the presence of net point charges which we do not consider here). Using the expression of the electric and magnetic fields in terms of $\mathbf{A}$, we can write
\begin{align*}
    \mathbf{S} &= -\dfrac{\partial \mathbf{A}}{\partial t} \times (\nabla \times \mathbf{A})= -\dfrac{1}{|\mathbf{r}|^2}\dfrac{\partial \dot{\mathbf{d}}}{\partial t} \times (\nabla \times \dot{\mathbf{d}}) -\dfrac{1}{|\mathbf{r}|}\dfrac{\partial \dot{\mathbf{d}}}{\partial t} \times (\nabla \dfrac{1}{|\mathbf{r}|} \times \dot{\mathbf{d}})= -\dfrac{1}{|\mathbf{r}|^2}\dfrac{\partial \dot{\mathbf{d}}}{\partial t} \times (\nabla \times \dot{\mathbf{d}}) + \left[\sim \dfrac{1}{|\mathbf{r}|^3}\right]= -\dfrac{1}{|\mathbf{r}|^2}\ddot{\mathbf{d}}(t') \times \left(\dfrac{\mathbf{r}}{|\mathbf{r}|} \times \ddot{\mathbf{d}}(t')\right) + \left[\sim \dfrac{1}{|\mathbf{r}|^3}\right]
\end{align*}
where we have used 
\begin{align*}
    &\dfrac{\partial \dot{\mathbf{d}}(t')}{\partial t} = \dfrac{\partial \dot{\mathbf{d}}(t')}{\partial t'} \dfrac{\partial t'}{\partial t} = \ddot{\mathbf{d}}(t'),\\
    &\dfrac{\partial}{\partial r_i} \dot{\mathbf{d}}(t') = \dfrac{\partial \dot{\mathbf{d}}(t')}{\partial t'} \dfrac{\partial t'}{\partial r_i} = -\ddot{\mathbf{d}}(t') \dfrac{r_i}{|\mathbf{r}|} \implies \epsilon_{ijk}\dfrac{\partial}{\partial r_i}  \dot{d_j}(t') =  -\epsilon_{ijk} \ddot{d_j}(t') \dfrac{r_i}{|\mathbf{r}|} \implies \nabla \times \dot{\mathbf{d}}(t') = -\dfrac{\mathbf{r}}{|\mathbf{r}|} \times \ddot{\mathbf{d}}(t').
\end{align*}
To compute the result of the double cross product, we set our spherical coordinate system to have $z$ axis aligned with $|\mathbf{\ddot{d}}|$ to get 
\begin{align*}
    \mathbf{S} &= \dfrac{1}{|\mathbf{r}|^2} \mathbf{\ddot{d}} \times \left(\dfrac{\mathbf{r}}{|\mathbf{r}|} \times \mathbf{\ddot{d}}\right) + \left[\sim \dfrac{1}{|\mathbf{r}|^3}\right] =  \dfrac{1}{|\mathbf{r}|^2} \left(\mathbf{\ddot{d}} \cdot \mathbf{\ddot{d}}\right) \dfrac{\mathbf{r}}{|\mathbf{r}|}  - \dfrac{1}{|\mathbf{r}|^2}  \left(\mathbf{\ddot{d}} \cdot \dfrac{\mathbf{r}}{|\mathbf{r}|} \right)\mathbf{\ddot{d}} + \left[\sim \dfrac{1}{|\mathbf{r}|^3}\right] =  \left(-\cos(\theta)\dfrac{\mathbf{\ddot{d}}}{|\mathbf{\ddot{d}}|} +  \dfrac{\mathbf{r}}{|\mathbf{r}|} \right)\dfrac{|\mathbf{\ddot{d}}|^2}{|\mathbf{r}|^2} + \left[\sim \dfrac{1}{|\mathbf{r}|^3}\right],\\
    \mathbf{n}_p \cdot \mathbf{S} &= \left(1-\cos(\theta)^2\right)\dfrac{|\mathbf{\ddot{d}}|^2}{|\mathbf{r}|^2} + \left[\sim \dfrac{1}{|\mathbf{r}|^3}\right].
\end{align*}
\subsection{Discussion}
\begin{enumerate}
    \item[i.] The power previously calculated does not depend on the radius $R$ at which we collect the emmited radiation for sufficiently large $R$. This is because the power scales localy by an inverse square law in the radial distance. This is consistent with the fact that waves propagating towards infinity traverse spheres at all $R$ with the same velocity. The conservation of the energy density they carry implies that the same amount of energy must flow trought all spheres. Furthermore, the constance of their velocity in the vacuum implies the rate of the flow only depends on $R$ trough the raterded time $t'$ (we can follow waves by looking at $t, \mathbf{r}$ that make $t'$ and see that the rate at which their energy traverse spheres is constant). 
    \item[ii.] For a system of $N$ point charges $q_\alpha$ with trajectories $\mathbf{r}'_\alpha(t')$, the retarded dipole moment has the following expression 
    \begin{align*}
        \mathbf{d}(t') = \sum_{\alpha=1}^N q_\alpha \mathbf{r}_\alpha'(t').  
    \end{align*}
    The radiated power at infinity found earlier only depends on the second retarded time derivative of $\mathbf{d}$ which reads 
    \begin{align*}
        \ddot{\mathbf{d}}(t') = \sum_{\alpha=1}^N q_\alpha \ddot{\mathbf{r}}_\alpha'(t').  
    \end{align*}
    and we see that the radiated power vanished if no charge is moving on an accelerated trajectory. 
    \item[iii.] Suppose we have a system of point charges that all have the same charge $q_\alpha$ to mass $m_\alpha$ ratio $s$ Using Newton's second law we find that for this system: 
    \begin{align*}
        \ddot{\mathbf{d}}(t') = \sum_{\alpha=1}^N q_\alpha \ddot{\mathbf{r}}_\alpha'(t') = \sum_{\alpha=1}^N \dfrac{q_\alpha}{m_\alpha} \mathbf{F}_\alpha(t') = s \sum_{\alpha=1}^N \mathbf{F}_\alpha(t') = 0.  
    \end{align*}
    The last sum vanished because no net external force is exerted on the system and the sum of internal forces must vanish by the action-reaction law.
    \item[iv.] A system with equal mass ratio is not physically relevant because it necessarely contains only positive (or negative) charges making it very unstable in nature. 
\end{enumerate}


\section{Beyond radiation}
\subsection{Three-vector Potential Refinement}
We now work on the second term of the approximation of $\mathbf{A}$ obtained in item A of the previous question. The goal of this section is to express this contribution to $\mathbf{A}$ in terms of 
\begin{align*}
    \mathbf{Q}_{ij} &= \int \text{d}^3r' \rho(\mathbf{r}', t')(3\mathbf{r}_i'\mathbf{r}_j' - \delta_{ij}|\mathbf{r}'|^2),\quad \textbf{Quadrupole Moment}\\
    \mathbf{m} &= \dfrac{1}{2}\int \text{d}^3r' \mathbf{r}' \times \mathbf{j}(\mathbf{r}', t').\quad \textbf{Magnetic Moment}
\end{align*}
Proceding in an analogous way to item B of the previous question we find
\begin{align*}
    \mathbf{A}^{r}_{j} &= \dfrac{\mathbf{r}_i}{|\mathbf{r}|^2} \cdot  \dfrac{\partial}{\partial t'}\int \text{d}^3r' \mathbf{r}'_i\  \mathbf{j}_j(t', \mathbf{r}') \\&= \dfrac{\mathbf{r}_i}{|\mathbf{r}|^2} \cdot  \dfrac{\partial}{\partial t'}\int \text{d}^3r' \mathbf{r}'_i \nabla(\mathbf{r}'_j)\ \cdot \  \mathbf{j}(t', \mathbf{r}') = \dfrac{\mathbf{r}_i}{|\mathbf{r}|^2} \cdot  \dfrac{\partial}{\partial t'}\int \text{d}^3r'  \left[\nabla \cdot (\mathbf{r}'_j \mathbf{r}'_i \ \mathbf{j}) - \mathbf{r}'_j \nabla \cdot (\mathbf{r}'_i \ \mathbf{j}(t', \mathbf{r}'))\right]\\
     &=- \dfrac{\mathbf{r}_i}{|\mathbf{r}|^2} \cdot  \dfrac{\partial}{\partial t'}\int \text{d}^3r'  \left[ \mathbf{r}'_j \nabla (\mathbf{r}'_i) \ \mathbf{j}(t', \mathbf{r}') + \mathbf{r}'_j \mathbf{r}'_i\nabla \cdot \mathbf{j}(t', \mathbf{r}')\right] \quad  \textbf{Divergence theorem}\\
     &=- \dfrac{\mathbf{r}_i}{|\mathbf{r}|^2} \cdot \int \text{d}^3r'  \left[ \mathbf{r}'_j \ \dot{\mathbf{j}}_i(t', \mathbf{r}') - \mathbf{r}'_j \mathbf{r}'_i \ddot{\rho}(t', \mathbf{r}')\right]. \quad  \textbf{Continuity Equation}
\end{align*}
Then, adding the fist line to the last one yields
\begin{align*}
    \mathbf{A}^{r}_{j} 
     &=\dfrac{\mathbf{r}_i}{2|\mathbf{r}|^2} \cdot \int \text{d}^3r'  \left[ \mathbf{r}'_i\  \dot{\mathbf{j}}_j(t', \mathbf{r}')-\mathbf{r}'_j \ \dot{\mathbf{j}}_i(t', \mathbf{r}') + \mathbf{r}'_j \mathbf{r}'_i \ddot{\rho}(t', \mathbf{r}')\right] 
\end{align*}
% USe reverse double cross product identity
\subsection{}
\subsection{}


\section{Acknowledgement}

}
\makereferences
%-------------------------------------------------------


%%%%%%%%%%%%%%%%%%%%%%%%
% Terminer le document %
%%%%%%%%%%%%%%%%%%%%%%%%
\end{document}