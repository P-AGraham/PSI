\documentclass[10pt, a4paper]{article}

%%%%%%%%%%%%%%
%  Packages  %
%%%%%%%%%%%%%%


\usepackage{page_format}
\usepackage{special}
\usepackage{hyperref}
\usepackage{tikz}
\usepackage[compat=1.1.0]{tikz-feynman}
\usepackage[font=small,labelfont=bf,
   justification=justified,
   format=plain]{caption}
%----------------------------------------------------------------------
%\usepackage{amssymb} % Mathematical fonts.
%\usepackage{amsfonts} % Mathematical fonts.
\usepackage[nice]{nicefrac} % Nicer fractions
\usepackage{braket} % Dirac Notation.
\usepackage{bbm} % More bold fonts.
%\usepackage{mathrsfs} % Mathematical fonts.
\usepackage{esint} % Integrals
\usepackage{cancel} % Allows to scratch expressions.
\usepackage{mathtools} % Tools for math formating.
\usepackage{slashed} % Allows to slash individual characters.
\usepackage{xargs} % Better handling of optional arguments for commands
%----------------------------------------------------------------------
%\usepackage{lmodern} % Fonts.
\usepackage{feyn} % Feynman Diagrams in mathmode

%%%%%%%%%%%%%%%%%%%%%%%%%%%
% Mathématiques et physique
%%%%%%%%%%%%%%%%%%%%%%%%%%%%
% SI Units -----------------------
% The package 'siunitx' causes unresolved crashes (as of 22/08/31)
\newcommand{\ampere}{\text{A}}
\newcommand{\bell}{\text{B}}
\newcommand{\celsius}{\degree\text{C}}
\newcommand{\coulomb}{\text{C}}
\newcommand{\degree}{\,^{\circ}}
\newcommand{\farad}{\text{F}}
\newcommand{\electro}{\text{e}}
\newcommand{\gram}{\text{g}}
\newcommand{\henry}{\text{H}}
\newcommand{\hertz}{\text{Hz}}
\newcommand{\hour}{\text{h}}
\newcommand{\joule}{\text{J}}
\newcommand{\kelvin}{\text{K}}
\newcommand{\meter}{\text{m}}
\newcommand{\minute}{\text{m}}
\newcommand{\mole}{\text{mol}}
\newcommand{\newton}{\text{N}}
\newcommand{\ohm}{\Omega}
\newcommand{\pascal}{\text{Pa}}
\newcommand{\rad}{\text{rad}}
\newcommand{\second}{\text{s}}
\newcommand{\tesla}{\text{T}}
\newcommand{\torr}{\text{Torr}}
\newcommand{\volt}{\text{V}}
\newcommand{\watt}{\text{W}}
%
\newcommand{\tera}{\text{T}}
\newcommand{\giga}{\text{G}}
\newcommand{\mega}{~\text{M}}
\newcommand{\kilo}{~\text{k}}
\newcommand{\deci}{\text{d}}
\newcommand{\centi}{\text{c}}
\newcommand{\milli}{\text{m}}
\newcommand{\micro}{\mu}
\newcommand{\nano}{\text{n}}
\newcommand{\pico}{\text{p}}
\newcommand{\femto}{\text{f}}
%
\newcommand{\units}[1]{\text{#1}}
\newcommand{\tothe}[1]{\textsuperscript{#1}}
%
\newcommand{\per}{\text{/}}
%
\newcommand{\Time}[3]{#1\hour~#2\minute~#3\second} % TODO Optional arguments.
\newcommand{\Angle}[3]{#1^{\circ}~#2'~#3''} % TODO Optional arguments.


% Better epsilon -----------------------
\let\oldepsilon\epsilon
\let\epsilon\varepsilon
\let\varepsilon\oldepsilon


% Better \bar -----------------------
\renewcommand{\bar}[1]{\mkern 1.5mu\overline{\mkern-1.5mu#1\mkern-1.5mu}\mkern 1.5mu}


% Équations -----------------------
\newcommand{\al}[1]{\begin{align} #1 \end{align}} % Numbered equation(s),
\newcommand{\eqn}[1]{\begin{align*} #1 \end{align*}} % Number-less equation(s),
\newcommand{\sys}[1]{\begin{dcases*} #1 \end{dcases*}} % System of equations.


% Exponents -----------------------
\newcommand{\Exp}[1]{\text{e}^{#1}}		% e^#
\newcommand{\E}[1]{\times 10^{#1}}		% X 10^#


% Delimiters -----------------------
\newcommand{\p}[1]{\left( #1 \right)}	% (#)
\newcommand{\cro}[1]{\left[ #1 \right]}	% [#]
\newcommand{\abs}[1]{\left| #1\right|}	% |#|
\newcommand{\avg}[1]{\left\langle #1 \right\rangle} % <#>
\newcommand{\acc}[1]{\left\lbrace #1 \right\rbrace} % {#}


% Vectors -----------------------
\newcommand{\ve}[1]{\mathbf{#1}} % Upright bold face.
\newcommand{\vu}[1]{\hat{\ve{#1}}} % Hat vector upright bold face
\newcommand{\tens}{\otimes} % Tensor product
\newcommand{\nablav}{\bm{\nabla}} % Bold gradient


% Trig. functions with automatic formating  -----------------------
\newcommandx{\Sin}[2][1={}]{\text{sin}^{#1}\!\p{#2}}
\newcommandx{\Cos}[2][1={}]{\text{cos}^{#1}\!\p{#2}}
\newcommandx{\Tan}[2][1={}]{\text{tan}^{#1}\!\p{#2}}
\newcommandx{\Csc}[2][1={}]{\text{csc}^{#1}\!\p{#2}}
\newcommandx{\Sec}[2][1={}]{\text{sec}^{#1}\!\p{#2}}
\newcommandx{\Cot}[2][1={}]{\text{cot}^{#1}\!\p{#2}}
\newcommandx{\Arcsin}[2][1={}]{\text{arcsin}^{#1}\!\p{#2}}
\newcommandx{\Arccos}[2][1={}]{\text{arccos}^{#1}\!\p{#2}}
\newcommandx{\Arctan}[2][1={}]{\text{arctan}^{#1}\!\p{#2}}
\newcommandx{\Sinh}[2][1={}]{\text{sinh}^{#1}\!\p{#2}}
\newcommandx{\Cosh}[2][1={}]{\text{cosh}^{#1}\!\p{#2}}
\newcommandx{\Tanh}[2][1={}]{\text{tanh}^{#1}\!\p{#2}}


% Matrices -----------------------
\newcommand{\mat}[1]{\begin{bmatrix} #1 \end{bmatrix}} % Matrices with hooks.
\newcommand{\pmat}[1]{\begin{pmatrix} #1 \end{pmatrix}} % Matrices with parentheses.
\newcommand{\deter}[1]{\abs{\begin{matrix} #1 \end{matrix}}} % Determinant.
\newcommandx{\mO}[2][1={}, 2={}]{ \def\temp{#2}\ifx\temp\empty\ve{O}_{#1}\else\ve{O}_{#1\times #2}\fi}% Zero matrix.
\newcommandx{\mI}[2][1={}, 2={}]{ \def\temp{#2}\ifx\temp\empty\ve{I}_{#1}\else\ve{O}_{#1\times #2}\fi}%  Identity matrix.
\newcommand{\Det}[1]{\text{det}\p{#1}} % det(#)
\newcommand{\Tr}[1]{\text{Tr}\p{#1}} % Tr(#)


% Derivatives -----------------------
\newcommand{\D}{\text{d}} % Differential 'd'.
\newcommandx{\dd}[3][1={},3={}]{\frac{\D^{#3}#1}{\D{#2}^{#3}}} % Total derivative according to #2, #1 is the function and #3 is the order.
\newcommand{\del}{\partial} % Partial 'd'.
\newcommandx{\ddp}[3][1={},3={}]{\frac{\del^{#3}#1}{\del{#2}^{#3}}} % Dérivée partielle selon #2, #1 est la fonction est #3 est l'ordre.
\newcommand{\eval}[1]{\left. {#1} \right|} % Bar on the right of expression.
\newcommand{\delbar}{\slashed{\del}} % Partial Inexact differential.
\newcommand{\dbar}{\dj}% Inexact differential.


% Integrals -----------------------
\newcommand{\intinf}{\int\displaylimits_{-\infty}^{\infty}} % From -00 to 00.
\newcommandx{\Int}[2][1={},2={}]{\int\displaylimits_{#1}^{#2}} % Faster bounded integrals.


% Complex numbers -----------------------
\renewcommand{\Re}[1]{\text{Re}\acc{#1}} % Re{#}
\renewcommand{\Im}[1]{\text{Im}\acc{#1}} % Im{#}


% Sets -----------------------
\newcommand{\N}{\mathbbm{N}} % Natural numbers.
\newcommand{\Z}{\mathbbm{Z}} % Integers.
\newcommand{\Q}{\mathbbm{Q}} % Rational numbers.
\newcommandx{\R}[1][1={}]{\mathbbm{R}^{#1}} % Real numbers.
\newcommandx{\C}[1][1={}]{\mathbbm{C}^{#1}} % Complex numbers.
\newcommandx{\F}[1][1={}]{\mathbbm{F}^{#1}} % Some field.
\newcommand{\M}[3]{\mathbb{M}_{#1\times#2}(#3)}	% Matrices.
\newcommand{\Po}[2]{\mathbb{P}_{#1}(#2)} % Polynomials.
\newcommand{\Lin}{\mathbb{L}} % Linear maps.


% Constants and physical symbols -----------------------
\newcommand{\eo}{\epsilon_0} % epsilon 0.
\renewcommand{\L}{\mathcal{L}} % Lagrangian.

\usepackage{listings,xcolor}

% Default fixed font does not support bold face
\DeclareFixedFont{\ttb}{T1}{txtt}{bx}{n}{12} % for bold
\DeclareFixedFont{\ttm}{T1}{txtt}{m}{n}{12}  % for normal

\lstset{language=Mathematica}
\lstset{basicstyle={\sffamily\footnotesize},
  numbers=left,
  numberstyle=\tiny\color{gray},
  numbersep=5pt,
  breaklines=true,
  captionpos={t},
  frame={lines},
  rulecolor=\color{black},
  framerule=0.5pt,
  columns=flexible,
  tabsize=2
}

\usepackage{color}
\definecolor{deepblue}{rgb}{0,0,0.5}
\definecolor{deepred}{rgb}{0.6,0,0}
\definecolor{deepgreen}{rgb}{0,0.5,0}

\newcommand\pythonstyle{\lstset{
language=Python,
basicstyle=\ttm,
morekeywords={self},              % Add keywords here
keywordstyle=\ttb\color{deepblue},
emph={MyClass,__init__},          % Custom highlighting
emphstyle=\ttb\color{deepred},    % Custom highlighting style
stringstyle=\color{deepgreen},
frame=tb,                         % Any extra options here
showstringspaces=false
}}

\lstnewenvironment{python}[1][]
{
\pythonstyle
\lstset{#1}
}
{}


% References
\usepackage{biblatex}
\addbibresource{ref.bib}


%%%%%%%%%%%%
%  Colors  %
%%%%%%%%%%%%
% ! EDIT HERE !
\colorlet{chaptercolor}{red!70!black} % Foreground color.
\colorlet{chaptercolorback}{red!10!white} % Background color


%%%%%%%%%%%%%%
% Page titre %
%%%%%%%%%%%%%%
\title{Homework 2 : Linearized gravity
} % Title of the assignement.
\author{\PA} % Your name(s).
\teacher{David Kubiznak and Ghazal Geshnizjani} % Your teacher's name.
\class{Relativity} % The class title.

\university{Perimeter Institute for Theoretical Physics} % University
\faculty{Perimeter Scholars International} % Faculty
%\departement{<Departement>} % Departement
\date{\today} % Date.


%%%%%%%%%%%%%%%%%%%%%%
% Begin the document %
%%%%%%%%%%%%%%%%%%%%%%
\begin{document}



% Make the title page.
\maketitlepage

% Make table of contents
\maketableofcontents

% Assignment starts here ----------------------------

\section{Linearized field equations}
\footnotesize{
Weak gravitational effects can be modeled as a perturbation of the flat Minkowski metric $\eta$. On the level of manifolds, this perturbation can be seen as a diffeomorphism $\phi : M \to M'$ mapping flat spacetime $M$ into a weakly curved manifold $M'$. A global coordinate chart $\psi : M \to \mathbb{R}^4$ on the flat spacetime can be converted to a coordinate chart $\psi'$ on the disformed manifold as $\psi' = \psi \circ \phi^{-1} : M' \to \mathbb{R}^4$. Taking the coordinates on $M$ to be cartesian, we work with the inherited coordinates on $M'$ as a starting point. In these coordinates, the full metric $g_{\mu\nu}$ can be Taylor expanded in a small parameter $\lambda$ as $g_{\mu\nu} = \eta_{\mu \nu} + h_{\mu \nu} + O(\lambda^2)$ where $h_{\mu \nu}$ is the perturbation depending linearly on $\lambda$. For all the following calculations, we drop the $O(\lambda^2)$ but keep in mind that everything represents a first-order expansion in $\lambda$.

To write the first-order contribution to the Einstein equations arising from this perturbation, we first compute the inverse metric. Expanding it in $\lambda$ around the inverse Minkowski metric, we have $g_{\mu\nu} = \eta^{\mu \nu} + f^{\mu \nu}$ and 
\begin{align*}
 \delta_{\rho}^{\nu} = g_{\rho \mu} g^{\mu\nu} = \eta_{\rho \mu}\eta^{\mu \nu} + h_{\rho \mu} \eta^{\mu \nu} + \eta_{\rho \mu} f^{\mu \nu} \iff f_\rho{}^{\nu} = -h_{\rho}{}^{\nu} \iff f^{\rho\nu} = -h^{\rho\nu}. 
\end{align*}
Then the expansion of the Christoffel symbols read
\begin{align*}
  \Gamma^{\sigma}{}_{\mu\nu} = \dfrac{1}{2}g^{\sigma \rho}(g_{\mu \rho, \nu}+g_{\rho \nu, \mu}-g_{\mu \nu, \rho}) = \dfrac{1}{2}(\eta^{\sigma \rho}-h^{\sigma \rho})(h_{\mu \rho, \nu}+h_{\rho \nu, \mu}-h_{\mu \nu, \rho}) = \dfrac{1}{2}\eta^{\sigma \rho}(h_{\mu \rho, \nu}+h_{\rho \nu, \mu}-h_{\mu \nu, \rho})
\end{align*}
because $\eta_{\mu\nu, \rho} = 0$ in cartesian coordinates. The Riemann tensor can now be expressed as 
\begin{align*}
  R^{\rho}{}_{\sigma\mu\nu} &= \Gamma^{\rho}{}_{\nu\sigma, \mu} - \Gamma^{\rho}{}_{\mu\sigma, \nu} + \Gamma^{\rho}{}_{\mu\lambda}\Gamma^{\lambda}{}_{\nu\sigma} - \Gamma^{\rho}{}_{\nu\lambda}\Gamma^{\lambda}{}_{\mu\sigma}\\ &= \Gamma^{\rho}{}_{\nu\sigma, \mu} - \Gamma^{\rho}{}_{\mu\sigma, \nu} = \dfrac{1}{2}\eta^{\rho \lambda}(h_{\nu \lambda, \sigma \mu}+h_{\lambda \sigma, \nu \mu}-h_{\nu \sigma, \lambda \mu}) - \dfrac{1}{2}\eta^{\rho \lambda}(h_{\mu \lambda, \sigma \nu}+h_{\lambda \sigma, \mu \nu}-h_{\mu \sigma, \lambda \nu})\\
  &= \dfrac{1}{2}\eta^{\rho \lambda}(h_{\nu \lambda, \sigma \mu}-h_{\nu \sigma, \lambda \mu} - h_{\mu \lambda, \sigma \nu} + h_{\mu \sigma, \lambda \nu}).
\end{align*}
Contracting the $\rho$ and $\mu$ indices, we get the following Ricci tensor:
\begin{align*}
  R_{\sigma \nu} &= \dfrac{1}{2}\eta^{\mu \lambda}(h_{\nu \lambda, \sigma \mu}-h_{\nu \sigma, \lambda \mu} - h_{\mu \lambda, \sigma \nu} + h_{\mu \sigma, \lambda \nu}) = \dfrac{1}{2}(h_{\nu}{}^{\mu}{}_{, \sigma \mu}-h_{\nu \sigma, \lambda}{}^{\lambda} - h^{\lambda}{}_{\lambda, \sigma \nu} + h^\mu{}_{\sigma, \mu \nu})\\
  &= \dfrac{1}{2}(h_{\nu}{}^{\mu}{}_{, \sigma \mu} + h^\mu{}_{\sigma, \mu \nu}-\square h_{\nu \sigma} - h_{, \sigma \nu} ), \quad h = h^{\lambda}{}_{\lambda} 
\end{align*}
where we used the fact raising indices with $g^{\mu\nu}$ for tensors proportionnal to $\lambda$ reduces to contracting them with $\eta^{\mu\nu}$ at first order in $\lambda$ (the $-h^{\mu \nu}$ term only contributes to second order). Contracting the remaining indices (with the Minkowski) metric yields the Ricci scalar 
\begin{align*}
  R = \eta^{\sigma \nu} R_{\sigma \nu} = \dfrac{1}{2}(h^{\sigma\mu}{}_{, \sigma \mu} + h^{\sigma \mu}{}_{, \mu \sigma}-\square h^{\nu}{}_{\nu} - h_{, \nu}{}^{\nu} ) = h^{\sigma\mu}{}_{, \sigma \mu} -\square h.
\end{align*}
Combining all the previous results, the linearised Einstein tensor can be written as 
\begin{align*}
  G_{\sigma \nu} = R_{\sigma \nu} - \frac{1}{2} \eta_{\sigma\nu} R = \dfrac{1}{2}(h_{\nu}{}^{\mu}{}_{, \sigma \mu} + h^\mu{}_{\sigma, \mu \nu}-\square h_{\nu \sigma} - h_{, \sigma \nu} - \eta_{\sigma\nu} h^{\rho\mu}{}_{, \rho \mu} + \eta_{\sigma\nu} \square h).
\end{align*}
We define $\bar{h}_{\sigma \nu} = h_{\sigma \nu} - \frac{1}{2}\eta_{\sigma \nu} h$ with trace  $\bar{h} = \eta^{\sigma \nu}\bar{h}_{\sigma \nu} = h - \frac{4}{2} h = -h$. With this in mind, the perturbation can be written as $h_{\sigma \nu} =  \bar{h}_{\sigma \nu} + \frac{1}{2}\eta_{\sigma \nu} (-\bar{h})$. Substitution of this form in the Einstein tensor leads to 
\begin{align*}
  G_{\sigma \nu} &= \dfrac{1}{2}(h_{\nu}{}^{\mu}{}_{, \sigma \mu} + h^\mu{}_{\sigma, \mu \nu}-\square h_{\nu \sigma} - h_{, \sigma \nu} - \eta_{\sigma\nu} h^{\rho\mu}{}_{, \rho \mu} + \eta_{\sigma\nu} \square h)\\
  &= \dfrac{1}{2}(\bar{h}_{\nu}{}^{\mu}{}_{, \sigma \mu}- \textcolor{blue}{\frac{1}{2} \bar{h}_{, \sigma \nu}} + \bar{h}^\mu{}_{\sigma, \mu \nu} - \textcolor{blue}{\frac{1}{2}\bar{h}_{, \sigma \nu}}-\square\bar{h}_{\sigma\nu} + \textcolor{red}{\frac{1}{2}\eta_{\sigma\nu} \square \bar{h}} + \textcolor{blue}{\bar{h}_{, \sigma \nu}}-\eta_{\sigma\nu}\bar{h}^{\rho\mu}{}_{, \rho \mu} + \textcolor{red}{\frac{1}{2}\eta_{\sigma\nu} \square\bar{h}} - \textcolor{red}{\eta_{\sigma\nu} \square \bar{h}})\\
  &= \dfrac{1}{2}(\bar{h}_{\nu}{}^{\mu}{}_{, \sigma \mu}+ \bar{h}^\mu{}_{\sigma, \mu \nu} -\square\bar{h}_{\sigma\nu} -\eta_{\sigma\nu}\bar{h}^{\rho\mu}{}_{, \rho \mu})
\end{align*}
with
\begin{align*}
  &h_{\nu}{}^{\mu}{}_{, \sigma \mu} = \bar{h}_{\nu}{}^{\mu}{}_{, \sigma \mu} - \frac{1}{2} \eta_{\nu}{}^{\mu} \bar{h}_{, \sigma \mu} = \bar{h}_{\nu}{}^{\mu}{}_{, \sigma \mu}- \frac{1}{2} \bar{h}_{, \sigma \nu}, \quad h^\mu{}_{\sigma, \mu \nu} = \bar{h}^\mu{}_{\sigma, \mu \nu} - \frac{1}{2}\eta^\mu{}_{\sigma} \bar{h}_{, \mu \nu} = \bar{h}^\mu{}_{\sigma, \mu \nu} - \frac{1}{2}\bar{h}_{, \sigma \nu}\\
  &\square h_{\sigma\nu} = \bar{h}_{\sigma\nu} - \frac{1}{2}\eta_{\sigma\nu} \square \bar{h}, \quad h_{, \sigma \nu} = -\bar{h}_{, \sigma \nu}, \quad \eta_{\sigma\nu} h^{\rho\mu}{}_{, \rho \mu} = \eta_{\sigma\nu}\bar{h}^{\rho\mu}{}_{, \rho \mu} - \frac{1}{2}\eta_{\sigma\nu}\eta^{\rho\mu} \bar{h}_{, \rho \mu} = \eta_{\sigma\nu}\bar{h}^{\rho\mu}{}_{, \rho \mu} - \frac{1}{2}\eta_{\sigma\nu} \square\bar{h}.
\end{align*}
Finally, the relation between the Einstein tensor and the stress-energy tensor $T_{\mu\nu}$ is provided by Einstein equations. We take $T_{\mu\nu}$ to be of the order of $\lambda$ consistently with the weak field on almost flat space ($T_{\mu\nu}$ has no zeroth order contribution) assumptions. The perturbation satisfies the equation 
\begin{align*}
  \dfrac{1}{2}(\bar{h}^{\mu}{}_{\nu, \sigma \mu}+ \bar{h}^\mu{}_{\sigma, \nu \mu} -\square\bar{h}_{\sigma\nu} -\eta_{\sigma\nu}\bar{h}^{\rho\mu}{}_{, \rho \mu}) = \dfrac{1}{2}( 2\bar{h}^\mu{}_{(\sigma, \nu) \mu} -\square\bar{h}_{\sigma\nu} -\eta_{\sigma\nu}\bar{h}^{\rho\mu}{}_{, \rho \mu}) = 8\pi G T_{\sigma \nu}
\end{align*}
with gravitational coupling strength $G$.
}


\section{Let’s simplify our lives}

\begin{enumerate}
  \item[(a)] Since coordinate transformations locally transform the metric components without changing the spacetime it describes, we can interpret them as gauge transformations on a tensor component field $g_{\mu\nu}$. To preserve the validity of our linearized expansion, we consider the effect of infinitesimal coordinate transformations $x^{\mu}{}'(x) = x^{\mu} - \xi^{\mu}(x)$ with $\xi$ at order in $\lambda$. This ensures that a coordinate change preserves $\eta_{\mu\nu}$ at zeroth order and sends $h_{\mu\nu}$ to a perturbation in the range satisfying the linearized Einstein equations.  The transformed components $h_{\mu\nu}'(x')$ will satisfy the equation and we recover a notion of linearized covariance. Relating the $g_{\mu\nu}(x)$ components and the gauge transformed components $g_{\mu\nu}'(x')$ at first order, we have 
  \begin{align*}
    \eta_{\mu\nu} + h_{\mu\nu}(x) = g_{\mu\nu}(x) &= x^{\mu}{}_{,\mu}' x^{\nu}{}_{,\nu'}' g_{\mu\nu}'(x'(x))\\
    &= (\delta^{\sigma}_\mu - \xi^{\sigma}{}_{,\mu}(x))(\delta^{\rho}_\nu- \xi^{\rho}{}_{,\nu}(x))(\eta_{\sigma\rho} + h_{\sigma\rho}'(x'(x)))\\
    &= \eta_{\mu\nu} + h_{\mu\nu}'(x'(x)) - \delta^{\sigma}_\mu \eta_{\sigma \rho} \xi^{\rho}{}_{,\nu}(x) - \delta^{\rho}_\nu \eta_{\sigma \rho} \xi^{\sigma}{}_{,\mu}(x)\\
    &= \eta_{\mu\nu} + h_{\mu\nu}'(x'(x)) - \xi_{\mu,\nu} - \xi_{\nu,\mu}
  \end{align*}
  Comparing the right and left-hand sides of this expression yields $h_{\mu\nu}'(x'(x)) = h_{\mu\nu}(x) + \xi_{\mu,\nu}(x) + \xi_{\nu,\mu}(x)$. To bring the dependency of $h_{\mu\nu}'$ to $x$ explicitly, we write the expansion $h_{\mu\nu}'(x'(x)) = h_{\mu\nu}'(x) + \xi^{\sigma}(x) h_{\mu\nu, \sigma}'(x)$ where the second term is second order in $\lambda$ and does not contribute so $h_{\mu\nu}'(x'(x)) = h_{\mu\nu}'(x)$.
  \item[(b)] Using the previous result, the gauge transformation of $\bar{h}_{\sigma \nu}$ to $\bar{h}_{\sigma \nu}'$ reads
  \begin{align*}
    \bar{h}_{\mu \nu}'(x) &= h_{\mu\nu}(x) + \xi_{\mu,\nu}(x) + \xi_{\nu,\mu}(x) - \frac{1}{2}\eta_{\mu \nu} \eta^{\sigma \rho} (h_{\sigma\rho}(x) + \xi_{\sigma,\rho}(x) + \xi_{\sigma,\rho}(x))\\
    &= \bar{h}_{\mu\nu}(x) + \xi_{\mu,\nu}(x) + \xi_{\nu,\mu}(x) - \eta_{\mu \nu} \xi_{\sigma,}{}^{\sigma}(x).
  \end{align*}
  Now we contract the $\mu$ index of $\bar{h}_{\mu \nu}$ with a derivative and get
  \begin{align*}
    \bar{h}'{}_{\mu \nu,}{}^{\mu}(x) &=  \bar{h}{}_{\mu \nu,}{}^{\mu}(x) + \xi_{\mu,\nu}{}^{\mu}(x) + \xi_{\nu,\mu}{}^{\mu}(x) - \eta_{\mu \nu}{}^{\mu} \xi_{\sigma,}{}^{\sigma}(x)\\
    &=  \bar{h}{}_{\mu \nu,}{}^{\mu}(x) + \xi_{\mu,\nu}{}^{\mu}(x) + \xi_{\nu,\mu}{}^{\mu}(x) -  \xi_{\sigma,\nu}{}^{\sigma}(x)\\
    &= \bar{h}{}_{\mu \nu,}{}^{\mu}(x) + \square \xi_{\nu}(x). 
  \end{align*} 
Choosing $\xi_\nu$ to make $\bar{h}'{}_{\mu \nu,}{}^{\mu}(x)$ vanish constitutes a choice of gauge called the \textit{De Donder gauge}. The coordinate transforms leading to this gauge are constrained by 
\begin{align*}
  \square \xi_{\nu}(x) = -\bar{h}{}_{\mu \nu,}{}^{\mu}(x) 
\end{align*}
which is a wave equation with $-\bar{h}{}_{\mu \nu,}{}^{\mu}(x)$ sources for each $\nu$. Given any starting $\bar{h}_{\mu \nu}$, we can compute the associated source and solve the wave equation to go to the De Donder gauge. In this gauge, the Einstein equations derived above become 
\begin{align*}
  8\pi G T_{\sigma \nu} = \dfrac{1}{2}(\bar{h}'_{\mu\sigma,}{}^\mu{}_\nu + \bar{h}'_{\mu\nu,}{}^\mu{}_\sigma -\square\bar{h}'_{\sigma\nu} -\eta_{\sigma\nu}(\bar{h}'_{\rho\mu,}{}^{\mu}){}^{\rho}) = -\frac{1}{2}\square\bar{h}'_{\sigma\nu} \iff \square\bar{h}'_{\sigma\nu} = -16\pi G T_{\sigma \nu}.
\end{align*}
In the following steps, we work in De Donder gauge and drop $'$ to simplify notation. 
\end{enumerate}


\newpage
\section{Gravitomagnetism}
\begin{enumerate}
  \item[(a)] The linearization of gravity works for $T_{\sigma \nu}$ of the order of $\lambda$ which is realised far from sources. Going further, the Newtonian limit is taken by approximating that the only significant $T_{\sigma \nu}$ component is mass density $\rho = T_{00}$. Then, all other components of Einstein equations have no significant sources at all times and vanish in the Newtonian limit. We can identify the Newtonian gravitationnal potential $\phi$ with $-\frac{1}{4} \bar{h}_{00}$ (or $h_{00} = \bar{h}_{00} - \frac{1}{2}\bar{h}_{00} = -2\phi$, $\bar{h} = \eta^{\mu\nu}\bar{h}_{\mu\nu} =-\bar{h}_{00}$ because only one diagonal element is non-zero). The Einstein equation associated with this component reads 
  \begin{align*}
    \square\bar{h}_{00} = -16\pi G T_{00} \iff 4\pi G \rho = -\left(-\frac{\partial^2}{\partial t^2} + \nabla^2 \right) \phi
  \end{align*}
  and in the quasi-static field limit (slowly changing $\phi$, not enough to emit considerable gravitational radiation, of the order of the field variations in celestial mechanics) we recover $\nabla^2 \phi = 4\pi G \rho$. The solution for $h_{\mu\nu}$ consistent with $\bar{h}_{\mu \nu}$ is 
  \begin{align*}
    h_{\mu \nu} =  \bar{h}_{\mu \nu} + \frac{1}{2}\eta_{\mu \nu} (-\bar{h}) = \begin{cases}
    \frac{1}{2}\bar{h}_{00}\delta_{ij}, \quad  (\mu, \nu) = (i, j)\\
    0, \quad (\mu, \nu) = (i, 0)\\
    \frac{1}{2}\bar{h}_{00}, \quad (\mu, \nu) = (0, 0).
    \end{cases}
  \end{align*}
  which leads to the linearized metric 
  \begin{align*}
    ds^2 = \left(-1+\frac{1}{2}\bar{h}_{00}\right)dt^2 + \left(1+\frac{1}{2}\bar{h}_{00}\right)(dx^2 + dy^2 + dz^2).
  \end{align*}
  We make sense of the $g_{ij}$ elements by comparing them to the weak field limit of the Schwarzschild for spherically symmetric energy density $\rho$. 
  
  We consider a point mass moving on a curve $\gamma: \mathbb{R} \to M'$. Its points are represented in the coordinate chart inherited from cartesian coordinates on $M$ by $x^\mu(\tau)$ parametrized by proper time $\tau$ (Lorentzian arc length for timelike velocities).  In the Newtonian limit, an infinitesimal proper time change on $\gamma$ reads 
  \begin{align*}
    -d\tau^2 =  -(1- 2\phi)dt^2 + \left(1+2\phi\right)\left(dx^2 + dy^2 + dz^2\right) \implies -1 =  -(1- 2\phi)\left(\frac{dt}{d\tau}\right)^2 + \left(1+2\phi\right)\left(\left(\frac{dx}{d\tau}\right)^2 + \left(\frac{dy}{d\tau}\right)^2 + \left(\frac{dz}{d\tau}\right)^2\right).
  \end{align*} 
  At leading order in $\lambda$ ($\sim$ neglecting gravitational time dilation and associated space effect) the proper time parametrization behaves the same way it does in Minkowski space. To go from Minkowsk-like proper time to Galilean-like absolute time we take $\frac{dx}{d\tau}, \frac{dy}{d\tau}, \frac{dz}{d\tau}$ to be small (neglecting special relativistic time dilation). This means that the Minkowski-like coordinate system is such that the three-velocity of the mass stays close to the time direction for all $\tau$ where the Newtonian limit applies. This three-velocity constraint reduces the previous equation to $1 = \frac{dt}{d\tau}$ implying parametrizing by proper time is equivalent to parametrizing by coordinate time $t$. A schematic way to write this conclusion is $x^{\mu}(\tau(t)) = x^{\mu}(t+ O(v^2) + O(\lambda)) = x_0^{\mu}(t) + O(v^2) + O(\lambda)$ and $\frac{d}{d\tau} = \frac{dt}{d\tau} \frac{d}{dt} = (1 + O(\lambda) + O(v^2)) \frac{d}{dt}$ where $v$ represents three-velocity (the dependency starts at $O(v^2)$ because of the Minkowski lorentz factor expansion). With this conclusion in mind, we can write the geodesic equation describing the free trajectory in $M'$ as follows
  \begin{align*}
    0 &= \frac{d^2 x^{\mu}(\tau)}{d \tau^2} + \Gamma^{\mu}{}_{\alpha \tau} \frac{dx^\alpha(\tau)}{d\tau} \frac{dx^\beta(\tau)}{d\tau}\\
    &= (1 + O(\lambda) + O(v^2))^2 \frac{d^2 x_0^{\mu}(t) + O(v^2) + O(\lambda)}{dt^2} + \Gamma^{\mu}{}_{\alpha \beta} (1 + O(\lambda) + O(v^2))^2\frac{dx_0^\alpha(t) + O(v^2) + O(\lambda)}{dt} \frac{dx_0^\beta(t)+O(\lambda) + O(v^2)}{dt}.
  \end{align*}
  Since $\Gamma^{\mu}{}_{\alpha \tau}$ is first order in $\lambda$ (see linearized expression given above), the leading order in $v$ and $\lambda$ of the geodesic equation is $0 = \frac{d^2 x^{\mu}(\tau)}{d t^2}$ (Newton's principle of inertia). To retrieve the dominant gravitationnal effets we go to first order in $\lambda$ and define $x^{\mu}(\tau(t)) = x_1^{\mu}(t) + O(v^2) + O(\lambda^2)$ to get 
  \begin{align*}
    0
    &\textcolor{blue}{=} (1 + O(\lambda) + O(v^2))^2 \frac{d^2 x_1^{\mu}(t) + O(v^2) + O(\lambda^2)}{dt^2} + \Gamma^{\mu}{}_{\alpha \beta} (1 + O(\lambda) + O(v^2))^2\frac{dx_1^\alpha(t) + O(v^2) + O(\lambda^2)}{dt} \frac{dx_1^\beta(t)+O(\lambda^2) + O(v^2)}{dt}\\
    &\text{Gravitationnal dilation effects vanish in the first term at $O(\lambda)$ because $x_1^{\mu}$ is already at $O(\lambda)$ and all other $O(\lambda)$ factors are neglected}\\
    &\textcolor{blue}{=} \frac{d^2 x_1^{\mu}(t)}{dt^2} + \Gamma^{\mu}{}_{\alpha \beta} \frac{dx_0^\alpha(t)}{dt} \frac{dx_0^\beta(t)}{dt}, \quad \text{For the $\Gamma^{\mu}{}_{\alpha \beta} = O(\lambda)$ term, only the zeroth order contributions $x_0^\mu$ are preserved}\\
    &\textcolor{blue}{=} \frac{d^2 x_1^{\mu}(t)}{dt^2} + \Gamma^{\mu}{}_{00} \frac{dx_0^0(t)}{dt} \frac{dx_0^0(t)}{dt}, \quad \text{$dx^i/d\tau = (1 + O(\lambda) + O(v^2))dx^i/dt = O(v) + O(\lambda)$ : spacial velocities factors vanish at $O(v^0)$}\\
    &\textcolor{blue}{=} \frac{d^2 x_1^{\mu}(t)}{dt^2} + \dfrac{1}{2}\eta^{\mu \rho}(h_{0 \rho, 0}+h_{\rho 0, 0}-h_{0 0, \rho}), \quad \text{principle of inertia at $O(\lambda^0) \impliedby x_0^0 = t$}\\
    &\textcolor{blue}{\implies} 0 = \frac{d^2 x_{1, i}(t)}{dt^2} - \dfrac{1}{2}h_{0 0,i} = \frac{d^2 x_{1, i}(t)}{dt^2} + \phi_{,i},\quad \text{lowering the index to get a gradient, $h_{00} = \bar{h}_{00} - \frac{1}{2}\bar{h}_{00} = -2\phi$}
  \end{align*}
  \item[(b)] If we allow significant energy flux ($T_{0i} = T_{i0}$ components) while keeping the stress ($T_{ij}$ components) negligible, the non-vanishing components of $\bar{h}_{\mu\nu}$ becomes $\bar{h}_{\mu0} = \bar{h}_{0\mu}$. These components can be associated with a four-potential $A_\mu = -\bar{h}_{\mu 0}/4$ sourced by the four-courent $J_\mu = -T_{0\mu}$. Writing Einstein's equations and the De Donder gauge condition for the nonzero components gives 
  \begin{align*}
  -4\square A_{\mu} = \square\bar{h}_{0\mu} = -16\pi G T_{0\mu} = 16\pi G J_{\mu} \iff \square A_{\mu} = -4\pi G J_{\mu}, \quad \bar{h}{}_{0\mu,}{}^{\mu} = -4A_{\mu,}{}^{\mu} = 0 \iff A_{\mu,}{}^{\mu} = 0
  \end{align*}
  which is analogous to Maxwell's equations for electromagnetism (an important difference remains through energy conditions that forbid negative charge densities $T_{00}$). As before we work in the quasi-static field limit where $A_{\mu, 0}$ is taken negligible. Again, the Newtonian limit is used to write the geodesic equation for a point mass up to $O(\lambda)$ and $O(v)$ (we go further than to extract leading order gravitomagnetic effects) as 
  \begin{align*}
    0
    &\textcolor{blue}{=} (1 + O(\lambda) + O(v^2))^2 \frac{d^2 x_1^{\mu}(t) + O(v^2) + O(\lambda^2)}{dt^2} + \Gamma^{\mu}{}_{\alpha \beta} (1 + O(\lambda) + O(v^2))^2\frac{dx_1^\alpha(t) + O(v^2) + O(\lambda^2)}{dt} \frac{dx_1^\beta(t)+O(\lambda^2) + O(v^2)}{dt}\\
    &\textcolor{blue}{=} \frac{d^2 x_1^{\mu}(t)}{dt^2} + \Gamma^{\mu}{}_{\alpha \beta} \frac{dx_0^\alpha(t)}{dt} \frac{dx_0^\beta(t)}{dt} =  \frac{d^2 x_1^{\mu}(t)}{dt^2} + \Gamma^{\mu}{}_{00} \frac{dx_0^0(t)}{dt} \frac{dx_0^0(t)}{dt} + \Gamma^{\mu}{}_{0i} \frac{dx_0^0(t)}{dt} v^i(t), \quad \text{with $v^i = \frac{dx_0^i(t)}{dt}$}\\
    &\textcolor{blue}{=} \frac{d^2 x_1^{\mu}(t)}{dt^2} + \dfrac{1}{2}\eta^{\mu \rho}(h_{0 \rho, 0}+h_{\rho 0, 0}-h_{0 0, \rho}) + \eta^{\mu \rho}(h_{\rho i, 0}+h_{\rho 0, i}-h_{0 i, \rho})v^i = \frac{d^2 x_1^{\mu}(t)}{dt^2} - \dfrac{1}{2}h_{0 0,} {}^\mu -4 (A^{\mu}{}_{, i}-A_{i,}{}^{\mu
    })v^i, \,\, h_{0i} = \bar{h}_{0i} - \frac{1}{2}\eta_{0i}\bar{h} = \bar{h}_{0i} \\
    &\textcolor{blue}{\implies} 0 = \frac{d^2 x_{1, j}(t)}{dt^2} - E_j - 4(\epsilon_{ij}{}^k B_k)v^i = \frac{d^2 x_{1, j}(t)}{dt^2} - E_j + 4 \epsilon_i {}^k{}_{j} v^i B_k.
  \end{align*}
  where we identified $\epsilon_{ij}{}^k B_k = A_{j, i}-A_{i, j}$ and $E_i = -\phi_{, i} - A_{i, 0} = -\phi_{, i} $ in analogy with the electromagnetic field extracted from the four-potential. The equivalent of Lorentz force was recovered. Its associated \textit{electric} charge $q$ equals the inertial mass $m$ of the particle and cancels with it. The coefficient of the magnetic term is $4$ times as big as the coefficient in electromagnetism and has a reversed sign.  
\end{enumerate}

\section{Acknowledgement}
Thanks to Nikhil for a discussion about the interpretation of the $h_{\mu\nu}$ solution obtained from the $\bar{h}_{\mu\nu}$.



% References
\makereferences
%-------------------------------------------------------


%%%%%%%%%%%%%%%%%%%%%%%%
% Terminer le document %
%%%%%%%%%%%%%%%%%%%%%%%%
\end{document}