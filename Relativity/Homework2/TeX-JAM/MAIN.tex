\documentclass[10pt, a4paper]{article}

%%%%%%%%%%%%%%
%  Packages  %
%%%%%%%%%%%%%%


\usepackage{page_format}
\usepackage{special}
\usepackage{hyperref}
\usepackage{tikz}
\usepackage[compat=1.1.0]{tikz-feynman}
\usepackage[font=small,labelfont=bf,
   justification=justified,
   format=plain]{caption}
\input{math_func}

\usepackage{listings,xcolor}

% Default fixed font does not support bold face
\DeclareFixedFont{\ttb}{T1}{txtt}{bx}{n}{12} % for bold
\DeclareFixedFont{\ttm}{T1}{txtt}{m}{n}{12}  % for normal

\lstset{language=Mathematica}
\lstset{basicstyle={\sffamily\footnotesize},
  numbers=left,
  numberstyle=\tiny\color{gray},
  numbersep=5pt,
  breaklines=true,
  captionpos={t},
  frame={lines},
  rulecolor=\color{black},
  framerule=0.5pt,
  columns=flexible,
  tabsize=2
}

\usepackage{color}
\definecolor{deepblue}{rgb}{0,0,0.5}
\definecolor{deepred}{rgb}{0.6,0,0}
\definecolor{deepgreen}{rgb}{0,0.5,0}

\newcommand\pythonstyle{\lstset{
language=Python,
basicstyle=\ttm,
morekeywords={self},              % Add keywords here
keywordstyle=\ttb\color{deepblue},
emph={MyClass,__init__},          % Custom highlighting
emphstyle=\ttb\color{deepred},    % Custom highlighting style
stringstyle=\color{deepgreen},
frame=tb,                         % Any extra options here
showstringspaces=false
}}

\lstnewenvironment{python}[1][]
{
\pythonstyle
\lstset{#1}
}
{}


% References
\usepackage{biblatex}
\addbibresource{ref.bib}


%%%%%%%%%%%%
%  Colors  %
%%%%%%%%%%%%
% ! EDIT HERE !
\colorlet{chaptercolor}{red!70!black} % Foreground color.
\colorlet{chaptercolorback}{red!10!white} % Background color


%%%%%%%%%%%%%%
% Page titre %
%%%%%%%%%%%%%%
\title{Homework 2 : Linearized gravity
} % Title of the assignement.
\author{\PA} % Your name(s).
\teacher{David Kubiznak and Ghazal Geshnizjani} % Your teacher's name.
\class{Relativity} % The class title.

\university{Perimeter Institute for Theoretical Physics} % University
\faculty{Perimeter Scholars International} % Faculty
%\departement{<Departement>} % Departement
\date{\today} % Date.


%%%%%%%%%%%%%%%%%%%%%%
% Begin the document %
%%%%%%%%%%%%%%%%%%%%%%
\begin{document}



% Make the title page.
\maketitlepage

% Make table of contents
\maketableofcontents

% Assignment starts here ----------------------------

\section{Linearized field equations}
\footnotesize{
Weak gravitationnal effects can be modeled as a perturbation of the flat Minkowski metric $\eta$. On the level of manifolds, this perturbation can be seen as a diffeomorphism $\phi : M \to M'$ mapping flat spacetime $M$ into a weakly curved manifold $M'$. A global coordinate chart $\psi : M \to \mathbb{R}^4$ on the flat spacetime can be converted to a coordinate chart $\psi'$ on the disformed manifold as $\psi' = \psi \circ \phi^{-1} : M' \to \mathbb{R}^4$. Taking the coordinates on $M$ to be cartesian, we work with the inherited coordinates on $M'$ as a staring point. In these coordinates, the full metric $g_{\mu\nu}$ can be Taylor expanded in a small parameter $\lambda$ as $g_{\mu\nu} = \eta_{\mu \nu} + h_{\mu \nu} + O(\lambda^2)$ where $h_{\mu \nu}$ is the perturbation depending linearly on $\lambda$. For all the following calculations, we drop the $O(\lambda^2)$ but keep in mind that everything represents a first order expansion in $\lambda$.

To write the fisrt order contribution to the Einstein equations arising from this perturbation, we first compute the inverse metric. Expanding it in $\lambda$ around the inverse minkowski metric, we have $g_{\mu\nu} = \eta^{\mu \nu} + f^{\mu \nu}$ and 
\begin{align*}
 \delta_{\rho}^{\nu} = g_{\rho \mu} g^{\mu\nu} = \eta_{\rho \mu}\eta^{\mu \nu} + h_{\rho \mu} \eta^{\mu \nu} + \eta_{\rho \mu} f^{\mu \nu} \iff f_\rho{}^{\nu} = -h_{\rho}{}^{\nu} \iff f^{\rho\nu} = -h^{\rho\nu}. 
\end{align*}
Then the expansion of the Christoffel symbols read
\begin{align*}
  \Gamma^{\sigma}{}_{\mu\nu} = \dfrac{1}{2}g^{\sigma \rho}(g_{\mu \rho, \nu}+g_{\rho \nu, \mu}-g_{\mu \nu, \rho}) = \dfrac{1}{2}(\eta^{\sigma \rho}-h^{\sigma \rho})(h_{\mu \rho, \nu}+h_{\rho \nu, \mu}-h_{\mu \nu, \rho}) = \dfrac{1}{2}\eta^{\sigma \rho}(h_{\mu \rho, \nu}+h_{\rho \nu, \mu}-h_{\mu \nu, \rho})
\end{align*}
because $\eta_{\mu\nu, \rho} = 0$ in cartesian coordinates. The Riemann tensor can now be expressed as 
\begin{align*}
  R^{\rho}{}_{\sigma\mu\nu} &= \Gamma^{\rho}{}_{\nu\sigma, \mu} - \Gamma^{\rho}{}_{\mu\sigma, \nu} + \Gamma^{\rho}{}_{\mu\lambda}\Gamma^{\lambda}{}_{\nu\sigma} - \Gamma^{\rho}{}_{\nu\lambda}\Gamma^{\lambda}{}_{\mu\sigma}\\ &= \Gamma^{\rho}{}_{\nu\sigma, \mu} - \Gamma^{\rho}{}_{\mu\sigma, \nu} = \dfrac{1}{2}\eta^{\rho \lambda}(h_{\nu \lambda, \sigma \mu}+h_{\lambda \sigma, \nu \mu}-h_{\nu \sigma, \lambda \mu}) - \dfrac{1}{2}\eta^{\rho \lambda}(h_{\mu \lambda, \sigma \nu}+h_{\lambda \sigma, \mu \nu}-h_{\mu \sigma, \lambda \nu})\\
  &= \dfrac{1}{2}\eta^{\rho \lambda}(h_{\nu \lambda, \sigma \mu}-h_{\nu \sigma, \lambda \mu} - h_{\mu \lambda, \sigma \nu} + h_{\mu \sigma, \lambda \nu}).
\end{align*}
Contracting the $\rho$ and $\mu$ indices, we get the following Ricci tensor:
\begin{align*}
  R_{\sigma \nu} &= \dfrac{1}{2}\eta^{\mu \lambda}(h_{\nu \lambda, \sigma \mu}-h_{\nu \sigma, \lambda \mu} - h_{\mu \lambda, \sigma \nu} + h_{\mu \sigma, \lambda \nu}) = \dfrac{1}{2}(h_{\nu}{}^{\mu}{}_{, \sigma \mu}-h_{\nu \sigma, \lambda}{}^{\lambda} - h^{\lambda}{}_{\lambda, \sigma \nu} + h^\mu{}_{\sigma, \mu \nu})\\
  &= \dfrac{1}{2}(h_{\nu}{}^{\mu}{}_{, \sigma \mu} + h^\mu{}_{\sigma, \mu \nu}-\square h_{\nu \sigma} - h_{, \sigma \nu} ), \quad h = h^{\lambda}{}_{\lambda} 
\end{align*}
where we used the fact raising indices with $g^{\mu\nu}$ for tensors proportionnal to $\lambda$ reduces to contracting them with $\eta^{\mu\nu}$ at first order in $\lambda$ (the $-h^{\mu \nu}$ term only contributes to second order). Contracting the remaining indices (with the Minkowski) metric yields the Ricci scalar 
\begin{align*}
  R = \eta^{\sigma \nu} R_{\sigma \nu} = \dfrac{1}{2}(h^{\sigma\mu}{}_{, \sigma \mu} + h^{\sigma \mu}{}_{, \mu \sigma}-\square h^{\nu}{}_{\nu} - h_{, \nu}{}^{\nu} ) = h^{\sigma\mu}{}_{, \sigma \mu} -\square h.
\end{align*}
Combining all the previous results, the linearised Einstein tensor can be written as 
\begin{align*}
  G_{\sigma \nu} = R_{\sigma \nu} - \frac{1}{2} \eta_{\sigma\nu} R = \dfrac{1}{2}(h_{\nu}{}^{\mu}{}_{, \sigma \mu} + h^\mu{}_{\sigma, \mu \nu}-\square h_{\nu \sigma} - h_{, \sigma \nu} - \eta_{\sigma\nu} h^{\rho\mu}{}_{, \rho \mu} + \eta_{\sigma\nu} \square h).
\end{align*}
We define $\bar{h}_{\sigma \nu} = h_{\sigma \nu} - \frac{1}{2}\eta_{\sigma \nu} h$ with trace  $\bar{h} = \eta^{\sigma \nu}\bar{h}_{\sigma \nu} = h - \frac{4}{2} h = -h$. With this in mind, the perturbation can be written as $h_{\sigma \nu} =  \bar{h}_{\sigma \nu} + \frac{1}{2}\eta_{\sigma \nu} (-\bar{h})$. Substitution of this form in the Einstein tensor leads to 
\begin{align*}
  G_{\sigma \nu} &= \dfrac{1}{2}(h_{\nu}{}^{\mu}{}_{, \sigma \mu} + h^\mu{}_{\sigma, \mu \nu}-\square h_{\nu \sigma} - h_{, \sigma \nu} - \eta_{\sigma\nu} h^{\rho\mu}{}_{, \rho \mu} + \eta_{\sigma\nu} \square h)\\
  &= \dfrac{1}{2}(\bar{h}_{\nu}{}^{\mu}{}_{, \sigma \mu}- \textcolor{blue}{\frac{1}{2} \bar{h}_{, \sigma \nu}} + \bar{h}^\mu{}_{\sigma, \mu \nu} - \textcolor{blue}{\frac{1}{2}\bar{h}_{, \sigma \nu}}-\square\bar{h}_{\sigma\nu} + \textcolor{red}{\frac{1}{2}\eta_{\sigma\nu} \square \bar{h}} + \textcolor{blue}{\bar{h}_{, \sigma \nu}}-\eta_{\sigma\nu}\bar{h}^{\rho\mu}{}_{, \rho \mu} + \textcolor{red}{\frac{1}{2}\eta_{\sigma\nu} \square\bar{h}} - \textcolor{red}{\eta_{\sigma\nu} \square \bar{h}})\\
  &= \dfrac{1}{2}(\bar{h}_{\nu}{}^{\mu}{}_{, \sigma \mu}+ \bar{h}^\mu{}_{\sigma, \mu \nu} -\square\bar{h}_{\sigma\nu} -\eta_{\sigma\nu}\bar{h}^{\rho\mu}{}_{, \rho \mu})
\end{align*}
with
\begin{align*}
  &h_{\nu}{}^{\mu}{}_{, \sigma \mu} = \bar{h}_{\nu}{}^{\mu}{}_{, \sigma \mu} - \frac{1}{2} \eta_{\nu}{}^{\mu} \bar{h}_{, \sigma \mu} = \bar{h}_{\nu}{}^{\mu}{}_{, \sigma \mu}- \frac{1}{2} \bar{h}_{, \sigma \nu}, \quad h^\mu{}_{\sigma, \mu \nu} = \bar{h}^\mu{}_{\sigma, \mu \nu} - \frac{1}{2}\eta^\mu{}_{\sigma} \bar{h}_{, \mu \nu} = \bar{h}^\mu{}_{\sigma, \mu \nu} - \frac{1}{2}\bar{h}_{, \sigma \nu}\\
  &\square h_{\sigma\nu} = \bar{h}_{\sigma\nu} - \frac{1}{2}\eta_{\sigma\nu} \square \bar{h}, \quad h_{, \sigma \nu} = -\bar{h}_{, \sigma \nu}, \quad \eta_{\sigma\nu} h^{\rho\mu}{}_{, \rho \mu} = \eta_{\sigma\nu}\bar{h}^{\rho\mu}{}_{, \rho \mu} - \frac{1}{2}\eta_{\sigma\nu}\eta^{\rho\mu} \bar{h}_{, \rho \mu} = \eta_{\sigma\nu}\bar{h}^{\rho\mu}{}_{, \rho \mu} - \frac{1}{2}\eta_{\sigma\nu} \square\bar{h}.
\end{align*}
Finally, the relation between the Einstein tensor and the stress-energy tensor $T_{\mu\nu}$ is provided by Einstein equations. We take $T_{\mu\nu}$ to be of the order of $\lambda$ consistently with the weak field on almost flat space ($T_{\mu\nu}$ has no zeroth order contribution) assumptions. The perturbation satisfies the equation 
\begin{align*}
  \dfrac{1}{2}(\bar{h}^{\mu}{}_{\nu, \sigma \mu}+ \bar{h}^\mu{}_{\sigma, \nu \mu} -\square\bar{h}_{\sigma\nu} -\eta_{\sigma\nu}\bar{h}^{\rho\mu}{}_{, \rho \mu}) = \dfrac{1}{2}( \bar{h}^\mu{}_{(\nu, \sigma) \mu} -\square\bar{h}_{\sigma\nu} -\eta_{\sigma\nu}\bar{h}^{\rho\mu}{}_{, \rho \mu}) = 8\pi G T_{\sigma \nu}
\end{align*}
with gravitationnal coupling strength $G$.
}


\section{Let’s simplify our lives}

\begin{enumerate}
  \item[(a)] Since coordinate transformations localy transform the metric components without changing the spacetime it describes, we can interpret them as gauge transformations on a tensor component field $g_{\mu\nu}$. To preserve the validity of our linearized expansion, we consider the effect of infinitesimal coordinate transformations $x^{\mu}{}'(x) = x^{\mu} - \xi^{\mu}(x)$ with $\xi$ at order in $\lambda$. This ensures that a coordinate change preserve $\eta_{\mu\nu}$ at zeroth order and sends $h_{\mu\nu}$ to a perturbation in the range satisfying the linearized Einstein equations.  The transformed components $h_{\mu\nu}'(x')$ will satisfy the equation and we recover a notion of linearized covariance. Relating the $g_{\mu\nu}(x)$ components and the gauge transformed components $g_{\mu\nu}'(x')$ at first order, we have 
  \begin{align*}
    \eta_{\mu\nu} + h_{\mu\nu}(x) = g_{\mu\nu}(x) &= x^{\mu}{}_{,\mu}' x^{\nu}{}_{,\nu'}' g_{\mu\nu}'(x'(x))\\
    &= (\delta^{\sigma}_\mu - \xi^{\sigma}{}_{,\mu}(x))(\delta^{\rho}_\nu- \xi^{\rho}{}_{,\nu}(x))(\eta_{\sigma\rho} + h_{\sigma\rho}'(x'(x)))\\
    &= \eta_{\mu\nu} + h_{\mu\nu}'(x'(x)) - \delta^{\sigma}_\mu \eta_{\sigma \rho} \xi^{\rho}{}_{,\nu}(x) - \delta^{\rho}_\nu \eta_{\sigma \rho} \xi^{\sigma}{}_{,\mu}(x)\\
    &= \eta_{\mu\nu} + h_{\mu\nu}'(x'(x)) - \xi_{\mu,\nu} - \xi_{\nu,\mu}
  \end{align*}
  Comparing the right and left hand sides of this expression yields $h_{\mu\nu}'(x'(x)) = h_{\mu\nu}(x) + \xi_{\mu,\nu}(x) + \xi_{\nu,\mu}(x)$. To bring the dependency of $h_{\mu\nu}'$ to $x$ explicitly, we write the expansion $h_{\mu\nu}'(x'(x)) = h_{\mu\nu}'(x) + \xi^{\sigma}(x) h_{\mu\nu, \sigma}'(x)$ where the $\xi^{\sigma}(x) h_{\mu\nu, \sigma}'(x)$ term is second order in $\lambda$ and does not contribute so $h_{\mu\nu}'(x'(x)) = h_{\mu\nu}'(x)$.
  \item[(b)] Using the previous result, the gauge transformation of $\bar{h}_{\sigma \nu}$ to $\bar{h}_{\sigma \nu}'$ reads
  \begin{align*}
    \bar{h}_{\mu \nu}'(x) &= h_{\mu\nu}(x) + \xi_{\mu,\nu}(x) + \xi_{\nu,\mu}(x) - \frac{1}{2}\eta_{\mu \nu} \eta^{\sigma \rho} (h_{\sigma\rho}(x) + \xi_{\sigma,\rho}(x) + \xi_{\sigma,\rho}(x))\\
    &= \bar{h}_{\mu\nu}(x) + \xi_{\mu,\nu}(x) + \xi_{\nu,\mu}(x) - \eta_{\mu \nu} \xi_{\sigma,}{}^{\sigma}(x).
  \end{align*}
  Now we contract the $\mu$ index of $\bar{h}_{\mu \nu}$ with a derivative and get
  \begin{align*}
    \bar{h}'{}_{\mu \nu,}{}^{\mu}(x) &=  \bar{h}{}_{\mu \nu,}{}^{\mu}(x) + \xi_{\mu,\nu}{}^{\mu}(x) + \xi_{\nu,\mu}{}^{\mu}(x) - \eta_{\mu \nu}{}^{\mu} \xi_{\sigma,}{}^{\sigma}(x)\\
    &=  \bar{h}{}_{\mu \nu,}{}^{\mu}(x) + \xi_{\mu,\nu}{}^{\mu}(x) + \xi_{\nu,\mu}{}^{\mu}(x) -  \xi_{\sigma,\nu}{}^{\sigma}(x)\\
    &= \bar{h}{}_{\mu \nu,}{}^{\mu}(x) + \square \xi_{\nu}(x). 
  \end{align*} 
Choosing $\xi_\nu$ to make $\bar{h}'{}_{\mu \nu,}{}^{\mu}(x)$ vanish constitutes a choice of gauge called the \textit{De Donder gauge}. The coordinate transforms leading to this gauge are constrained by 
\begin{align*}
  \square \xi_{\nu}(x) = -\bar{h}{}_{\mu \nu,}{}^{\mu}(x) 
\end{align*}
which is a wave equation with $-\bar{h}{}_{\mu \nu,}{}^{\mu}(x)$ sources for each $\nu$. Given any starting $\bar{h}_{\mu \nu}$, we can compute the associated source and solve the wave quation to go to the De Donder gauge. 
\end{enumerate}

\section{Gravitomagnetism}
\begin{enumerate}
  \item[(a)]
  \item[(b)]
\end{enumerate}

\section{Acknowledgement}



% References
\makereferences
%-------------------------------------------------------


%%%%%%%%%%%%%%%%%%%%%%%%
% Terminer le document %
%%%%%%%%%%%%%%%%%%%%%%%%
\end{document}