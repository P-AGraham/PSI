\documentclass[10pt, a4paper]{article}

%%%%%%%%%%%%%%
%  Packages  %
%%%%%%%%%%%%%%


\usepackage{page_format}
\usepackage{special}
\usepackage{hyperref}
\usepackage{tikz}
\usepackage[compat=1.1.0]{tikz-feynman}
\usepackage[font=small,labelfont=bf,
   justification=justified,
   format=plain]{caption}
%----------------------------------------------------------------------
%\usepackage{amssymb} % Mathematical fonts.
%\usepackage{amsfonts} % Mathematical fonts.
\usepackage[nice]{nicefrac} % Nicer fractions
\usepackage{braket} % Dirac Notation.
\usepackage{bbm} % More bold fonts.
%\usepackage{mathrsfs} % Mathematical fonts.
\usepackage{esint} % Integrals
\usepackage{cancel} % Allows to scratch expressions.
\usepackage{mathtools} % Tools for math formating.
\usepackage{slashed} % Allows to slash individual characters.
\usepackage{xargs} % Better handling of optional arguments for commands
%----------------------------------------------------------------------
%\usepackage{lmodern} % Fonts.
\usepackage{feyn} % Feynman Diagrams in mathmode

%%%%%%%%%%%%%%%%%%%%%%%%%%%
% Mathématiques et physique
%%%%%%%%%%%%%%%%%%%%%%%%%%%%
% SI Units -----------------------
% The package 'siunitx' causes unresolved crashes (as of 22/08/31)
\newcommand{\ampere}{\text{A}}
\newcommand{\bell}{\text{B}}
\newcommand{\celsius}{\degree\text{C}}
\newcommand{\coulomb}{\text{C}}
\newcommand{\degree}{\,^{\circ}}
\newcommand{\farad}{\text{F}}
\newcommand{\electro}{\text{e}}
\newcommand{\gram}{\text{g}}
\newcommand{\henry}{\text{H}}
\newcommand{\hertz}{\text{Hz}}
\newcommand{\hour}{\text{h}}
\newcommand{\joule}{\text{J}}
\newcommand{\kelvin}{\text{K}}
\newcommand{\meter}{\text{m}}
\newcommand{\minute}{\text{m}}
\newcommand{\mole}{\text{mol}}
\newcommand{\newton}{\text{N}}
\newcommand{\ohm}{\Omega}
\newcommand{\pascal}{\text{Pa}}
\newcommand{\rad}{\text{rad}}
\newcommand{\second}{\text{s}}
\newcommand{\tesla}{\text{T}}
\newcommand{\torr}{\text{Torr}}
\newcommand{\volt}{\text{V}}
\newcommand{\watt}{\text{W}}
%
\newcommand{\tera}{\text{T}}
\newcommand{\giga}{\text{G}}
\newcommand{\mega}{~\text{M}}
\newcommand{\kilo}{~\text{k}}
\newcommand{\deci}{\text{d}}
\newcommand{\centi}{\text{c}}
\newcommand{\milli}{\text{m}}
\newcommand{\micro}{\mu}
\newcommand{\nano}{\text{n}}
\newcommand{\pico}{\text{p}}
\newcommand{\femto}{\text{f}}
%
\newcommand{\units}[1]{\text{#1}}
\newcommand{\tothe}[1]{\textsuperscript{#1}}
%
\newcommand{\per}{\text{/}}
%
\newcommand{\Time}[3]{#1\hour~#2\minute~#3\second} % TODO Optional arguments.
\newcommand{\Angle}[3]{#1^{\circ}~#2'~#3''} % TODO Optional arguments.


% Better epsilon -----------------------
\let\oldepsilon\epsilon
\let\epsilon\varepsilon
\let\varepsilon\oldepsilon


% Better \bar -----------------------
\renewcommand{\bar}[1]{\mkern 1.5mu\overline{\mkern-1.5mu#1\mkern-1.5mu}\mkern 1.5mu}


% Équations -----------------------
\newcommand{\al}[1]{\begin{align} #1 \end{align}} % Numbered equation(s),
\newcommand{\eqn}[1]{\begin{align*} #1 \end{align*}} % Number-less equation(s),
\newcommand{\sys}[1]{\begin{dcases*} #1 \end{dcases*}} % System of equations.


% Exponents -----------------------
\newcommand{\Exp}[1]{\text{e}^{#1}}		% e^#
\newcommand{\E}[1]{\times 10^{#1}}		% X 10^#


% Delimiters -----------------------
\newcommand{\p}[1]{\left( #1 \right)}	% (#)
\newcommand{\cro}[1]{\left[ #1 \right]}	% [#]
\newcommand{\abs}[1]{\left| #1\right|}	% |#|
\newcommand{\avg}[1]{\left\langle #1 \right\rangle} % <#>
\newcommand{\acc}[1]{\left\lbrace #1 \right\rbrace} % {#}


% Vectors -----------------------
\newcommand{\ve}[1]{\mathbf{#1}} % Upright bold face.
\newcommand{\vu}[1]{\hat{\ve{#1}}} % Hat vector upright bold face
\newcommand{\tens}{\otimes} % Tensor product
\newcommand{\nablav}{\bm{\nabla}} % Bold gradient


% Trig. functions with automatic formating  -----------------------
\newcommandx{\Sin}[2][1={}]{\text{sin}^{#1}\!\p{#2}}
\newcommandx{\Cos}[2][1={}]{\text{cos}^{#1}\!\p{#2}}
\newcommandx{\Tan}[2][1={}]{\text{tan}^{#1}\!\p{#2}}
\newcommandx{\Csc}[2][1={}]{\text{csc}^{#1}\!\p{#2}}
\newcommandx{\Sec}[2][1={}]{\text{sec}^{#1}\!\p{#2}}
\newcommandx{\Cot}[2][1={}]{\text{cot}^{#1}\!\p{#2}}
\newcommandx{\Arcsin}[2][1={}]{\text{arcsin}^{#1}\!\p{#2}}
\newcommandx{\Arccos}[2][1={}]{\text{arccos}^{#1}\!\p{#2}}
\newcommandx{\Arctan}[2][1={}]{\text{arctan}^{#1}\!\p{#2}}
\newcommandx{\Sinh}[2][1={}]{\text{sinh}^{#1}\!\p{#2}}
\newcommandx{\Cosh}[2][1={}]{\text{cosh}^{#1}\!\p{#2}}
\newcommandx{\Tanh}[2][1={}]{\text{tanh}^{#1}\!\p{#2}}


% Matrices -----------------------
\newcommand{\mat}[1]{\begin{bmatrix} #1 \end{bmatrix}} % Matrices with hooks.
\newcommand{\pmat}[1]{\begin{pmatrix} #1 \end{pmatrix}} % Matrices with parentheses.
\newcommand{\deter}[1]{\abs{\begin{matrix} #1 \end{matrix}}} % Determinant.
\newcommandx{\mO}[2][1={}, 2={}]{ \def\temp{#2}\ifx\temp\empty\ve{O}_{#1}\else\ve{O}_{#1\times #2}\fi}% Zero matrix.
\newcommandx{\mI}[2][1={}, 2={}]{ \def\temp{#2}\ifx\temp\empty\ve{I}_{#1}\else\ve{O}_{#1\times #2}\fi}%  Identity matrix.
\newcommand{\Det}[1]{\text{det}\p{#1}} % det(#)
\newcommand{\Tr}[1]{\text{Tr}\p{#1}} % Tr(#)


% Derivatives -----------------------
\newcommand{\D}{\text{d}} % Differential 'd'.
\newcommandx{\dd}[3][1={},3={}]{\frac{\D^{#3}#1}{\D{#2}^{#3}}} % Total derivative according to #2, #1 is the function and #3 is the order.
\newcommand{\del}{\partial} % Partial 'd'.
\newcommandx{\ddp}[3][1={},3={}]{\frac{\del^{#3}#1}{\del{#2}^{#3}}} % Dérivée partielle selon #2, #1 est la fonction est #3 est l'ordre.
\newcommand{\eval}[1]{\left. {#1} \right|} % Bar on the right of expression.
\newcommand{\delbar}{\slashed{\del}} % Partial Inexact differential.
\newcommand{\dbar}{\dj}% Inexact differential.


% Integrals -----------------------
\newcommand{\intinf}{\int\displaylimits_{-\infty}^{\infty}} % From -00 to 00.
\newcommandx{\Int}[2][1={},2={}]{\int\displaylimits_{#1}^{#2}} % Faster bounded integrals.


% Complex numbers -----------------------
\renewcommand{\Re}[1]{\text{Re}\acc{#1}} % Re{#}
\renewcommand{\Im}[1]{\text{Im}\acc{#1}} % Im{#}


% Sets -----------------------
\newcommand{\N}{\mathbbm{N}} % Natural numbers.
\newcommand{\Z}{\mathbbm{Z}} % Integers.
\newcommand{\Q}{\mathbbm{Q}} % Rational numbers.
\newcommandx{\R}[1][1={}]{\mathbbm{R}^{#1}} % Real numbers.
\newcommandx{\C}[1][1={}]{\mathbbm{C}^{#1}} % Complex numbers.
\newcommandx{\F}[1][1={}]{\mathbbm{F}^{#1}} % Some field.
\newcommand{\M}[3]{\mathbb{M}_{#1\times#2}(#3)}	% Matrices.
\newcommand{\Po}[2]{\mathbb{P}_{#1}(#2)} % Polynomials.
\newcommand{\Lin}{\mathbb{L}} % Linear maps.


% Constants and physical symbols -----------------------
\newcommand{\eo}{\epsilon_0} % epsilon 0.
\renewcommand{\L}{\mathcal{L}} % Lagrangian.

\usepackage{listings,xcolor}

% Default fixed font does not support bold face
\DeclareFixedFont{\ttb}{T1}{txtt}{bx}{n}{12} % for bold
\DeclareFixedFont{\ttm}{T1}{txtt}{m}{n}{12}  % for normal

\lstset{language=Mathematica}
\lstset{basicstyle={\sffamily\footnotesize},
  numbers=left,
  numberstyle=\tiny\color{gray},
  numbersep=5pt,
  breaklines=true,
  captionpos={t},
  frame={lines},
  rulecolor=\color{black},
  framerule=0.5pt,
  columns=flexible,
  tabsize=2
}

\usepackage{color}
\definecolor{deepblue}{rgb}{0,0,0.5}
\definecolor{deepred}{rgb}{0.6,0,0}
\definecolor{deepgreen}{rgb}{0,0.5,0}

\newcommand\pythonstyle{\lstset{
language=Python,
basicstyle=\footnotesize,
morekeywords={self},              % Add keywords here
keywordstyle=\color{deepblue},
emph={MyClass,__init__},          % Custom highlighting
emphstyle=\color{deepred},    % Custom highlighting style
stringstyle=\color{deepgreen},
frame=tb,                         % Any extra options here
showstringspaces=false
}}

\lstnewenvironment{python}[1][]
{
\pythonstyle
\lstset{#1}
}
{}

% References
\usepackage{biblatex}
\addbibresource{ref.bib}


%%%%%%%%%%%%
%  Colors  %
%%%%%%%%%%%%
% ! EDIT HERE !
\colorlet{chaptercolor}{red!70!black} % Foreground color.
\colorlet{chaptercolorback}{red!10!white} % Background color


%%%%%%%%%%%%%%
% Page titre %
%%%%%%%%%%%%%%
\title{Homework 2} % Title of the assignement.
\author{\PA} % Your name(s).
\teacher{Emilie Huffman and Giuseppe Sellaroli} % Your teacher's name.
\class{Statistical Physics} % The class title.

\university{Perimeter Institute for Theoretical Physics} % University
\faculty{Perimeter Scholars International} % Faculty
%\departement{<Departement>} % Departement
\date{\today} % Date.


%%%%%%%%%%%%%%%%%%%%%%
% Begin the document %
%%%%%%%%%%%%%%%%%%%%%%
\begin{document}

% Make the title page.
\maketitlepage

% Make table of contents
\maketableofcontents

% Assignment starts here ----------------------------

\footnotesize{
\section{Gaussian model for $T<T_c$}
\begin{enumerate}
  \item[(a)] The Ising model can be formulated as a statistical field theory on a lattice using the Hubbard–Stratonovich transformation. The auxiliary field takes values $\phi_i$ (organized in a vector $\phi$) on sites $i$ at position $x_i$ of a square lattice containing a total of $N$ sites. The connectivity of the Ising interaction is encoded in a translationaly symmetric matrix with elements $B_{ij} = B(x_i - x_j)$. At zero magnetic field, the theory is parameterized by inverse temperature $\beta$ and a constant $A$. The statistical field theory is given by the following partition function and action:
  \begin{align*}
    Z=\sqrt{\operatorname{det}\left(\frac{2 \beta A^2 B}{\pi}\right)} \int_{\mathbb{R}^N} \mathrm{~d}^N \phi e^{-S(\phi)}, \quad S(\phi)=\frac{\beta A^2}{2} \phi^{\mathrm{t}} B \phi-\sum_i \ln \left(\cosh \left(\beta A(B \phi)_i\right)\right).
  \end{align*}
  We are interested in the model resulting from a gaussian approximation of $S(\phi)$ around its minimizing field configuration given at each site by $\psi_j$. Up to second order in deviations form this minimum, the expansion of the discrete field reads
  \begin{align*}
    S(\phi) \approx S(\psi)+\frac{1}{2} \sum_{i, j}(\phi-\psi)_i(\phi-\psi)_j \frac{\partial^2 S}{\partial \phi_i \partial \phi_j}(\psi). 
  \end{align*}
  Because $B$ encodes translationaly symmetric nearest neighbour interaction, $B_{ij}$ can be expressed \cite{CitekeyBook} as a Fourier sum on a single crystal momentum $k$. We have 
  \begin{align*}
    B_{i, j} = \frac{1}{N}\sum_{k} B_k e^{-k \cdot (x_i - x_j)}  \implies \sum_{i} B_{i, j} = \frac{1}{N}\sum_{k} B_k e^{-k \cdot x_j} \left(\sum_i e^{-k \cdot x_i}\right) = \frac{1}{N}\sum_{k} B_k e^{-k \cdot x_j} N\delta_{k, 0} = B_0.
  \end{align*}
  where we used the exponential sum representation of the $\delta_{k, 0}$ \cite{CitekeyBook}. A similar result holds for the inverse of $B$. Indeed
  \begin{align*}
    (B^{-1})_{i, j} = \frac{1}{N} \sum_k \frac{1}{B_k} e^{i k \cdot\left(x_i-x_j\right)}  \implies \sum_{i} (B^{-1})_{i, j} = \frac{1}{N}\sum_{k} \frac{1}{B_k} e^{-k \cdot x_j} \left(\sum_i e^{-k \cdot x_i}\right) = \frac{1}{N}\sum_{k} \frac{1}{B_0} e^{-k \cdot x_j} N\delta_{k, 0} = \frac{1}{B_0}
  \end{align*}
  We note that $B_0$ is related to the critical temperature of the gaussian model by $B_0 = k_B T_c$.  
  \item[(b)] Using the fact $S(\phi)$ is a function of $N$ real variables $\phi_i$, we have that its minimal is realized for $\phi_i = \psi_i$ such that the derivatives $\left.\frac{\partial S}{\partial \phi}\right|_{\psi}$ simultaneously vanish. This corresponds to 
  \begin{align*}
    0 = \left.\dfrac{\partial S}{\partial \phi_k}\right|_{\psi} &= \frac{\beta A^2}{2} \sum_{i, j} \phi_j B_{ij} \delta_{ik} + \frac{\beta A^2}{2} \sum_{i, j} \delta_{jk} B_{ij} \phi_i -\sum_i \tanh \left(\beta A(B \phi)_i\right) \sum_j \beta A B_{ij} \delta_{jk}\\
    &=  \frac{\beta A^2}{2} \sum_{j} \phi_j B_{kj} + \frac{\beta A^2}{2} \sum_{i} B_{ik} \phi_i -\sum_i \tanh \left(\beta A(B \phi)_i\right) \sum_k \beta A B_{ik} \\
    &=  \beta A^2 \sum_{i} \phi_i B_{ki}- B_0 \beta A \sum_i \tanh \left(\beta A(B \phi)_i\right)\quad \text{(using (a) and $B_{i, j} = B_{j, i}$)}\\
    &\iff  \sum_{i} A \phi_i B_{ki} = B_0 \sum_i \tanh \left(A\frac{1}{T}(B \phi)_i\right).
  \end{align*}
  Since the right hand side of the previous expression is independant of $k$ the left hand side also is implying there exists a number $\bar{\psi}$ such that $B_0 \bar{\psi} = \sum_{i} \phi_i B_{ki}$. This constitutes a linear systems of equations and, since $B_{ki}$ has non-zero determinant, we have the solution $\phi_i = \sum_{k} B_0 \bar{\psi} (B^{-1})_{ki} = \bar{\psi} B_0/B_0$ (same minimizing field value at all sites). Denoting $M = A \bar{\psi}$ we recover the familiar mean field theory self-consistency relation 
  \begin{align*}
    M \sum_{i} B_{ki} = B_0 \sum_i \tanh \left(\frac{1}{k_B T}M \sum_j B_{ij}\right) \iff M = \sum_i \tanh \left(\frac{T_c}{T}M\right). 
  \end{align*}
  \item[(c)] 
  \item[(d)] 
  \item[(e)]
  \item[(f)]
\end{enumerate}

\section{Acknowledgement}

}

% References
\makereferences
%-------------------------------------------------------


%%%%%%%%%%%%%%%%%%%%%%%%
% Terminer le document %
%%%%%%%%%%%%%%%%%%%%%%%%
\end{document}