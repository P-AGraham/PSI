\documentclass[10pt, a4paper]{article}

%%%%%%%%%%%%%%
%  Packages  %
%%%%%%%%%%%%%%


\usepackage{page_format}
\usepackage{special}
\usepackage{hyperref}
\usepackage{tikz}
\usepackage[compat=1.1.0]{tikz-feynman}
\usepackage[font=small,labelfont=bf,
   justification=justified,
   format=plain]{caption}
\input{math_func}

\usepackage{listings,xcolor}

% Default fixed font does not support bold face
\DeclareFixedFont{\ttb}{T1}{txtt}{bx}{n}{12} % for bold
\DeclareFixedFont{\ttm}{T1}{txtt}{m}{n}{12}  % for normal

\lstset{language=Mathematica}
\lstset{basicstyle={\sffamily\footnotesize},
  numbers=left,
  numberstyle=\tiny\color{gray},
  numbersep=5pt,
  breaklines=true,
  captionpos={t},
  frame={lines},
  rulecolor=\color{black},
  framerule=0.5pt,
  columns=flexible,
  tabsize=2
}

\usepackage{color}
\definecolor{deepblue}{rgb}{0,0,0.5}
\definecolor{deepred}{rgb}{0.6,0,0}
\definecolor{deepgreen}{rgb}{0,0.5,0}

\newcommand\pythonstyle{\lstset{
language=Python,
basicstyle=\footnotesize,
morekeywords={self},              % Add keywords here
keywordstyle=\color{deepblue},
emph={MyClass,__init__},          % Custom highlighting
emphstyle=\color{deepred},    % Custom highlighting style
stringstyle=\color{deepgreen},
frame=tb,                         % Any extra options here
showstringspaces=false
}}

\lstnewenvironment{python}[1][]
{
\pythonstyle
\lstset{#1}
}
{}

% References
\usepackage{biblatex}
\addbibresource{ref.bib}


%%%%%%%%%%%%
%  Colors  %
%%%%%%%%%%%%
% ! EDIT HERE !
\colorlet{chaptercolor}{red!70!black} % Foreground color.
\colorlet{chaptercolorback}{red!10!white} % Background color


%%%%%%%%%%%%%%
% Page titre %
%%%%%%%%%%%%%%
\title{Homework 2} % Title of the assignement.
\author{\PA} % Your name(s).
\teacher{Emilie Huffman and Giuseppe Sellaroli} % Your teacher's name.
\class{Statistical Physics} % The class title.

\university{Perimeter Institute for Theoretical Physics} % University
\faculty{Perimeter Scholars International} % Faculty
%\departement{<Departement>} % Departement
\date{\today} % Date.


%%%%%%%%%%%%%%%%%%%%%%
% Begin the document %
%%%%%%%%%%%%%%%%%%%%%%
\begin{document}

% Make the title page.
\maketitlepage

% Make table of contents
\maketableofcontents

% Assignment starts here ----------------------------

\footnotesize{
\section{Gaussian model for $T<T_c$}
\begin{enumerate}
  \item[(a)] The Ising model can be formulated as a statistical field theory on a lattice using the Hubbard–Stratonovich transformation. The auxiliary field takes values $\phi_i$ (organized in a vector $\phi$) on sites $i$ at position $x_i$ of a square lattice containing a total of $N$ sites. The connectivity of the Ising interaction is encoded in a translationaly symmetric matrix with elements $B_{ij} = B(x_i - x_j)$. At zero magnetic field, the theory is parameterized by inverse temperature $\beta$ and a constant $A$. The statistical field theory is given by the following partition function and action:
  \begin{align*}
    Z=\sqrt{\operatorname{det}\left(\frac{2 \beta A^2 B}{\pi}\right)} \int_{\mathbb{R}^N} \mathrm{~d}^N \phi e^{-S(\phi)}, \quad S(\phi)=\frac{\beta A^2}{2} \phi^{\mathrm{t}} B \phi-\sum_i \ln \left(\cosh \left(\beta A(B \phi)_i\right)\right).
  \end{align*}
  We are interested in the model resulting from a gaussian approximation of $S(\phi)$ around its minimizing field configuration given at each site by $\psi_j$. Up to second order in deviations form this minimum, the expansion of the discrete field reads
  \begin{align*}
    S(\phi) \approx S(\psi)+\frac{1}{2} \sum_{i, j}(\phi-\psi)_i(\phi-\psi)_j \frac{\partial^2 S}{\partial \phi_i \partial \phi_j}(\psi). 
  \end{align*}
  Because $B$ encodes translationaly symmetric nearest neighbour interaction, $B_{ij}$ can be expressed \cite{CitekeyBook} as a Fourier sum on a single crystal momentum $k$. We have 
  \begin{align*}
    B_{i, j} = \frac{1}{N}\sum_{k} B_k e^{-k \cdot (x_i - x_j)}  \implies \sum_{i} B_{i, j} = \frac{1}{N}\sum_{k} B_k e^{-k \cdot x_j} \left(\sum_i e^{-k \cdot x_i}\right) = \frac{1}{N}\sum_{k} B_k e^{-k \cdot x_j} N\delta_{k, 0} = B_0.
  \end{align*}
  where we used the exponential sum representation of the $\delta_{k, 0}$ \cite{CitekeyBook}. A similar result holds for the inverse of $B$. Indeed
  \begin{align*}
    (B^{-1})_{i, j} = \frac{1}{N} \sum_k \frac{1}{B_k} e^{i k \cdot\left(x_i-x_j\right)}  \implies \sum_{i} (B^{-1})_{i, j} = \frac{1}{N}\sum_{k} \frac{1}{B_k} e^{-k \cdot x_j} \left(\sum_i e^{-k \cdot x_i}\right) = \frac{1}{N}\sum_{k} \frac{1}{B_0} e^{-k \cdot x_j} N\delta_{k, 0} = \frac{1}{B_0}
  \end{align*}
  We note that $B_0$ is related to the critical temperature of the gaussian model by $B_0 = k_B T_c$.  
  \item[(b)] Using the fact $S(\phi)$ is a function of $N$ real variables $\phi_i$, we have that its minimal is realized for $\phi_i = \psi_i$ such that the derivatives $\left.\frac{\partial S}{\partial \phi}\right|_{\psi}$ simultaneously vanish. This corresponds to 
  \begin{align*}
    0 = \left.\dfrac{\partial S}{\partial \phi_k}\right|_{\psi} &= \frac{\beta A^2}{2} \sum_{i, j} \phi_j B_{ij} \delta_{ik} + \frac{\beta A^2}{2} \sum_{i, j} \delta_{jk} B_{ij} \phi_i -\sum_i \tanh \left(\beta A(B \phi)_i\right) \sum_j \beta A B_{ij} \delta_{jk}\\
    &=  \frac{\beta A^2}{2} \sum_{j} \phi_j B_{kj} + \frac{\beta A^2}{2} \sum_{i} B_{ik} \phi_i -\sum_i \tanh \left(\beta A(B \phi)_i\right) \sum_k \beta A B_{ik} \\
    &=  \beta A^2 \sum_{i} \phi_i B_{ki}- B_0 \beta A \sum_i \tanh \left(\beta A(B \phi)_i\right)\quad \text{(using (a) and $B_{i, j} = B_{j, i}$)}\\
    &\iff  \sum_{i} A \phi_i B_{ki} = B_0 \sum_i \tanh \left(A\frac{1}{T}(B \phi)_i\right).
  \end{align*}
  Since the right hand side of the previous expression is independant of $k$ the left hand side also is implying there exists a number $\bar{\psi}$ such that $B_0 \bar{\psi} = \sum_{i} \phi_i B_{ki}$. This constitutes a linear systems of equations and, since $B_{ki}$ has non-zero determinant, we have the solution $\phi_i = \sum_{k} B_0 \bar{\psi} (B^{-1})_{ki} = \bar{\psi} B_0/B_0$ (same minimizing field value at all sites). Denoting $M = A \bar{\psi}$ we recover the familiar mean field theory self-consistency relation 
  \begin{align*}
    M \sum_{i} B_{ki} = B_0 \sum_i \tanh \left(\frac{1}{k_B T}M \sum_j B_{ij}\right) \iff M = \sum_i \tanh \left(\frac{T_c}{T}M\right). 
  \end{align*}
  \item[(c)] 
  \item[(d)] 
  \item[(e)]
  \item[(f)]
\end{enumerate}

\section{Acknowledgement}

}

% References
\makereferences
%-------------------------------------------------------


%%%%%%%%%%%%%%%%%%%%%%%%
% Terminer le document %
%%%%%%%%%%%%%%%%%%%%%%%%
\end{document}