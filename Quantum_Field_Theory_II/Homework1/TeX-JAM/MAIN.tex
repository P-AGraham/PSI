\documentclass[10pt, a4paper]{article}

%%%%%%%%%%%%%%
%  Packages  %
%%%%%%%%%%%%%%


\usepackage{page_format}
\usepackage{special}
\usepackage{hyperref}
\usepackage{tikz}
\usepackage[compat=1.1.0]{tikz-feynman}
\input{math_func}

\usepackage{slashed}

% References
\usepackage{biblatex}
\addbibresource{ref.bib}


%%%%%%%%%%%%
%  Colors  %
%%%%%%%%%%%%
% ! EDIT HERE !
\colorlet{chaptercolor}{red!70!black} % Foreground color.
\colorlet{chaptercolorback}{red!10!white} % Background color

%%%%%%%%%%%%%%
% Page titre %
%%%%%%%%%%%%%%
\title{Homework 1} % Title of the assignement.
\author{\PA} % Your name(s).
\teacher{Francois David and Dan Wohns} % Your teacher's name.
\class{Quantum Field Theory II} % The class title.

\university{Perimeter Institute for Theoretical Physics} % University
\faculty{Perimeter Scholars International} % Faculty
%\departement{<Departement>} % Departement
\date{\today} % Date.


%%%%%%%%%%%%%%%%%%%%%%
% Begin the document %
%%%%%%%%%%%%%%%%%%%%%%
\begin{document}

\section{Gross-Neveu Model}

\begin{enumerate}
  \item[(a)] The Gross-Neveu model is a $1+1$ dimensionnal model describing a collection of $N$ fermionic fields $\psi^a$ (and a supressed Dirac index ranging from $1$ to $2$) indexed by a color index $a$ ranging from $1$ to $N$. The fermions each contribute a free massless Dirac term $\bar{\psi}_a i \slashed{\partial} \psi_a$ (summation on $a$ is implied) and are coupled with an interaction term $g^2 (\psi_a \psi^a)/(2N)$ with coupling constant $g$. The full lagrangian density reads
  \begin{align*}
    \mathcal{L}_{\mathrm{GN}}=\bar{\psi}_a i \slashed{\partial} \psi^a+\frac{g^2}{2 N}\left(\bar{\psi}_a \psi^a\right)^2. 
  \end{align*}
  We work in natural units where $\hbar = 1, c = 1$ making all quantitities of interest gain dimension of a power of mass. For action, length and time these powers are respectively $0$, $-1$ and $-1$. Since the action is dimensionless and obtained by integrating the lagrangian density $\mathcal{L}_{\mathrm{GN}}$ over $1+1$ spacetime dimensions, we have $0 = [\mathcal{L}_{\mathrm{GN}}] +  [dx^2] = [\mathcal{L}_{\mathrm{GN}}] - 2 \implies [\mathcal{L}_{\mathrm{GN}}] = 2$. Knowing the dimension of the lagrangian density, we use the kinetic term to compute the dimension of the fields. Since the derivatives have the inverse dimension of their associated variables, we have $2 = 2[\psi^a] + [\slashed{\partial}] = 2[\psi^a] + 1 \implies [\psi^a] = 1/2$. Finally, the dimension of the coupling constant can be extracted using the interaction term. Indeed,  $2 = 4 [\psi^a] + 2 [g] = 2 + 2 [g] \implies [g] = 0$ making the coupling constant dimensionless. 
  \item[(b)] The slashed derivative $\slashed{\partial}$ appearing in the lagrangian density is the result of a contraction of the $\gamma^\mu = (\gamma^0, \gamma^1)$ with the partial derivatives $\partial_\mu = (\partial_0, \partial_1)$. In $1+1$ dimensionnal spacetime we can form a clifford algebra with $\gamma^0 = \sigma^2, \gamma^1 = i \sigma^1$ where $\sigma^2, \sigma^1$ are Pauli matrices. We also have $\gamma^5 = \gamma^0 \gamma^1 = \sigma^3$. Following the properties of clifford algebras, we have that $\gamma^5$ anticommutes with all $\gamma^\mu$ matrices. All the $\gamma^\mu$ matrices are multiplied together in $\gamma^5$ and any $\gamma^\mu$ commutes with itself, but anticommutes with the remaining matrix in the product leading to $\{\gamma^5, \gamma^\mu\} = 0$. The Dirac adjoint of the fermionic fields appearing in the lagrangian density is defined to be $\bar{\psi}_a = \psi^{a\dagger} \gamma^0$.  We relate $\psi^{a}$ and a transformed field $\phi^{a}$ by $\psi^{a} = \gamma^5 \phi^{a}$. In terms of the transformed field, the  Dirac adjoint of $\psi^{a}$ becomes $\bar{\psi}_a = \phi^{a\dagger}\gamma^{5\dagger} \gamma^0 = -\phi^{a\dagger} \gamma^0 \gamma^{5} = -\bar{\phi}_a \gamma^5$ because $\gamma^5 = \sigma^3$ is real. Using these expressions, we can express $\mathcal{L}_{\mathrm{GN}}$ in terms of the transformed field as follows: 
  \begin{align*}
    \mathcal{L}_{\mathrm{GN}}=-\bar{\phi}_a \gamma^5 i \slashed{\partial} \gamma^5 \phi^a+\frac{g^2}{2 N}\left(-\bar{\phi}_a \gamma^5 \gamma^5 \phi^a\right)^2 = \bar{\phi}_a i \slashed{\partial}  \phi^a+\frac{g^2}{2 N}\left(\bar{\phi}_a \phi^a\right)^2  
  \end{align*}
  where we used $\{\gamma^5, \gamma^\mu\} = 0$ in the first term and $\gamma^5 = 1$ in the second term. We notice that the new lagrangian density is identical to the $\psi^a$ one and the transformation from $\psi^a$ to $\phi^a$ is a discrete symmetry. A mass $m$ term $m \bar{\psi}_a \psi^a$ would transform to $-m \bar{\phi}_a \gamma^5 \gamma^5 \phi^a = -m \bar{\phi}_a \phi^a$ which is not identical to the initial term and therefore breaks the considered symmetry. 
  \item[(c)]
  \item[(d)]
  \item[(e)]
  \item[(f)] 
  \item[(g)]
  \item[(h)]
\end{enumerate}

% Make the title page.
\maketitlepage

% Make table of contents
\maketableofcontents

% Assignment starts here ----------------------------



\section{Acknowledgement}
I worked on my own for this assignment.


% References
\makereferences
%-------------------------------------------------------


%%%%%%%%%%%%%%%%%%%%%%%%
% Terminer le document %
%%%%%%%%%%%%%%%%%%%%%%%%
\end{document}